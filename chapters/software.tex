\clearpage
\chapter{Software}
\label{sec:software}

    This chapter presents the software developed within Exa-MA, focusing on features, mathematics, functionalities, publications, acknowledgments, and contact details.

    First, we present some general statistics about the software identified for Exa-MA that will present some benchmarking results. 
    Other software (new or existing) will be included in the future version of this document.
    
        \paragraph{Architectures}
        The following pie chart~\ref{fig:arch} shows the distribution of hardware architectures used.
        \begin{figure}[H]
\centering
\begin{tikzpicture}
\pie[text=legend, color={red, orange}, sum=auto]{13/CPU, 7/GPU}
\end{tikzpicture}
\caption{Distribution of hardware architectures}
\label{fig:arch}
\end{figure}

        \paragraph{Programming Languages}
        The following pie chart~\ref{fig:languages} shows the distribution of programming languages used, highlighting the variety of computational solutions employed.
        \begin{figure}[H]
\centering
\begin{tikzpicture}
\pie[text=legend, color={red, orange, yellow, lime, skyblue}, sum=auto]{15/C++, 1/C\#, 3/C, 4/Fortran, 4/Python}
\end{tikzpicture}
\caption{Distribution of programming languages}
\label{fig:languages}
\end{figure}

        \paragraph{Parallelism Technology}
        The pie chart~\ref{fig:parallelism} below represents the parallelism techniques used in Exa-MA software selected for this document.
        \begin{figure}[H]
\centering
\begin{tikzpicture}
\pie[text=legend, color={red, orange, yellow, lime, skyblue, pink}, sum=auto]{7/Multithread, 13/MPI, 6/GPU, 1/Parallelism - C++, 1/Task based, 1/Chapel}
\end{tikzpicture}
\caption{Distribution of parallelism technologies}
\label{fig:parallelism}
\end{figure}

        \paragraph{Data Formats}
        The chart~\ref{fig:data} shows the supported data formats, for flexibility and compatibility in data handling, supported by Exa-MA software selected for this document.
        \begin{figure}[H]
\centering
\begin{tikzpicture}
\pie[text=legend, color={red, orange, yellow, lime, skyblue, pink, cyan, magenta, peach, lavender}, sum=auto]{2/XML, 5/HDF5, 2/JSON, 2/Ensight, 5/None, 1/YAML, 1/Data-management system, 6/VTK, 6/in-house format, 4/Gmsh and associated formats, 2/MED, 1/MFront}
\end{tikzpicture}
\caption{Distribution of data formats}
\label{fig:data}
\end{figure}

        \paragraph{DevOps - CI/CD}
        The pie chart~\ref{fig:devops-cicd} below displays the support of continuous integration and deployment practices as well as continuous benchmarking, showcasing systematic software updates, quality maintenance and performance regression.
        \begin{figure}[H]
\centering
\begin{tikzpicture}
\pie[text=legend, color={red, orange, yellow, lime}, sum=auto]{11/Continuous Integration, 1/Continuous Benchmarking, 1/Continuous Delivery, 4/None}
\end{tikzpicture}
\caption{Distribution of DevOps CI/CD/CD}
\label{fig:devops-cicd}
\end{figure}

        \paragraph{DevOps - Packaging}
        The next chart~\ref{fig:devops-packaging} shows different packaging methods used, which help in the distribution and management of software.
        \begin{figure}[H]
\centering
\begin{tikzpicture}
\pie[text=legend, color={red, orange, yellow, lime, skyblue, pink}, sum=auto]{10/None, 5/Debian-based, 2/Fedora, 2/Spack, 1/GUIX, 1/Other}
\end{tikzpicture}
\caption{Distribution of DevOps Packaging}
\label{fig:devops-packaging}
\end{figure}

        \paragraph{DevOps - Containers}
        The pie chart~\ref{fig:devops-containers} displays the use of container technologies, which help encapsulate the software to run reliably in various environments.
        \begin{figure}[H]
\centering
\begin{tikzpicture}
\pie[text=legend, color={red, orange, yellow}, sum=auto]{12/None, 2/Apptainer/Singularity, 2/Docker}
\end{tikzpicture}
\caption{Distribution of DevOps Containers}
\label{fig:devops-containers}
\end{figure}

        \paragraph{DevOps - Testing}
        The following pie chart~\ref{fig:devops-testing} details the testing practices adopted, illustrating the commitment to software reliability and functionality.
        \begin{figure}[H]
\centering
\begin{tikzpicture}
\pie[text=legend, color={red, orange, yellow, lime}, sum=auto]{8/Unit, 7/Verification, 3/Validation, 6/None}
\end{tikzpicture}
\caption{Distribution of DevOps Testing}
\label{fig:devops-testing}
\end{figure}
\section{Software: Arcane Framework}
\label{sec:Arcane Framework:software}



\begin{table}[h!]
    \centering
    { \setlength{\parindent}{0pt}
    \def\arraystretch{1.25}
    \arrayrulecolor{numpexgray}
    {\fontsize{9}{11}\selectfont
    \begin{tabular}{!{\color{numpexgray}\vrule}p{.4\textwidth}!{\color{numpexgray}\vrule}p{.6\textwidth}!{\color{numpexgray}\vrule}}
        \rowcolor{numpexgray}{\rule{0pt}{2.5ex}\color{white}\bf Field} & {\rule{0pt}{2.5ex}\color{white}\bf Details} \\
        \rowcolor{white}\textbf{Consortium} & \begin{tabular}{l}
CEA\\
IFPEN\\
\end{tabular} \\
        \rowcolor{numpexlightergray}\textbf{Exa-MA Partners} & \begin{tabular}{l}
CEA\\
\end{tabular} \\
        \rowcolor{white}\textbf{Contact Emails} & \begin{tabular}{l}
lydie.grospellier@cea.fr\\
\end{tabular} \\
        \rowcolor{numpexlightergray}\textbf{Supported Architectures} & \begin{tabular}{l}
CPU\\
GPU\\
\end{tabular} \\
        \rowcolor{white}\textbf{Repository} & \href{https://github.com/arcaneframework/framework}{https://github.com/arcaneframework/framework} \\
        \rowcolor{numpexlightergray}\textbf{License} & \begin{tabular}{l}
OSS: apache-2\\
\end{tabular} \\
        \bottomrule
    \end{tabular}
    }}
    \caption{Arcane Framework Information}
\end{table}

\subsection{Software summary}
\label{sec:Arcane Framework:summary}
Detailed overview not available.



\subsection{Purpose}
\label{sec:Arcane Framework:purpose}
Purpose not available.

\subsection{Programming and Computational Environment}
\label{sec::Arcane Framework:environment_capabilities}


The following table summarizes these aspects for Arcane Framework, providing a  view of its programming and computational capabilities.

\begin{table}[h!]
    \centering
    {
    \setlength{\parindent}{0pt}
    \def\arraystretch{1.25}
    \arrayrulecolor{numpexgray}
    {\fontsize{9}{11}\selectfont
    \begin{tabular}{lp{.3\textwidth}p{.5\textwidth}}
        \rowcolor{numpexgray}{\rule{0pt}{2.5ex}\color{white}\bf Category}  & {\rule{0pt}{2.5ex}\color{white}\bf Details} & {\rule{0pt}{2.5ex}\color{white}\bf Description}\\
        \rowcolor{white}Languages  & \begin{tabular}{l}
C++\\
C\#\\
\end{tabular} & Programming languages and language standards supported by the software \\
        \rowcolor{numpexlightergray}Parallelism  & \begin{tabular}{l}
GPU\\
MPI\\
Multithread\\
\end{tabular} & Parallel computing methods and frameworks utilized by the software.\\
        \rowcolor{white}Data Formats  & \begin{tabular}{l}
Ensight\\
HDF5\\
JSON\\
XML\\
\end{tabular} & Data formats that the software can handle or produce.\\
        \rowcolor{numpexlightergray}Resilience  & \begin{tabular}{l}
Checkpoint restart\\
\end{tabular} & Fault tolerance and recovery mechanisms employed by the software.\\
        \rowcolor{white}DevOps & \begin{tabular}{l}
Continuous Integration\\
\end{tabular} & Outlines the development and operational practices including continuous integration, containerization, and testing methodologies.  \\
        \rowcolor{numpexlightergray}Packaging  & \begin{tabular}{l}
None\\
\end{tabular} & Software packaging and distribution.\\
        \rowcolor{white}Testing  & \begin{tabular}{l}
Unit\\
Verification\\
\end{tabular} & Testing methodologies employed to ensure software quality and correctness.\\
        \rowcolor{numpexlightergray}Containerization  & \begin{tabular}{l}
None\\
\end{tabular} & Container technologies used to package and deploy the software.\\
        \rowcolor{white}Interfaces  & \begin{tabular}{l}
None\\
\end{tabular} & List of software Arcane Framework has interfaces with.\\
        \bottomrule
    \end{tabular}
    }}
    \caption{Arcane Framework programming and computational environment}
\end{table}



\subsection{Mathematics}
\label{sec:Arcane Framework:mathematics}
Mathematics not available.

In this section, provide a summary the mathematics used in the software.


\subsection{Relevant Publications}
\label{sec:Arcane Framework:publications}

Here is a list of relevant publications related to the software:


\subsection{Acknowledgements}
\label{sec::Arcane Framework:acknowledgements}

The software has been developed with the support of the following funding agencies and institutions: 




Acknowledgements not available.



\section{Software: CGAL}
\label{sec:CGAL:software}



\begin{table}[h!]
    \centering
    { \setlength{\parindent}{0pt}
    \def\arraystretch{1.25}
    \arrayrulecolor{numpexgray}
    {\fontsize{9}{11}\selectfont
    \begin{tabular}{!{\color{numpexgray}\vrule}p{.4\textwidth}!{\color{numpexgray}\vrule}p{.6\textwidth}!{\color{numpexgray}\vrule}}
        \rowcolor{numpexgray}{\rule{0pt}{2.5ex}\color{white}\bf Field} & {\rule{0pt}{2.5ex}\color{white}\bf Details} \\
        \rowcolor{white}\textbf{Consortium} & \begin{tabular}{l}
Inria\\
\end{tabular} \\
        \rowcolor{numpexlightergray}\textbf{Exa-MA Partners} & \begin{tabular}{l}
Inria CA\\
\end{tabular} \\
        \rowcolor{white}\textbf{Contact Emails} & \begin{tabular}{l}
christos.georgiadis@inria.fr\\
pierre.alliez@inria.fr\\
\end{tabular} \\
        \rowcolor{numpexlightergray}\textbf{Supported Architectures} & \begin{tabular}{l}
CPU Only\\
\end{tabular} \\
        \rowcolor{white}\textbf{Repository} & \href{https://github.com/CGAL}{https://github.com/CGAL} \\
        \rowcolor{numpexlightergray}\textbf{License} & \begin{tabular}{l}
OSS:: GPL v*\\
OSS:: LGPL v*\\
\end{tabular} \\
        \rowcolor{white}\textbf{Bottlenecks roadmap} & \begin{tabular}{l}
B10 - Scientific Productivity\\
B11 - Reproducibility and Replicability of Computation\\
B6 - Data Management\\
B7 - Exascale Algorithms\\
\end{tabular} \\
        \bottomrule
    \end{tabular}
    }}
    \caption{CGAL Information}
\end{table}

\subsection{Software summary}
\label{sec:CGAL:summary}
%Detailed overview not available.

CGAL (Computational Geometry Algorithms Library) is an open-source software project designed to provide 
numerically reliable software components (algorithms and geometric data structures) for use in 2D, 3D, or arbitrary dimensions.
These components include convex hulls, triangulations, Boolean operations, intersection calculations, mesh generation, 
3D point cloud processing, and more. CGAL’s main design features, along with various components, are utilized in industrial 
robotics and digital engineering simulations. For industrial applications, CGAL offers reliable, interoperable components 
that save development time by eliminating the need to reinvent the wheel, allowing users to focus on the business specializations 
that deliver the most value.




\subsection{Purpose}
\label{sec:CGAL:purpose}
%Purpose not available.

The purpose of CGAL is to offer developers tools for solving complex geometric problems in the form of a C++ templated library. 
CGAL is used in fields like CAD, robotics, and scientific computing, offering components for tasks like mesh generation, 
spatial searching, and geometry processing. CGAL is available under a dual licensing scheme: for integration into other open-source software, 
it is provided under LGPL or GPL licenses, depending on the component. For proprietary or commercial projects, 
licenses can be purchased, with options tailored for academic, research, or industrial customers.

\subsection{Programming and Computational Environment}
\label{sec::CGAL:environment_capabilities}

%christos todo 


The following table summarizes these aspects for CGAL, providing a  view of its programming and computational capabilities.

\begin{table}[h!]
    \centering
    {
    \setlength{\parindent}{0pt}
    \def\arraystretch{1.25}
    \arrayrulecolor{numpexgray}
    {\fontsize{9}{11}\selectfont
    \begin{tabular}{lp{.3\textwidth}p{.5\textwidth}}
        \rowcolor{numpexgray}{\rule{0pt}{2.5ex}\color{white}\bf Category}  & {\rule{0pt}{2.5ex}\color{white}\bf Details} & {\rule{0pt}{2.5ex}\color{white}\bf Description}\\
        \rowcolor{white}Languages  & \begin{tabular}{l}
C++\\
\end{tabular} & Programming languages and language standards supported by the software \\
        \rowcolor{numpexlightergray}Parallelism  & \begin{tabular}{l}
Multithread\\
\end{tabular} & Parallel computing methods and frameworks utilized by the software.\\
        \rowcolor{white}Data Formats  & \begin{tabular}{l}
None\\
\end{tabular} & Data formats that the software can handle or produce.\\
        \rowcolor{numpexlightergray}Resilience  & \begin{tabular}{l}
None\\
\end{tabular} & Fault tolerance and recovery mechanisms employed by the software.\\
        \rowcolor{white}DevOps & \begin{tabular}{l}
Continuous Integration\\
\end{tabular} & Outlines the development and operational practices including continuous integration, containerization, and testing methodologies.  \\
        \rowcolor{numpexlightergray}Packaging  & \begin{tabular}{l}
Debian\\
Fedora\\
Spack\\
Ubuntu\\
\end{tabular} & Software packaging and distribution.\\
        \rowcolor{white}Testing  & \begin{tabular}{l}
Unit\\
\end{tabular} & Testing methodologies employed to ensure software quality and correctness.\\
        \rowcolor{numpexlightergray}Containerization  & \begin{tabular}{l}
None\\
\end{tabular} & Container technologies used to package and deploy the software.\\
        \rowcolor{white}Interfaces  & \begin{tabular}{l}
None\\
\end{tabular} & List of software CGAL has interfaces with.\\
        \bottomrule
    \end{tabular}
    }}
    \caption{CGAL programming and computational environment}
\end{table}



\subsection{Mathematics}
\label{sec:CGAL:mathematics}
%Mathematics not available.
%
%In this section, provide a summary the mathematics used in the software.


Numerical robustness is a fundamental concern in geometric computing, even more so than in other types of numerical methods. 
In geometric algorithms, slight inaccuracies in numerical computations can lead to significant errors, such as incorrect topological 
configurations or degeneracies that disrupt the algorithm's logic. 
The CGAL offers a flexible and powerful solution to this problem by following the exact computation paradigm (\url{https://www.cgal.org/exact.html}),
enabling users to avoid rounding errors and ensure robust algorithms.

CGAL leverages generic programming, enabling the use of different components through a C++ templated environment. 
This approach allows algorithms and data structures to be flexible and reusable across various geometric scenarios. 
For instance, a 2D convex hull algorithm in CGAL can be applied to an arbitrary 3D plane by utilizing appropriate templated parameters. 
This templated design ensures that developers can extend or customize components easily to 
meet specific requirements.


\subsection{Relevant Publications}
\label{sec:CGAL:publications}

%Here is a list of relevant publications related to the software:

%christos todo: add items in zotero and cite them here (also links to CGAL packages)

\url{https://inria.hal.science/hal-01071759}

\url{https://inria.hal.science/hal-03688637}


\subsection{Acknowledgements}
\label{sec::CGAL:acknowledgements}

%The software has been developed with the support of the following funding agencies and institutions: 
%
%
%Acknowledgements not available.

%\url{https://www.cgal.org/partners.html}


CGAL has been originally funded by European Union's information technologies programme Esprit, by Project 21957 - CGAL, 
with the project partners Utrecht University (The Netherlands), ETH Zurich (Switzerland), Freie Universitaet Berlin (Germany), 
INRIA Sophia-Antipolis (France), Martin-Luther-University Halle-Wittenberg (Germany), Max-Planck-Institute Saarbruecken (Germany), 
RISC Linz (Austria), and Tel-Aviv University (Israel).

Following that, CGAL has been supported by several European Research Programs:
\begin{itemize}
\item GALIA - Project 28155 - GALIA.
\item ECG -  Project IST-2000-26473 - ECG.
\item ACS - Project IST-006413 - ACS
\item Aim@Shape - Project IST NoE-506766
\item GUDHI - FP7-IDEAS-ERC 339025
\end{itemize}

Commercial licenses to CGAL are provided by the GeometryFactory spin-off company.



\section{Software: Composyx}
\label{sec:Composyx:software}

\begin{itemize}
    \item \textbf{Contact Email(s):} gilles.marait@inria.fr
    \item \textbf{Supported Architecture(s):} HYBRID
    \item \textbf{Repository Link:} \href{https://gitlab.inria.fr/composyx/composyx}{https://gitlab.inria.fr/composyx/composyx}
\end{itemize}

\subsection{Software summary}
\label{sec:Composyx:summary}
Detailed overview not available.



\subsection{Purpose}
\label{sec:Composyx:purpose}
Purpose not available.



\subsection{Mathematics}
\label{sec:Composyx:mathematics}
Mathematics not available.


\subsection{Relevant Publications}
\label{sec:Composyx:publications}

\subsection{Acknowledgements}
\label{sec::Composyx:acknowledgements}

Acknowledgements not available.



\section{Software: Feel++}
\label{sec:Feel++:software}



\begin{table}[h!]
    \centering
    { \setlength{\parindent}{0pt}
    \def\arraystretch{1.25}
    \arrayrulecolor{numpexgray}
    {\fontsize{9}{11}\selectfont
    \begin{tabular}{!{\color{numpexgray}\vrule}p{.4\textwidth}!{\color{numpexgray}\vrule}p{.6\textwidth}!{\color{numpexgray}\vrule}}
        \rowcolor{numpexgray}{\rule{0pt}{2.5ex}\color{white}\bf Field} & {\rule{0pt}{2.5ex}\color{white}\bf Details} \\
        \rowcolor{white}\textbf{Consortium} & \begin{tabular}{l}
Feel++ Consortium\\
\end{tabular} \\
        \rowcolor{numpexlightergray}\textbf{Exa-MA Partners} & \begin{tabular}{l}
CNRS\\
Inria Grenoble\\
Unistra\\
\end{tabular} \\
        \rowcolor{white}\textbf{Contact Emails} & \begin{tabular}{l}
christophe.prudhomme@cemosis.fr\\
vincent.chabannes@cemosis.fr\\
\end{tabular} \\
        \rowcolor{numpexlightergray}\textbf{Supported Architectures} & \begin{tabular}{l}
CPU Only\\
\end{tabular} \\
        \rowcolor{white}\textbf{Repository} & \href{https://github.com/feelpp/feelpp}{https://github.com/feelpp/feelpp} \\
        \rowcolor{numpexlightergray}\textbf{License} & \begin{tabular}{l}
OSS:: GPL v*\\
OSS:: LGPL v*\\
\end{tabular} \\
        \rowcolor{white}\textbf{Bottlenecks roadmap} & \begin{tabular}{l}
B10 - Scientific Productivity\\
B11 - Reproducibility and Replicability of Computation\\
B12 - Pre/Post Processing and In-Situ Processing\\
B2 - Interconnect Technology\\
B6 - Data Management\\
B7 - Exascale Algorithms\\
\end{tabular} \\
        \bottomrule
    \end{tabular}
    }}
    \caption{Feel++ Information}
\end{table}

\subsection{Software summary}
\label{sec:Feel++:summary}
Detailed overview not available.



\subsection{Purpose}
\label{sec:Feel++:purpose}
Purpose not available.

\subsection{Programming and Computational Environment}
\label{sec::Feel++:environment_capabilities}


The following table summarizes these aspects for Feel++, providing a  view of its programming and computational capabilities.

\begin{table}[h!]
    \centering
    {
    \setlength{\parindent}{0pt}
    \def\arraystretch{1.25}
    \arrayrulecolor{numpexgray}
    {\fontsize{9}{11}\selectfont
    \begin{tabular}{lp{.3\textwidth}p{.5\textwidth}}
        \rowcolor{numpexgray}{\rule{0pt}{2.5ex}\color{white}\bf Category}  & {\rule{0pt}{2.5ex}\color{white}\bf Details} & {\rule{0pt}{2.5ex}\color{white}\bf Description}\\
        \rowcolor{white}Languages  & \begin{tabular}{l}
C++\\
C++17\\
C++20\\
Python\\
\end{tabular} & Programming languages and language standards supported by the software \\
        \rowcolor{numpexlightergray}Parallelism  & \begin{tabular}{l}
MPI\\
Parallelism - C++\\
Task based\\
\end{tabular} & Parallel computing methods and frameworks utilized by the software.\\
        \rowcolor{white}Data Formats  & \begin{tabular}{l}
Data-management system\\
Ensight\\
Gmsh and associated formats\\
HDF5\\
JSON\\
VTK\\
VTK\\
YAML\\
in-house format\\
\end{tabular} & Data formats that the software can handle or produce.\\
        \rowcolor{numpexlightergray}Resilience  & \begin{tabular}{l}
Checkpoint restart\\
\end{tabular} & Fault tolerance and recovery mechanisms employed by the software.\\
        \rowcolor{white}DevOps & \begin{tabular}{l}
Continuous Benchmarking\\
Continuous Delivery\\
Continuous Integration\\
\end{tabular} & Outlines the development and operational practices including continuous integration, containerization, and testing methodologies.  \\
        \rowcolor{numpexlightergray}Packaging  & \begin{tabular}{l}
Debian\\
Fedora\\
Spack\\
Ubuntu\\
\end{tabular} & Software packaging and distribution.\\
        \rowcolor{white}Testing  & \begin{tabular}{l}
Unit\\
Validation\\
Verification\\
\end{tabular} & Testing methodologies employed to ensure software quality and correctness.\\
        \rowcolor{numpexlightergray}Containerization  & \begin{tabular}{l}
Docker\\
Singularity\\
\end{tabular} & Container technologies used to package and deploy the software.\\
        \rowcolor{white}Interfaces  & \begin{tabular}{l}
Dymola/OpenModelica/FMU\\
HPdomain decomposition methods\\
MMG/ParMMG\\
OpenTurns\\
PETSc\\
Salome\\
\end{tabular} & List of software Feel++ has interfaces with.\\
        \bottomrule
    \end{tabular}
    }}
    \caption{Feel++ programming and computational environment}
\end{table}



\subsection{Mathematics}
\label{sec:Feel++:mathematics}
Mathematics not available.

In this section, provide a summary the mathematics used in the software.


\subsection{Relevant Publications}
\label{sec:Feel++:publications}

Here is a list of relevant publications related to the software:


\subsection{Acknowledgements}
\label{sec::Feel++:acknowledgements}

The software has been developed with the support of the following funding agencies and institutions: 




Acknowledgements not available.



\section{Software: FreeFem++}
\label{sec:FreeFem++:software}



\begin{itemize}
    \item \textbf{Contact Email(s):} frederic.hecht@sorbonne-universite.fr, pierre.jolivet@sorbonne-universite.fr
    \item \textbf{Supported Architecture(s):} CPU
    \item \textbf{Repository Link:} \href{https://github.com/FreeFem/FreeFem-sources}{https://github.com/FreeFem/FreeFem-sources}
\end{itemize}

\subsection{Software summary}
\label{sec:FreeFem++:summary}
Detailed overview not available.



\subsection{Purpose}
\label{sec:FreeFem++:purpose}
Purpose not available.



\subsection{Mathematics}
\label{sec:FreeFem++:mathematics}
Mathematics not available.


\subsection{Relevant Publications}
\label{sec:FreeFem++:publications}

\subsection{Acknowledgements}
\label{sec::FreeFem++:acknowledgements}

Acknowledgements not available.



\section{Software: Hawen}
\label{sec:Hawen:software}



\begin{table}[h!]
    \centering
    { \setlength{\parindent}{0pt}
    \def\arraystretch{1.25}
    \arrayrulecolor{numpexgray}
    {\fontsize{9}{11}\selectfont
    \begin{tabular}{!{\color{numpexgray}\vrule}p{.4\textwidth}!{\color{numpexgray}\vrule}p{.6\textwidth}!{\color{numpexgray}\vrule}}
        \rowcolor{numpexgray}{\rule{0pt}{2.5ex}\color{white}\bf Field} & {\rule{0pt}{2.5ex}\color{white}\bf Details} \\
        \rowcolor{white}\textbf{Consortium} & \begin{tabular}{l}
Inria\\
\end{tabular} \\
        \rowcolor{numpexlightergray}\textbf{Exa-MA Partners} & \begin{tabular}{l}
Inria BXSO\\
\end{tabular} \\
        \rowcolor{white}\textbf{Contact Emails} & \begin{tabular}{l}
florian.faucher@inria.fr\\
\end{tabular} \\
        \rowcolor{numpexlightergray}\textbf{Supported Architectures} & \begin{tabular}{l}
CPU Only\\
\end{tabular} \\
        \rowcolor{white}\textbf{Repository} & \href{https://gitlab.com/ffaucher/hawen}{https://gitlab.com/ffaucher/hawen} \\
        \rowcolor{numpexlightergray}\textbf{License} & \begin{tabular}{l}
OSS:: GPL v*\\
\end{tabular} \\
        \rowcolor{white}\textbf{Bottlenecks roadmap} & \begin{tabular}{l}
B10 - Scientific Productivity\\
B11 - Reproducibility and Replicability of Computation\\
B6 - Data Management\\
B7 - Exascale Algorithms\\
\end{tabular} \\
        \bottomrule
    \end{tabular}
    }}
    \caption{Hawen Information}
\end{table}

% ---------------------------------------------
\newcommand{\hawen}{\textsc{Hawen}}
\subsection{Software summary}
\label{sec:Hawen:summary}
% ---------------------------------------------

Software \hawen~(\url{https://ffaucher.gitlab.io/hawen-website/})
considers the time-harmonic modeling of mechanical waves, and the 
associated quantitative inverse wave problems.
The code uses the Hybridizable Discontinuous Galerkin method for the 
discretization. 
It relies on iterative minimization algorithm to solve the quantitative
inverse wave problem via nonlinear optimization.



\subsection{Purpose}
\label{sec:Hawen:purpose}

The major applications of \hawen~are to solve large-scale inverse 
problems, in particular in the context of Earth's imaging and 
helioseismology.


\subsection{Programming and Computational Environment}
\label{sec::Hawen:environment_capabilities}


The following table~\ref{table:hawen-environment} summarizes these aspects for \hawen, providing a  view of its programming and computational capabilities.

\begin{table}[h!]
    \centering
    {
    \setlength{\parindent}{0pt}
    \def\arraystretch{1.25}
    \arrayrulecolor{numpexgray}
    {\fontsize{9}{11}\selectfont
    \begin{tabular}{lp{.3\textwidth}p{.5\textwidth}}
        \rowcolor{numpexgray}{\rule{0pt}{2.5ex}\color{white}\bf Category}  & {\rule{0pt}{2.5ex}\color{white}\bf Details} & {\rule{0pt}{2.5ex}\color{white}\bf Description}\\
        \rowcolor{white}Languages  & \begin{tabular}{l}
Fortran\\
\end{tabular} & Programming languages and language standards supported by the software \\
        \rowcolor{numpexlightergray}Parallelism  & \begin{tabular}{l}
MPI\\
Multithread\\
\end{tabular} & Parallel computing methods and frameworks utilized by the software.\\
        \rowcolor{white}Data Formats  & \begin{tabular}{l}
Gmsh and associated formats\\
VTK\\
in-house binary format\\
\end{tabular} & Data formats that the software can handle or produce.\\
        \rowcolor{numpexlightergray}Resilience  & \begin{tabular}{l}
None\\
\end{tabular} & Fault tolerance and recovery mechanisms employed by the software.\\
        \rowcolor{white}DevOps & \begin{tabular}{l}
None\\
\end{tabular} & Outlines the development and operational practices including continuous integration, containerization, and testing methodologies.  \\
        \rowcolor{numpexlightergray}Packaging  & \begin{tabular}{l}
None\\
\end{tabular} & Software packaging and distribution.\\
        \rowcolor{white}Testing  & \begin{tabular}{l}
Analytic solutions\\
\end{tabular} & Testing methodologies employed to ensure software quality and correctness.\\
        \rowcolor{numpexlightergray}Containerization  & \begin{tabular}{l}
None\\
\end{tabular} & Container technologies used to package and deploy the software.\\
        \rowcolor{white}Interfaces  & \begin{tabular}{l}
MUMPS\\ 
Metis\\ 
ARPACK (optional)\\
PARPACK (optional)\\
\end{tabular} & List of software Hawen has interfaces with.\\
        \bottomrule
    \end{tabular}
    }}
    \caption{Hawen programming and computational environment}
    \label{table:hawen-environment}
\end{table}



\subsection{Mathematics}
\label{sec:Hawen:mathematics}
Mathematics not available.

In this section, provide a summary the mathematics used in the software.


\subsection{Relevant Publications}
\label{sec:Hawen:publications}

Here is a list of relevant publications related to the software:
\begin{itemize}
\item \cite{Hawen2021}: Reference of the software in the journal of open-source software;
\item \cite{Faucher2020adjoint}:
      Mathematical details of the adjoint-state method in the framework 
      of Hybridizable Discontinuous Galerkin discretization method.
      It provides the computational steps for the implementation of the 
      inverse problem.
\item \cite{Pham2024stabilization}: Details of the numerical implementation 
      of the HDG method for anisotropic elasticity.
\item \cite{Faucher2019FRgWIGeo,Faucher2020DAS}: 
      Use of the software in the context of seismic imaging.
\item \cite{Pham2020Siam,Pham2024assembling}:
      Use of the software in the context of helioseismology.
\item \cite{Faucher2023viscoacoustic}: 
      Use of the software in the context of viscoacoustics ultrasound imaging.
\item \cite{Liu2024,Benitez2024}: 
      Use of the software in the context of data science and
      for benchmarks.
\end{itemize}


\subsection{Acknowledgements}
\label{sec::Hawen:acknowledgements}

The software has been developed with the support of the following funding agencies and institutions: 

\begin{itemize}
  \item Since 2021, F. Faucher is part of the team Makutu of INRIA Bordeaux, at the 
                    University of Pau and Pays de l'Adour.
  \item 2024--2029, F. Faucher acknowledges support of the European Research Council 
                    with ERC-StG Project INCORWAVE -- grant 101116288.
  \item 2019--2021, F. Faucher acknowledges funding by the Austrian Science Fund (FWF) 
        under the Lise Meitner grant allocation M2791-N.
\end{itemize}



\input{software/hpdomain-decomposition-methods/hpdomain-decomposition-methods.tex}
\section{Software: MaHyCo}
\label{sec:MaHyCo:software}



\begin{itemize}
    \item \textbf{Contact Email(s):} jean-philippe.perlat@cea.fr
    \item \textbf{Supported Architecture(s):} HYBRID
    \item \textbf{Repository Link:} \href{URL not visible}{URL not visible}
\end{itemize}

\subsection{Software summary}
\label{sec:MaHyCo:summary}
Detailed overview not available.



\subsection{Purpose}
\label{sec:MaHyCo:purpose}
Purpose not available.



\subsection{Mathematics}
\label{sec:MaHyCo:mathematics}
Mathematics not available.


\subsection{Relevant Publications}
\label{sec:MaHyCo:publications}

\subsection{Acknowledgements}
\label{sec::MaHyCo:acknowledgements}

Acknowledgements not available.



\section{Software: MANTA}
\label{sec:MANTA:software}



\begin{table}[h!]
    \centering
    { \setlength{\parindent}{0pt}
    \def\arraystretch{1.25}
    \arrayrulecolor{numpexgray}
    {\fontsize{9}{11}\selectfont
    \begin{tabular}{!{\color{numpexgray}\vrule}p{.4\textwidth}!{\color{numpexgray}\vrule}p{.6\textwidth}!{\color{numpexgray}\vrule}}
        \rowcolor{numpexgray}{\rule{0pt}{2.5ex}\color{white}\bf Field} & {\rule{0pt}{2.5ex}\color{white}\bf Details} \\
        \rowcolor{white}\textbf{Consortium} & \begin{tabular}{l}
CEA + consortium in development (see EUROPLEXUS)\\
\end{tabular} \\
        \rowcolor{numpexlightergray}\textbf{Exa-MA Partners} & \begin{tabular}{l}
CEA\\
\end{tabular} \\
        \rowcolor{white}\textbf{Contact Emails} & \begin{tabular}{l}
olivier.jamond@cea.fr\\
\end{tabular} \\
        \rowcolor{numpexlightergray}\textbf{Supported Architectures} & \begin{tabular}{l}
CPU Only\\
\end{tabular} \\
        \rowcolor{white}\textbf{Repository} & \href{None}{None} \\
        \rowcolor{numpexlightergray}\textbf{License} & \begin{tabular}{l}
GPL-V3 (we may switch to LGPL)\\
\end{tabular} \\
        \rowcolor{white}\textbf{Bottlenecks roadmap} & \begin{tabular}{l}
B10 - Scientific Productivity\\
B11 - Reproducibility and Replicability of Computation\\
B6 - Data Management\\
B7 - Exascale Algorithms\\
\end{tabular} \\
        \bottomrule
    \end{tabular}
    }}
    \caption{MANTA Information}
\end{table}

\subsection{Software summary}
\label{sec:MANTA:summary}

MANTA (Mechanical Numerical Toolbox for advanced Application) is an open-source
effort from the French Alternative Energies and Atomic Energy Commission (CEA) to develop a
multiphysics solver for quasi-static and fast-transient simulations of fluids and solids. MANTA aims
to replace the 40 years-old Cast3M and Europlexus solvers and provide larger physical modeling
abilities using up to date technologies. It aims at being used in a massively parallel computation context.

MANTA's functionalities are built over a very generic "core layer" which should be able to deal 
with any set of PDEs and any "mesh-based" numerical method. So whereas being developed mainly in the 
field of mechanics, it can be used easily to deal with other physics.

\subsection{Purpose}
\label{sec:MANTA:purpose}

The project has been designed to meet the following objectives:
\begin{itemize}
    \item Being able to simulate complex industrial systems: this implies a great flexibility to be able to handle the complexity of an industrial system in a single calculation.
    \item High performance computing.
    \item "Automatic parallelism": new functionalities should be developed without bothering about parallelism.
    \item Provide a clean, simple and stable Application Programming Interface (API) in C++ and python.
    \item Generic and flexible to be used by researchers in other fields of numerical methods than mechanics.
    \item Quality assurance, robustness and reliability compatible with safety-critical studies in the
nuclear industry.
    \item Maintainability over decades.
\end{itemize}

MANTA targets two main kinds of users:
\begin{itemize}
    \item The mechanical engineers or researchers which exploit the output of numerical simulations
to design or analyze physical system of interest. In view of such a user, MANTA provides
a so-called end-user layer which offers a clean and easy API (both in C++ and python).
Most numerical details are hidden by default. Also, a very important point is that its API
is meant to be very stable in time.
    \item The researcher in the field of numerical methods which would like to implement and test
various algorithms. The MANTA so-called core-layer provides a generic and flexible way to
implement a new unstructured-mesh-based numerical method dealing with a given set of
Partial Differential Equations (PDE).
\end{itemize}

\subsection{Programming and Computational Environment}
\label{sec::MANTA:environment_capabilities}

MANTA is developed using a very standard "feature-branch" kind of collaborative workflow. The source code is hosted on `gitlab.com` (at this time in a private project, but soon in a public one). The code is built using `CMake`, uses `spack` to manage its dependencies. A very "modular" `CMake` architecture, borrowed from `VTK` has been developed: the source code shows as a set of interdependent "modules" which can be enabled/disabled by the user through a configuration file. Each "module" enabled triggers the enabling of the modules on which it depends, external third-parties, set of tests, ... The code is tested in several "test configurations" (compiler suite, linux distribution, compilation options, ...) inside docker images whose definitions are stored in the source repository. From an quality assurance point of view, we should be able to recompile and retest any commit of the code in the same way as it has been validated by the CI process. The code is mainly developed in `C++-20` (at this time, but we will follow the C++ new standards in the future), and can handle some functions in `Fortran`.

MANTA can be run on any linux machine, from a personal laptop to a large computation cluster. At this time it is not planned to be ported to windows. At this time, the code is MPI-only (only openMPI is supported in the build process at this time), but will start to be developed soon to handle performance portability. MANTA can be used through several ways:
\begin{itemize}
    \item By cloning the repository and compiling it. In this case, we provide scripts to synchronize everything from a local machine connected to internet to a computation cluster having no or limited access to internet.
    \item We provide an automatically up-to-date `apptainer` image in which MANTA is already built.
    \item We are working on a "`spack` recipe" that we will soon push to `spack`'s database.
\end{itemize}

\begin{table}[h!]
    \centering
    {
    \setlength{\parindent}{0pt}
    \def\arraystretch{1.25}
    \arrayrulecolor{numpexgray}
    {\fontsize{9}{11}\selectfont
    \begin{tabular}{lp{.3\textwidth}p{.5\textwidth}}
        \rowcolor{numpexgray}{\rule{0pt}{2.5ex}\color{white}\bf Category}  & {\rule{0pt}{2.5ex}\color{white}\bf Details} & {\rule{0pt}{2.5ex}\color{white}\bf Description}\\
        \rowcolor{white}Languages  & \begin{tabular}{l}
C++\\
\end{tabular} & Programming languages and language standards supported by the software \\
        \rowcolor{numpexlightergray}Parallelism  & \begin{tabular}{l}
MPI\\
\end{tabular} & Parallel computing methods and frameworks utilized by the software.\\
        \rowcolor{white}Data Formats  & \begin{tabular}{l}
Gmsh and associated formats\\
MED\\
MFront\\
VTK\\
\end{tabular} & Data formats that the software can handle or produce.\\
        \rowcolor{numpexlightergray}Resilience  & \begin{tabular}{l}
None\\
\end{tabular} & Fault tolerance and recovery mechanisms employed by the software.\\
        \rowcolor{white}DevOps & \begin{tabular}{l}
Continuous Integration\\
\end{tabular} & Outlines the development and operational practices including continuous integration, containerization, and testing methodologies.  \\
        \rowcolor{numpexlightergray}Packaging  & \begin{tabular}{l}
Apptainer image, spack recipe\\
\end{tabular} & Software packaging and distribution.\\
        \rowcolor{white}Testing  & \begin{tabular}{l}
Non-regression\\
Verification\\
Validation\\
Use of several \\
"test configurations" inside \\docker images
\end{tabular} & Testing methodologies employed to ensure software quality and correctness.\\
        \rowcolor{numpexlightergray}Containerization  & \begin{tabular}{l}
Apptainer\\
\end{tabular} & Container technologies used to package and deploy the software.\\
        \rowcolor{white}Interfaces  & \begin{tabular}{l}
Most significant 3rd parties:\\
PETSc, SLEPc\\
moab\\
zoltan\\
eigen\\
mfront/mgis\\
vtk\\
MEDcoupling\\
nanobind
\end{tabular} & List of software MANTA has interfaces with.\\
        \bottomrule
    \end{tabular}
    }}
    \caption{MANTA programming and computational environment}
\end{table}



\subsection{Mathematics}
\label{sec:MANTA:mathematics}

MANTA's "core layer" is fully generic to handle PDEs and related numerical methods which can be formalized as the assembling of distributed linear systems through the spatial integration on meshes, and their resolution. Some auxiliary linear systems can be attached to a linear system to introduce dual unknowns allowing to impose some constraints on the primal unknowns. We ends up with a saddle-point linear system of the form:

{
\newcommand{\m}[1]{\boldsymbol{#1}}
\renewcommand{\v}[1]{\boldsymbol{#1}}
\renewcommand{\t}[1]{\underline{#1}}
\renewcommand{\d}[1]{\, \mathrm{d}#1}
\begin{equation}
   \begin{bmatrix}
     \m{A}&\m{C}^T_1&\hdots&\m{C}^T_q \\
     \m{C}^T_1&&& \\
     \vdots &&\m{0}& \\
     \m{C}^T_q &&&\\
   \end{bmatrix}
   \begin{bmatrix}
     \v{X} \\
     \v{\lambda}_1  \\
     \vdots\\
     \v{\lambda}_q \\
   \end{bmatrix}
   =
   \begin{bmatrix}
     \v{B} \\
     \v{D}_1  \\
     \vdots\\
     \v{D}_q \\
   \end{bmatrix}
\end{equation}


MANTA provides internally different methods to "eliminate" the dual unknown $p$ for any auxiliary linear system: $\m{A}$ and $\m{B}$ are modified so that one obtain the exact or an approximated (depending on the method) solution $\v{X}$ of the problem, but with the unknown $\v{\lambda}_p$ removed.

The linear systems are assembled by spatial integration over some sets of mesh entities in a very classical way:
\begin{equation}
  \m{M}=\sum_i \mathcal{A}_i \int_{E_i} \m{m}(\t{x}) \d{\t{x}}
\end{equation}
Here $\mathcal{A}_i$ is an "assembling operator" which maps local degrees of freedom for the entity $E_i$ to degrees of freedom of the problem. $\m{m}$ is the integrand function whose value is a dense matrix. The integral evaluation is approximated by means of quadrature rules and some mappings to reference cells:
\begin{equation}
  \m{M}=\sum_i \mathcal{A}_i \sum_{j} w_j \m{m}(\t{\xi}_j) |\det(\t{\phi}_i(\t{\xi}_j))| \text{ , where } (\t{x}\in E_i) = \t{\phi}_i(\xi)
\end{equation}
The definition of the integrand $\m{m}$ and the assembling operator $\mathcal{A}_i$ are the two main entry points in the generic algorithm by which the new functionalities are developed.

}

\subsection{Relevant Publications}
\label{sec:MANTA:publications}
Jamond, O., Lelong, N., Brooking, G., Helfer, T., Prabel, B., Prat, R., \& Jaccon, A. (2024, May). MANTA: an industrial-strength open-source high performance explicit and implicit multi-physics solver. In 16ème Colloque National en Calcul de Structures.

Jamond, O., Lelong, N., Fourmont, A., Bluthé, J., Breuze, M., Bouda, P., ...\& Prabel, B. (2022, May). MANTA: un code HPC généraliste pour la simulation de problèmes complexes en mécanique. In CSMA 2022 15ème Colloque National en Calcul des Structures. 


\subsection{Acknowledgements}
\label{sec::MANTA:acknowledgements}

The software has been developed with the support of the following funding agencies and institutions: 




Acknowledgements not available.



\section{Software: MMG/ParMMG}
\label{sec:MMG/ParMMG:software}



\begin{table}[h!]
    \centering
    { \setlength{\parindent}{0pt}
    \def\arraystretch{1.25}
    \arrayrulecolor{numpexgray}
    {\fontsize{9}{11}\selectfont
    \begin{tabular}{!{\color{numpexgray}\vrule}p{.4\textwidth}!{\color{numpexgray}\vrule}p{.6\textwidth}!{\color{numpexgray}\vrule}}
        \rowcolor{numpexgray}{\rule{0pt}{2.5ex}\color{white}\bf Field} & {\rule{0pt}{2.5ex}\color{white}\bf Details} \\
        \rowcolor{white}\textbf{Consortium} & \begin{tabular}{l}
Inria\\
\end{tabular} \\
        \rowcolor{numpexlightergray}\textbf{Exa-MA Partners} & \begin{tabular}{l}
Inria BXSO\\
Sorbonne U\\
U Grenoble Alpes\\
\end{tabular} \\
        \rowcolor{white}\textbf{Contact Emails} & \begin{tabular}{l}
None\\
\end{tabular} \\
        \rowcolor{numpexlightergray}\textbf{Supported Architectures} & \begin{tabular}{l}
CPU Only\\
\end{tabular} \\
        \rowcolor{white}\textbf{Repository} & \href{None}{None} \\
        \rowcolor{numpexlightergray}\textbf{License} & \begin{tabular}{l}
None\\
\end{tabular} \\
        \bottomrule
    \end{tabular}
    }}
    \caption{MMG/ParMMG Information}
\end{table}

\subsection{Software summary}
\label{sec:MMG/ParMMG:summary}
Detailed overview not available.



\subsection{Purpose}
\label{sec:MMG/ParMMG:purpose}
Purpose not available.

\subsection{Programming and Computational Environment}
\label{sec::MMG/ParMMG:environment_capabilities}


The following table summarizes these aspects for MMG/ParMMG, providing a  view of its programming and computational capabilities.

\begin{table}[h!]
    \centering
    {
    \setlength{\parindent}{0pt}
    \def\arraystretch{1.25}
    \arrayrulecolor{numpexgray}
    {\fontsize{9}{11}\selectfont
    \begin{tabular}{lp{.3\textwidth}p{.5\textwidth}}
        \rowcolor{numpexgray}{\rule{0pt}{2.5ex}\color{white}\bf Category}  & {\rule{0pt}{2.5ex}\color{white}\bf Details} & {\rule{0pt}{2.5ex}\color{white}\bf Description}\\
        \rowcolor{white}Languages  & \begin{tabular}{l}
C\\
Fortran\\
\end{tabular} & Programming languages and language standards supported by the software \\
        \rowcolor{numpexlightergray}Parallelism  & \begin{tabular}{l}
MPI\\
\end{tabular} & Parallel computing methods and frameworks utilized by the software.\\
        \rowcolor{white}Data Formats  & \begin{tabular}{l}
in-house format\\
\end{tabular} & Data formats that the software can handle or produce.\\
        \rowcolor{numpexlightergray}Resilience  & \begin{tabular}{l}
None\\
\end{tabular} & Fault tolerance and recovery mechanisms employed by the software.\\
        \rowcolor{white}DevOps & \begin{tabular}{l}
Continuous Integration\\
\end{tabular} & Outlines the development and operational practices including continuous integration, containerization, and testing methodologies.  \\
        \rowcolor{numpexlightergray}Packaging  & \begin{tabular}{l}
None\\
\end{tabular} & Software packaging and distribution.\\
        \rowcolor{white}Testing  & \begin{tabular}{l}
None\\
\end{tabular} & Testing methodologies employed to ensure software quality and correctness.\\
        \rowcolor{numpexlightergray}Containerization  & \begin{tabular}{l}
None\\
\end{tabular} & Container technologies used to package and deploy the software.\\
        \rowcolor{white}Interfaces  & \begin{tabular}{l}
None\\
\end{tabular} & List of software MMG/ParMMG has interfaces with.\\
        \bottomrule
    \end{tabular}
    }}
    \caption{MMG/ParMMG programming and computational environment}
\end{table}



\subsection{Mathematics}
\label{sec:MMG/ParMMG:mathematics}
Mathematics not available.

In this section, provide a summary the mathematics used in the software.


\subsection{Relevant Publications}
\label{sec:MMG/ParMMG:publications}

Here is a list of relevant publications related to the software:


\subsection{Acknowledgements}
\label{sec::MMG/ParMMG:acknowledgements}

The software has been developed with the support of the following funding agencies and institutions: 




Acknowledgements not available.



\section{Software: pBB}
\label{sec:pBB:software}



\begin{table}[h!]
    \centering
    { \setlength{\parindent}{0pt}
    \def\arraystretch{1.25}
    \arrayrulecolor{numpexgray}
    {\fontsize{9}{11}\selectfont
    \begin{tabular}{!{\color{numpexgray}\vrule}p{.4\textwidth}!{\color{numpexgray}\vrule}p{.6\textwidth}!{\color{numpexgray}\vrule}}
        \rowcolor{numpexgray}{\rule{0pt}{2.5ex}\color{white}\bf Field} & {\rule{0pt}{2.5ex}\color{white}\bf Details} \\
        \rowcolor{white}\textbf{Consortium} & \begin{tabular}{l}
Université de Lille\\
\end{tabular} \\
        \rowcolor{numpexlightergray}\textbf{Exa-MA Partners} & \begin{tabular}{l}
Inria Lille\\
\end{tabular} \\
        \rowcolor{white}\textbf{Contact Emails} & \begin{tabular}{l}
nouredine.melab@univ-lille.fr\\
\end{tabular} \\
        \rowcolor{numpexlightergray}\textbf{Supported Architectures} & \begin{tabular}{l}
CPU or GPU\\
\end{tabular} \\
        \rowcolor{white}\textbf{Repository} & \href{https://gitlab.inria.fr/jgmys/permutationbb}{https://gitlab.inria.fr/jgmys/permutationbb} \\
        \rowcolor{numpexlightergray}\textbf{License} & \begin{tabular}{l}
OSS: Cecill-*\\
\end{tabular} \\
        \rowcolor{white}\textbf{Bottlenecks roadmap} & \begin{tabular}{l}
B10 - Scientific Productivity\\
B11 - Reproducibility and Replicability of Computation\\
B6 - Data Management\\
B7 - Exascale Algorithms\\
\end{tabular} \\
        \bottomrule
    \end{tabular}
    }}
    \caption{pBB Information}
\end{table}

\subsection{Software summary}
\label{sec:pBB:summary}
Detailed overview not available.



\subsection{Purpose}
\label{sec:pBB:purpose}
Purpose not available.

\subsection{Programming and Computational Environment}
\label{sec::pBB:environment_capabilities}


The following table summarizes these aspects for pBB, providing a  view of its programming and computational capabilities.

\begin{table}[h!]
    \centering
    {
    \setlength{\parindent}{0pt}
    \def\arraystretch{1.25}
    \arrayrulecolor{numpexgray}
    {\fontsize{9}{11}\selectfont
    \begin{tabular}{lp{.3\textwidth}p{.5\textwidth}}
        \rowcolor{numpexgray}{\rule{0pt}{2.5ex}\color{white}\bf Category}  & {\rule{0pt}{2.5ex}\color{white}\bf Details} & {\rule{0pt}{2.5ex}\color{white}\bf Description}\\
        \rowcolor{white}Languages  & \begin{tabular}{l}
C++\\
\end{tabular} & Programming languages and language standards supported by the software \\
        \rowcolor{numpexlightergray}Parallelism  & \begin{tabular}{l}
Chapel\\
GPU\\
MPI\\
Multithread\\
\end{tabular} & Parallel computing methods and frameworks utilized by the software.\\
        \rowcolor{white}Data Formats  & \begin{tabular}{l}
None\\
\end{tabular} & Data formats that the software can handle or produce.\\
        \rowcolor{numpexlightergray}Resilience  & \begin{tabular}{l}
Checkpoint restart\\
\end{tabular} & Fault tolerance and recovery mechanisms employed by the software.\\
        \rowcolor{white}DevOps & \begin{tabular}{l}
None\\
\end{tabular} & Outlines the development and operational practices including continuous integration, containerization, and testing methodologies.  \\
        \rowcolor{numpexlightergray}Packaging  & \begin{tabular}{l}
None\\
\end{tabular} & Software packaging and distribution.\\
        \rowcolor{white}Testing  & \begin{tabular}{l}
None\\
\end{tabular} & Testing methodologies employed to ensure software quality and correctness.\\
        \rowcolor{numpexlightergray}Containerization  & \begin{tabular}{l}
None\\
\end{tabular} & Container technologies used to package and deploy the software.\\
        \rowcolor{white}Interfaces  & \begin{tabular}{l}
None\\
\end{tabular} & List of software pBB has interfaces with.\\
        \bottomrule
    \end{tabular}
    }}
    \caption{pBB programming and computational environment}
\end{table}



\subsection{Mathematics}
\label{sec:pBB:mathematics}
Mathematics not available.

In this section, provide a summary the mathematics used in the software.


\subsection{Relevant Publications}
\label{sec:pBB:publications}

Here is a list of relevant publications related to the software:


\subsection{Acknowledgements}
\label{sec::pBB:acknowledgements}

The software has been developed with the support of the following funding agencies and institutions: 




Acknowledgements not available.



\section{Software: Samurai}
\label{sec:Samurai:software}



\begin{table}[h!]
    \centering
    { \setlength{\parindent}{0pt}
    \def\arraystretch{1.25}
    \arrayrulecolor{numpexgray}
    {\fontsize{9}{11}\selectfont
    \begin{tabular}{!{\color{numpexgray}\vrule}p{.4\textwidth}!{\color{numpexgray}\vrule}p{.6\textwidth}!{\color{numpexgray}\vrule}}
        \rowcolor{numpexgray}{\rule{0pt}{2.5ex}\color{white}\bf Field} & {\rule{0pt}{2.5ex}\color{white}\bf Details} \\
        \rowcolor{white}\textbf{Consortium} & \begin{tabular}{l}
IP Paris\\
\end{tabular} \\
        \rowcolor{numpexlightergray}\textbf{Exa-MA Partners} & \begin{tabular}{l}
CEA\\
IPP\\
\end{tabular} \\
        \rowcolor{white}\textbf{Contact Emails} & \begin{tabular}{l}
Loic Gouarin\\
\end{tabular} \\
        \rowcolor{numpexlightergray}\textbf{Supported Architectures} & \begin{tabular}{l}
CPU Only\\
\end{tabular} \\
        \rowcolor{white}\textbf{Repository} & \href{https://github.com/hpc-maths/samurai}{https://github.com/hpc-maths/samurai} \\
        \rowcolor{numpexlightergray}\textbf{License} & \begin{tabular}{l}
OSS::BSD\\
\end{tabular} \\
        \rowcolor{white}\textbf{Bottlenecks roadmap} & \begin{tabular}{l}
B10 - Scientific Productivity\\
B11 - Reproducibility and Replicability of Computation\\
B6 - Data Management\\
B7 - Exascale Algorithms\\
\end{tabular} \\
        \bottomrule
    \end{tabular}
    }}
    \caption{Samurai Information}
\end{table}

\subsection{Software summary}
\label{sec:Samurai:summary}

samurai is an open source software package written in modern C++ (C++17 and soon C++20), enabling the representation of sparse Cartesian meshes with different levels of resolution in a compressed way, using an interval representation. Resolution refers to Cartesian cells of the same size. A set algebra is provided to manage the operators that can intervene between these meshes. These include intersections, unions, differences and translations.
It is then possible to attach scalar and vector fields to these meshes and perform operations on these fields. Access operators facilitate field manipulation according to resolution levels and coordinates.

This data structure can then be used to implement spatial and temporal schemes. The~\cref{fig:Samurai:architecture} shows the different layers involved in samurai. Some of them are currently being implemented.

\begin{figure}[h!]
    \centering
    \includegraphics[width=0.8\textwidth]{graphics/samurai/samurai.png}
    \caption{Samurai architecture}
    \label{fig:Samurai:architecture}
\end{figure}

\href{https://github.com/hpc-maths/ponio}{ponio} is an open source software developed at the HPC@Maths team at CMAP (Ecole polytechnique). The aim of ponio is to provide a set of schemes in time for solving a whole collection of ODEs and PDEs. The simplest is the combination of an operator separation strategy and a method of line involving various classical time integrators like Runge-Kutta methods, or optimized ones (RADAU5, ROCK4) and also operator splitting methods as well as IMEX schemes; the long-term objective is also to be able to tackle innovative adaptive code coupling techniques through an interface as well as classes of time-space coupled schemes (Lax-Wendroff, OSMP, time-space coupled IMEX with good asymptotic preserving and stability properties...).

The design principles of samurai are the following:
\begin{enumerate}
    \item Compress the mesh according to the level-wise spatial connectivity along each Cartesian axis.
    \item Achieve fast look-up for a cell into the structure, especially for parents and neighbors. This is particularly useful when utilizing numerical schemes such as Finite Volumes, \emph{etc.} on the hybrid mesh.
    \item Maximize the memory contiguity of the stored data to allow for caching and vectorization (contrarily to the $z$-curve).
    \item Facilitate inter-level operations which are common in many numerical techniques (\emph{e.g.} multiresolution).
    \item Allow for a time evolution of the hybrid mesh (\emph{via} AMR or multiresolution)  efficiently.
    \item Give the possibility of writing numerical schemes in a transparent way as one were on a uniform mesh.
\end{enumerate}

To give an overview of the compression capabilities of samurai, the~\cref{tab:Samurai:compression} shows the number of cells needed to represent a Cartesian mesh defined by the simple-2d example found in the p4est library (\cite{burstedde_p4est_2011}) and illustrated in~\cref{fig:Samurai:simple2d}.

\begin{figure}[h!]
    \centering
    \includegraphics[width=0.8\textwidth]{graphics/samurai/p4est_3.png}
    \caption{simple-2d test from p4est library}
    \label{fig:Samurai:simple2d}
\end{figure}

\begin{table}[h!]
    \centering
    {
        \setlength{\parindent}{0pt}
        \def\arraystretch{1.25}
        \arrayrulecolor{numpexgray}
        {
            \fontsize{9}{11}\selectfont
            \begin{tabular}{llllll}

    \rowcolor{numpexgray}{\rule{0pt}{2.5ex}\color{white}\bf Level} &  {\rule{0pt}{2.5ex}\color{white}\bf Num. of cells }  &{\rule{0pt}{2.5ex}\color{white}\bf p4est} &  {\rule{0pt}{2.5ex}\color{white}\bf samurai (leaves) } &{\rule{0pt}{2.5ex}\color{white}\bf samurai (all)} &  {\rule{0pt}{2.5ex}\color{white}\bf ratio }\\

    \rowcolor{white} 9 & 66379 & 2.57 Mb & 33.68 Kb & 121 Kb & 21.24 \\
    \rowcolor{numpexlightergray} 10 & 263767 & 10.25 Mb & 66.64 Kb & 236.8 Kb & 43.28 \\
    \rowcolor{white} 11 & 1051747 & 40.96 Mb & 132.36 Kb & 467.24 Kb & 87.66 \\
    \rowcolor{numpexlightergray} 12 & 4200559 & 163.75 Mb & 263.6 Kb & 927 Kb & 176.64 \\
    \rowcolor{white} 13 & 16789627 & 654.86 Mb & 525.9 Kb & 1.85 Mb & 353.98 \\
    \rowcolor{numpexlightergray} 14 & 67133575 & 2.61 Gb & 1.05 Mb & 3.68 Mb & 709.24 \\
\end{tabular}
        }
    }
    \caption{WP1: Compression rate between samurai and p4est meshes}
    \label{tab:Samurai:compression}
\end{table}

\subsection{Purpose}
\label{sec:Samurai:purpose}

Based on this new data structure, samurai's objective is to be able to easily describe AMR mesh adaptation methods with a heuristic refinement criterion, or multiresolution methods based on a wavelet base decomposition (\cite{cohen_fully_2003}). Multiresolution, although more complicated to implement, offers greater robustness than AMR methods, since the refinement criterion is based solely on the calculation of a detail derived from the wavelet decomposition. It therefore provides finer control over the error made between the fine solution everywhere and the adapted solution whatever the physical problem studied. Most of available software are based on AMR methods (AMReX \cite{zhang_amrex_2021}, Dyablo \cite{delorme_novel_nodate} and others , \cite{dubey_survey_2014-1}). Only few of them are based on multiresolution methods (Murphy \cite{gillis_murphy---scalable_2022}, Wabbit \cite{krah_wavelet_2022}) and used cell-based structure.

Block-based AMR methods have good memory contiguity thanks to their patch-based hierarchical data structure. This makes it possible to use the vectorization of modern processors. However, to be effective, the patches need to be large enough. It is therefore possible to refine more than necessary.

Cell-based AMR methods lose this memory contiguity, as the mesh is now flattened. A tree-like data structure is therefore required. To restore good arithmetic intensity, it is customary to place cell blocks in the tree leaves. Here again, we refine more than is necessary.

In comparison, the samurai data structure maintains memory contiguity in one direction by using intervals. This is the same as with AMR block-based methods. This ensures that modern processors remain vectorized. What's more, the data structure allows refinement only where necessary. This means no more refinement than is necessary, while maintaining good arithmetic intensity. samurai therefore combines the advantages of the two previous data structures.

\subsection{Programming and Computational Environment}
\label{sec::Samurai:environment_capabilities}


The following table summarizes these aspects for Samurai, providing a  view of its programming and computational capabilities.

\begin{table}[h!]
    \centering
    {
    \setlength{\parindent}{0pt}
    \def\arraystretch{1.25}
    \arrayrulecolor{numpexgray}
    {\fontsize{9}{11}\selectfont
    \begin{tabular}{lp{.3\textwidth}p{.5\textwidth}}
        \rowcolor{numpexgray}{\rule{0pt}{2.5ex}\color{white}\bf Category}  & {\rule{0pt}{2.5ex}\color{white}\bf Details} & {\rule{0pt}{2.5ex}\color{white}\bf Description}\\
        \rowcolor{white}Languages  & \begin{tabular}{l}
C++17\\
\end{tabular} & Programming languages and language standards supported by the software \\
        \rowcolor{numpexlightergray}Parallelism  & \begin{tabular}{l}
MPI\\
Multithread\\
\end{tabular} & Parallel computing methods and frameworks utilized by the software.\\
        \rowcolor{white}Data Formats  & \begin{tabular}{l}
HDF5\\
\end{tabular} & Data formats that the software can handle or produce.\\
        \rowcolor{numpexlightergray}Resilience  & \begin{tabular}{l}
None\\
\end{tabular} & Fault tolerance and recovery mechanisms employed by the software.\\
        \rowcolor{white}DevOps & \begin{tabular}{l}
Continuous Delivery\\
Continuous Integration\\
\end{tabular} & Outlines the development and operational practices including continuous integration, containerization, and testing methodologies.  \\
        \rowcolor{numpexlightergray}Packaging  & \begin{tabular}{l}
Other\\
\end{tabular} & Software packaging and distribution.\\
        \rowcolor{white}Testing  & \begin{tabular}{l}
Functional\\
Unit\\
Validation\\
Verification\\
\end{tabular} & Testing methodologies employed to ensure software quality and correctness.\\
        \rowcolor{numpexlightergray}Containerization  & \begin{tabular}{l}
None\\
\end{tabular} & Container technologies used to package and deploy the software.\\
        \rowcolor{white}Interfaces  & \begin{tabular}{l}
PETSc\\
\end{tabular} & List of software Samurai has interfaces with.\\
        \bottomrule
    \end{tabular}
    }}
    \caption{Samurai programming and computational environment}
\end{table}



\subsection{Mathematics}
\label{sec:Samurai:mathematics}

samurai provides a set of operators for working with Cartesian grids of varying resolutions. Two distinct categories of operators can be identified:

\begin{itemize}
\item Prediction operators are employed to calculate the value of a field on a fine grid based on the value of a field on a coarse grid.
\item Projection operators are used to determine the value of a field on a coarse grid by using the value of a field on a fine grid.
\end{itemize}

The aforementioned operators are currently defined for control volumes up to order 11. The size of the associated stencils is automatically accounted for by samurai. In a near future, the same operators will be defined for finite differences and discountinuous Galerkin methods.

It is thus possible to reconstruct the solution at any desired resolution level. This enables the solution to be found at the finest level, as well as calculations to be performed involving two different AMR meshes. To illustrate this purpose, we can imagine a case where one mesh is used to solve the Navier-Stokes equations, while another is employed to simulate the propagation of a pollutant via an advection equation, wherein the velocity is derived from the velocity provided by the Navier-Stokes equations. The first mesh is adapted using the velocity field, whereas the second mesh is adapted using the pollutant concentration (\cite{nguessan_high_2021}).

In addition to these resolution-level operators, samurai provides an API for defining finite-volume operators that can be used for both explicit and implicit methods. The defined FVM operator available with samurai are

\begin{itemize}
    \item linear homogeneous operator
    \item linear heterogeneous operator
    \item non-linear operator
\end{itemize}

In the following, we will illustrate the use of samurai with a simple example of a linear homogeneous operator. The operator is a scalar Laplacian operator defined as follows:

Since we have

\begin{equation*}
\int_V \Delta u = \int_{\partial V} \nabla u\cdot \mathbf{n},
\end{equation*}

the flux function to implement is a discrete version of $\nabla u\cdot \mathbf{n}$.
Here, we choose the normal gradient of the first order, requiring a stencil of two cells.
This is enough to write the static configuration:

\begin{listing}[ht]
\begin{minted}[
    linenos,                % Line numbers
    fontsize=\scriptsize,        % Reduce font size
    bgcolor=bgcolor,        % Slightly gray background
    frame=lines,            % Delimiters around the code
    framesep=2mm,           % Space between code and frame
    rulecolor=\color{gray}, % Color of the frame
    breaklines              % Allow line breaks in long lines
  ]{cpp}
auto u = samurai::make_field<1>("u", mesh); // scalar field

using cfg = samurai::FluxConfig<SchemeType::LinearHomogeneous,
                                1,            // output_field_size
                                2,            // stencil_size
                                decltype(u)>; // input_field_type
\end{minted}
\end{listing}

Now, denoting by $V_L$ (left) and $V_R$ (right) the stencil cells and $F$ their interface, the discrete flux from $V_L$ to $V_R$ writes

\begin{equation*}
    \mathcal{F}_h(u_h)_{|F} := \frac{u_R-u_L}{h},
\end{equation*}

where $u_L$ and $u_R$ are the finite volume approximations of $u$ in the respective cells, and $h$ is the cell length.
%Referring to formula :eq:`linear_comb`, the coefficients in the linear combination of $(u_L, u_R)$ correspond to $(-1/h, 1/h)$.
The flux function then writes:

\begin{listing}[ht]
    \begin{minted}[
        linenos,                % Line numbers
        fontsize=\scriptsize,        % Reduce font size
        bgcolor=bgcolor,        % Slightly gray background
        frame=lines,            % Delimiters around the code
        framesep=2mm,           % Space between code and frame
        rulecolor=\color{gray}, % Color of the frame
        breaklines              % Allow line breaks in long lines
      ]{cpp}
samurai::FluxDefinition<cfg> gradient([](double h)
{
    static constexpr std::size_t L = 0; // left
    static constexpr std::size_t R = 1; // right

    samurai::FluxStencilCoeffs<cfg> c;
    c[L] = -1/h;
    c[R] =  1/h;
    return c;
});
\end{minted}
\end{listing}

First of all, remark that we have declared only one flux function for all directions.
We could have written as many functions as directions:
they would have been identical, except that we would have replaced the name of the constants
\verb!L=0, R=1! with \verb!B=0, T=1! (bottom, top) and \verb!B=0, F=1! (back, front) to better reflect the actual direction currently managed.
The indexes 0 and 1 actually refer to the configured stencil.
In this case, no particular stencil has been defined, so the default ones are used: in the x-direction of a 3D space,
it is \verb!{{0,0,0}, {1,0,0}}!, i.e. the current cell at index 0 (which we call \verb!L!) and its right neighbor at index 1 (which we call \verb!R!).

Finally, the operator must be constructed from the flux definition by the instruction

\begin{listing}[ht]
    \begin{minted}[
        linenos,                % Line numbers
        fontsize=\scriptsize,        % Reduce font size
        bgcolor=bgcolor,        % Slightly gray background
        frame=lines,            % Delimiters around the code
        framesep=2mm,           % Space between code and frame
        rulecolor=\color{gray}, % Color of the frame
        breaklines              % Allow line breaks in long lines
      ]{cpp}
auto laplacian = samurai::make_flux_based_scheme(gradient);
\end{minted}
\end{listing}

samurai uses lazy evaluation to have concise and readable equations. The following code snippet shows how to apply the operator to a field to solve the heat equation in implicit and explicit ways using backward Euler time scheme

\begin{listing}[ht]
    \begin{minted}[
        linenos,                % Line numbers
        fontsize=\scriptsize,        % Reduce font size
        bgcolor=bgcolor,        % Slightly gray background
        frame=lines,            % Delimiters around the code
        framesep=2mm,           % Space between code and frame
        rulecolor=\color{gray}, % Color of the frame
        breaklines              % Allow line breaks in long lines
      ]{cpp}
auto unp1 = samurai::make_field<1>("unp1", mesh);
if (explicit_scheme)
{
    unp1 = u - dt * laplacian(u);
}
else
{
    auto back_euler = id + dt * laplacian;
    samurai::petsc::solve(back_euler, unp1, u); // solves the linear equation   [Id + dt*Diff](unp1) = u
}
\end{minted}
\end{listing}

It can be observed that the implicit case is constructed and solved using PETSc.

These operators can then be used in adapted mesh refinement methods.

Other spatial discretization methods will soon be proposed in samurai, such as finite differences and discontinuous Galerkin methods based on the same operator definition approach.

\subsection{Relevant Publications}
\label{sec:Samurai:publications}

Here is a list of relevant publications related to the software:

\begin{itemize}
    \item \cite{bellotti_multidimensional_2022}: this article explains how to use the adaptive multiresolution (MR) approach based on wavelets with lattice Boltzmann methods.
    \item \cite{bellotti_multiresolution-based_2022}: in this article, an error analysis is proposed. For the purpose of validating this error analysis, we conduct a series of test cases for various schemes and scalar and systems of conservation laws, where solutions with shocks are to be found and local mesh adaptation is especially relevant. Theoretical estimates are retrieved while a reduced memory footprint is observed.
\end{itemize}

\subsection{Acknowledgements}
\label{sec::Samurai:acknowledgements}

The software has been developed with the support of the following funding agencies and institutions:
\begin{itemize}
    \item \'Ecole polytechnique
    \item CNRS
    \item CIEDS
 \end{itemize}

\section{Software: Scimba}
\label{sec:Scimba:software}



\begin{itemize}
    \item \textbf{Contact Email(s):} emmanuel.franck@inria.fr, matthieu.boileau@math.unistra.fr, victor.michel-dansac@inria.fr
    \item \textbf{Supported Architecture(s):} GPU
    \item \textbf{Repository Link:} \href{https://gitlab.inria.fr/scimba/scimba}{https://gitlab.inria.fr/scimba/scimba}
\end{itemize}

\subsection{Software summary}
\label{sec:Scimba:summary}
Detailed overview not available.



\subsection{Purpose}
\label{sec:Scimba:purpose}
Purpose not available.



\subsection{Mathematics}
\label{sec:Scimba:mathematics}
Mathematics not available.


\subsection{Relevant Publications}
\label{sec:Scimba:publications}

\subsection{Acknowledgements}
\label{sec::Scimba:acknowledgements}

Acknowledgements not available.



\section{Software: TRUST Platform}
\label{sec:TRUST Platform:software}



\begin{itemize}
    \item \textbf{Contact Email(s):} pierre.ledac@cea.fr
    \item \textbf{Supported Architecture(s):} HYBRID
    \item \textbf{Repository Link:} \href{https://github.com/cea-trust-platform}{https://github.com/cea-trust-platform}
\end{itemize}

\subsection{Software summary}
\label{sec:TRUST Platform:summary}
Detailed overview not available.



\subsection{Purpose}
\label{sec:TRUST Platform:purpose}
Purpose not available.



\subsection{Mathematics}
\label{sec:TRUST Platform:mathematics}
Mathematics not available.


\subsection{Relevant Publications}
\label{sec:TRUST Platform:publications}

\subsection{Acknowledgements}
\label{sec::TRUST Platform:acknowledgements}

Acknowledgements not available.



\section{Software: Zellij}
\label{sec:Zellij:software}



\begin{itemize}
    \item \textbf{Contact Email(s):} el-ghazali.talbi@univ-lille.fr
    \item \textbf{Supported Architecture(s):} GPU
    \item \textbf{Repository Link:} \href{https://github.com/ThomasFirmin/zellij}{https://github.com/ThomasFirmin/zellij}
\end{itemize}

\subsection{Software summary}
\label{sec:Zellij:summary}
Detailed overview not available.



\subsection{Purpose}
\label{sec:Zellij:purpose}
Purpose not available.



\subsection{Mathematics}
\label{sec:Zellij:mathematics}
Mathematics not available.


\subsection{Relevant Publications}
\label{sec:Zellij:publications}

\subsection{Acknowledgements}
\label{sec::Zellij:acknowledgements}

Acknowledgements not available.




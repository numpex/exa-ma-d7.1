
\section{Objectives \& Context}
% • Which Exa-MA bottlenecks (B#) WP2 addresses  
% • Relation to overall strategy (surrogate modeling, physics-informed ML)  
% • List of key tasks T2.1–T2.6

WP2 is concerned  with the development of model  order reduction (MOR)
methods,  fast  surrogate  solvers, and  Scientific  Machine  Learning
(SciML) methods.  In particular, we study non-intrusive approaches for
designing ultra-fast  surrogate models  of complex  physical problems,
and  strategies for  leveraging  these surrogates.   We consider  {\bf
  data-driven} techniques including  comparisons between reduced basis
methods and  methods based on  neural networks  (NN), as well  as {\bf
  model-driven} techniques  with a  focus on physics-based  NN models.
In practice, WP2 comprises the following tasks:
\begin{itemize}
\item[T2.1]  Surrogate models  based on  physics-driven Deep  Learning
  (DL).   Artificial Neural  Networks  (ANNs) approaches  that do  not
  require labeled data  sets for their training  have been extensively
  studied  in recent  years.  The  concept of  Physics-Informed Neural
  Networks (PINNs) is currently the most  popular and is used to solve
  PDEs,  fractional  equations, integral-differential  equations,  and
  stochastic PDEs. They  have a great potential of  application to the
  solution of non-linear, time-dependent,  parameter dependent PDEs or
  even  to inverse  problems.   However, PINNs  struggle  to scale  to
  problems  with   larger  domains   and  more   complex,  multi-scale
  solutions, partly due to the spectral bias of NNs and the increasing
  size of the underlying PINN  optimization function. To overcome such
  isses, we propose to combine PINNs with domain decomposition methods
  (DDM).   Specific  topics  that  are addressed  in  this  task  are:
  reduction  of  training  cost, influence  of  network  architecture,
  influence of loss definition and optimization strategy; 
\item[T2.2]  PDE  operator  learning with  Neural  Operators.   Neural
  Operators (NO)  are NNs specifically dedicated  to the approximation
  of inverse PDE operators.  These  methods, such as DeepOnet and FNO,
  allow for  fast computation  of PDE  solutions for  various physical
  problems.  In  this task  we  study  NO networks  theoretically  and
  methodologically,  focusing   on  applying  them  to   more  complex
  nonlinear problems and difficult geometries,  and working on ways to
  guarantee the quality of results;
\item  [T2.3]  Data-driven  MOR. Recent  approaches  include  building
  nonlinear  reduction  methods,  such   as  using  auto-encoders  and
  projections or nonlinear corrections to classical methods. This task
  aims  to  study theoretical  aspects  and  propose new  methods  for
  nonlinear  reduction   on  difficult   geometries,  hyper-reduction,
  closure,  and  learning  techniques   for  multiple  objective  cost
  functions,  and also  to investigate  stochastic versions  with deep
  generative latent models;  
\item[T2.4]  Non-intrusive   reduced  basis  methods   for  parametric
  problems In this  task, we study Non-Intrusive  Reduced Basis (NIRB)
  methods   to  reduce   the  computational   cost  of   high-fidelity
  codes. These methods include  operator inference methods that enable
  the automated affine or  higher-order decomposition of operators and
  two-grid methods. The offline part of these methods is expensive but
  performed  only  once,   while  the  online  phase   can  be  vastly
  accelerated  by  orders  of  magnitude. Both  steady  and  evolution
  problems are considered  and in this task.  We also  plan to combine
  these  strategies  with  parallel  in  time  methods  for  parabolic
  problems;
\item[T2.5]  Mixing  low-  and  high-fidelity  models.   Multifidelity
  modeling  (MFM)  uses  a  hierarchy of  lower  fidelity  models,  or
  surrogate  models,  to reduce  the  computational  cost of  repeated
  simulations  with high-fidelity  model.  The  lower fidelity  models
  must, in  general, be chosen  for each  problem, but there  are some
  general principles  that can be  followed that  we focus on  in this
  task. Scaling is, once again, a central issue and argues in favor of
  the Monte Carlo type approaches that we consider in this task;
\item[T2.6]  Real-time  models  with super  resolution  methods  allow
  computing an interpolate, which is a fine representation of pictures
  or functions  using a  coarse representation. Deep  Learning methods
  provide impressive results in this direction with a nice perspective
  for  PDEs.  In  this task,  we propose  to extend  these methods  to
  complex  PDEs   but  also  to  stochastic   processes  (using  deep,
  generative  models) and  on unstructured  grids (using  graph neural
  networks).   These   approaches  would  allow   very  low-resolution
  simulations to be carried out and corrected in real time.
\end{itemize}  

\section{Scientific Advances}
% • Advance 1 (T2.1): …  
% • Advance 2 (T2.2): …  
% • Advance 3 (T2.3): …  
% • etc.

During the reporting period, we have progressed on the following tasks: 

\begin{itemize}
\item[T2.1] In the  framework of a collaboration  between the Atlantis
  team (St\'ephane  Lanteri), the Macaron team  (Victor Michel-Dansac)
  and Eindhoven  University of  Technology (Victorita Dolean),  and in
  the context of  the Ph.D.  thesis of Daria  Hrebenshchykova, we have
  studied  a PINN-based  approach  for simulating  high frequency  and
  multiscale frequency-domain wave propagation  problems. For that, we
  are building upon the MLFBPINN (Multi-Level Finite Basis PINN) model
  recently introduced by  Victorita Dolean {\em et al.},  We propose a
  novel variant  of MLFBPINN where a  local PINN in each  subdomain is
  trained with a PML (Prefectly-Matched  Layer) on the whole subdomain
  boundary.  This  is an ongoing  work. In the Macaron  team (Emmanuel
  Franck,  Victor  Michel  Dansac  and  Laurent  Navoret),  two  other
  NN-based methods  have been  developed.  Firstly,  a semi-Lagrangian
  sequential  in time  method coupled  with a  preconditioned gradient
  method,  which  allows  to   compute  efficiently  high  dimensional
  (physical  or parametric)  advection-diffusion problems  and kinetic
  problems such as those modeled with a Vlasov type equation.  In this
  method   the   collocation   points   are   transported   with   the
  characteristics and then projected at each time on a NN space.  This
  method beats the classical approaches fro this type of problems when
  the dimension is 5 or above.  Seondly, in collaboration with the WP4
  (Emmanuel Franck, Victor Michel  Dansac, Amaury Bilières and Yannick
  Privat),  the Macaron  team has  proposed a  new shape  optimization
  method  for  parametric problems  where  the  PDE solution  and  the
  mapping between a reference geometry and a new one are computed by a
  NN.  Additionally using the like between  NNs and ODEs we impose the
  volume conservation  using a symplectic  NN.  This method  is faster
  and gives more accurate results than those provided by classical FEM
  method.
\item[T2.2]  The Macaron  team (Joubine  Aghili, Emmanuel  Francka and
  Victor Michel  Dansac), has proposed  a simple method  to accelerate
  the convergence of nonlinear solvers for PDEs like the Newton method
  using a Fourier  Neural Operator (FNO).  For that, a  FNO is trained
  using data and discrete residual  loss to predict the solution using
  the  source and  the spatial  coefficient for  a nonlinear  elliptic
  equation.  This  FNO produces  an initial  condition for  the Newton
  method,  which allows  to capture  very quickly  the coarse  scales.
  This approach allows to avoid the non-convergence situations that in
  can face when using the Newton  method, and allows to win between 80
  and 200\% of CPU time.
\item[T2.3] In  the framework of  a collaboration between  the Macaron
  team (Victor Michel Dansac, Vincent  Vigon and Emmanuel Franck), the
  Atlantis  team  (St\'ephane Lanteri)  and  the  Makutu team  (Hélène
  Barucq) and  in teh  context of the  postdoctoral project  of Truing
  Hung, a  nonlinear reduced-order  modeling method has  been proposed
  for dealing with elliptic PDEs, which  is based on a neural implicit
  representation  that  allows  to   evaluate  the  solution  and  the
  derivatives  at each  point without  any interpolation  process.  To
  increase the accuracy a two-step  method is introduced where one fit
  the  encoder and  decoder firstly  with  data, and  then refine  the
  decoding  process  with  a  physical  loss  and  a  Greedy  training
  algorithm. The final validation of the method is in ongoing.
\item[T2.4] In the framework of a collaboration between the University
  of  Strasbourg  (Christophe  Prud’homme   and  Joubine  Aghili)  and
  Sorbonne University (Yvon Maday), the Ph.D. thesis of Hassan Ballout
  aims  at  overcoming  the  limitations  of  linear  order  reduction
  methods,  which  become  ineffective  and very  costly  for  certain
  classes   of  problems,   particularly  those   involving  transport
  phenomena.   These  limitations  are formalized  by  the  Kolmogorov
  $n$-width barrier.  In our research,  we are trying to overcome this
  barrier  by  using  nonlinear   approximation  techniques.   We  are
  focusing on the Nonlinear Compressive  Reduced Basis Method.  We are
  participating in  the analysis and  development of the  method, both
  theoretically and practically. The  whole process will be integrated
  and  validated  on  benchmarks  in   CFD  and  heat  transfer,  with
  optimization for exascale computing.
\item[T2.5] First, we  note that T2.5 did not get  funding from NumPEx
  so far. However,  we have proposed new approaches  based on coupling
  with NNs.  In  the framework of a collaboration  between the Macaron
  team (Laurent Navoret, Victor Michel Dansac and Emmanuel Franck) and
  the Makutu  team (Hélène  Barucq and  Florian Faucher),  an original
  multifidelity  modeling approach  for parametric  problems has  been
  devised.  The  idea is to  capture coarse scales  with a PINN  for a
  large-scale  parametric problem,  and then  correct this  prediction
  online with a FEM.  This multifidelity approach makes it possible to
  calculate a set of accurate  solutions for a parametric problem with
  much  lower CPU  time.   In  fact, the  training  is  valid for  all
  solutions, and the correction requires only a very coarse mesh. This
  approach could  be generalized  to more  complicated PDEs  and other
  physical contexts.
\end{itemize}

We also note that T2.6  did not get funding from NumPEx so far.

\section{Application Showcase}
% 
%\input{chapters/applications/specs/app-scimba-rom-1}  
% \input additional app specs as needed

\textbf{Frequency-domain wave propagation modeling with PINNs} \\

As a first  application showcase, we propose to develop  a hierachy of
wave propagation  problems modeled by  the Hemholtz equation in  2D by
increasing   the  frequency   of  the   wave  excitation   source  and
cconsidering  different  types  of heterogeneity  of  the  propagation
medium,  from  multilayered  configurations  with  piecewise  constant
variation  of the  physical parameter,  to continuously  varying medim
caharacteristic  as  in  acoustic  wave propagation  in  a  geological
medium.

\textbf{Plasma modeling with neural network based methods }\\

Tokamak MHD is a highly HPC-intensive application, especially in terms
of  memory consumption.  Indeed, tokamak  MHD codes  are implicit  and
involve the inversion of very large matrices. In this application, the
goal  is to  build a  3D neural  network–based solver  to compute  the
equilibrium (Grad–Shafranov  equation) and simplified MHD  models. The
approach relies  on time-sequential neural  methods of the  PINN type,
combined with  natural gradient  optimization, adaptive  sampling, and
adaptive neural networks.

\textbf{Hybrid numerical for high-dimensional PDEs}\\

We  propose  to  develop  a demonstrator  of  numerical  methods  that
hybridize  machine learning  with classical  approaches. The  baseline
code will  be a  DG (Discontinuous Galerkin)  solver in  which several
numerical  modules  will be  made  learnable  through neural  networks
(including   mesh    adaptation,   basis   functions,    fluxes,   and
preconditioners). The key idea is that the learning components capture
the large-scale structures, while the  numerical solver focuses on the
unresolved  scales. These  approaches  are  particularly intended  for
solving PDEs in high physical  dimensions (such as radiative transfer)
or with parametric dependencies.

\section{Software Framework Contributions}
% • PINN frameworks: …  
% • Neural Operators: …  
% • Reduced‐basis toolboxes: …  
% • NIRB methods: …

\subsection{Main Software Framework: Scimba and Feel++}

Scimba is a  library that implements PINNs or  sequential time methods
(discrete PINNs,  neural Galerkin  or semi-Lagrangian).  The  code can
handle complex  geometries, adaptive sampling,  and GPU-preconditioned
optimizers.   It can  use different  types  of models  such as  neural
networks,   classical   or    nonlinear   kernel   methods,   low-rank
approximations,  etc.   The  core  library of  Scimba  is  written  in
Pytorch, and a simplified version centered around neural approaches in
Jax is  currently being developed.  The code also implements  EDO flow
learning methods.  NO methods will also be implemented in the future.

\subsection{Additional Software Framework}



\section{Preliminary Benchmarks \& Trends}
% • Surrogate model speedups placeholder  
% • Error vs.\ basis size placeholder  
% • Regression test pass-rate placeholder

The  studies  conducted in  T2.1  to  T2.5  are  all in  progress  and
preliminary results are  available only for validation  purpose of the
novel methodologies under development. 

\section{Roadmap \& Next Steps}
% • Deliverables D2.x to complete by Mxx  
% • New apps to integrate into WP7 CI/CD  
% • Dependencies on WP7 infrastructure (containers, ReFrame)  

The following deliverables are planned: 
\begin{itemize}
\item D2.2-S : Surrogate modeling toolbox - final version planned for M60
\item D2.3-BR : Benchmarking - final version planned for M60
\end{itemize}

Benchmarkinkg activities will require interactions with WP7 and in particular T7.1.
%!TEX root = ../exa-ma-d7.1.tex

\chapter{Software Production Methodology}
\label{chap:methodology}

\section{Introduction}
This chapter presents the updated methodology for benchmarking, testing, and validating Exa-MA software in preparation for exascale deployment. In continuity with v1.1.1 and the Exa-MA Application Frameworks \& Infrastructure Guide), we focus on operational practices and measurable outcomes. We introduce:

\begin{itemize}
  \item \textbf{Energy efficiency metrics} alongside performance and scalability.
  \item \textbf{Expanded hardware coverage}: including EuroHPC systems and multi-GPU clusters.
  \item \textbf{Automated CI/CD pipelines} for nightly benchmarks and non-regression.
  \item \textbf{Enhanced data management}: DOIs, Zenodo archiving, FAIR datasets.
  \item \textbf{Structured application definitions}: mini-apps, extended mini-apps, demonstrators, proxy-apps as per the application framework.
  \item \textbf{Advanced fault tolerance}: in-situ fault injection, ULFM, scalable checkpointing.
\end{itemize}

This methodology ensures that all Exa-MA applications—whether mini-apps or full demonstrators—adhere to a unified set of practices for reproducibility, portability, and high performance.

\section{Key Objectives and Bottlenecks}
We revisit our original aims under WP7 and map them to the primary bottlenecks (B1--B13); this mapping guides prioritization and evaluation:

\begin{itemize}
  \item \textbf{Performance \& Scalability} (\ac{B2}, \ac{B7})
  \item \textbf{Energy Efficiency} (\ac{B1})
  \item \textbf{Reproducibility \& Portability} (\ac{B6}, \ac{B11})
  \item \textbf{Automation \& CI/CD}
  \item \textbf{Data Management \& I/O} (\ac{B6}, \ac{B9}, \ac{B10})
  \item \textbf{Fault Tolerance \& Resilience} (\ac{B8}, \ac{B12})
\end{itemize}

\section{Structured Application Definitions}
Following the Application Framework, we use the following categories to align scope with testing depth and reporting:

\begin{description}
  \item[Mini-App] Isolates a single kernel or algorithm (e.g., mesh processing in FEEL++).
  \item[Extended Mini-App] Mini-app with supporting scripts for I/O and data generation.
  \item[Demonstrator] Integrated multi-WP workflow illustrating a real-world use case.
  \item[Proxy-App] Representative full-stack workload spanning three or more WPs for system-level tests.
\end{description}

Each application is documented with core metadata (ID, name, WP, partners), scientific purpose, technical metadata (framework, languages), inputs/outputs, metrics, status, and planning, ensuring traceability and unified benchmarking.

\section{Methodology Overview}
Our approach rests on four pillars:
\begin{enumerate}
  \item \textbf{Benchmark Phases:} Baseline, Scalability, Energy, Hybrid.
  \item \textbf{Testing Levels:} Non-regression, Verification, Validation.
  \item \textbf{Automation:} ReFrame-based workflows integrated into GitHub Actions or GitLab CI.
  \item \textbf{Data Management:} DOIs, Zenodo archival, FAIR principles, HDF5/DAOS frameworks.
\end{enumerate}

\section{Benchmark Phases and Metrics}
\subsection{Baseline Performance}
Establish the reference on a single node, reporting time-to-solution and memory footprint; these results anchor scaling and energy studies.

\subsection{Scalability Tests}
Exercise weak and strong scaling across large CPU/GPU counts; compare efficiency relative to the baseline.

\subsection{Energy Efficiency}
Profile power via RAPL/NVML and report energy-to-solution metrics (e.g., joules per operation, flops per watt) under comparable workloads.

\subsection{Hybrid Workloads}
Run concurrent CPU+GPU workloads to assess balance and end-to-end throughput on heterogeneous nodes.

\section{Automation and CI/CD}
We codify the methodology in CI/CD; ReFrame test suites run on merges and on nightly schedules:
\begin{itemize}
  \item Container builds (Spack, Singularity, Docker).
  \item Execution of all benchmark phases.
  \item Result collection in Prometheus and Grafana dashboards.
  \item Automated summary report generation (PDF/Markdown).
\end{itemize}

\section{Data Management and Reproducibility}
All inputs, outputs, and logs are versioned with DOIs in Zenodo. Datasets and results comply with FAIR principles, using HDF5 (and DAOS) for scalable, fault-tolerant I/O.

\section{Fault Tolerance and Resilience}
Beyond checkpoint/restart, we complement with:
\begin{itemize}
  \item ULFM-based MPI recovery.
  \item SCR-based scalable checkpointing.
  \item In-situ fault injection at multiple levels.
\end{itemize}

\section{Toolchain and Technologies}
Profilers: Extrae, Score-P, Nsight, Arm MAP. \\
Regression and orchestration: ReFrame. \\
Packaging: Spack, Guix-HPC, Docker, Singularity. \\
Visualization: Paraver, Vampir, Grafana.

\section{Demonstrator Pipelines}
Define Level 1–3 demonstrators with container recipes and test classifications (non-regression, verification, validation), hardware targets, and acceptance criteria.

\section{Conclusion and Outlook}
This updated methodology, implemented immediately, will be refined in future versions based on lessons learned. The v3 deliverable will include expanded energy benchmarks, CI/CD maturity, and advanced fault-tolerance strategies.

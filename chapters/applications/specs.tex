%!TEX root = ../../../exa-ma-d7.1.tex
% chapters/applications/specs.tex
\section{Application Specifications}
\label{sec:apps-specs}

All Exa‑MA applications are documented via a standardized spreadsheet template. The metadata recorded here underpins traceability, reproducibility, and consistent benchmarking, and directly drives CI pipelines and reports.

\subsection{Core Metadata}
Identifiers and ownership establish responsibility and uniquely reference each application.
\begin{itemize}
  \item \texttt{id}: Unique identifier for the application.  
  \item \texttt{name}: Descriptive title.  
  \item \texttt{partners}: Contributing institutions or teams.  
  \item \texttt{responsible}: Permanent lead or PI.  
  \item \texttt{work\_package}: WP(s) implementing the algorithm.  
  \item \texttt{application\_type}: \{mini-app, extended mini-app, demonstrator, proxy-app\}.
\end{itemize}

\subsection{Scientific Purpose \& Methods}
Each specification states the scientific objective and summarizes the algorithms, with clear links to the contributing work packages.
\begin{itemize}
  \item \texttt{purpose}: Short description of scientific goal.  
  \item \texttt{method\_algorithm\_WPx}: Summary of algorithms under test (e.g., DG, ROM, Newton–Krylov).
\end{itemize}

\subsection{Technical Metadata}
The software and parallel stack document portability and scaling characteristics.
\begin{itemize}
  \item \texttt{framework}: Software framework (FEEL++, MFEM, PyTorch…).  
  \item \texttt{parallel\_framework}: MPI, OpenMP, CUDA, Kokkos.  
  \item \texttt{languages}: C++, Python, Fortran, etc.
\end{itemize}

\subsection{Inputs \& Outputs}
Inputs and outputs specify data formats and serialization choices to enable reuse and downstream analysis.
\begin{itemize}
  \item \texttt{inputs}: File formats (VTK, HDF5, JSON, MatrixMarket…).  
  \item \texttt{outputs}: Data products (HDF5 fields, CSV logs, VTK views…).  
  \item \texttt{metadata\_format}: JSON, YAML, XML.
\end{itemize}

\subsection{Metrics}
Common metrics enable verification and performance comparisons across applications.
\begin{itemize}
  \item Verification: \texttt{analytical-solution-match}, \texttt{manufactured-solution}.  
  \item Performance: \texttt{time-to-solution}, \texttt{strong-scalability}, \texttt{roofline-analysis}.  
  \item Energy \& Efficiency: \texttt{energy-to-solution}, \texttt{flops-per-watt}.  
  \item SciML: \texttt{projection-error}, \texttt{train-loss}, \texttt{inference-latency}.
\end{itemize}

\begin{longtable}{@{}p{0.25\textwidth}p{0.70\textwidth}@{}}
\toprule
\textbf{Metric} & \textbf{Description} \\
\midrule
\endhead

analytical-solution-match & Compare results to a known analytical solution. \\
conservation-check         & Validate conservation properties (e.g., mass, energy). \\
manufactured-solution      & Use the Method of Manufactured Solutions (MMS) for code verification. \\
regression-test-pass       & Pass/fail indicator from automated non-regression test suite. \\
benchmark-verification     & Benchmarks verifying numerical behavior under expected conditions. \\
strong-scalability         & Performance trend under fixed problem size as resources increase. \\
weak-scalability           & Performance trend as problem size scales proportionally with resources. \\
roofline-analysis          & Comparison of achieved FLOPS vs. hardware roofline limits. \\
time-to-solution           & Wall-clock runtime for solving a reference problem. \\
throughput                 & Number of simulations (or tasks) completed per unit time. \\
gpu-acceleration           & Degree of performance gain from GPU execution. \\
heterogeneous-support      & Ability to exploit mixed CPU/GPU or diverse architectures. \\
precision-robustness       & Stability of results across single/mixed/double precision. \\
energy-to-solution         & Total energy consumed during a simulation run. \\
power-profile              & Time-resolved power consumption trace. \\
flops-per-watt             & Efficiency in floating-point operations per watt. \\
memory-footprint           & Maximum or average memory usage during execution. \\
io-scaling                 & I/O time trend under parallel scaling scenarios. \\
load-balance-efficiency    & Effectiveness of workload distribution across resources. \\
communication-overhead     & Time spent in inter-process communication (e.g., MPI). \\
projection-error           & Error between reduced-order and full-order model outputs. \\
relative-error             & Normalized error measure in ROM/SciML contexts. \\
eigenvalue-decay           & Decay rate of eigenvalues in reduced-basis methods. \\
basis-size                 & Number of modes retained in reduced-order models. \\
online-speedup             & Speedup ratio of ROM inference vs. full model. \\
offline-cost               & Time/resources required for training or setup (ROM/SciML). \\
train-loss                 & Loss function value during SciML model training. \\
test-error                 & Generalization error on unseen data for SciML. \\
prediction-error           & Error in SciML predictions vs. ground truth. \\
inference-latency          & Time required for a single model inference or output. \\
model-size                 & Size of the trained model (parameter count or file size). \\
training-time              & Total time to train a model or construct a ROM basis. \\
robustness-to-noise        & Stability of predictions under perturbed inputs. \\
data-efficiency            & Accuracy achieved per number of training samples. \\
physics-constraint-error   & Violation magnitude of physical constraints in PINNs/SciML. \\
PINN-residual-loss         & PDE residual norm in Physics-Informed Neural Network runs. \\

\bottomrule
\end{longtable}

\subsection{Status \& Scope}
Status and scope indicate maturity and the breadth of benchmarking.
\begin{itemize}
  \item \texttt{status}: \{spec-only, prototype-ready, verified, benchmark-ready, packaged, …\}  
  \item \texttt{benchmark\_scope}: \{single-node, multi-node, multi-gpu, full-system, not-benchmarked\}
\end{itemize}

\subsection{Planning \& Documentation}
Planning fields anchor timelines and provide pointers to source code and documentation.
\begin{itemize}
  \item \texttt{spec\_due}, \texttt{proto\_due}: Dates for spec and prototype.  
  \item \texttt{repo\_url}, \texttt{tex\_url}: Links to code and docs.  
  \item \texttt{notes}: Coordination remarks.
\end{itemize}
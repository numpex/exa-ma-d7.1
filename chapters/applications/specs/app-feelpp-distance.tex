%!TEX root = ../../../../exa-ma-d7.1.tex

\section{Compute distance to a range of entities}


Code by Céline \textbf{todo}

\Cref{tab:app-feelpp-template} describes the specifications of the application.

\begin{table}[ht]
    \centering
    \begin{tblr}{
        colspec = {l X[10cm]},
        row{odd} = {numpexlightergray},
        hlines = {0.1pt, numpexgray},
        vlines = {numpexgray},
        row{1} = {numpexgray, fg=white, font=\bfseries},
    }
        Field & Details \\
        id & \texttt{app-template} \\
        name &  \\
        Partners &  \\
        PC &  \\
        Responsible (Permanent) &  \\
        WP7 Engineer & \\
        work\_package & \\
        application\_type & \\
        purpose & \\
        Method-Algorithm WP1 & \\
        Method-Algorithm WP2 & \\
        Method-Algorithm WP3 & \\
        Method-Algorithm WP4 & \\
        Method-Algorithm WP5 & \\
        Method-Algorithm WP6 & \\
        WP7 & \\
        outputs & \\
        metrics & \\
        status & \\
        Benchmark scope & \\
        Framework & \\
        parallel\_framework & \\
        spec\_due & \\
        proto\_due & \\
        repo\_url & \url{https://github.com/numpex/??}\\
    \end{tblr}
    \caption{Description of the demonstrator \texttt{app-template}.}
    \label{tab:app-template}
\end{table}


%%%%%%%%%%%%%%%%%%%%%%%%%%%%%%%%%%%%%%%%%%%%%%%%%%%%%%%%%%%%%%%%%%%%%%%%%%%%%%%%%%%%%%%%%%%%%%%%%%%%%%%%%%%%%%%%%%%%%%%%


\subsection{Description of the benchmark}

Description of the benchmark, with some figures.


%%%%%%%%%%%%%%%%%%%%%%%%%%%%%%%%%%%%%%%%%%%%%%%%%%%%%%%%%%%%%%%%%%%%%%%%%%%%%%%%%%%%%%%%%%%%%%%%%%%%%%%%%%%%%%%%%%%%%%%%


\subsection{Benchmarking tools used}

Description of the benchmarking tools used, as well as the metrics that are used.


%%%%%%%%%%%%%%%%%%%%%%%%%%%%%%%%%%%%%%%%%%%%%%%%%%%%%%%%%%%%%%%%%%%%%%%%%%%%%%%%%%%%%%%%%%%%%%%%%%%%%%%%%%%%%%%%%%%%%%%%


\subsection{Input/Output Dataset Description}


\subsubsection{Input Data:}
  \begin{itemize}
  \item Meshes: ...
  \item Setup: ...
  \item Sif image: ...
  \item ...
  \end{itemize}

Include table of figure that are relevant (size of the meshes, etc...)

\subsubsection{Output Data:}

The output includes the computed values of validation measure in CSV files format, export visualization files (mesh, partitioning, temperature), and the time taken to perform each simulation step.
Metric:
\begin{itemize}
    \item \texttt{benchmark-verification},
    \item \texttt{strong-scalability},
    \item \texttt{weak-scalability},
    \item \texttt{io-scaling}: time taken by the application to read meshes, and to write the solution on disk (especially reading).
\end{itemize}




%%%%%%%%%%%%%%%%%%%%%%%%%%%%%%%%%%%%%%%%%%%%%%%%%%%%%%%%%%%%%%%%%%%%%%%%%%%%%%%%%%%%%%%%%%%%%%%%%%%%%%%%%%%%%%%%%%%%%%%%

\subsection{Results summary}


Insert the numerical solution, and other figures obtained thought benchmarking measures.

%!TEX root = ../../../../exa-ma-d7.1.tex

\section{Mini App: Test I/O at large scale}


This mini application tests the input and output of data for \Feelpp.
\Cref{tab:app-feelpp-io} describes the specifications of the application.

\begin{table}[ht]
    \centering
    \begin{tblr}{
        colspec = {l X[10cm]},
        row{odd} = {numpexlightergray},
        hlines = {0.1pt, numpexgray},
        vlines = {numpexgray},
        row{1} = {numpexgray, fg=white, font=\bfseries},
    }
        Field & Details \\
        id & \texttt{app-feelpp-io} \\
        name & Test I/O at large scale \\
        Partners &  Unistra \\
        PC & PC1 - ExaMA, PC2 - WP2 \\
        Responsible (Permanent) &  V. Chabannes, C. Prud'homme \\
        WP7 Engineer & Thomas Saigre (UNISTRA) \\
        work\_package & WP1 \\
        application\_type & mini-app \\
        % purpose & \\
        % Method-Algorithm WP1 & \\
        % Method-Algorithm WP2 & \\
        % Method-Algorithm WP3 & \\
        % Method-Algorithm WP4 & \\
        % Method-Algorithm WP5 & \\
        % Method-Algorithm WP6 & \\
        % WP7 & \\
        % outputs & \\
        % metrics & \\
        % status & \\
        % Benchmark scope & \\
        Framework & Feel++, PETSc \\
        parallel\_framework & MPI \\
        % spec\_due & \\
        % proto\_due & \\
        repo\_url & \url{https://github.com/numpex/apps-feelpp}\\
    \end{tblr}
    \caption{Description of the demonstrator \texttt{app-feelpp-io}.}
    \label{tab:app-feelpp-io}
\end{table}


%%%%%%%%%%%%%%%%%%%%%%%%%%%%%%%%%%%%%%%%%%%%%%%%%%%%%%%%%%%%%%%%%%%%%%%%%%%%%%%%%%%%%%%%%%%%%%%%%%%%%%%%%%%%%%%%%%%%%%%%


\subsection{Description of the benchmark}

This application consists in a small code that runs a given number of time the following steps:
\begin{itemize}
    \item Load a mesh $\mathcal{M}$ with a discretization size $h$,
    \item Create a functional space $X_h = P_{\text{c},h}^d$ associated to the desired order of discretization $d$,
    \item Create a vector $u\in X_h$, and export it to the disk,
    \item Write the loaded mesh on disk.
\end{itemize}

This process it iterated a given number of times.

%%%%%%%%%%%%%%%%%%%%%%%%%%%%%%%%%%%%%%%%%%%%%%%%%%%%%%%%%%%%%%%%%%%%%%%%%%%%%%%%%%%%%%%%%%%%%%%%%%%%%%%%%%%%%%%%%%%%%%%%


\subsection{Benchmarking tools used}

This mini app can be run with any mesh family.
The results presented in this document are obtained with a 3D mesh of a realistic geometry of a human eyeball \cite{saigre_model_2024}, with various level of refinement.
The steps to obtain this mesh, stemming from a human eyeball CAD are presented in \cite{chabannes_3d_2024}.
\Cref{fig:spec:app-feelpp:eye2brain:mesh} presents the mesh, and the level disparity of the refinement according to the region considered.

\begin{figure}
  \centering
  \includegraphics[width=0.6\textwidth]{feelpp/feelpp-benchmark-eyemesh.png}
  \caption{Mesh discretization, with the various regions of the eye.}
  \label{fig:spec:app-feelpp:eye2brain:mesh}
\end{figure}


%%%%%%%%%%%%%%%%%%%%%%%%%%%%%%%%%%%%%%%%%%%%%%%%%%%%%%%%%%%%%%%%%%%%%%%%%%%%%%%%%%%%%%%%%%%%%%%%%%%%%%%%%%%%%%%%%%%%%%%%


\subsection{Input/Output Dataset Description}


\subsubsection{Input Data:}
  \begin{itemize}
    \item Meshes: We have generated meshes of various level of refinement, denoted from \texttt{M2} to \texttt{M5}.
      The characteristics of the meshes, as well as the number of degree of freedom for various order of discretization is presented in \Cref{tab:spec:app-feelpp:eye2brain:mesh_stats}.
      All the meshes are pre-partitionned, with the application \texttt{feelpp\_mesh\_partitionner}\footnote{\url{https://docs.feelpp.org/user/latest/using/tools/mesh_partitioner.html}}
    \item Setup: use the function provided by the library \Feelpp, such as mesh manipulation and data exporter.
      The full source code is available in \texttt{src} directory of the GitHub repository \texttt{apps-feelpp}.
    \item Sif image: \texttt{feelpp:v0.111.0-preview.10-noble-sif}  (stored in the GitHub registry of \Feelpp)
  \end{itemize}

\SetTblrInner{rowsep=0pt}
\begin{table}[!ht]
    \centering
    \resizebox{\textwidth}{!}{
    \begin{tblr}{
        colspec={c*{9}{Q[c, cmd=\pgfmathprintnumber]}},
        vlines={numpexgray},
        hlines={numpexgray},
        row{1,2}={bg=numpexgray, fg=white, font=\bfseries, halign=c, cmd=\normalfont},
        rowhead=2,
    }
    \SetCell[c=4]{c}{Mesh properties} & & & & \SetCell[c=6]{c}{Number of degrees of freedom} & &\\
        Tag & \# points & \# edge & \# faces & $\mathP_1$ & $\mathP_2$ & $\mathP_3$ & $\mathP_4$ & $\mathP_5$ & $\mathP_6$ \\
        \texttt{M2} & 120581  & 762245   & 1273293  & 120581  & 882826   & 2918364   & 6858823   & 13335831  & 22981016 \\
        \texttt{M3} & 207845  & 1373087  & 2318292  & 207845  & 1580932  & 5272311   & 12435031  & 24222141  & 41786690 \\
        \texttt{M4} & 995906  & 6874446  & 11717323 & 995906  & 7870352  & 26462121  & 62609995  & 122152756 & 210929186 \\
        \texttt{M5} & 7360346 & 51295490 & 87714470 & 7360346 & 58655836 & 197665796 & 468169551 & 913946426 & 1578775746 \\
    \end{tblr}
    }

  \caption{Statistics on meshes of the eye and number of degrees of freedom with respect to finite element approximation}
  \label{tab:spec:app-feelpp:eye2brain:mesh_stats}
\end{table}

\subsubsection{Output Data:}

The monitored quantities are the average times for each of the steps run in the loop described above:
\begin{inparaenum}[\it(i)]
\item \texttt{time\_loadMesh}, the time taken by the application to load the mesh from the disk;
\item \texttt{time\_createFunctionSapce}, the time taken to create the functional space $X_h$ or order $d$;
\item \texttt{time\_saveMesh}, the time taken to write the mesh data on the disk, in a new file;
\item \texttt{time\_export}, the time to export the vector $u\in X_h$ to disk, using the \texttt{exporter} functionality of \Feelpp.
\end{inparaenum}
The times are stored in a JSON file during execution.


%%%%%%%%%%%%%%%%%%%%%%%%%%%%%%%%%%%%%%%%%%%%%%%%%%%%%%%%%%%%%%%%%%%%%%%%%%%%%%%%%%%%%%%%%%%%%%%%%%%%%%%%%%%%%%%%%%%%%%%%

\subsection{Results summary}


\newcommand{\plot}[2][(0.99,0.99)]{
  \begin{tikzpicture}
    \begin{axis}[
      % width=\textwidth, height=0.6172\textwidth,
      xlabel={ Number of tasks }, ylabel={ Time [s] },
      xtick=data, xtick align=outside,
      ymode=log,
      ymajorgrids=true, yminorgrids=true,
      xmajorgrids=true,
      xticklabels from table={#2}{resources.tasks},
      cycle list name=color list, legend style={at={#1},anchor=north east}
    ]
      \addplot+[mark=*, color=customdarkblue] table [x expr=\coordindex, y=M2] {#2} ;
      \addlegendentry{ \texttt{M2} }
      \addplot+[mark=*, color=customcyan] table [x expr=\coordindex, y=M3] {#2} ;
      \addlegendentry{ \texttt{M3} }
      \addplot+[mark=*, color=customorange] table [x expr=\coordindex, y=M4] {#2} ;
      \addlegendentry{ \texttt{M4} }
      \addplot+[mark=*, color=custompurple] table [x expr=\coordindex, y=M5] {#2} ;
      \addlegendentry{ \texttt{M5} }
    \end{axis}
  \end{tikzpicture}
  \vspace{-2em}
}

\pgfplotstableread[col sep=comma]{chapters/applications/specs/data/app-feelpp-io/loadMeshP1.csv}\dataLoadMesh
% \pgfplotstableread[col sep=comma]{chapters/applications/specs/data/app-feelpp-io/loadMeshP2.csv}\dataLoadMesh
% \pgfplotstableread[col sep=comma]{chapters/applications/specs/data/app-feelpp-io/loadMeshP3.csv}\dataLoadMesh
% \pgfplotstableread[col sep=comma]{chapters/applications/specs/data/app-feelpp-io/loadMeshP4.csv}\dataLoadMesh
% \pgfplotstableread[col sep=comma]{chapters/applications/specs/data/app-feelpp-io/loadMeshP5.csv}\dataLoadMesh

\pgfplotstableread[col sep=comma]{chapters/applications/specs/data/app-feelpp-io/functionSpaceP1.csv}\dataFunctionSpace
% \pgfplotstableread[col sep=comma]{chapters/applications/specs/data/app-feelpp-io/functionSpaceP2.csv}\dataFunctionSpace
% \pgfplotstableread[col sep=comma]{chapters/applications/specs/data/app-feelpp-io/functionSpaceP3.csv}\dataFunctionSpace
% \pgfplotstableread[col sep=comma]{chapters/applications/specs/data/app-feelpp-io/functionSpaceP4.csv}\dataFunctionSpace
% \pgfplotstableread[col sep=comma]{chapters/applications/specs/data/app-feelpp-io/functionSpaceP5.csv}\dataFunctionSpace

\pgfplotstableread[col sep=comma]{chapters/applications/specs/data/app-feelpp-io/saveMeshP1.csv}\dataSaveMesh
% \pgfplotstableread[col sep=comma]{chapters/applications/specs/data/app-feelpp-io/saveMeshP2.csv}\dataSaveMesh
% \pgfplotstableread[col sep=comma]{chapters/applications/specs/data/app-feelpp-io/saveMeshP3.csv}\dataSaveMesh
% \pgfplotstableread[col sep=comma]{chapters/applications/specs/data/app-feelpp-io/saveMeshP4.csv}\dataSaveMesh
% \pgfplotstableread[col sep=comma]{chapters/applications/specs/data/app-feelpp-io/saveMeshP5.csv}\dataSaveMesh

\pgfplotstableread[col sep=comma]{chapters/applications/specs/data/app-feelpp-io/exportDataP1.csv}\dataExportData
% \pgfplotstableread[col sep=comma]{chapters/applications/specs/data/app-feelpp-io/exportDataP2.csv}\dataExportData
% \pgfplotstableread[col sep=comma]{chapters/applications/specs/data/app-feelpp-io/exportDataP3.csv}\dataExportData
% \pgfplotstableread[col sep=comma]{chapters/applications/specs/data/app-feelpp-io/exportDataP4.csv}\dataExportData
% \pgfplotstableread[col sep=comma]{chapters/applications/specs/data/app-feelpp-io/exportDataP5.csv}\dataExportData


\begin{figure}
  \begin{subfigure}[c]{0.49\textwidth}
    \centering
    \plot{\dataLoadMesh}
    \caption{ Load the mesh. }
  \end{subfigure}
  \begin{subfigure}[c]{0.49\textwidth}
    \centering
    \plot{\dataFunctionSpace}
    \caption{ Time to create function space $X_h$. }
  \end{subfigure}
  \begin{subfigure}[c]{0.49\textwidth}
    \centering
    \plot[(0.99,0.35)]{\dataSaveMesh}
    \caption{ Time to save mesh on disk. }
  \end{subfigure}
  \begin{subfigure}[c]{0.49\textwidth}
    \centering
    \plot[(0.99,0.35)]{\dataExportData}
    \caption{ Write  the vector $u$ on disk. }
  \end{subfigure}
  \caption{Profiling times for the mini-app \texttt{app-feelpp-io}, run on Gaya, at a fixed discretization order ($\mathP_1$).}
\end{figure}


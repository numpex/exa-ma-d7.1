%!TEX root = ../../../../exa-ma-d7.1.tex

\section{Test I/O at large scale}


This mini application tests the input and output of data for \feelpp.
\Cref{tab:app-feelpp-io} describes the specifications of the application.

\begin{table}[ht]
    \centering
    \begin{tblr}{
        colspec = {l X[10cm]},
        row{odd} = {numpexlightergray},
        hlines = {0.1pt, numpexgray},
        vlines = {numpexgray},
        row{1} = {numpexgray, fg=white, font=\bfseries},
    }
        Field & Details \\
        id & \texttt{app-feelpp-io} \\
        name & Test I/O at large scale \\
        Partners &  Unistra \\
        PC & PC1 - ExaMA, PC2 - WP2 \\
        Responsible (Permanent) &  V. Chabannes, C. Prud'homme \\
        WP7 Engineer & Thomas Saigre (UNISTRA) \\
        work\_package & WP1 \\
        application\_type & mini-app \\
        % purpose & \\
        % Method-Algorithm WP1 & \\
        % Method-Algorithm WP2 & \\
        % Method-Algorithm WP3 & \\
        % Method-Algorithm WP4 & \\
        % Method-Algorithm WP5 & \\
        % Method-Algorithm WP6 & \\
        % WP7 & \\
        % outputs & \\
        % metrics & \\
        % status & \\
        % Benchmark scope & \\
        Framework & Feel++, PETSc \\
        parallel\_framework & MPI \\
        % spec\_due & \\
        % proto\_due & \\
        repo\_url & \url{https://github.com/numpex/apps-feelpp}\\
    \end{tblr}
    \caption{Description of the demonstrator \texttt{app-feelpp-io}.}
    \label{tab:app-feelpp-io}
\end{table}


%%%%%%%%%%%%%%%%%%%%%%%%%%%%%%%%%%%%%%%%%%%%%%%%%%%%%%%%%%%%%%%%%%%%%%%%%%%%%%%%%%%%%%%%%%%%%%%%%%%%%%%%%%%%%%%%%%%%%%%%


\subsection{Description of the benchmark}

This application consists in a small code that runs a given number of time the following steps:
\begin{itemize}
    \item Load a mesh $\mathcal{M}$ with a discretization size $h$,
    \item Create a functional space $X_h = P_{\text{c},h}^d$ associated to the desired order of discretization $d$,
    \item Create a vector $u\in X_h$, and export it to the disk,
    \item Write the loaded mesh on disk.
\end{itemize}

This process it iterated a given number of times.

%%%%%%%%%%%%%%%%%%%%%%%%%%%%%%%%%%%%%%%%%%%%%%%%%%%%%%%%%%%%%%%%%%%%%%%%%%%%%%%%%%%%%%%%%%%%%%%%%%%%%%%%%%%%%%%%%%%%%%%%


\subsection{Benchmarking tools used}

Description of the benchmarking tools used, as well as the metrics that are used.


%%%%%%%%%%%%%%%%%%%%%%%%%%%%%%%%%%%%%%%%%%%%%%%%%%%%%%%%%%%%%%%%%%%%%%%%%%%%%%%%%%%%%%%%%%%%%%%%%%%%%%%%%%%%%%%%%%%%%%%%


\subsection{Input/Output Dataset Description}


\subsubsection{Input Data:}
  \begin{itemize}
  \item Meshes: ...
  \item Setup: ...
  \item Sif image: ...
  \item ...
  \end{itemize}

\SetTblrInner{rowsep=0pt}
\begin{table}[!ht]
    \centering
    \resizebox{\textwidth}{!}{
    \begin{tblr}{
        colspec={c*{9}{Q[c, cmd=\pgfmathprintnumber]}},
        vlines={numpexgray},
        hlines={numpexgray},
        row{1,2}={bg=numpexgray, fg=white, font=\bfseries, halign=c, cmd=\normalfont},
        rowhead=2, % This option excludes the first two rows from the column command
    }
    \SetCell[c=4]{c}{Mesh properties} & & & & \SetCell[c=6]{c}{Number of degrees of freedom} & &\\
        Tag & \# points & \# edge & \# faces & $\mathP_1$ & $\mathP_2$ & $\mathP_3$ & $\mathP_4$ & $\mathP_5$ & $\mathP_6$ \\
        \texttt{M2} & 120581  & 762245   & 1273293  & 120581  & 882826   & 2918364   & 6858823   & 13335831  & 22981016 \\
        \texttt{M3} & 207845  & 1373087  & 2318292  & 207845  & 1580932  & 5272311   & 12435031  & 24222141  & 41786690 \\
        \texttt{M4} & 995906  & 6874446  & 11717323 & 995906  & 7870352  & 26462121  & 62609995  & 122152756 & 210929186 \\
        \texttt{M5} & 7360346 & 51295490 & 87714470 & 7360346 & 58655836 & 197665796 & 468169551 & 913946426 & 1578775746 \\
    \end{tblr}
    }

  \caption{Statistics on meshes of the eye and number of degrees of freedom with respect to finite element approximation}
  \label{tab:spec:app-feelpp:eye2brain:mesh_stats}
\end{table}

\subsubsection{Output Data:}

The output includes the computed values of validation measure in CSV files format, export visualization files (mesh, partitioning, temperature), and the time taken to perform each simulation step.
Metric:
\begin{itemize}
    \item \texttt{benchmark-verification},
    \item \texttt{strong-scalability},
    \item \texttt{weak-scalability},
    \item \texttt{io-scaling}: time taken by the application to read meshes, and to write the solution on disk (especially reading).
\end{itemize}




%%%%%%%%%%%%%%%%%%%%%%%%%%%%%%%%%%%%%%%%%%%%%%%%%%%%%%%%%%%%%%%%%%%%%%%%%%%%%%%%%%%%%%%%%%%%%%%%%%%%%%%%%%%%%%%%%%%%%%%%

\subsection{Results summary}


Insert the numerical solution, and other figures obtained thourght benchmarking measures.

%!TEX root = ../../../../exa-ma-d7.1.tex

\section{Mini App: Test I/O at large scale}
\label{sec:app:specs:app-feelpp-io}

This mini-app tests large-scale I/O for \Feelpp.
The I/O operations measured here are representative of those performed in the demonstrators described in \Cref{sec:app:specs:app-feelpp-discr-1,sec:app:specs:app-feelpp-discr-2}, where mesh loading and field export can become performance bottlenecks for large-scale simulations.

\Cref{tab:app-feelpp-io} describes the specifications of the application.

\begin{table}[ht]
    \centering
    \begin{tblr}{
        colspec = {l X[10cm]},
        row{odd} = {numpexlightergray},
        hlines = {0.1pt, numpexgray},
        vlines = {numpexgray},
        row{1} = {numpexgray, fg=white, font=\bfseries},
    }
        Field & Details \\
        id & \texttt{app-feelpp-io} \\
        name & Test I/O at large scale \\
        Partners &  Unistra \\
        PC & PC1 - ExaMA, PC2 - ExaSoft \\
        Responsible (Permanent) &  V. Chabannes, C. Prud'homme \\
        WP7 Engineer & Thomas Saigre (UNISTRA) \\
        work\_package & WP1, WP6 \\
        application\_type & mini-app \\
        purpose & Benchmark parallel I/O performance for mesh loading and field export operations \\
        Method-Algorithm WP1 & finite element spaces, mesh partitioning \\
        Method-Algorithm WP2 & \\
        Method-Algorithm WP3 & \\
        Method-Algorithm WP4 & \\
        Method-Algorithm WP5 & \\
        Method-Algorithm WP6 & parallel I/O, HDF5, file system performance \\
        WP7 & \\
        outputs & HDF5 data files, VTK/EnSight exports, JSON performance metrics \\
        metrics & \texttt{io-scaling}, \texttt{strong-scalability} \\
        status & benchmark-ready \\
        Benchmark scope & io-performance, scaling-analysis \\
        Framework & Feel++, PETSc \\
        parallel\_framework & MPI \\
        spec\_due & 6/15/2025 \\
        proto\_due & 7/1/2025 \\
        repo\_url & \url{https://github.com/numpex/apps-feelpp}\\
    \end{tblr}
    \caption{Description of the demonstrator \texttt{app-feelpp-io}.}
    \label{tab:app-feelpp-io}
\end{table}


%%%%%%%%%%%%%%%%%%%%%%%%%%%%%%%%%%%%%%%%%%%%%%%%%%%%%%%%%%%%%%%%%%%%%%%%%%%%%%%%%%%%%%%%%%%%%%%%%%%%%%%%%%%%%%%%%%%%%%%%


\subsection{Description of the benchmark}

This mini-app repeats the following steps a given number of times:
\begin{itemize}
    \item Load a mesh $\mathcal{M}$ with discretization size $h$,
    \item Create a function space $X_h = P_{\text{c},h}^d$ for the desired discretization order $d$,
    \item Create a vector-valued function space $\mathbf{X}_h = \left[P_{\text{c},h}^d\right]^3$,
    \item Create a vector $u\in X_h$ and export it to disk,
    \item Create a vector $\mathbf{u}\in\mathbf{X}_h$ and export it to disk,
    \item Write the loaded mesh to disk.
\end{itemize}

Exports are performed in two ways:
\begin{inparaenum}[\it (i)]
  \item write at format HDF5 on the disk directly,
  \item use the \texttt{exporter} functionality of \Feelpp.
\end{inparaenum}
For the second, we provide more details.
The two fields are interpolated to a $\mathP_1$ field before being saved on the disk.
The mesh is also exported in this process.
The time measured is the time to perform projections of $u$ and $\mathbf{u}$ and to save the files.

This whole process is iterated a given number of times.

%%%%%%%%%%%%%%%%%%%%%%%%%%%%%%%%%%%%%%%%%%%%%%%%%%%%%%%%%%%%%%%%%%%%%%%%%%%%%%%%%%%%%%%%%%%%%%%%%%%%%%%%%%%%%%%%%%%%%%%%


\subsection{Benchmarking tools used}

This mini app can be run with any mesh family.
The results presented in this document are obtained with a 3D mesh of a realistic geometry of a human eyeball \cite{saigre_model_2024}, with various level of refinement.
The steps to obtain this mesh, stemming from a human eyeball CAD are presented in \cite{chabannes_3d_2024}.
\Cref{fig:spec:app-feelpp:eye2brain:mesh} presents the mesh, and the level disparity of the refinement according to the region considered.

\begin{figure}
  \centering
  \includegraphics[width=0.6\textwidth]{feelpp/feelpp-benchmark-eyemesh.png}
  \caption{Mesh discretization, with the various regions of the eye.}
  \label{fig:spec:app-feelpp:eye2brain:mesh}
\end{figure}


%%%%%%%%%%%%%%%%%%%%%%%%%%%%%%%%%%%%%%%%%%%%%%%%%%%%%%%%%%%%%%%%%%%%%%%%%%%%%%%%%%%%%%%%%%%%%%%%%%%%%%%%%%%%%%%%%%%%%%%%


\subsection{Input/Output Dataset Description}


\subsubsection{Input Data:}
  \begin{itemize}
    \item Meshes: We have generated meshes of various refinement levels, denoted \texttt{M2} to \texttt{M5}.
      The characteristics of the meshes, as well as the number of degrees of freedom for various discretization orders, are presented in \Cref{tab:spec:app-feelpp:eye2brain:mesh_stats}.
      All meshes are pre-partitioned with the application \texttt{feelpp\_mesh\_partitionner}\footnote{\url{https://docs.feelpp.org/user/latest/using/tools/mesh_partitioner.html}}
    \item Setup: use the function provided by the library \Feelpp, such as mesh manipulation and data exporter.
      The full source code is available in \texttt{src} directory of the GitHub repository \texttt{apps-feelpp}.
    \item Sif image: \texttt{feelpp:v0.111.0-preview.10-noble-sif}  (stored in the GitHub registry of \Feelpp)
  \end{itemize}

\SetTblrInner{rowsep=0pt}
\begin{table}[!ht]
    \centering
    \resizebox{\textwidth}{!}{
    \begin{tblr}{
        colspec={c*{9}{Q[c, cmd=\pgfmathprintnumber]}},
        vlines={numpexgray},
        hlines={numpexgray},
        row{1,2}={bg=numpexgray, fg=white, font=\bfseries, halign=c, cmd=\normalfont},
        rowhead=2,
    }
    \SetCell[c=4]{c}{Mesh properties} & & & & \SetCell[c=6]{c}{Number of degrees of freedom} & &\\
        Tag & \# points & \# edge & \# faces & $\mathP_1$ & $\mathP_2$ & $\mathP_3$ & $\mathP_4$ & $\mathP_5$ & $\mathP_6$ \\
        \texttt{M2} & 120581  & 762245   & 1273293  & 120581  & 882826   & 2918364   & 6858823   & 13335831  & 22981016 \\
        \texttt{M3} & 207845  & 1373087  & 2318292  & 207845  & 1580932  & 5272311   & 12435031  & 24222141  & 41786690 \\
        \texttt{M4} & 995906  & 6874446  & 11717323 & 995906  & 7870352  & 26462121  & 62609995  & 122152756 & 210929186 \\
        \texttt{M5} & 7360346 & 51295490 & 87714470 & 7360346 & 58655836 & 197665796 & 468169551 & 913946426 & 1578775746 \\
    \end{tblr}
    }

  \caption{Statistics on meshes of the eye and number of degrees of freedom with respect to finite element approximation.}
  \label{tab:spec:app-feelpp:eye2brain:mesh_stats}
\end{table}

\subsubsection{Output Data:}

The monitored quantities are the average times for each of the steps run in the loop described above:
\begin{inparaenum}[\it(i)]
\item \texttt{time\_loadMesh}, the time taken by the application to load the mesh from the disk;
\item \texttt{time\_createFunctionSpace}, the time taken to create the functional space $X_h$ or order $d$;
\item \texttt{time\_saveMesh}, the time taken to write the mesh data on the disk, in a new file;
\item \texttt{time\_export}, the time to export the vector $u\in X_h$ to disk, using the \texttt{exporter} functionality of \Feelpp.
\end{inparaenum}
The times are stored in a JSON file during execution.


%%%%%%%%%%%%%%%%%%%%%%%%%%%%%%%%%%%%%%%%%%%%%%%%%%%%%%%%%%%%%%%%%%%%%%%%%%%%%%%%%%%%%%%%%%%%%%%%%%%%%%%%%%%%%%%%%%%%%%%%

\subsection{Results summary}



We present in \Cref{fig:specs:app-feelpp-io:np} results over the mesh family for various numbers of parallel tasks.
The mini-app is run on the cluster Gaya.
The results presented are obtained with the degree of discretization $\mathP_1$.
Similar behavior is observed for other discretization orders.

Note that some points are missing, either because the memory required exceeded what was available on a node of Gaya, or because the iterative process took longer than the job time limit (1 hour).

We see that mesh reading and function-space creation scale well with the number of parallel processes, as shown in \Cref{fig:specs:app-feelpp-io:np:loadmesh,fig:specs:app-feelpp-io:np:functionspace,fig:specs:app-feelpp-io:np:functionspaceV}, with a limit reached at $\texttt{np}=96$ for the small meshes.

Conversely, the steps writing data to disk do not scale with more parallel processes.
Times increase (\Cref{fig:specs:app-feelpp-io:np:savemesh,fig:specs:app-feelpp-io:np:exportfield,fig:specs:app-feelpp-io:np:exportfieldv}) or stagnate (\Cref{fig:specs:app-feelpp-io:np:export}).
This loss of performance stems from coordination among parallel processes when writing to a single destination.
Because the mesh is pre-partitioned, the reading step does not suffer the same issue.

\import{chapters/applications/specs/data/app-feelpp-io}{figure-np.tex}


\vspace{\baselineskip}

Now, we present in \Cref{fig:specs:app-feelpp-io:disc} the evolution of the time taken by each step, depending on the degree of discretization considered.
Specifically, we see in \Cref{fig:specs:app-feelpp-io:disc:loadmesh,fig:specs:app-feelpp-io:disc:savemesh} that the mesh manipulation is indeed independent of the degree of discretization.

Conversely, other steps are impacted by the discretization order through the increased number of degrees of freedom (see \Cref{tab:spec:app-feelpp:eye2brain:mesh_stats}).
In \Cref{fig:specs:app-feelpp-io:disc:functionspace,fig:specs:app-feelpp-io:disc:functionspaceV}, time increases roughly linearly with order, with a larger increase for disk writes (\Cref{fig:specs:app-feelpp-io:disc:exportfield,fig:specs:app-feelpp-io:disc:exportfieldv,fig:specs:app-feelpp-io:disc:export}).

\import{chapters/applications/specs/data/app-feelpp-io}{figure-order.tex}

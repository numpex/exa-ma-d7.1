%!TEX root = ../../../../exa-ma-d7.1.tex

\section{Demonstrator of fluid flow}


The specifications are identical to what was presented in \Cref{sec:app:specs:app-feelpp-discr-1}.
In this test case, we use the \Feelpp toolbox fluid.
\Cref{tab:app-feelpp-discr-1} describes the specifications of the application.


\subsection{Description of the benchmark}

The benchmark presented here, denoted by \emph{FDA benchmark}, that was proposed by the US Food and Drug Administration (FDA) in \cite{hariharan_multilaboratory_2011} to assess the stability, accuracy and robustness of computational methods.

The geometry is shown in \Cref{fig:spec:app-feelpp-discr-2:fda:geometry}

\begin{figure}[!ht]
  \centering
  \def\svgwidth{\textwidth}
  \import{graphics/feelpp}{feelpp-fda-2D-geometry.pdf_tex}
  \caption{FDA benchmark geometry description, adapted from Hariharan et al., 2011.}
  \label{fig:spec:app-feelpp-discr-2:fda:geometry}
\end{figure}



\subsection{Benchmarking tools used}


The performance tools integrated into the \Feelpp-toolboxes framework were used to measure the execution time.






\subsection{Input/Output Dataset Description}


\subsubsection{Input Data:}
  \begin{itemize}
  \item Meshes: We have generated three levels of mesh called \texttt{M1}, \texttt{M2}
    and \texttt{M3}. These meshes are stored in GMSH format. The statistics can be found in \Cref{tab:spec:app-feelpp-discr-1:thermal_bridges:discr_stat}. We have also prepared for
    each mesh level a collection of partitioned mesh.
    The format used is an in-house mesh format of \Feelpp based on
    JSON+HDF5 file type.
    The GMSH meshes and the partitioned meshes can be found on our Girder
    database management, in the \Feelpp collections.
  \item Setup: Use standard setup of \Feelpp toolboxes. It corresponds to a cfg
    file and JSON file. These config files are present in the GitHub of \Feelpp.
  \item Sif image: \texttt{feelpp:v0.111.0-preview.10-noble-sif}  (stored in the GitHub registry of \Feelpp)
  \end{itemize}

\subsubsection{Output Data:}



%%%%%%%%%%%%%%%%%%%%%%%%%%%%%%%%%%%%%%%%%%%%%%%%%%%%%%%%%%%%%%%%%%%%%%%%%%%%%%%%%%%%%%%%%%%%%%%%%%%%%%%%%%%%%%%%%%%%%%%%

\subsection{Results summary}



\subsubsection{Numerical solution}

\todo{Insert numerical solution obtained from the toolbox, and the partitioning of the mesh}



\subsubsection{Mesh converging}

\todo{Insert results of mesh converging, with comparison with the reference values of the benchmark \texttt{benchmark-verification}}



\subsection{Times of execution}

\todo{Insert results of mesh converging, with comparison with the reference values of the benchmark \texttt{benchmark-verification}}


\subsubsection{Scalability on IO}

\todo{Insert results of mesh converging, with comparison with the reference values of the benchmark \texttt{benchmark-verification}}


\todo{Add results}

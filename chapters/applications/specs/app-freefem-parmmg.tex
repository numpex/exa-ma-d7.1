%!TEX root = ../../../../exa-ma-d7.1.tex

\section{Demonstrator: Parallel Anisotropic Mesh Adaptation with ParMmg}
\label{sec:app:specs:app-freefem-parmmg}

This demonstrator evaluates ParMmg for distributed anisotropic mesh adaptation within the FreeFEM++ framework. The benchmark uses the classic Fichera corner problem to assess adaptive mesh refinement strategies and end‑to‑end workflow integration.

\Cref{tab:app-freefem-parmmg} describes the specifications of the application.

\begin{table}[ht]
    \centering
    \begin{tblr}{
        colspec = {l X[10cm]},
        row{odd} = {numpexlightergray},
        hlines = {0.1pt, numpexgray},
        vlines = {numpexgray},
        row{1} = {numpexgray, fg=white, font=\bfseries},
    }
        Field & Details \\
        id & \texttt{app-freefem-parmmg} \\
        name & Fichera corner with ParMmg \\
        Partners & Sorbonne U, Inria PARIS \\
        PC & PC1 - ExaMA \\
        Responsible (Permanent) & P. Jolivet; P.-H. Tournier \\
        WP7 Engineer & \\
        work\_package & WP1, WP3 \\
        application\_type & extended-mini-app \\
        purpose & Test ParMmg performance for distributed anisotropic mesh adaptation \\
        Method-Algorithm WP1 & mesh adaptation, unstructured mesh, finite element \\
        Method-Algorithm WP2 & \\
        Method-Algorithm WP3 & preconditioning \\
        Method-Algorithm WP4 & \\
        Method-Algorithm WP5 & \\
        Method-Algorithm WP6 & \\
        WP7 & \\
        outputs & in-house, YAML config, VTK, PETSc binary, XML reports \\
        metrics & \texttt{strong-scalability}, \texttt{weak-scalability}, \texttt{io-scaling} \\
        status & benchmark-ready \\
        Benchmark scope & multi-node, method-verification, workflow-component \\
        Framework & Freefem++, MMG/ParMMG, PETSc \\
        parallel\_framework & MPI \\
        spec\_due & 6/15/2025 \\
        proto\_due & \\
        repo\_url & \\
    \end{tblr}
    \caption{Description of the demonstrator \texttt{app-freefem-parmmg}.}
    \label{tab:app-freefem-parmmg}
\end{table}

\subsection{Description of the benchmark}

The Fichera corner is a classical benchmark for adaptive mesh refinement featuring a re‑entrant corner singularity. The domain is a cube with one corner removed; we solve Laplace's equation with appropriate boundary conditions.

Key features tested:
\begin{itemize}
\item \textbf{Anisotropic adaptation:} ParMmg adapts the mesh based on solution gradients and Hessian information
\item \textbf{Load balancing:} Dynamic redistribution as mesh evolves during adaptation cycles
\item \textbf{Quality preservation:} Maintaining element quality during parallel remeshing
\item \textbf{Workflow integration:} Coupling solve-adapt-solve cycles with FreeFEM++
\end{itemize}

\subsection{Benchmarking methodology}

Adaptive refinement cycles:
\begin{enumerate}
\item Solve PDE on current mesh using FreeFEM++
\item Compute metric tensor field from solution
\item Call ParMmg for parallel mesh adaptation
\item Interpolate solution to new mesh
\item Repeat until convergence or maximum refinement level
\end{enumerate}

Performance metrics focus on:
\begin{itemize}
\item Time per adaptation cycle
\item Mesh quality evolution
\item Load balancing efficiency
\item Solver convergence on adapted meshes
\end{itemize}

\subsection{Expected results}

The demonstrator validates:
\begin{itemize}
\item Effective anisotropic mesh adaptation near singularities
\item Strong scaling up to 128-256 cores
\item Weak scaling with proportional mesh growth
\item I/O performance for mesh and solution data
\end{itemize}

\textbf{Status:} \texttt{benchmark-ready} - validated and ready for systematic HPC benchmarking.

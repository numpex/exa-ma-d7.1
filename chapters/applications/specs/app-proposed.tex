%!TEX root = ../../../../exa-ma-d7.1.tex

\section{Proposed Applications - Early Stage Specifications}
\label{sec:app:specs:proposed}

This section describes applications currently in brainstorming or early specification stages. These represent planned future work within the \exama project.

\subsection{Optimization and Inverse Problems}

\subsubsection{app-eki: Ensemble Kalman Inversion}
\label{sec:app:specs:app-eki}

\textbf{Status:} Proposed (spec-only) | \textbf{Partners:} Inria BXSO, U Grenoble Alpes | \textbf{Target:} 1/15/2027

\textbf{Purpose:} Demonstrate feasibility of ensemble Kalman filter based inversion of a simplified wave equation for geophysics applications.

\textbf{Key Methods:}
\begin{itemize}
\item WP1: Spectral element, dG/hdG discretization
\item WP4: Stochastic inverse problems, ensemble-based data assimilation
\item WP6: Uncertainty propagation
\end{itemize}

\textbf{Framework:} PETSc, GeoS | \textbf{Parallel:} MPI (ensemble mode)

\textbf{Metrics:} manufactured-solution validation, weak-scalability

\textbf{Scientific Context:} Ensemble Kalman inversion provides a derivative-free approach to inverse problems by treating inversion as a filtering problem. The method is particularly attractive for high-dimensional parameter spaces where gradient computation is expensive or unavailable.

---

\subsubsection{app-zellij: Continuous Optimization Solver}
\label{sec:app:specs:app-zellij}

\textbf{Status:} Brainstorming | \textbf{Partners:} Inria Lille | \textbf{Responsible:} E-G. Talbi

\textbf{Purpose:} Solver for continuous optimization problems with focus on WP5 algorithms.

\textbf{Key Methods:}
\begin{itemize}
\item WP5: Iterative optimization methods, metaheuristics
\end{itemize}

\textbf{Framework:} Zellij | \textbf{Status:} Early brainstorming phase

\textbf{Note:} This application is in early conceptual development. Further specification required before implementation can begin.

---

\subsubsection{app-ffpp-shape-opt-fem: Shape Optimization with FEM}
\label{sec:app:specs:app-ffpp-shape-opt-fem}

\textbf{Status:} Brainstorming | \textbf{Partners:} Unistra | \textbf{Responsible:} Victor Dansac

\textbf{Purpose:} Demonstrate new shape optimization algorithm with volume preservation using finite element formulation.

\textbf{Key Methods:}
\begin{itemize}
\item WP5: Iterative methods, shape optimization, volume preservation
\end{itemize}

\textbf{Framework:} FreeFEM++ | \textbf{Parallel:} Sequential (initial version)

\textbf{Scientific Context:} Volume-preserving shape optimization is important for applications where total volume must remain constant (e.g., aircraft design, biological shape evolution). The FEM formulation provides flexibility in handling complex geometries.

---

\subsubsection{app-scimba-shape-opt-nn: Shape Optimization with Neural Networks}
\label{sec:app:specs:app-scimba-shape-opt-nn}

\textbf{Status:} Brainstorming | \textbf{Partners:} Unistra | \textbf{Responsible:} Victor Dansac

\textbf{Purpose:} Demonstrate new shape optimization algorithm with volume preservation using neural network formulation.

\textbf{Key Methods:}
\begin{itemize}
\item WP5: Shape optimization, iterative methods, neural network surrogates
\end{itemize}

\textbf{Framework:} Scimba | \textbf{Parallel:} GPU

\textbf{Scientific Context:} Neural network-based shape representation offers potential advantages over traditional mesh-based approaches: smooth parameterization, automatic differentiability, and natural regularization through network architecture.

---

\subsection{Machine Learning for Scientific Computing}

\subsubsection{app-scimba-plasma: Plasma Simulation with Neural Networks}
\label{sec:app:specs:app-scimba-plasma}

\textbf{Status:} Proposed | \textbf{Partners:} Unistra | \textbf{Responsible:} Emmanuel Franck

\textbf{Purpose:} Demonstrate neural network algorithms for 3D tokamak plasma simulation (mini-app).

\textbf{Key Methods:}
\begin{itemize}
\item WP2: Neural network-based PDE solvers, physics-informed neural networks
\end{itemize}

\textbf{Framework:} Scimba | \textbf{Application Domain:} Plasma physics, fusion energy

\textbf{Scientific Context:} Plasma simulation in tokamaks requires solving high-dimensional kinetic equations. Neural networks offer potential for learning reduced models that capture essential physics while being computationally tractable.

---

\subsubsection{app-scimba-hybrid: Hybrid Numerical Methods}
\label{sec:app:specs:app-scimba-hybrid}

\textbf{Status:} Proposed | \textbf{Partners:} Unistra | \textbf{Responsible:} Emmanuel Franck

\textbf{Purpose:} Test numerical hybrid methods on high-dimensional physical and parametric PDE problems (demonstrator).

\textbf{Key Methods:}
\begin{itemize}
\item WP2: Hybrid numerical-ML methods, high-dimensional PDEs
\end{itemize}

\textbf{Framework:} Scimba

\textbf{Scientific Context:} Hybrid approaches combine traditional numerical methods (FEM, FV, spectral) with machine learning components. This can provide the best of both worlds: guaranteed conservation properties and numerical stability from classical methods, with efficiency and adaptivity from ML.

---

\subsection{Feel++ Toolboxes at Scale}

\subsubsection{app-feelpp-toolbox: Multi-Physics Toolbox Benchmark}
\label{sec:app:specs:app-feelpp-toolbox}

\textbf{Status:} Proposed | \textbf{Partners:} Unistra | \textbf{Responsible:} V. Chabannes; C. Prud'homme

\textbf{Purpose:} Test \Feelpp toolboxes at scale across multiple physics domains (demonstrator).

\textbf{Test Cases:}
\begin{itemize}
\item \textbf{Heat:} Eye2brain and thermal bridges applications
\item \textbf{Solid mechanics:} NAFEMS LE-10 benchmark
\item \textbf{Thermoelectric:} HiFiMagnet application
\item \textbf{Fluid:} FDA benchmark
\end{itemize}

\textbf{Key Methods:}
\begin{itemize}
\item WP1: Multi-physics coupling, finite elements
\item WP3: Domain decomposition, iterative solvers
\end{itemize}

\textbf{Framework:} Feel++, PETSc | \textbf{Target date:} 6/15/2025

\textbf{Note:} This demonstrator consolidates multiple physics applications to provide a comprehensive evaluation of the \Feelpp toolbox ecosystem at HPC scale.

---

\subsection{Development Timeline}

The proposed applications follow a staged development approach:

\begin{enumerate}
\item \textbf{Q2 2025:} Complete specifications for spec-only applications (EKI, NLCRB, toolboxes)
\item \textbf{Q3 2025:} Initial prototypes for brainstorming applications (Zellij, shape optimization)
\item \textbf{Q4 2025 - Q1 2026:} Full implementation of spec-only applications
\item \textbf{Q2 2026:} Integration and benchmarking of Scimba applications
\item \textbf{Q3 2026:} Validation and documentation for all new demonstrators
\end{enumerate}

\subsection{Notes on Status Definitions}

\begin{itemize}
\item \textbf{brainstorming:} Initial concept phase, requirements not yet finalized
\item \textbf{spec-only:} Specification complete, implementation not yet started
\item \textbf{in-development:} Active implementation ongoing
\item \textbf{benchmark-ready:} Validated and ready for systematic HPC benchmarking
\end{itemize}


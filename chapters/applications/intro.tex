%!TEX root = ../../../exa-ma-d7.1.tex

% chapters/applications/intro.tex
\section{Introduction \& Scope}
\label{sec:apps-intro}

Work Package 7 (WP7) provides the “glue” for Exa-MA, delivering the CI/CD framework, containerized environments, packaging standards, benchmarking suite, training materials, and automated deployment pipelines that bind together the methodological developments from WP1-WP6 into reproducible, high-performance demonstrators on supercomputers.

This section defines the types of applications within Exa-MA and lays out the purpose, goals, and classification scheme that ensures consistency across mini-apps, extended mini-apps, demonstrators, and proxy-apps.

\subsection{Purpose \& Goals}
\begin{itemize}
  \item Align application research scope for WP1-WP6 with project KPIs.
  \item Agree on key performance metrics and measurement methods.
  \item Confirm timelines for specification delivery, prototype runs, and benchmarking in D7.1 v2.
\end{itemize}

\subsection{Types of Applications}
To ensure coherence across use-cases and scalability targets, Exa-MA classifies applications into four categories:
\begin{description}
  \item[Mini-App:] A focused test of a single method or kernel (e.g., FEEL++ mesh operation).
  \item[Extended Mini-App:] Mini-App plus supporting scripts (I/O stress tests, data generation).
  \item[Demonstrator:] Integrated workflow spanning multiple WPs to showcase a real-world scientific scenario.
  \item[Proxy-App:] Representative workload combining three or more WPs to emulate full-stack exascale behavior.
\end{description}
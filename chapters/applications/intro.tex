%!TEX root = ../../../exa-ma-d7.1.tex

% chapters/applications/intro.tex
\section{Introduction \& Scope}
\label{sec:apps-intro}

Work Package 7 (WP7) provides the “glue” for Exa-MA, delivering the CI/CD framework, containerized environments, packaging standards, benchmarking suite, training materials, and automated deployment pipelines that bind together the methodological developments from WP1-WP6 into reproducible, high-performance demonstrators on supercomputers.

This section defines the types of applications within Exa-MA and lays out their purpose, goals, and a classification scheme that keeps mini‑apps, extended mini‑apps, demonstrators, and proxy‑apps consistent and tied to measurable outcomes under WP7’s shared framework and CI.

\subsection{Purpose \& Goals}
We align expectations across WPs by clarifying objectives, shared metrics, and delivery milestones for specifications, prototypes, and benchmarks.
\begin{itemize}
  \item Align application research scope for WP1-WP6 with project KPIs.
  \item Agree on key performance metrics and measurement methods.
  \item Confirm timelines for specification delivery, prototype runs, and benchmarking in D7.1 v2.
\end{itemize}

\subsection{Types of Applications}
Each category comes with a clear scope, intended level of realism, and role in benchmarking and integration across the project.
To ensure coherence across use-cases and scalability targets, Exa-MA classifies applications into four categories:
\begin{description}
  \item[Mini-App:] A focused test of a single method or kernel (e.g., FEEL++ mesh operation).
  \item[Extended Mini-App:] Mini-App plus supporting scripts (I/O stress tests, data generation).
  \item[Demonstrator:] Integrated workflow spanning multiple WPs to showcase a real-world scientific scenario.
  \item[Proxy-App:] Representative workload combining three or more WPs to emulate full-stack exascale behavior.
\end{description}
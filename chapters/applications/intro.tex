%!TEX root = ../../../exa-ma-d7.1.tex

% chapters/applications/intro.tex
\section{Introduction \& Scope}
\label{sec:apps-intro}

Work Package 7 (WP7) provides the “glue” for Exa-MA, delivering the CI/CD framework, containerized environments, packaging standards, benchmarking suite, training materials, and automated deployment pipelines that bind together the methodological developments from WP1-WP6 into reproducible, high-performance demonstrators on supercomputers.

This section defines application categories and their role within WP7’s shared framework and CI, keeping mini‑apps, extended mini‑apps, demonstrators, and proxy‑apps consistent and tied to measurable outcomes.

\subsection{Purpose \& Goals}
We align expectations across WPs by clarifying objectives, shared metrics, and delivery milestones:
\begin{itemize}
  \item Align scope across WP1–WP6 with project KPIs.
  \item Agree on key performance metrics and measurement methods.
  \item Confirm timelines for specs, prototypes, and D7.1 v2 benchmarking.
\end{itemize}

\subsection{Types of Applications}
We classify applications into four categories to align scope with benchmarking and integration across the project:
\begin{description}
  \item[Mini-App:] A focused test of a single method or kernel (e.g., FEEL++ mesh operation).
  \item[Extended Mini-App:] Mini-App plus supporting scripts (I/O stress tests, data generation).
  \item[Demonstrator:] Integrated workflow spanning multiple WPs to showcase a real-world scientific scenario.
  \item[Proxy-App:] Representative workload combining three or more WPs to emulate full-stack exascale behavior.
\end{description}

%!TEX root = ../../../exa-ma-d7.1.tex

% chapters/applications/framework.tex
\section{Application Frameworks \& Best Practices}
\label{sec:apps-framework}

High‑performance Exa‑MA applications—mini‑apps through demonstrators—follow a unified framework to ensure reproducibility, portability, and maintainability, with minimal templates and CI hooks to apply the same conventions everywhere.

\subsection{Repository Layout}
A common structure makes code, tests, and artifacts predictable across projects and easy to navigate.
All repositories conform to:
\begin{verbatim}
src/                 % core code
tests/               % unit, regression, convergence tests
doc/                 % user documentation (Markdown, Sphinx, LaTeX)
containers/          % Docker/Apptainer recipes
.spack/              % Spack environments (dev, benchmark)
.github/workflows/   % CI/CD definitions (build, test, benchmark, lint)
\end{verbatim}

\subsection{Environment Management}
Pinned environments ensure that development and benchmarking run with identical dependencies and settings.
\begin{itemize}
  \item Use \textbf{Spack} to pin exact dependency versions; commit \texttt{spack.lock}.
  \item Provide separate envs under \texttt{.spack/dev/} and \texttt{.spack/benchmark/}.
  \item Document activation in \texttt{.spack/README.md}.
\end{itemize}

\subsection{CI/CD Pipelines}
Automated workflows standardize building, testing, and benchmarking, and leverage shared actions across repositories.
\begin{itemize}
  \item Define workflows for \texttt{build}, \texttt{test}, \texttt{benchmark}, \texttt{lint}.
  \item Reuse WP7 shared actions via \texttt{uses: numpex/wp7-actions/...}.
  \item On merge to \texttt{develop} and nightly, run full benchmark suite and push metrics to Grafana.
\end{itemize}

\subsection{Containerization \& Packaging}
Container images and versioning conventions provide reproducible execution on laptops and supercomputers alike.
\begin{itemize}
  \item Multi-stage Dockerfiles / Apptainer recipes in \texttt{containers/}.
  \item Base images tagged \texttt{<proj>-v<semver>-<date>}.
  \item Leverage Spack inside containers for reproducibility.
\end{itemize}

\subsection{Framework Examples}
Templates accelerate setup while preserving profiling and testing conventions.
\begin{description}
  \item[FEEL++] Clone from \texttt{feelpp-project} template: preconfigured \texttt{src/}, \texttt{.spack/}, CI, containers.
  \item[MFEM/FreeFEM++] Apply the same layout and CI conventions.
  \item[PETSc] Use \texttt{-log\_view} profiling and conform to the \texttt{tests/} and \texttt{containers/} structure.
\end{description}

\subsection{Best Practices}
These practices avoid configuration drift and enable unattended benchmarking.
\begin{itemize}
  \item Pin all versions; avoid hard-coded paths.
  \item Automate benchmarks on PRs (smoke) and nightly (full).
  \item Tag releases via GitHub Releases, include PDF report and container digests.
  \item Link all artifacts (code, containers, datasets) via DOIs for traceability.
\end{itemize}
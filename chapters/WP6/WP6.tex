%!TEX root = ../../../exa-ma-d7.1.tex
% \chapter{Work Package 6: Uncertainty Quantification}
% \label{chap:wp6}

\section{Objectives \& Context}
% • Which Exa-MA bottlenecks (B#) WP6 addresses  
% • Role in overall strategy (high-dimensional UQ, sensitivity analysis, surrogate‐based UQ)  
% • List of key tasks T6.1–T6.4

Uncertainty Quantification (UQ) includes several steps, from uncertainty propagation to sensitivity
analysis, to understand the important often correlated factors and then aim to reduce uncertainty
through modeling improvements in WP1, finally establishing robust inversion or optimization under
uncertainties. Enabling UQ in the different scales of a multiscale model remains a challenge that
exascale computing will help address. The modeling of uncertainties may be strongly (e.g., in the case
of stochastic equations) or weakly entangled into the models, calling for specific strategies but
systematically requiring the evaluation of extremely high-dimensional integrals. Again, multi-fidelity
stochastic or deterministic modeling will enable innovative tractable algorithms.

T6.1 Kernel-based sensitivity analysis for high-dimensional data and integral computing\\
To handle very high-dimensional and multivariate data implied in exascale applications, tractable and
relevant extensions of sensitivity analysis built upon kernel-based dependence measures will be
developed. In addition, optimized computing schemes of high dimensional integrals will be
investigated, in support of the uncertainty propagation step.

T6.2 UQ in a PDE solving framework \\
The propagation of uncertainties (on the initial conditions,
on the coefficients) in complex PDE solutions requires many simulations. Machine learning and
stochastic spectral methods could provide suitable approximations of the solutions but their calibration
can exceed by a few orders of magnitude the size of the underlying problems, making HPC strategies
mandatory.

T6.3 Surrogate modeling for UQ\\
 Complex multi-physics and/or multi-scale problems often involve
coupled, nested and chained numerical codes. The building and calibration of a global metamodel
assembling all prediction uncertainties is a formidable task that requires HPC.

T6.4 Acceleration of the bricks of the UQ process steps by leveraging exascale calculations\\
The methodological developments of Tasks 6.1 to 6.3 will be integrated, in the opensource platforms
Uranie and OpenTURNS dedicated to uncertainty quantification, taking advantage of the exascale
computational properties. Benchmarking on exascale applications will be conducted.


\section{Scientific Advances}
% • Advance 1 (T6.1): …  
% • Advance 2 (T6.2): …  
% • Advance 3 (T6.3): …  
% • etc.

T6.2 UQ in a PDE solving framework \\ 
The development of high-fidelity numerical simulations of physical phenomena increases the dimensionality of both inputs and outputs. While traditional surrogate modeling methods study numerical simulators with one-dimensional output, exascale applications will provides output of very high-dimension such as physical spatial fields. Moreover, such simulators often guarantees the respect of some physical constraints, such that incompressibility for fluid dynamics. Our achievement this year is to propose a multi-output Gaussian process surrogate modeling technique that respect linear physical constraints and scales well with output dimension. A first presentation of the results has been done at \textit{Journées de statistique} 2025. 


T6.3 Surrogate modeling for UQ\\
Neural networks are widely used for their strong representational capacity but face key limitations: the need for a large number of training points (which is often limited in the context of numerical simulation), a lack of explanability and confidence assessment.
Gaussian processes (GPs) are widely used as a metamodel for emulating time-consuming computer codes because of  their properties, which are particularly interesting when the data are derived from costly simulations:\\
- they can be interpolating,\\
- they are highly predictive even with few available observations,\\
- they allow to work with several input types,\\
- and most importantly, they intrinsically quantify prediction uncertainty, crucial for reliability, optimization, and sensitivity analysis.

Most existing GP literature focuses on low-dimensional vector inputs. However, many industrial applications involve meshes, represented as graphs with potentially hundreds of thousands of nodes. Conventional methods struggle with such structures.

Our first achievement was to design approaches that learn directly from graphs, making them applicable beyond meshes to molecules, transport, or social networks.
Since Gaussian process regression is a kernel-based method,
a new graph kernel, the Sliced Wasserstein Weisfeiler-Lehman (SWWL), was developed to address this challenge. It combines \cite{carpintero2024}:\\
	1.	a continuous Weisfeiler-Lehman scheme to build node embeddings,\\
	2.	a Sliced Wasserstein distance to compare graph representations, ensuring positive definiteness of the kernel and significantly reducing computational complexity.

A second challenge involves the outputs, which are not scalars but fields or signals defined on graph nodes of varying sizes. To handle this, a methodology was proposed, combining regularized optimal transport, dimensionality reduction, and GPs indexed by graphs \cite{carpintero2025}.

Additionally, many problems involve categorical variables (e.g., material types). GPs offer flexible ways to model correlations in such spaces. A comparative study was conducted to recommend best practices. Block correlation structures work well when groups of levels are known. When they are not, a clustering-based procedure, relying on the conditional output distributions for each level, can infer these groups effectively without prior kernel training.
Some of these procedures are, however, computationally intensive as the standard optimization routines used with their default hyperparameters (tolerance, maximal number of evaluations) fail to provide good results, while the use of more demanding values of the hyperparameters gives good results. This is when HPC becomes critical.

\section{Application Showcase}
% \input{chapters/applications/specs/app-uq-uranie-1}  
% \input additional app specs as needed

\paragraph{Applications mapping.}
\begin{itemize}
	\item Ensemble Kalman Inversion (proposed, with UQ): Section~\ref{sec:app:specs:app-eki}.
\end{itemize}

\section{Software Framework Contributions}
% • Uranie enhancements: …  
% • OpenTURNS integration: …  
% • Kernel-based sensitivity module: …  
% • High-dimensional integral computation library: …

\paragraph{Frameworks mapping.}
\begin{itemize}
	\item Uranie: Section~\ref{sec:Uranie:software}\,; OpenTURNS (if applicable); Scimba for surrogate UQ: Section~\ref{sec:Scimba:software}.
\end{itemize}

During the period end of 2024 to Septembre 2025, two new updates of the URANIE platform have been developed. 
URANIE v4.9.0 main improvements are:
\begin{itemize}
	\item \textbf{Methods developments:}
	\begin{itemize}
		\item Add the \texttt{Sensitivity::TSobolRank} class to compute Sobol indexes for given data (based on Chatterjee method)
    	\item Add the ants colony optimization method in Vizir (\texttt{Reoptimizer} module)
    	\item Add the CMAES optimization method in Vizir (\texttt{Reoptimizer} module)
    	\item Add \texttt{drawCopulogram} which looks like Pairs plot with rank scatterplot on the upper triangle
	\end{itemize}
	\item \textbf{Corrections \& modifications:}
	\begin{itemize}
		\item Add an aggregator to extract value from a \texttt{TSensitivity} object with the method \texttt{getValue(order, input)}
    	\item Simplify the way to run permutation in HSIC (\texttt{nperm=100} option)
    	\item Python2 is now deprecated (if you still want to use it, please use previous version)
    	\item Add an aggregator to extract the scalar value of a quantile
	\end{itemize}
	\item \textbf{User manual:}
	\begin{itemize}
		\item Update information about the compatibility with Python
   		\item Add a Macro for \texttt{TSobolRank}
    	\item Minor modification to explain how to use HSIC
    	\item Add the new way to extract results from sensitivity analysis using \texttt{getValue} 
	\end{itemize}
\end{itemize}
URANIE v4.10.0 main improvements are:
\begin{itemize}
	\item \textbf{Methods developments:}
	\begin{itemize}
		\item Add the \texttt{Sensitivity::TCramerVonMises} class to compute Cramer Von Mises indexes
    	\item Add a new module to \texttt{TMetropHasting} to continue an existing MH computation for a given number of additional iterations and implement minor adjustments to facilitate restarting the \texttt{TMetropHasting} constructor with cleared \texttt{tds} attributes
    	\item Add the possibility for \texttt{DataServer} to read (\texttt{fileDataReadCSV}) and export CSV (\texttt{exportDataCSV}) files
    	\item Add an Ishigami binary (mainly for training sessions)
    	\item Relauncher: improvement for launching complex codes (sharing of files between code, MPI parallel code call) (beta version -- not yet documented)
    	\item Launcher: Use of \texttt{Boost.Process} which should allow better scaling when using a large number of CPUs
	\end{itemize}
	\item \textbf{Corrections \& modifications:}
	\begin{itemize}
		\item Correction and improvement of the \texttt{THSIC} class
	\end{itemize}
	\item \textbf{User manual:}
	\begin{itemize}
		\item Add a Macro for \texttt{TCramerVonMises}
	\end{itemize}
\end{itemize}


\section{Preliminary Benchmarks \& Trends}
% • UQ scaling placeholder  
% • Sensitivity analysis performance placeholder  
% • Regression test pass-rate placeholder


Designing categorical kernels is a major challenge for Gaussian process regression with continuous and categorical inputs. Despite previous studies, it is difficult to identify a preferred method, either because the evaluation metrics,
the optimization procedure, or the datasets change depending on the study. In particular, reproducible code is rarely available. We are carrying out a reproducible comparative study of all existing categorical kernels
on many of the test cases investigated so far. We also propose new evaluation metrics inspired by the optimization
community, which provide quantitative rankings of the methods across several tasks. 
We will soon make available a package that proposes python routines for the best methods and their best versions.

\paragraph{Benchmarks mapping.}
\begin{itemize}
	\item Sensitivity analysis kernels (HSIC, Sobol, Cramér–von Mises) and UQ pipelines integrated in WP7 (T7.1) and connected to the corresponding application benchmarks where available.
\end{itemize}



\section{Roadmap \& Next Steps}
% • Deliverables D6.x to complete by Mxx  
% • New UQ apps to integrate into WP7 CI/CD  
% • Dependencies on WP7 infrastructure (containers, ReFrame)  

T6.2 UQ in a PDE solving framework \\ 
After proposing a efficient surrogate modeling of physical fields, we will focus on uncertainty quantification of high-dimensional random variables. Indeed, exascale computation will produce more and more complex simulation outputs, and a simple scalar risk measure quantity will be to coarse to represent wisely the uncertainty in the simulated physical fields. The work to be done is to define a theoretically sound and suitable mathematical notion of quantiles in high-dimension. The first lines of research on this topic will be the notion of statistical depth and optimal transport. 


\section{Software: Uranie}
\label{sec:WP5:Uranie:software}

\begin{table}[h!]
    \centering
    { \setlength{\parindent}{0pt}
    \def\arraystretch{1.25}
    \arrayrulecolor{numpexgray}
    {\fontsize{9}{11}\selectfont
    \begin{tabular}{!{\color{numpexgray}\vrule}p{.4\textwidth}!{\color{numpexgray}\vrule}p{.6\textwidth}!{\color{numpexgray}\vrule}}
        \rowcolor{numpexgray}{\rule{0pt}{2.5ex}\color{white}\bf Field} & {\rule{0pt}{2.5ex}\color{white}\bf Details} \\
        \rowcolor{white}\textbf{Consortium} & \begin{tabular}{l}
CEA\\
\end{tabular} \\
        \rowcolor{numpexlightergray}\textbf{Exa-MA Partners} & \begin{tabular}{l}
CEA\\
\end{tabular} \\
        \rowcolor{white}\textbf{Contact Emails} & \begin{tabular}{l}
clement.gauchy@cea.fr\\
jean-baptiste.blanchard@cea.fr\\
rudy.chocat@cea.fr\\
\end{tabular} \\
        \rowcolor{numpexlightergray}\textbf{Supported Architectures} & \begin{tabular}{l}
CPU Only\\
\end{tabular} \\
        \rowcolor{white}\textbf{Repository} & \href{https://uranie.cea.fr}{https://uranie.cea.fr} \\
        \rowcolor{numpexlightergray}\textbf{License} & \begin{tabular}{l}
OSS:: LGPL v*\\
\end{tabular} \\
        \rowcolor{white}\textbf{Bottlenecks roadmap} & \begin{tabular}{l}
None\\
\end{tabular} \\
        \bottomrule
    \end{tabular}
    }}
    \caption{WP5: Uranie Information}
\end{table}

\subsection{Software Overview}
\label{sec:WP5:Uranie:summary}

Uranie platform is based on ROOT and has by consequence a lot of ROOT characteristics such as:
\begin{itemize}
    \item interactive C++ interpreter (Cling) based on LLVM and Clang 
    \item Python interface (PyROOT)
    \item SQL database access 
\end{itemize}
Uranie is organized in different modules, each devoted to a specific task in the Uncertainty Quantification (UQ) framework. 
The modules devoted to optimization are listed in table \ref{tab:WP5:Uranie:features}  

\begin{table}[h!]
    \centering
    { 
        \setlength{\parindent}{0pt}
        \def\arraystretch{1.25}
        \arrayrulecolor{numpexgray}
        {
            \fontsize{9}{11}\selectfont
            \begin{tabular}{!{\color{numpexgray}\vrule}p{.25\linewidth}!{\color{numpexgray}\vrule}p{.6885\linewidth}!{\color{numpexgray}\vrule}}
    
    \rowcolor{numpexgray}{\rule{0pt}{2.5ex}\color{white}\bf Features} &  {\rule{0pt}{2.5ex}\color{white}\bf Short Description }\\ 
    
\rowcolor{white} Optimizer and Reoptimizer modules & The Optimizer and Reoptimizer libraries are dedicated to optimisation and model calibration. Model calibration consists in setting up the degrees of freedom of a model so that future simulations will optimally fit an experimental database. The optimisation is a complex procedure and several techniques are available to perform single-criterion or multi criteria one, with and without constraint. \\
\rowcolor{numpexlightergray} Optimization with surrogate model module & The MetaModelOptim library is a library dedicated to optimisation techniques coupling the generation of surrogate models (in particular the kriging one) and the evolutionnary algorithms to get an EGO-like approach. \\ 
\rowcolor{white} Calibration module & The Calibration library is more a dedicated module that is used to get the best estimations of some of the parameter of a specific model under consideration. This module provides different techniques relying on their own hypothesis on the model but all of these methods need data to perform this calibration.
\end{tabular}
        }
    }
    \caption{WP5: Uranie Features}
    \label{tab:WP5:Uranie:features}
\end{table}


\subsection{Parallel Capabilities}
\label{sec:WP5:Uranie:performances}


\begin{itemize}
    \item The parallel programming environment of URANIE leverages MPI, PTHREAD
    and CUDA to exploit the full potential of parallel computing. 
    On our platform, MPI is used to distribute simulations across different nodes,
    ensuring efficient scalability and enabling the platform to handle complex, large-scale computations. It is mainly use for the launching of external code.
    On the other hand, CUDA use GPU's capabilities in Artificial Neural Network of URANIE.
    URANIE use also LibSSH for launching code on different cluster (in the module TLauncher).

    \item The parallel computation environment of our platform is built on a HPC architecture designed to maximize computational power and efficiency
    using both distributed and shared memory parallelism. URANIE is used on CEA/TGCC supercomputers such as IRESNE.

    \item URANIE allows performing simulations in parallel for uncertainty quantification.
    \item \textbf{Scalability:} The scalability is constant because each the software distributes the simulation. if we add resource, they are devoted to run new simulations.
\end{itemize}



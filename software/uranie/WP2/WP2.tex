\section{Software: Uranie}
\label{sec:WP2:Uranie:software}

\begin{table}[h!]
    \centering
    { \setlength{\parindent}{0pt}
    \def\arraystretch{1.25}
    \arrayrulecolor{numpexgray}
    {\fontsize{9}{11}\selectfont
    \begin{tabular}{!{\color{numpexgray}\vrule}p{.4\textwidth}!{\color{numpexgray}\vrule}p{.6\textwidth}!{\color{numpexgray}\vrule}}
        \rowcolor{numpexgray}{\rule{0pt}{2.5ex}\color{white}\bf Field} & {\rule{0pt}{2.5ex}\color{white}\bf Details} \\
        \rowcolor{white}\textbf{Consortium} & \begin{tabular}{l}
CEA\\
\end{tabular} \\
        \rowcolor{numpexlightergray}\textbf{Exa-MA Partners} & \begin{tabular}{l}
CEA\\
\end{tabular} \\
        \rowcolor{white}\textbf{Contact Emails} & \begin{tabular}{l}
clement.gauchy@cea.fr\\
jean-baptiste.blanchard@cea.fr\\
rudy.chocat@cea.fr\\
\end{tabular} \\
        \rowcolor{numpexlightergray}\textbf{Supported Architectures} & \begin{tabular}{l}
CPU Only\\
\end{tabular} \\
        \rowcolor{white}\textbf{Repository} & \href{https://sourceforge.net/projects/uranie/}{https://sourceforge.net/projects/uranie/} \\
        \rowcolor{numpexlightergray}\textbf{License} & \begin{tabular}{l}
OSS:: LGPL v*\\
\end{tabular} \\
        \rowcolor{white}\textbf{Bottlenecks roadmap} & \begin{tabular}{l}
None\\
\end{tabular} \\
        \bottomrule
    \end{tabular}
    }}
    \caption{WP2: Uranie Information}
\end{table}

\subsection{Software Overview}
\label{sec:WP2:Uranie:summary}
Uranie platform is based on ROOT and has by consequence a lot of ROOT characteristics such as:
\begin{itemize}
    \item interactive C++ interpreter (Cling) based on LLVM and Clang 
    \item Python interface (PyROOT)
    \item SQL database access 
\end{itemize}
Uranie is organized in different modules, each devoted to a specific task in the Uncertainty Quantification (UQ) framework. 
The different surrogate modeling techniques available in the Modeler module are listed in table \ref{tab:WP2:Uranie:features}  

\begin{table}[h!]
    \centering
    { 
        \setlength{\parindent}{0pt}
        \def\arraystretch{1.25}
        \arrayrulecolor{numpexgray}
        {
            \fontsize{9}{11}\selectfont
            \begin{tabular}{!{\color{numpexgray}\vrule}p{.25\linewidth}!{\color{numpexgray}\vrule}p{.6885\linewidth}!{\color{numpexgray}\vrule}}
    
    \rowcolor{numpexgray}{\rule{0pt}{2.5ex}\color{white}\bf Features} &  {\rule{0pt}{2.5ex}\color{white}\bf Short Description }\\ 
    
\rowcolor{white} Polynomial chaos expansion & The basic idea of polynomial chaos expansion is that any square-integrable function can be written as $f(x) = \sum_{\alpha} f_{\alpha} \Psi_{\alpha}(x)$ where $\{f_{alpha}\}$ are the PC coefficients, $\{\Psi_{alpha}\}$ is the orthogonal polynomial-basis. $\alpha$ corresponds to a multi-index whose dimension is equal to the dimension of vector $x$ and whose L1 norm $\lVert \alpha \rVert_1$ is the degree of the resulting polynomial. \\
\rowcolor{numpexlightergray} Artificial Neural Networks & The artificial neural networks done within Uranie need input from \texttt{OPT++} and can also benefit from the computation power of graphical process unit (GPU) if available. Their implementation is done through the \texttt{TANNModeler} Uranie-class \\ 
\rowcolor{white} Kriging & First developed for geostatistic needs, the kriging method, named after D. Krige and also called Gaussian Process (GP) regression is another way to construct an surrogate model of a deterministic function. Its interesting features of GP are that it provides uncertainty along with its prediction and that it can interpolate the training data which is very useful for surrogate modeling of deterministic functions. 
\end{tabular}
        }
    }
    \caption{WP2: Uranie Features}
    \label{tab:WP2:Uranie:features}
\end{table}


\subsection{Parallel Capabilities}
\label{sec:WP2:Uranie:performances}


\begin{itemize}
    \item The parallel programming environment of URANIE leverages MPI, PTHREAD
    and CUDA to exploit the full potential of parallel computing. 
    On our platform, MPI is used to distribute simulations across different nodes,
    ensuring efficient scalability and enabling the platform to handle complex, large-scale computations. It is mainly use for the launching of external code.
    On the other hand, CUDA use GPU's capabilities in Artificial Neural Network of URANIE.
    URANIE use also LibSSH for launching code on different cluster (in the module TLauncher).

    \item The parallel computation environment of our platform is built on a HPC architecture designed to maximize computational power and efficiency
    using both distributed and shared memory parallelism. URANIE is used on CEA/TGCC supercomputers such as IRESNE.

    \item URANIE allows performing simulations in parallel for uncertainty quantification.
    \item \textbf{Scalability:} The scalability is constant because each the software distributes the simulation. if we add resource, they are devoted to run new simulations.
\end{itemize}


\subsection{Initial Performance Metrics}
\label{sec:WP2:Uranie:metrics}

This section provides a summary of initial performance benchmarks performed in the context of WP2. It ensures reproducibility by detailing input/output datasets, benchmarking tools, and the results. All data should be publicly available, ideally with a DOI for future reference.

\begin{itemize}
    \item \textbf{Overall Performance:} Summarize the software's computational performance, energy efficiency, and scalability results across different architectures (e.g., CPU, GPU, hybrid systems).
    \item \textbf{Input/Output Dataset:} Provide a detailed description of the dataset used for the benchmark, including:
        \begin{itemize}
            \item Input dataset size, structure, and format (e.g., CSV, HDF5, NetCDF).
            \item Output dataset format and key results.
            \item Location of the dataset (e.g., GitHub repository, institutional repository, or open access platform).
            \item DOI or permanent link for accessing the dataset.
        \end{itemize}
    \item \textbf{open-data Access:} Indicate whether the datasets used for the benchmark are open access, and provide a DOI or a direct link for download. Where applicable, highlight any licensing constraints.
    \item \textbf{Challenges:} Identify any significant bottlenecks or challenges observed during the benchmarking process, including data handling and computational performance.
    \item \textbf{Future Improvements:} Outline areas for optimization, including dataset handling, memory usage, or algorithmic efficiency, to address identified challenges.
\end{itemize}

\subsubsection{Benchmark \#1}
\begin{itemize}
    \item \textbf{Description:} Briefly describe the benchmark case, including the problem size, target architecture (e.g., CPU, GPU), and the input data. Mention the specific goals of the benchmark (e.g., testing scalability, energy efficiency).
    \item \textbf{Benchmarking Tools Used:} List the tools used for performance analysis, such as Extrae, Score-P, TAU, Vampir, or Nsight, and specify what metrics were measured (e.g., execution time, FLOPS, energy consumption).
    \item \textbf{Input/Output Dataset Description:}
        \begin{itemize}
            \item \textbf{Input Data:} Describe the input dataset (size, format, data type) and provide a DOI or link to access it.
            \item \textbf{Output Data:} Specify the structure of the results (e.g., memory usage, runtime logs) and how they can be accessed or replicated.
            \item \textbf{Data Repository:} Indicate where the data is stored (e.g., Zenodo, institutional repository) and provide a DOI or URL for accessing the data.
        \end{itemize}
    \item \textbf{Results Summary:} Include a summary of key metrics (execution time, memory usage, FLOPS) and their comparison across architectures (e.g., CPU, GPU).
    \item \textbf{Challenges Identified:} Describe any bottlenecks encountered (e.g., memory usage, parallelization inefficiencies) and how they impacted the benchmark.
\end{itemize}

\subsection{12-Month Roadmap}
\label{sec:WP2:Uranie:roadmap}

In this section, describe the roadmap for improving benchmarks and addressing the challenges identified. This should include:
\begin{itemize}
    \item \textbf{Data Improvements:} Plans for improving input/output data management, including making datasets more accessible and ensuring reproducibility through open-data initiatives.
    \item \textbf{Methodology Application:} Implementation of the benchmarking methodology proposed in this deliverable to streamline reproducibility and dataset management.
    \item \textbf{Results Retention:} Plans to maintain benchmark results in a publicly accessible repository with appropriate metadata and documentation, ensuring long-term usability.
\end{itemize}

In~\cref{tab:WP2:Uranie:bottlenecks}, we briefly discuss the bottleneck roadmap associated to the software and relevant to the work package.

\begin{table}[h!]
    \centering
    
    

    \centering
    { 
        \setlength{\parindent}{0pt}
        \def\arraystretch{1.25}
        \arrayrulecolor{numpexgray}
        {
            \fontsize{9}{11}\selectfont
            \begin{tabular}{!{\color{numpexgray}\vrule}p{.25\linewidth}!{\color{numpexgray}\vrule}p{.6885\linewidth}!{\color{numpexgray}\vrule}}
    
    \rowcolor{numpexgray}{\rule{0pt}{2.5ex}\color{white}\bf Bottlenecks} &  {\rule{0pt}{2.5ex}\color{white}\bf Short Description }\\ 
    
\rowcolor{white}    None & provide short description here \\
\end{tabular}
        }
    }
    \caption{WP2: Uranie plan with Respect to Relevant Bottlenecks}
    \label{tab:WP2:Uranie:bottlenecks}
\end{table}
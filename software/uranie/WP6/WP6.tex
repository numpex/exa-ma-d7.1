\section{Software: Uranie}
\label{sec:WP6:Uranie:software}

\begin{table}[h!]
    \centering
    { \setlength{\parindent}{0pt}
    \def\arraystretch{1.25}
    \arrayrulecolor{numpexgray}
    {\fontsize{9}{11}\selectfont
    \begin{tabular}{!{\color{numpexgray}\vrule}p{.4\textwidth}!{\color{numpexgray}\vrule}p{.6\textwidth}!{\color{numpexgray}\vrule}}
        \rowcolor{numpexgray}{\rule{0pt}{2.5ex}\color{white}\bf Field} & {\rule{0pt}{2.5ex}\color{white}\bf Details} \\
        \rowcolor{white}\textbf{Consortium} & \begin{tabular}{l}
CEA\\
\end{tabular} \\
        \rowcolor{numpexlightergray}\textbf{Exa-MA Partners} & \begin{tabular}{l}
CEA\\
\end{tabular} \\
        \rowcolor{white}\textbf{Contact Emails} & \begin{tabular}{l}
clement.gauchy@cea.fr\\
jean-baptiste.blanchard@cea.fr\\
rudy.chocat@cea.fr\\
\end{tabular} \\
        \rowcolor{numpexlightergray}\textbf{Supported Architectures} & \begin{tabular}{l}
CPU Only\\
GPU 
\end{tabular} \\
        \rowcolor{white}\textbf{Repository} & \href{https://uranie.cea.fr}{https://uranie.cea.fr} \\
        \rowcolor{numpexlightergray}\textbf{License} & \begin{tabular}{l}
OSS:: LGPL v*\\
\end{tabular} \\
        \rowcolor{white}\textbf{Bottlenecks roadmap} & \begin{tabular}{l}
Coupling with parallelized software \\
\end{tabular} \\
        \bottomrule
    \end{tabular}
    }}
    \caption{WP6: Uranie Information}
\end{table}

\subsection{Software Overview}
\label{sec:WP6:Uranie:summary}

Uranie platform is based on ROOT and has by consequence a lot of ROOT characteristics such as:
\begin{itemize}
    \item interactive C++ interpreter (Cling) based on LLVM and Clang 
    \item Python interface (PyROOT)
    \item SQL database access 
\end{itemize}
Uranie is organized in different modules, each devoted to a specific task in the Uncertainty Quantification (UQ) framework. 
The modules devoted to Uncertainty Quantification (UQ) are listed in table \ref{tab:WP6:Uranie:features}  

\begin{table}[h!]
    \centering
    { 
        \setlength{\parindent}{0pt}
        \def\arraystretch{1.25}
        \arrayrulecolor{numpexgray}
        {
            \fontsize{9}{11}\selectfont
            \begin{tabular}{!{\color{numpexgray}\vrule}p{.25\linewidth}!{\color{numpexgray}\vrule}p{.6885\linewidth}!{\color{numpexgray}\vrule}}
    
    \rowcolor{numpexgray}{\rule{0pt}{2.5ex}\color{white}\bf Features} &  {\rule{0pt}{2.5ex}\color{white}\bf Short Description }\\ 
    
\rowcolor{white}  DataServer module & The DataServer module is the heart of the Uranie platform, and contains the DataServer class. It contains all the necessary information on the variables in a problem (names, units, probability distributions, data files, etc.) and enables basic statistical operations to be performed. \\
\rowcolor{numpexlightergray}  Sampler module & This module contains sampling algorithms, such as Monte Carlo or Latin Hypercube Sampling. \\
\rowcolor{white} Launcher \& Relauncher modules & The Launcher and Relauncher module apply an analytical function (python or C++), external simulation code or any combination of these to the contents of a DataServer. The contents of the DataServer can result from a design of experiments generated using one of the sampling techniques, or can be loaded from an external source (ASCII file, SQL database, etc.). Calculations can be easily launched on a cluster or parallelized using different paradigms: fork, shared memory (thread), separate memory (MPI)... \\
\rowcolor{numpexlightergray} Sensitivity module: & The Sensitivity module is used to perform a sensitivity analysis of one of the output responses in relation to the input factors. Estimation of Sobol' indices and HSIC indices are implemented in this module.
\end{tabular}
        }
    }
    \caption{WP6: Uranie Features}
    \label{tab:WP6:Uranie:features}
\end{table}


\subsection{Parallel Capabilities}
\label{sec:WP6:Uranie:performances}


\begin{itemize}
    \item The parallel programming environment of URANIE leverages MPI, PTHREAD
    and CUDA to exploit the full potential of parallel computing. 
    On our platform, MPI is used to distribute simulations across different nodes,
    ensuring efficient scalability and enabling the platform to handle complex, large-scale computations. It is mainly use for the launching of external code.
    On the other hand, CUDA use GPU's capabilities in Artificial Neural Network of URANIE.
    URANIE use also LibSSH for launching code on different cluster (in the module TLauncher).

    \item The parallel computation environment of our platform is built on a HPC architecture designed to maximize computational power and efficiency
    using both distributed and shared memory parallelism. URANIE is used on CEA/TGCC supercomputers such as IRESNE.

    \item URANIE allows performing simulations in parallel for uncertainty quantification.
    \item \textbf{Scalability:} The scalability is constant because each the software distributes the simulation. if we add resource, they are devoted to run new simulations.
    % \item \textbf{Integration with Other Systems:} Describe how the software integrates with other numerical libraries in the Exa-MA framework.
\end{itemize}


\subsection{Initial Performance Metrics}
\label{sec:WP6:Uranie:metrics}

This section provides a summary of initial performance benchmarks performed in the context of WP6. It ensures reproducibility by detailing input/output datasets, benchmarking tools, and the results. All data should be publicly available, ideally with a DOI for future reference.

\begin{itemize}
    \item \textbf{Overall Performance:} Summarize the software's computational performance, energy efficiency, and scalability results across different architectures (e.g., CPU, GPU, hybrid systems).
    \item \textbf{Input/Output Dataset:} Provide a detailed description of the dataset used for the benchmark, including:
        \begin{itemize}
            \item Input dataset size, structure, and format (e.g., CSV, HDF5, NetCDF).
            \item Output dataset format and key results.
            \item Location of the dataset (e.g., GitHub repository, institutional repository, or open access platform).
            \item DOI or permanent link for accessing the dataset.
        \end{itemize}
    \item \textbf{open-data Access:} Indicate whether the datasets used for the benchmark are open access, and provide a DOI or a direct link for download. Where applicable, highlight any licensing constraints.
    \item \textbf{Challenges:} Identify any significant bottlenecks or challenges observed during the benchmarking process, including data handling and computational performance.
    \item \textbf{Future Improvements:} Outline areas for optimization, including dataset handling, memory usage, or algorithmic efficiency, to address identified challenges.
\end{itemize}

\subsubsection{Benchmark \#1}
\begin{itemize}
    \item \textbf{Description:} Briefly describe the benchmark case, including the problem size, target architecture (e.g., CPU, GPU), and the input data. Mention the specific goals of the benchmark (e.g., testing scalability, energy efficiency).
    \item \textbf{Benchmarking Tools Used:} List the tools used for performance analysis, such as Extrae, Score-P, TAU, Vampir, or Nsight, and specify what metrics were measured (e.g., execution time, FLOPS, energy consumption).
    \item \textbf{Input/Output Dataset Description:}
        \begin{itemize}
            \item \textbf{Input Data:} Describe the input dataset (size, format, data type) and provide a DOI or link to access it.
            \item \textbf{Output Data:} Specify the structure of the results (e.g., memory usage, runtime logs) and how they can be accessed or replicated.
            \item \textbf{Data Repository:} Indicate where the data is stored (e.g., Zenodo, institutional repository) and provide a DOI or URL for accessing the data.
        \end{itemize}
    \item \textbf{Results Summary:} Include a summary of key metrics (execution time, memory usage, FLOPS) and their comparison across architectures (e.g., CPU, GPU).
    \item \textbf{Challenges Identified:} Describe any bottlenecks encountered (e.g., memory usage, parallelization inefficiencies) and how they impacted the benchmark.
\end{itemize}

\subsection{12-Month Roadmap}
\label{sec:WP6:Uranie:roadmap}

In the context of exascale computation, Uranie will have to perform uncertainty quantification on more complex simulation software, that will be heavily parallelized using for instance MPI or OpenMP, and run on supercomputers. For the specific case of uncertainty propagation. Ensemble-runs of a simulation software has to be performed, and this can be tricky when the software is parallelized. Memory storage will also be challenging in the exascale era and "on the fly" handling of the output data generated by the simulation software has to be performed by Uranie. In the next deliverable, an adaptation of the Relauncher module using the ICoCo API (\url{https://github.com/cea-trust-platform/icoco-coupling}) will be proposed and illustrated on an uncertainty propagation task using TRUST software, with on the fly processing of the data generated using TRUST Python API.  

\begin{table}[h!]
    \centering
    
    

    \centering
    { 
        \setlength{\parindent}{0pt}
        \def\arraystretch{1.25}
        \arrayrulecolor{numpexgray}
        {
            \fontsize{9}{11}\selectfont
            \begin{tabular}{!{\color{numpexgray}\vrule}p{.25\linewidth}!{\color{numpexgray}\vrule}p{.6885\linewidth}!{\color{numpexgray}\vrule}}
    
    \rowcolor{numpexgray}{\rule{0pt}{2.5ex}\color{white}\bf Bottlenecks} &  {\rule{0pt}{2.5ex}\color{white}\bf Short Description }\\ 
    
\rowcolor{white} Coupling with parallelized software  & Nowadays more and more simulation software are parralelized, sometimes with shared memory. Moreover, multiphysics simulations implies chained or coupled simulation software. Uranie has to be capable to perform UQ studies on such simulation software. \\
\end{tabular}
        }
    }
    \caption{WP6: Uranie plan with Respect to Relevant Bottlenecks}
    \label{tab:WP6:Uranie:bottlenecks}
\end{table}
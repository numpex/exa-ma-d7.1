\section{Software: Uranie}
\label{sec:Uranie:software}



\begin{table}[ht!]
    \centering
    { \setlength{\parindent}{0pt}
    \def\arraystretch{1.25}
    \arrayrulecolor{numpexgray}
    {\fontsize{9}{11}\selectfont
    \begin{tabular}{!{\color{numpexgray}\vrule}p{.4\textwidth}!{\color{numpexgray}\vrule}p{.6\textwidth}!{\color{numpexgray}\vrule}}
        \rowcolor{numpexgray}{\rule{0pt}{2.5ex}\color{white}\bf Field} & {\rule{0pt}{2.5ex}\color{white}\bf Details} \\
        \rowcolor{white}\textbf{Consortium} & \begin{tabular}{l}
CEA\\
\end{tabular} \\
        \rowcolor{numpexlightergray}\textbf{Exa-MA Partners} & \begin{tabular}{l}
CEA\\
\end{tabular} \\
        \rowcolor{white}\textbf{Contact Emails} & \begin{tabular}{l}
clement.gauchy@cea.fr\\
jean-baptiste.blanchard@cea.fr\\
rudy.chocat@cea.fr\\
\end{tabular} \\
        \rowcolor{numpexlightergray}\textbf{Supported Architectures} & \begin{tabular}{l}
CPU, GPU\\
\end{tabular} \\
        \rowcolor{white}\textbf{Repository} & \href{https://uranie.cea.fr}{https://uranie.cea.fr} \\
        \rowcolor{numpexlightergray}\textbf{License} & \begin{tabular}{l}
OSS:: LGPL \\
\end{tabular} \\
        \rowcolor{white}\textbf{Bottlenecks roadmap} & \begin{tabular}{l}
None\\
\end{tabular} \\
        \bottomrule
    \end{tabular}
    }}
    \caption{Uranie Information}
\end{table}

\subsection{Software summary}
\label{sec:Uranie:summary}

Uranie platform is based on ROOT and has by consequence a lot of ROOT characteristics such as:
\begin{itemize}
    \item Interactive C++ interpreter (Cling) based on LLVM and Clang
    \item Python interface (PyROOT)
    \item SQL database access
\end{itemize}


\subsection{Purpose}
\label{sec:Uranie:purpose}

Uranie (the version under discussion here being v4.9.0) is a software dedicated to perform studies on uncertainty propagation, sensitivity analysis and surrogate model generation and calibration, based on ROOT (the corresponding version being v6.32.00). The motivation for the development of Uranie is the VVUQ (Verification, Validation and Uncertainty Quantification) approach for conceiving a numerical model of real physical phenomena of interests. Uranie is developed such that it interfaces well with CEA internal numerical simulation software.


\begin{table}[ht!]
    \centering
    {
    \setlength{\parindent}{0pt}
    \def\arraystretch{1.25}
    \arrayrulecolor{numpexgray}
    {\fontsize{9}{11}\selectfont
    \begin{tabular}{lp{.3\textwidth}p{.5\textwidth}}
        \rowcolor{numpexgray}{\rule{0pt}{2.5ex}\color{white}\bf Category}  & {\rule{0pt}{2.5ex}\color{white}\bf Details} & {\rule{0pt}{2.5ex}\color{white}\bf Description}\\
        \rowcolor{white}Languages  & \begin{tabular}{l}
C++\\
Python\\
\end{tabular} & Programming languages and language standards supported by the software \\
        \rowcolor{numpexlightergray}Parallelism  & \begin{tabular}{l}
GPU\\
MPI\\
Multithread\\
\end{tabular} & Parallel computing methods and frameworks utilized by the software.\\
        \rowcolor{white}Data Formats  & \begin{tabular}{l}
SALOME format\\
ROOT\\
SQL\\
\end{tabular} & Data formats that the software can handle or produce.\\
        \rowcolor{numpexlightergray}Resilience  & \begin{tabular}{l}
None\\
\end{tabular} & Fault tolerance and recovery mechanisms employed by the software.\\
        \rowcolor{white}DevOps & \begin{tabular}{l}
Continuous Integration\\
\end{tabular} & Outlines the development and operational practices including continuous integration, containerization, and testing methodologies.  \\
        \rowcolor{numpexlightergray}Packaging  & \begin{tabular}{l}
None\\
\end{tabular} & Software packaging and distribution.\\
        \rowcolor{white}Testing  & \begin{tabular}{l}
Unit\\
Validation\\
Verification\\
\end{tabular} & Testing methodologies employed to ensure software quality and correctness.\\
        \rowcolor{numpexlightergray}Containerization  & \begin{tabular}{l}
None\\
\end{tabular} & Container technologies used to package and deploy the software.\\
        \rowcolor{white}Interfaces  & \begin{tabular}{l}
Salome \\
Trust \\
\end{tabular} & List of software Uranie has interfaces with.\\
        \bottomrule
    \end{tabular}
    }}
    \caption{Uranie programming and computational environment}
\end{table}


\subsection{Application entry points}
\label{sec:Uranie:apps}
Uranie is used as the VVUQ and calibration backbone for several activities; see Chapter~\ref{chap:applications}:
\begin{itemize}
        \item Ensemble Kalman Inversion (calibration and UQ, proposed): Section~\ref{sec:app:specs:app-eki}.
\end{itemize}



\subsection{Mathematics}
\label{sec:Uranie:mathematics}
The mathematics used in Uranie are related to:
\begin{itemize}
        \item Surrogate modeling \& machine learning techniques: Gaussian process regression, neural networks, polynomial chaos expansion
        \item Optimization: EGO-like algorithm and ants colony based metaheuristics
        \item Calibration: Latent parameters estimations \& Bayesian algorithms
        \item Sensitivity analysis: HSIC indices estimation, Sobol' indices.
\end{itemize}


\subsection{Relevant Publications}
\label{sec:Uranie:publications}

Here is a relevant publication used to cite Uranie:

\begin{itemize}
   \item \fullcite{blanchard_uranie_2019}
\end{itemize}


\subsection{Acknowledgements}
\label{sec::Uranie:acknowledgements}

The software has been developed in \href{https://www.cea.fr/}{CEA} and funded by the SIMU research program.




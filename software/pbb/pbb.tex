\section{Software: pBB}
\label{sec:pBB:software}

\begin{table}[h!]
    \centering
    { \setlength{\parindent}{0pt}
    \def\arraystretch{1.25}
    \arrayrulecolor{numpexgray}
    {\fontsize{9}{11}\selectfont
    \begin{tabular}{!{\color{numpexgray}\vrule}p{.4\textwidth}!{\color{numpexgray}\vrule}p{.6\textwidth}!{\color{numpexgray}\vrule}}
        \rowcolor{numpexgray}{\rule{0pt}{2.5ex}\color{white}\bf Field} & {\rule{0pt}{2.5ex}\color{white}\bf Details} \\
        \rowcolor{white}\textbf{Consortium} & \begin{tabular}{l}
Université de Lille\\
\end{tabular} \\
        \rowcolor{numpexlightergray}\textbf{Exa-MA Partners} & \begin{tabular}{l}
Inria Lille\\
\end{tabular} \\
        \rowcolor{white}\textbf{Contact Emails} & \begin{tabular}{l}
nouredine.melab@univ-lille.fr\\
\end{tabular} \\
        \rowcolor{numpexlightergray}\textbf{Supported Architectures} & \begin{tabular}{l}
CPU or GPU\\
\end{tabular} \\
        \rowcolor{white}\textbf{Repository} & 
        \begin{tabular}{l}
             \href{https://gitlab.inria.fr/jgmys/permutationbb}{https://gitlab.inria.fr/jgmys/permutationbb}\\
            \href{https://github.com/Guillaume-Helbecque/P3D-DFS}{https://github.com/Guillaume-Helbecque/P3D-DFS} \\
        \end{tabular} \\
        \rowcolor{numpexlightergray}\textbf{License} & \begin{tabular}{l}
OSS: Cecill-*\\
\end{tabular} \\
        \rowcolor{white}\textbf{Bottlenecks roadmap} & \begin{tabular}{l}
B10 - Scientific Productivity\\
B11 - Reproducibility and Replicability of Computation\\
%B6 - Data Management\\
B7 - Exascale Algorithms\\
\end{tabular} \\
        \bottomrule
    \end{tabular}
    }}
    \caption{pBB Information}
\end{table}

\subsection{Software summary}
\label{sec:pBB:summary}

pBB is initially an implementation of a massively parallel Branch-and-Bound (B\&B) algorithm for the exact resolution of permutation-based optimization problems, like Permutation Flow-shop Scheduling (see \url{https://gitlab.inria.fr/jgmys/permutationbb}). pBB is designed using the bare-metal MPI+X approach. 
First, pBB has been extended to improve its genericity w.r.t optimization problems than can be solved, going beyond the permutation ones, like Knapsack problems. A new data structure named distBag-DFS is proposed for that purpose. 
In addition, a PGAS-guided design approach is used to improve its software productivity-awareness (see \url{https://github.com/Guillaume-Helbecque/P3D-DFS}). The Chapel language is used for this implementation of pBB meeting these genericity and productivity objectives. 

\subsection{Purpose}
\label{sec:pBB:purpose}

The purpose of pBB is to provide an efficient, scalable, and general framework for solving exact optimization problems using B\&B. Three main properties characterize pBB:
\begin{itemize}
    \item \textbf{Generalization}: our goal is to build a framework which is applicable to various optimization problems (e.g., Flow-shop Scheduling, Knapsack problems).

    \item \textbf{Massively parallel}: a transparent and efficient parallel implementation of the algorithms on various architectures (e.g., multicore CPUs, GPUs, clusters) is carried out. The main challenge is the parallelization of the tree search component of the framework. Many parallel tree search algorithms can be considered.

    \item \textbf{Productivity}: pBB enhances productivity by unifying inter-node, intra-node, and GPU levels through a PGAS-based design approach. This simplifies the development and maintenance of parallel algorithms, allowing developers to focus on high-level design while ensuring efficient execution across diverse architectures.
\end{itemize}

\subsection{Programming and Computational Environment}
\label{sec::pBB:environment_capabilities}

The following table summarizes these aspects for pBB, providing a  view of its programming and computational capabilities.

\begin{table}[h!]
    \centering
    {
    \setlength{\parindent}{0pt}
    \def\arraystretch{1.25}
    \arrayrulecolor{numpexgray}
    {\fontsize{9}{11}\selectfont
    \begin{tabular}{lp{.3\textwidth}p{.5\textwidth}}
        \rowcolor{numpexgray}{\rule{0pt}{2.5ex}\color{white}\bf Category}  & {\rule{0pt}{2.5ex}\color{white}\bf Details} & {\rule{0pt}{2.5ex}\color{white}\bf Description}\\
        \rowcolor{white}Languages  & \begin{tabular}{l}
C++\\
Chapel\\
\end{tabular} & Programming languages and language standards supported by the software \\
        \rowcolor{numpexlightergray}Parallelism  & \begin{tabular}{l}
Chapel\\
CUDA and HIP\\
MPI\\
OpenMP and PThreads\\
\end{tabular} & Parallel computing methods and frameworks utilized by the software.\\
        \rowcolor{white}Data Formats  & \begin{tabular}{l}
None\\
\end{tabular} & Data formats that the software can handle or produce.\\
        \rowcolor{numpexlightergray}Resilience  & \begin{tabular}{l}
Checkpoint restart\\
\end{tabular} & Fault tolerance and recovery mechanisms employed by the software.\\
        \rowcolor{white}DevOps & \begin{tabular}{l}
None\\
\end{tabular} & Outlines the development and operational practices including continuous integration, containerization, and testing methodologies.  \\
        \rowcolor{numpexlightergray}Packaging  & \begin{tabular}{l}
None\\
\end{tabular} & Software packaging and distribution.\\
        \rowcolor{white}Testing  & \begin{tabular}{l}
None\\
\end{tabular} & Testing methodologies employed to ensure software quality and correctness.\\
        \rowcolor{numpexlightergray}Containerization  & \begin{tabular}{l}
None\\
\end{tabular} & Container technologies used to package and deploy the software.\\
        \rowcolor{white}Interfaces  & \begin{tabular}{l}
None\\
\end{tabular} & List of software pBB has interfaces with.\\
        \bottomrule
    \end{tabular}
    }}
    \caption{pBB programming and computational environment}
\end{table}

\subsection{Mathematics}
\label{sec:pBB:mathematics}

Combinatorial Optimization Problems (COP) consist in finding an object within a finite (or countably infinite) set which is optimal according to a given criterion. Formally, a COP can be defined as a couple $(X, f)$, where $X$ is the search space and $f:X\rightarrow \mathbb{R}$ the objective function to be minimized or maximized. Constraints that must be fulfilled by a feasible solution $x\in X$ can be incorporated in the definition of the search space $X$ or the objective function $f$. The objective function $f$ takes its values in a totally ordered set, usually the set of real numbers or integers. The value $f(x)$ measures the cost (e.g., quality, time, benefit) of solution $x\in X$. The goal is to find one or multiple solution(s) $x^*\in X$ that are feasible and satisfy $f(x^*)\leq f(x), \forall x\in X$ in the case of minimization, or $f(x^*)\geq f(x), \forall x\in X$ in the case of maximization.

\subsection{Relevant Publications}
\label{sec:pBB:publications}

Here is a list of relevant publications related to the software:

Jan Gmys. 2022. Exactly Solving Hard Permutation Flowshop Scheduling Problems on Peta-Scale GPU-Accelerated Supercomputers. INFORMS J. on Computing 34, 5, 2502–2522. https://doi.org/10.1287/ijoc.2022.1193

This publication presents significant advancements in the field of combinatorial optimization, specifically targeting makespan minimization in permutation Flowshop Scheduling, a notoriously hard problem. Leveraging a GPU-accelerated B\&B approach, pBB allows to solve 11 previously unsolved benchmark instances from Taillard's 1993 benchmarks. By utilizing the computational power of peta-scale high-performance computing platforms, the study demonstrates how parallel search techniques can efficiently traverse highly irregular search trees on distributed systems, providing key insights for optimization researchers focused on leveraging GPUs and multicore processors for large-scale problem-solving.

Guillaume Helbecque, Tiago Carneiro, Nouredine Melab, Jan Gmys, Pascal Bouvry. PGAS Data Structure for Unbalanced Tree-Based Algorithms at Scale. Computational Science – ICCS 2024. 2024. https://doi.org/10.1007/978-3-031-63759-9\_13

This publication focuses on advancing the design of parallel algorithms for modern supercomputers by introducing a PGAS-based data structure and a Work-Stealing mechanism. These innovations target unbalanced tree exploration using depth-first search, a common challenge in parallel tree-based algorithms. Implemented in Chapel, the proposed solution demonstrates promising scalability in single-node backtracking experiments using the Unbalanced Tree Search benchmark. Additionally, large-scale experiments using B\&B for Flowshop Scheduling reveal significant strong scaling efficiency on 400 computer nodes (51,200 CPU cores), making it relevant for researchers interested in optimizing parallel workloads on distributed systems.

Helbecque G, Gmys J, Melab N, Carneiro T, Bouvry P. Parallel distributed productivity-aware tree-search using Chapel. Concurrency Computat Pract Exper. 2023; 35(27):e7874. https://doi.org/10.1002/cpe.7874

This publication explores the design and implementation of pBB, using the PGAS Chapel programming language, particularly suited for exascale computing. It exploits the PGAS data structure of previous publication. Experimental results, using up to 4,096 CPU cores, compare Chapel's implementation to OpenMP (intra-node) and MPI+X (inter-node) counterparts, highlighting competitive performance in both shared and distributed memory settings. The study underscores Chapel as a viable alternative to traditional parallel programming models for exascale-aware applications.

\subsection{Acknowledgements}
\label{sec::pBB:acknowledgements}

The software has been developed with the support of the following funding agencies and institutions: Université de Lille, Inria Lille, ANR, and FNR.
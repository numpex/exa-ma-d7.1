\section{Software: pBB}
\label{sec:pBB:software}

\begin{table}[h!]
    \centering
    { \setlength{\parindent}{0pt}
    \def\arraystretch{1.25}
    \arrayrulecolor{numpexgray}
    {\fontsize{9}{11}\selectfont
    \begin{tabular}{!{\color{numpexgray}\vrule}p{.4\textwidth}!{\color{numpexgray}\vrule}p{.6\textwidth}!{\color{numpexgray}\vrule}}
        \rowcolor{numpexgray}{\rule{0pt}{2.5ex}\color{white}\bf Field} & {\rule{0pt}{2.5ex}\color{white}\bf Details} \\
        \rowcolor{white}\textbf{Consortium} & \begin{tabular}{l}
Université de Lille\\
\end{tabular} \\
        \rowcolor{numpexlightergray}\textbf{Exa-MA Partners} & \begin{tabular}{l}
Inria Lille\\
\end{tabular} \\
        \rowcolor{white}\textbf{Contact Emails} & \begin{tabular}{l}
nouredine.melab@univ-lille.fr\\
\end{tabular} \\
        \rowcolor{numpexlightergray}\textbf{Supported Architectures} & \begin{tabular}{l}
CPU or GPU\\
\end{tabular} \\
        \rowcolor{white}\textbf{Repository} & 
        \begin{tabular}{l}
             \href{https://gitlab.inria.fr/jgmys/permutationbb}{https://gitlab.inria.fr/jgmys/permutationbb}\\
            \href{https://github.com/Guillaume-Helbecque/P3D-DFS}{https://github.com/Guillaume-Helbecque/P3D-DFS} \\
        \end{tabular} \\
        \rowcolor{numpexlightergray}\textbf{License} & \begin{tabular}{l}
OSS: Cecill-*\\
\end{tabular} \\
        \rowcolor{white}\textbf{Bottlenecks roadmap} & \begin{tabular}{l}
B10 - Scientific Productivity\\
Productivity is first related to the different parallel levels of supercomputers including the inter-node, intra-node and GPU levels. In pBB, productivity is ensured using the PGAS-based design approach. The Chapel language is used for the implementation of PGAS. In this latter, the different parallel levels are unified. Second, pBB provides a data structure (P3D-DFS) that can be used in a productive way for different optimization problems.  
B11 - Reproducibility and Replicability of Computation\\
%B6 - Data Management\\
B7 - Exascale Algorithms\\
\end{tabular} \\
        \bottomrule
    \end{tabular}
    }}
    \caption{pBB Information}
\end{table}

\subsection{Software summary}
\label{sec:pBB:summary}

pBB is initially an implementation of a massively parallel Branch-and-Bound (B\&B) algorithm for the exact resolution of permutation-based optimization problems, like Permutation Flow-shop Scheduling (see \url{https://gitlab.inria.fr/jgmys/permutationbb}). pBB is designed using the bare-metal MPI+X approach. 
First, pBB has been extended to improve its genericity w.r.t optimization problems than can be solved, going beyond the permutation ones, like Knapsack problems. A new data structure named distBag-DFS is proposed for that purpose. 
In addition, a PGAS-guided design approach is used to improve its software productivity-awareness (see \url{https://github.com/Guillaume-Helbecque/P3D-DFS}). The Chapel language is used for this implementation of pBB meeting these genericity and productivity objectives. 

\subsection{Purpose}
\label{sec:pBB:purpose}
Three main properties characterize pBB:
\begin{itemize}
    \item \textbf{Generalization}: our goal is to build a framework which unifies and generalizes various popular fractal decomposition-based algorithms from different communities (e.g. global optimization, reinforcement learning, computational intelligence).

    \item

    \item \textbf{Massively parallel}: a transparent and efficient parallel implementation of the algorithms on various architectures (e.g. multicores, GPUs) is carried out. The main challenge is the parallelization of the tree search component of the framework. Many parallel tree search algorithms can be considered.
\end{itemize}

\subsection{Programming and Computational Environment}
\label{sec::pBB:environment_capabilities}

The following table summarizes these aspects for pBB, providing a  view of its programming and computational capabilities.

\begin{table}[h!]
    \centering
    {
    \setlength{\parindent}{0pt}
    \def\arraystretch{1.25}
    \arrayrulecolor{numpexgray}
    {\fontsize{9}{11}\selectfont
    \begin{tabular}{lp{.3\textwidth}p{.5\textwidth}}
        \rowcolor{numpexgray}{\rule{0pt}{2.5ex}\color{white}\bf Category}  & {\rule{0pt}{2.5ex}\color{white}\bf Details} & {\rule{0pt}{2.5ex}\color{white}\bf Description}\\
        \rowcolor{white}Languages  & \begin{tabular}{l}
C++\\
Chapel\\
\end{tabular} & Programming languages and language standards supported by the software \\
        \rowcolor{numpexlightergray}Parallelism  & \begin{tabular}{l}
Chapel\\
CUDA and HIP\\
MPI\\
OpenMP and PThreads\\
\end{tabular} & Parallel computing methods and frameworks utilized by the software.\\
        \rowcolor{white}Data Formats  & \begin{tabular}{l}
None\\
\end{tabular} & Data formats that the software can handle or produce.\\
        \rowcolor{numpexlightergray}Resilience  & \begin{tabular}{l}
Checkpoint restart\\
\end{tabular} & Fault tolerance and recovery mechanisms employed by the software.\\
        \rowcolor{white}DevOps & \begin{tabular}{l}
None\\
\end{tabular} & Outlines the development and operational practices including continuous integration, containerization, and testing methodologies.  \\
        \rowcolor{numpexlightergray}Packaging  & \begin{tabular}{l}
None\\
\end{tabular} & Software packaging and distribution.\\
        \rowcolor{white}Testing  & \begin{tabular}{l}
None\\
\end{tabular} & Testing methodologies employed to ensure software quality and correctness.\\
        \rowcolor{numpexlightergray}Containerization  & \begin{tabular}{l}
None\\
\end{tabular} & Container technologies used to package and deploy the software.\\
        \rowcolor{white}Interfaces  & \begin{tabular}{l}
None\\
\end{tabular} & List of software pBB has interfaces with.\\
        \bottomrule
    \end{tabular}
    }}
    \caption{pBB programming and computational environment}
\end{table}

\subsection{Mathematics}
\label{sec:pBB:mathematics}

Combinatorial Optimization Problems (COP) consist in finding an object within a finite (or countably infinite) set which is optimal according to a given criterion. Formally, a COP can be defined as a couple $(X, f)$, where $X$ is the search space and $f:X\rightarrow \mathbb{R}$ the objective function to be minimized or maximized. Constraints that must be fulfilled by a feasible solution $x\in X$ can be incorporated in the definition of the search space $X$ or the objective function $f$. The objective function $f$ takes its values in a totally ordered set, usually the set of real numbers or integers. The value $f(x)$ measures the cost (e.g., quality, time, benefit) of solution $x\in X$. The goal is to find one or multiple solution(s) $x^*\in X$ that are feasible and satisfy $f(x^*)\leq f(x), \forall x\in X$ in the case of minimization, or $f(x^*)\geq f(x), \forall x\in X$ in the case of maximization.

\subsection{Relevant Publications}
\label{sec:pBB:publications}

Here is a list of relevant publications related to the software:

Guillaume Helbecque, Tiago Carneiro, Nouredine Melab, Jan Gmys, Pascal Bouvry. PGAS Data Structure for Unbalanced Tree-Based Algorithms at Scale. Computational Science – ICCS 2024. 2024. https://doi.org/10.1007/978-3-031-63759-9\_13

Helbecque G, Gmys J, Melab N, Carneiro T, Bouvry P. Parallel distributed productivity-aware tree-search using Chapel. Concurrency Computat Pract Exper. 2023; 35(27):e7874. https://doi.org/10.1002/cpe.7874

\subsection{Acknowledgements}
\label{sec::pBB:acknowledgements}

The software has been developed with the support of the following funding agencies and institutions: 

Acknowledgements not available.



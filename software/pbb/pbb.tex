\section{Software: pBB}
\label{sec:pBB:software}


\begin{table}[h!]
    \centering
pBB is initially an implementation of a massively parallel Branch&Bound algorithm for the exact resolution of permutation-based optimization problems, like Permutation Flow-shop Scheduling (see https://gitlab.inria.fr/jgmys/permutationbb). pBB is designed using the bare-metal MPI+X approach. First, pBB has been extended to improve its genericity w.r.t optimization problems than can be solved, going beyond the permutation ones. A new data structure named P3D-DFS is proposed for that purpose. In addition, a PGAS-guided design approach is used to improve its software productivity-awareness (see https://github.com/Guillaume-Helbecque). The Chapel language is used for the implementation of pBB meeting these genericity and productivity objectives. 
    { \setlength{\parindent}{0pt}
    \def\arraystretch{1.25}
    \arrayrulecolor{numpexgray}
    {\fontsize{9}{11}\selectfont
    \begin{tabular}{!{\color{numpexgray}\vrule}p{.4\textwidth}!{\color{numpexgray}\vrule}p{.6\textwidth}!{\color{numpexgray}\vrule}}
        \rowcolor{numpexgray}{\rule{0pt}{2.5ex}\color{white}\bf Field} & {\rule{0pt}{2.5ex}\color{white}\bf Details} \\
        \rowcolor{white}\textbf{Consortium} & \begin{tabular}{l}
Université de Lille\\
\end{tabular} \\
        \rowcolor{numpexlightergray}\textbf{Exa-MA Partners} & \begin{tabular}{l}
Inria Lille\\
\end{tabular} \\
        \rowcolor{white}\textbf{Contact Emails} & \begin{tabular}{l}
nouredine.melab@univ-lille.fr\\
\end{tabular} \\
        \rowcolor{numpexlightergray}\textbf{Supported Architectures} & \begin{tabular}{l}
CPU or GPU\\
\end{tabular} \\
        \rowcolor{white}\textbf{Repository} & \href{https://gitlab.inria.fr/jgmys/permutationbb}{https://gitlab.inria.fr/jgmys/permutationbb} \\
        \rowcolor{numpexlightergray}\textbf{License} & \begin{tabular}{l}
OSS: Cecill-*\\
\end{tabular} \\
        \rowcolor{white}\textbf{Bottlenecks roadmap} & \begin{tabular}{l}
B10 - Scientific Productivity\\
Productivity is first related to the different parallel levels of supercomputers including the inter-node, intra-node and GPU levels. In pBB, productivity is ensured using the PGAS-based design approach. The Chapel language is used for the implementation of PGAS. In this latter, the different parallel levels are unified. Second, pBB provides a data structure (P3D-DFS) that can be used in a productive way for different optimization problems.  
B11 - Reproducibility and Replicability of Computation\\
%B6 - Data Management\\
B7 - Exascale Algorithms\\
\end{tabular} \\
        \bottomrule
    \end{tabular}
    }}
    \caption{pBB Information}
\end{table}

\subsection{Software summary}
\label{sec:pBB:summary}
%Detailed overview not available.

\subsection{Purpose}
\label{sec:pBB:purpose}
Purpose not available.

\subsection{Programming and Computational Environment}
\label{sec::pBB:environment_capabilities}


The following table summarizes these aspects for pBB, providing a  view of its programming and computational capabilities.

\begin{table}[h!]
    \centering
    {
    \setlength{\parindent}{0pt}
    \def\arraystretch{1.25}
    \arrayrulecolor{numpexgray}
    {\fontsize{9}{11}\selectfont
    \begin{tabular}{lp{.3\textwidth}p{.5\textwidth}}
        \rowcolor{numpexgray}{\rule{0pt}{2.5ex}\color{white}\bf Category}  & {\rule{0pt}{2.5ex}\color{white}\bf Details} & {\rule{0pt}{2.5ex}\color{white}\bf Description}\\
        \rowcolor{white}Languages  & \begin{tabular}{l}
C++\\
\end{tabular} & Programming languages and language standards supported by the software \\
        \rowcolor{numpexlightergray}Parallelism  & \begin{tabular}{l}
Chapel\\
CUDA and HIP\\
MPI\\
OpenMP and PThreads\\
\end{tabular} & Parallel computing methods and frameworks utilized by the software.\\
        \rowcolor{white}Data Formats  & \begin{tabular}{l}
None\\
\end{tabular} & Data formats that the software can handle or produce.\\
        \rowcolor{numpexlightergray}Resilience  & \begin{tabular}{l}
Checkpoint restart\\
\end{tabular} & Fault tolerance and recovery mechanisms employed by the software.\\
        \rowcolor{white}DevOps & \begin{tabular}{l}
None\\
\end{tabular} & Outlines the development and operational practices including continuous integration, containerization, and testing methodologies.  \\
        \rowcolor{numpexlightergray}Packaging  & \begin{tabular}{l}
None\\
\end{tabular} & Software packaging and distribution.\\
        \rowcolor{white}Testing  & \begin{tabular}{l}
None\\
\end{tabular} & Testing methodologies employed to ensure software quality and correctness.\\
        \rowcolor{numpexlightergray}Containerization  & \begin{tabular}{l}
None\\
\end{tabular} & Container technologies used to package and deploy the software.\\
        \rowcolor{white}Interfaces  & \begin{tabular}{l}
None\\
\end{tabular} & List of software pBB has interfaces with.\\
        \bottomrule
    \end{tabular}
    }}
    \caption{pBB programming and computational environment}
\end{table}

\subsection{Mathematics}
\label{sec:pBB:mathematics}
Mathematics not available.

In this section, provide a summary the mathematics used in the software.

\subsection{Relevant Publications}
\label{sec:pBB:publications}

Here is a list of relevant publications related to the software:

\subsection{Acknowledgements}
\label{sec::pBB:acknowledgements}

The software has been developed with the support of the following funding agencies and institutions: 

Acknowledgements not available.



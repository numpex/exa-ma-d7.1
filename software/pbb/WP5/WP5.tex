\section{Software: pBB}
\label{sec:WP5:pBB:software}

\begin{table}[h!]
    \centering
    { \setlength{\parindent}{0pt}
    \def\arraystretch{1.25}
    \arrayrulecolor{numpexgray}
    {\fontsize{9}{11}\selectfont
    \begin{tabular}{!{\color{numpexgray}\vrule}p{.4\textwidth}!{\color{numpexgray}\vrule}p{.6\textwidth}!{\color{numpexgray}\vrule}}
        \rowcolor{numpexgray}{\rule{0pt}{2.5ex}\color{white}\bf Field} & {\rule{0pt}{2.5ex}\color{white}\bf Details} \\
        \rowcolor{white}\textbf{Consortium} & \begin{tabular}{l}
Université de Lille\\
\end{tabular} \\
        \rowcolor{numpexlightergray}\textbf{Exa-MA Partners} & \begin{tabular}{l}
Inria Lille\\
\end{tabular} \\
        \rowcolor{white}\textbf{Contact Emails} & \begin{tabular}{l}
nouredine.melab@univ-lille.fr\\
\end{tabular} \\
        \rowcolor{numpexlightergray}\textbf{Supported Architectures} & \begin{tabular}{l}
CPU or GPU\\
\end{tabular} \\
        \rowcolor{white}\textbf{Repository} & \begin{tabular}{l}
             \href{https://gitlab.inria.fr/jgmys/permutationbb}{https://gitlab.inria.fr/jgmys/permutationbb}\\
            \href{https://github.com/Guillaume-Helbecque/P3D-DFS}{https://github.com/Guillaume-Helbecque/P3D-DFS} \\
        \end{tabular} \\
        \rowcolor{numpexlightergray}\textbf{License} & \begin{tabular}{l}
OSS: Cecill-*\\
\end{tabular} \\
        \rowcolor{white}\textbf{Bottlenecks roadmap} & \begin{tabular}{l}
B10 - Scientific Productivity\\
B11 - Reproducibility and Replicability of Computation\\
%B6 - Data Management\\
B7 - Exascale Algorithms\\
\end{tabular} \\
        \bottomrule
    \end{tabular}
    }}
    \caption{WP5: pBB Information}
\end{table}

\subsection{Software Overview}
\label{sec:WP5:pBB:summary}

pBB is initially an implementation of a massively parallel Branch-and-Bound (B\&B) algorithm for the exact resolution of permutation-based optimization problems, like Permutation Flow-shop Scheduling (see \url{https://gitlab.inria.fr/jgmys/permutationbb}). pBB is designed using the bare-metal MPI+X approach. 
First, pBB has been extended to improve its genericity w.r.t optimization problems than can be solved, going beyond the permutation ones, like Knapsack problems. A new data structure named distBag-DFS is proposed for that purpose. 
In addition, a PGAS-guided design approach is used to improve its software productivity-awareness (see \url{https://github.com/Guillaume-Helbecque/P3D-DFS}). The Chapel language is used for this implementation of pBB meeting these genericity and productivity objectives. 

In~\cref{tab:WP5:pBB:features} we provide a summary of the software features relevant to the work package which are briefly discussed.

\begin{table}[h!]
    \centering
    { 
        \setlength{\parindent}{0pt}
        \def\arraystretch{1.25}
        \arrayrulecolor{numpexgray}
        {
            \fontsize{9}{11}\selectfont
            \begin{tabular}{!{\color{numpexgray}\vrule}p{.25\linewidth}!{\color{numpexgray}\vrule}p{.6885\linewidth}!{\color{numpexgray}\vrule}}
    
    \rowcolor{numpexgray}{\rule{0pt}{2.5ex}\color{white}\bf Features} &  {\rule{0pt}{2.5ex}\color{white}\bf Short Description }\\ 
    
\rowcolor{white} Branch-and-Bound algorithms & Branch-and-Bound is a general-purpose algorithm for solving optimization problems. It systematically explores the solution space using four key operators: branching, bounding, selection, and pruning. Branching divides the problem into smaller subproblems. Bounding calculates bounds to estimate the potential of subproblems. Selection chooses the most promising subproblems for further exploration. Pruning discards subproblems that cannot yield a better solution than the current best. \\
\end{tabular}
        }
    }
    \caption{WP5: pBB Features}
    \label{tab:WP5:pBB:features}
\end{table}

\subsection{Parallel Capabilities}
\label{sec:WP5:pBB:performances}

\begin{itemize}
    \item pBB includes several parallel implementations at different parallel levels: multi-core, GPU and multi-GPU, (distributed) cluster. For the MPI+X approach, several parallel environments are used including MPI, OpenMP and PThreads, Cuda and HIP. For the PGAS approach, all parallel levels are mananged using the Chapel programming language.
    \item pBB has been used to solve optimization problems on Grid'5000, Meluxina (Luxembourg) and LUMI (Finland) EuroHPC supercomputers.
    %\item  
    \item \textbf{Scalability:} pBB is scalable at the intra-node level (multi-core) as well as inter-node level (distributed). However, its scalability w.r.t GPUs has to be improved.
    %\item \textbf{Integration with Other Systems:} Describe how the software integrates with other numerical libraries in the Exa-MA framework.
\end{itemize}

\subsection{Initial Performance Metrics}
\label{sec:WP5:pBB:metrics}

This section provides a summary of initial performance benchmarks performed in the context of WP5. %It ensures reproducibility by detailing input/output datasets, benchmarking tools, and the results. All data should be publicly available, ideally with a DOI for future reference.

\begin{itemize}
    \item \textbf{Overall Performance:}
    pBB presented significant advancements in the field of combinatorial optimization, specifically targeting makespan minimization in permutation Flowshop Scheduling, a notoriously hard problem. Leveraging up to 384 NVIDIA V100 GPUs, pBB allowed to solve 11 previously unsolved benchmark instances from Taillard's 1993 benchmarks. By utilizing the computational power of peta-scale high-performance computing platforms, the study demonstrates how parallel search techniques can efficiently traverse highly irregular search trees on distributed systems.
    On a large-scale CPU-based systems, pBB also demonstrates significant performance, as 50\% of strong scaling efficiency is achieved using 400 computer nodes (51,200 CPU cores).
    \item \textbf{Input/Output Dataset:} %Provide a detailed description of the dataset used for the benchmark, including:
        \begin{itemize}
            \item Optimization problems to solve are defined as functions: ${f: \mathcal{X} \rightarrow \mathbb{R}}$, where $\hat{x}$ is the optimum, $f$ the objective function, and $\mathcal{X}$ is the search space incorporating a set of constraints.
            \item The output is mainly the solutions found (i.e, vectors of integers) and their quality, along with some execution statistics (e.g., size of the explored tree).
            \item Our main dataset consists in well-known permutation Flowshop Scheduling instances introduced in: Taillard, E., Benchmarks for basic scheduling problems, \textit{European Journal of Operational Research}, 64, 2, 278-285, 1993. \url{https://doi.org/10.1016/0377-2217(93)90182-M}. Instances are generated using the generator described in the reference.
            %\item DOI or permanent link for accessing the dataset.
        \end{itemize}
    \item \textbf{open-data Access:} Links are provided in the previous section.
    \item \textbf{Challenges:} 
    The main challenges of the benchmarking process are twofold:
    \begin{itemize}
        \item Scalability of parallel optimization algorithms: it consists in analyzing and improving the obtained speedups of the various parallel models that can be designed.
        \item Characteristics of the target optimization problem: it consists in studying the characteristics of the optimization problems (e.g. dimension, cost of the objective function) that affect the performance of the algorithms.
    \end{itemize}
    \item \textbf{Future Improvements:} 
    The future improvements will concern the following aspects:
    \begin{itemize}
        \item Fault tolerance aspects have to be deepened.
        \item In addition to standard benchmarks of optimization problems, it will be interesting to consider real-life complex applications.
    \end{itemize}
\end{itemize}

\subsubsection{Benchmark \#1}
\begin{itemize}
    \item \textbf{Description:} Our main benchmark consists in well-known permutation Flowshop Scheduling instances introduced in: Taillard, E., Benchmarks for basic scheduling problems, \textit{European Journal of Operational Research}, 64, 2, 278-285, 1993. \url{https://doi.org/10.1016/0377-2217(93)90182-M}. It includes many instance sizes, ranging from 5 machines $\times$ 20 jobs to 20 machines $\times$ 500 jobs. This benchmark is mainly used for testing performance scalability.
    \item \textbf{Benchmarking Tools Used:} The execution time and speed-up metrics are mainly used in our experiments.
    \item \textbf{Input/Output Dataset Description:}
        \begin{itemize}
            \item \textbf{Input Data:} Instances are generated using the generator described in the above reference.
            \item \textbf{Output Data:} The output data contain statistics to check correctness (e.g., size of the explored tree, optimum found) as well as the total execution time of the algorithm.
            %\item \textbf{Data Repository:} Indicate where the data is stored (e.g., Zenodo, institutional repository) and provide a DOI or URL for accessing the data.
        \end{itemize}
    %\item \textbf{Results Summary:} Include a summary of key metrics (execution time, memory usage, FLOPS) and their comparison across architectures (e.g., CPU, GPU).
    %\item \textbf{Challenges Identified:} Describe any bottlenecks encountered (e.g., memory usage, parallelization inefficiencies) and how they impacted the benchmark.
\end{itemize}

\subsection{12-Month Roadmap}
\label{sec:WP5:pBB:roadmap}

In this section, we describe the roadmap for improving benchmarks and addressing the challenges identified. %This should include:
\begin{itemize}
    \item \textbf{Data Improvements:} There are no issues concerning the improvements of input/output data management. All datasets are accessible and reproducibility is ensured through open-data initiatives.
    %\item \textbf{Methodology Application:} Implementation of the benchmarking methodology proposed in this deliverable to streamline reproducibility and dataset management.
    \item \textbf{Results Retention:} Plans to maintain benchmark results in a publicly accessible repository with appropriate metadata and documentation, ensuring long-term usability. All obtained results are open access and are published in HAL, conferences and journals.
\end{itemize}

In~\cref{tab:WP5:pBB:bottlenecks}, we briefly discuss the bottleneck roadmap associated to the software and relevant to the work package.

\begin{table}[h!]
    \centering
    
    

    \centering
    { 
        \setlength{\parindent}{0pt}
        \def\arraystretch{1.25}
        \arrayrulecolor{numpexgray}
        {
            \fontsize{9}{11}\selectfont
            \begin{tabular}{!{\color{numpexgray}\vrule}p{.25\linewidth}!{\color{numpexgray}\vrule}p{.6885\linewidth}!{\color{numpexgray}\vrule}}
    
    \rowcolor{numpexgray}{\rule{0pt}{2.5ex}\color{white}\bf Bottlenecks} &  {\rule{0pt}{2.5ex}\color{white}\bf Short Description }\\ 
    
\rowcolor{white}    B10 - Scientific Productivity & provide short description here \\
\rowcolor{numpexlightergray}    B11 - Reproducibility and Replicability of Computation & Provide validation quantities for each benchmark. \\
%\rowcolor{white}    B6 - Data Management & provide short description here \\
\rowcolor{white}    B7 - Exascale Algorithms & Fault tolerance aspects will be deepened. \\
\end{tabular}
        }
    }
    \caption{WP5: pBB plan with Respect to Relevant Bottlenecks}
    \label{tab:WP5:pBB:bottlenecks}
\end{table}
\section{Software: Scimba}
\label{sec:Scimba:software}



\begin{table}[h!]
    \centering
    { \setlength{\parindent}{0pt}
        \def\arraystretch{1.25}
        \arrayrulecolor{numpexgray}
        {\fontsize{9}{11}\selectfont
            \begin{tabular}{!{\color{numpexgray}\vrule}p{.4\textwidth}!{\color{numpexgray}\vrule}p{.6\textwidth}!{\color{numpexgray}\vrule}}
                \rowcolor{numpexgray}{\rule{0pt}{2.5ex}\color{white}\bf Field}
                 & {\rule{0pt}{2.5ex}\color{white}\bf Details}                                         \\
                \rowcolor{white}\textbf{Consortium}
                 & \begin{tabular}{l}
                       INRIA   \\
                       UNISTRA \\
                   \end{tabular}                                                                   \\
                \rowcolor{numpexlightergray}\textbf{Exa-MA Partners}
                 & \begin{tabular}{l}
                       Unistra \\
                   \end{tabular}                                                                   \\
                \rowcolor{white}\textbf{Contact Emails}
                 & \begin{tabular}{l}
                       emmanuel.franck@inria.fr      \\
                       victor.michel-dansac@inria.fr \\
                   \end{tabular}                                                       \\
                \rowcolor{numpexlightergray}\textbf{Supported Architectures}
                 & \begin{tabular}{l}
                       CPU or GPU \\
                   \end{tabular}                                                                   \\
                \rowcolor{white}\textbf{Repository}
                 & \href{https://gitlab.inria.fr/scimba/scimba}{https://gitlab.inria.fr/scimba/scimba} \\
                \rowcolor{numpexlightergray}\textbf{License}
                 & \begin{tabular}{l}
                       MIT \\
                   \end{tabular}                                                                   \\
                \rowcolor{white}\textbf{Bottlenecks roadmap}
                 & \begin{tabular}{l}
                       B10 - Scientific Productivity                          \\
                       B11 - Reproducibility and Replicability of Computation \\
                       B6 - Data Management                                   \\
                       B7 - Exascale Algorithms                               \\
                   \end{tabular}                              \\
                \bottomrule
            \end{tabular}
        }}
    \caption{Scimba Information}
\end{table}

\subsection{Software summary}
\label{sec:Scimba:summary}

Scimba\footnote{\url{https://gitlab.inria.fr/scimba/scimba}} is a software package that provides a framework for solving partial differential equations using physics-informed learning. It is mainly developed by Inria researchers (E. Franck and V. Michel-Dansac), with support from CNRS and the University of Strasbourg.

\subsection{Purpose}
\label{sec:Scimba:purpose}

The purpose of Scimba is to provide a readily-available
library for physics-informed machine learning.
It has two main objectives:
\begin{itemize}
    \item being used as a tool for solving PDEs
          with differentiable classical or ML-based methods,
          before exploiting their numerical solutions
          and differentiable properties;
    \item provide an easy framework in which users
          may develop new differentiable numerical methods
          (ML-based or classical).
\end{itemize}
As such, it aims at being easy to install and use,
while providing existing ML-based and classical methods
(see~\Cref{sec:Scimba:mathematics}).
It also aims at providing good performance,
using the intrinsic prarallelizability of
ML-based methods and libraries.

\subsection{Programming and Computational Environment}
\label{sec::Scimba:environment_capabilities}

The following table summarizes these aspects for Scimba, providing a  view of its programming and computational capabilities.

\begin{table}[h!]
    \centering
    {
        \setlength{\parindent}{0pt}
        \def\arraystretch{1.25}
        \arrayrulecolor{numpexgray}
        {\fontsize{9}{11}\selectfont
            \begin{tabular}{lp{.3\textwidth}p{.5\textwidth}}
                \rowcolor{numpexgray}{\rule{0pt}{2.5ex}\color{white}\bf Category}
                 & {\rule{0pt}{2.5ex}\color{white}\bf Details}
                 & {\rule{0pt}{2.5ex}\color{white}\bf Description}                                                                                    \\
                \rowcolor{white}Languages
                 & \begin{tabular}{l}
                       Python \\
                   \end{tabular}
                 & Scimba is written in \texttt{Python}, using the \texttt{pytorch} library for ML and differentiable computing.                      \\
                \rowcolor{numpexlightergray}Parallelism
                 & \begin{tabular}{l}
                       GPU, CPU (shared) \\
                   \end{tabular}
                 & Scimba takes advantage of the intrinsic parallelization capabilities of \texttt{pytorch} (shared-memory CPU and GPU, by default).  \\
                \rowcolor{white}Data Formats
                 & input: none;
                output: \texttt{.pth} \& image files
                 & Scimba outputs \texttt{.pth} files containing neural networks weights, as well as image files representing the equation solutions. \\
                \rowcolor{numpexlightergray}Resilience
                 & \begin{tabular}{l}
                       None \\
                   \end{tabular}
                 & None \\
                \rowcolor{white}DevOps
                 & \begin{tabular}{l}
                       None \\
                   \end{tabular}
                 & None  \\
                \rowcolor{numpexlightergray}Packaging
                 & \begin{tabular}{l}
                       pip, GitHub \\
                   \end{tabular}
                 & Scimba is packaged on pip (\texttt{pip install scimba}) and available on gitlab (\url{https://gitlab.inria.fr/scimba/scimba})  \\
                \rowcolor{white}Testing
                 & \begin{tabular}{l}
                       pytest \\
                   \end{tabular}
                 & Tests are provided for core functionalities, available by running \texttt{pytest}. \\
                \rowcolor{numpexlightergray}Containerization
                 & \begin{tabular}{l}
                       None \\
                   \end{tabular}
                 & None \\
                \rowcolor{white}Interfaces
                 & \begin{tabular}{l}
                       \texttt{Feel++} \\
                   \end{tabular}
                 & Scimba interfaces with Feel++: solution data produced with Feel++ has successfully been
                 used to train neural networks in Scimba. \\
                \bottomrule
            \end{tabular}
        }}
    \caption{Scimba programming and computational environment}
\end{table}


\subsection{Application entry points}
\label{sec:Scimba:apps}
Scimba is used to instantiate differentiable solvers and hybrid workflows; see Chapter~\ref{chap:applications}:
\begin{itemize}
    \item Physics-informed shape optimization with neural networks (proposed): Section~\ref{sec:app:specs:app-scimba-shape-opt-nn}.
    \item Data-driven plasma surrogates for reduced models (proposed): Section~\ref{sec:app:specs:app-scimba-plasma}.
    \item Hybrid numerical/ML solvers (proposed): Section~\ref{sec:app:specs:app-scimba-hybrid}.
\end{itemize}


\subsection{Mathematics}
\label{sec:Scimba:mathematics}
% Mathematics not available.

% In this section, provide a summary the mathematics used in the software.

This software mainly uses physics-informed
learning~\cite{karniadakis_physics-informed_2021}
to solve partial differential equations.
More specifically, it solves partial differential equations of the form
\begin{equation*}
    \begin{cases}
        \mathcal{D}(u, x, t,; \mu) = 0,
         & \text{in } \Omega \times (0,T) \times \mathbb{M},          \\
        \mathcal{B}(u, x, t; \mu) = 0,
         & \text{on } \partial \Omega \times (0,T) \times \mathbb{M}, \\
        u(x, 0; \mu) = u_0(x; \mu),
         & \text{on } \Omega \times \mathbb{M},                       \\
    \end{cases}
\end{equation*}
with $\mathcal{D}$ a differential operator, $\mathcal{B}$ a boundary condition operator,
$u_0$ an initial condition, $\Omega$ the spatial domain,
$T$ the final time, and $\mathbb{M}$ the parameter space.
The solution $u$ is approximated with a variety of physics-informed neural networks,
which are trained to minimize the residual of the PDE,
the boundary conditions and the initial conditions.
An example of a physics-informed loss function is:
\begin{equation*}
    \begin{aligned}
        \mathcal{L} & =
        \int_\Omega \int_0^T \int_{\mathbb{M}}
        ||\mathcal{D}(u, x, t; \mu)||^2 \,
        \mathrm{d}\mu \, \mathrm{d}t \, \mathrm{d}x \\
                    & +
        \int_{\partial\Omega} \int_0^T \int_{\mathbb{M}}
        ||\mathcal{B}(u, x, t; \mu)||^2 \,
        \mathrm{d}\mu \, \mathrm{d}t \, \mathrm{d}x
        +
        \int_{\Omega} \int_{\mathbb{M}}
        ||u(x, 0; \mu) - u_0(x;\mu)||^2 \,
        \mathrm{d}\mu \, \mathrm{d}x.
    \end{aligned}
\end{equation*}
Other data-driven methods
(e.g. DeepONets~\cite{lu_learning_2021})
or numerical methods
(e.g. Neural Galerkin~\cite{bruna_neural_2024})
are also available in Scimba.

% \subsection{Relevant Publications}
% \label{sec:Scimba:publications}

% Here is a list of relevant publications related to the software:


\subsection{Acknowledgements}
\label{sec::Scimba:acknowledgements}

Scimba has mainly been developed by Inria research scientists.
Other contributors include students and researchers
from the University of Strasbourg and CEA Cadarache.
For the moment, no specific funding has been dedicated to Scimba.
Exa-MA has supported the internships of Marie Sengler (Strasbourg) and Daria Hrebenshchykova (Sophia-Antipolis), who both worked with Scimba.


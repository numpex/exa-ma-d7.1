


\subsubsection{Benchmark \#\counter{feelppWP1benchcounter}: Contact Mechanics}

\paragraph{Description}
This benchmark simulates the dynamic unilateral contact between an elastic bouncing 
ball and a rigid horizontal wall, presented in \cite{chouly_explicit_2018}. 
The full model, combining ray-tracing, the Signorini contact mechanics, and the dynamics of elastic bodies 
is presented in \cite{van_landeghem_motion_nodate}. \\


\paragraph{Benchmarking Tools Used}

The simulations are conducted on the Gaya supercomputer. The execution time of the 
following tasks is monitored:

\begin{inparaenum}[\it (i)]
    \item Mesh: loading and initialization of the non-partitioned mesh,
    \item Data Structures: initialization of data structures,
    \item Ray-tracing: collision detection using ray-tracing,
    \item Assembly: construction of the dynamic algebraic system, 
    \item Solve: solving the non-linear algebraic system, and
    \item Post process: exporting the vectorial displacement field, the scalar contact displacement and the scalar contact pressure at each time iteration.
\end{inparaenum}

\paragraph{Input/Output Dataset Description}

\begin{itemize}
    \item \textbf{Input Data:} As input, we consider the same mesh for all simulations, 
    using $P_1$ Lagrange elements for the vectorial unknown displacement field. The mesh 
    characteristics, the number of mesh elements and the number of degrees of freedom, 
    are provided in \Cref{tab:feelpp:mesh:contact}. Additionally, the input data includes the configuration files necessary to run the simulations.
    \item \textbf{Output Data:} The output dataset includes the time evolution of 
    the displacement field of the elastic body, as well as the time evolution of 
    the contact displacement and pressure. In addition, the execution times for 
    the different tasks are stored.
\end{itemize}



\begin{table}[!ht]
    \centering
    { \setlength{\parindent}{0pt}
    \def\arraystretch{1.25}
    \arrayrulecolor{numpexgray}
    {\fontsize{9}{11}\selectfont
    \begin{tabular}{!{\color{numpexgray}\vrule}c!{\color{numpexgray}\vrule}c!{\color{numpexgray}\vrule}c!{\color{numpexgray}\vrule}c!{\color{numpexgray}\vrule}c!{\color{numpexgray}\vrule}c!{\color{numpexgray}\vrule}c!{\color{numpexgray}\vrule}c!{\color{numpexgray}\vrule}}
        \rowcolor{numpexgray}
        \multicolumn{3}{c!{\color{numpexgray}\vrule}}{\color{white}\bf Mesh properties} & \multicolumn{1}{c!{\color{numpexgray}\vrule}}{\color{white}\bf Number of degrees of freedom} \\
        \hline
        \rowcolor{numpexgray} {\color{white}\bf $h_\text{min}$} & {\color{white}\bf $h_\text{max}$} & {\color{white}\bf \# elements} & {\color{white}\bf $\vct{u}$} \\
        \pgfmathprintnumber{0.19269262925729186} & \pgfmathprintnumber{0.46595156749445504} & \pgfmathprintnumber{21675942} & \pgfmathprintnumber{9208203} \\
        \hline
    \end{tabular}
    }}
    \caption{Characteristics of the mesh and the number of degrees of freedom for the vectorial displacement field $\vct{u}$ with $P_1$ discretization.}%
    \label{tab:feelpp:mesh:contact}
\end{table}

\paragraph{Results Summary}

The results for the computational time and relative computational time for the 
different tasks and varying numbers of processors are presented in ~\Cref{fig:feelpp:wp1:contact:time}, 
and ~\Cref{fig:feelpp:wp1:contact:time-rel}. The bars show the results using $P_1$ Lagrange 
elements for the unknown displacement field. 


We observe that the resolution of the dynamic system constitutes the majority of 
the computational time, and its relative time increases with the number of cores. 
%The communication between nodes and synchronization points become predominant as the number of cores increases.
With the increase in the number of cores, the absolute execution time related to 
data structure initialization, ray-tracing, assembly, and post-processing decreases. 
The mesh loading time remains constant, as it is not partitioned at the input.


\pgfplotstableread{\currfiledir/data/contact-time.dat}\dataContact

\begin{figure}
    \centering
    \begin{subfigure}[b]{\textwidth}
    \begin{tikzpicture}
        \begin{axis}[
            width=\textwidth, height=8cm,
            xlabel={Nproc}, ylabel={Computational time [s]},
            xtick={0,1,2,3,4}, xticklabels={32,64,128,256,384},
            legend style={at={(0.5,-0.18)}, anchor=north, legend columns=-1},
            ymajorgrids=true, yminorgrids=true, ymin=0,
            bar width=7pt, ybar stacked,
            %ymode=log,
            % title={Computational time for the 3D case},
        ]
        \addplot+[ybar, bar width=0.2, fill=customdarkblue, draw=black, point meta=y] table [x=x, y=mesh] {\dataContact};
        % \addlegendentry{Mesh}
        \addplot+[ybar, bar width=0.2, fill=customcyan, draw=black, point meta=y] table [x=x, y=data] {\dataContact};
        % \addlegendentry{Data Structures}
        \addplot+[ybar, bar width=0.2, fill=black, draw=black, point meta=y] table [x=x, y expr={300*\thisrow{raytracing}} ] {\dataContact};
        % \addlegendentry{Assembly}
        \addplot+[ybar, bar width=0.2, fill=customorange, draw=black, point meta=y] table [x=x, y=assembly ] {\dataContact};

        \addplot+[ybar, bar width=0.2, fill=custompurple, draw=black, point meta=y] table [x=x, y=solve] {\dataContact};
        % \addlegendentry{Solve}
        \addplot+[ybar, bar width=0.2, fill=customgreen, draw=black, point meta=y] table [x=x, y=postprocess] {\dataContact};
        % \addlegendentry{Post process}

        \end{axis}
        \end{tikzpicture}
    \caption{Computational time for the contact mechanics testcase.}
    \label{fig:feelpp:wp1:contact:time}
    \end{subfigure}
    \hfill
    \begin{subfigure}[b]{\textwidth}
    \begin{tikzpicture}
        \begin{axis}[
            width=\textwidth, height=8cm,
            xlabel={Nproc}, ylabel={Relative computational time [\%]},
            xtick={0,1,2,3,4}, xticklabels={32,64,128,256,384},
            legend style={at={(0.5,-0.18)}, anchor=north, legend columns=-1},
            ymajorgrids=true, yminorgrids=true,
            bar width=7pt, ybar stacked,
            ymin=0, ymax=100,
            % title={Relative computational time for the 3D case},
        ]

        % Compute the relative time for each component by dividing by the total time
        % using the correct column names from the initial plot
        \addplot+[ybar, bar width=0.2, fill=customdarkblue, draw=black, point meta=y]
            table [x=x, y expr={100*\thisrow{mesh}/(\thisrow{mesh} + \thisrow{data} + 300*\thisrow{raytracing} + \thisrow{assembly} + \thisrow{solve} + \thisrow{postprocess} )}] {\dataContact};
        \addlegendentry{Mesh}
        \addplot+[ybar, bar width=0.2, fill=customcyan, draw=black, point meta=y]
            table [x=x, y expr={100*\thisrow{data}/(\thisrow{mesh} + \thisrow{data} + 300*\thisrow{raytracing} + \thisrow{assembly} + \thisrow{solve} + \thisrow{postprocess} )}] {\dataContact};
        \addlegendentry{Data Structures}
        \addplot+[ybar, bar width=0.2, fill=black, draw=black, point meta=y]
            table [x=x, y expr={3000*\thisrow{raytracing}/(\thisrow{mesh} + \thisrow{data} + 300*\thisrow{raytracing} + \thisrow{assembly} + \thisrow{solve} + \thisrow{postprocess} )}] {\dataContact};
        \addlegendentry{Ray-tracing}
        \addplot+[ybar, bar width=0.2, fill=customorange, draw=black, point meta=y]
            table [x=x, y expr={100*\thisrow{assembly}/(\thisrow{mesh} + \thisrow{data} + 300*\thisrow{raytracing} + \thisrow{assembly} + \thisrow{solve} + \thisrow{postprocess} )}] {\dataContact};
        \addlegendentry{Assembly}
        \addplot+[ybar, bar width=0.2, fill=custompurple, draw=black, point meta=y]
            table [x=x, y expr={100*\thisrow{solve}/(\thisrow{mesh} + \thisrow{data} + 300*\thisrow{raytracing} + \thisrow{assembly} + \thisrow{solve} + \thisrow{postprocess} )}] {\dataContact};
        \addlegendentry{Solve}
        \addplot+[ybar, bar width=0.2, fill=customgreen, draw=black, point meta=y]
            table [x=x, y expr={100*\thisrow{postprocess}/(\thisrow{mesh} + \thisrow{data} + 300*\thisrow{raytracing} + \thisrow{assembly} + \thisrow{solve} + \thisrow{postprocess} )}] {\dataContact};
        \addlegendentry{Post process}

        \end{axis}
    \end{tikzpicture}

    \caption{Relative computational time for the contact mechanics testcase.}
    \label{fig:feelpp:wp1:contact:time-rel}
\end{subfigure}
\caption{Absolute (\Cref{fig:feelpp:wp1:contact:time}) and relative (\Cref{fig:feelpp:wp1:contact:time-rel}) computational time for the various tasks using  $P_1$ Lagrange elements for the vectorial displacement field.}
\label{fig:feelpp:wp1:contactbenchmark:time}
\end{figure}



\subsection{12-Month Roadmap}
\label{sec:WP1:Feelpp:roadmap}

For the next 12 month, we plan to focus on the following aspects of the benchmarking process:
\begin{description}
    \item[Data Improvements] Unify the input and output data format and structure to facilitate comparison and analysis. In particular, we wish to design dataset architecture that holds all necessary information for the benchmarking process as well as reference output results if any.  We have started an effort to continuously benchmark our software using reframe and collect the information in a database including a visualisation tool through a website.
    \item[Methodology Application] Several aspects will be developed:
    \begin{itemize}
        \item include HdG methods and high order methods in the current benchmarks, we didn't have the time to include them in the current results.
        \item task based parallelism using the runtime environment \ac{specx}:
        \begin{itemize}
            \item add multithreading support in various steps of the computational pipeline.
            \item enable distribution of some work load on GPU (eg. Ray Tracing, Assemly and Solve steps)
        \end{itemize}
        \item investigate the use of Kokkos to define portable and performant kernels.
        \item Improve Ray tracing parallel performance at large scale
        \item Improve I/O performance at large scale leveraging the HDF5 data format and the parallel I/O capabilities of the library.
        \item Improve partitioning and load balancing of the mesh at large scale.
        \item Improve parallel mesh adaptation at large scale using ParMMG
    \end{itemize}
    \item[Results Retention] We use two data management platforms Girder and Zenodo to store the data and results of the benchmarks following the methdology in~\cref{sec:methodology-intro}.
\end{description}

In~\Cref{tab:WP1:Feelpp:bottlenecks}, we briefly discuss the bottleneck roadmap associated to the software and relevant to the work package.

\begin{table}[!ht]
    \centering

    {
        \setlength{\parindent}{0pt}
        \def\arraystretch{1.25}
        \arrayrulecolor{numpexgray}
        {
            \fontsize{9}{11}\selectfont
            \begin{tabular}{!{\color{numpexgray}\vrule}p{.25\linewidth}!{\color{numpexgray}\vrule}p{.6885\linewidth}!{\color{numpexgray}\vrule}}

    \rowcolor{numpexgray}{\rule{0pt}{2.5ex}\color{white}\bf Bottlenecks} &  {\rule{0pt}{2.5ex}\color{white}\bf Short Description }\\

\rowcolor{white}    B10 - Scientific Productivity & Containerization and packaging are enabled as well as \ac{CI}/\ac{CD}. \ac{CB} will be enabled very soon.\\
\rowcolor{numpexlightergray}    B11 - Reproducibility and Replicability of Computation & building upon data management and scientific productivity improvements to enable reproducibility \\
\rowcolor{white}    B12 - Pre/Post Processing and In-Situ Processing & improve I/O using HDF5 and MPI I/O and possibly used framework from \ac{PC3}\\
\rowcolor{numpexlightergray}    B6 - Data Management & dataset creation and management are being improved to satisfy the methodology in~\cref{sec:methodology-intro}. Girder~\cref{sec:arch:girder:unistra} and~\cref{sec:arch:zenodo}  will be used to store our dataset and enable FAIR principles\\
\rowcolor{white}    B7 - Exascale Algorithms & enable Ray Tracing, cG, HdG and spectral element methods  on GPU; enable new partitioning strategies and load balancing; enable advanced profiling using score-P, EzTrace; improve memory management; enable parallel in time strategies  \\
\hline
\end{tabular}
        }
    }
    \caption{WP1: \Feelpp plan with Respect to Relevant Bottlenecks}
    \label{tab:WP1:Feelpp:bottlenecks}
\end{table}

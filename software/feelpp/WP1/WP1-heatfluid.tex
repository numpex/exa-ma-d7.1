



\subsubsection{Benchmark \#\counter{feelppWP1benchcounter}: HeatFluid Coupling}
\label{sec:WP1:Feelpp:benchmark4}

\newcommand{\vct}[1]{\vec{#1}}
\newcommand{\mat}[1]{\underline{\underline{#1}}}


% \emph{enlever tous les détails, et laisser les références, expliciter la liste de maillage et la machine où est faite le bench, et la mise en donnée (paramètrisation)}

% \emph{dans wp3: parler du préconditionner}


\paragraph{Description}
This benchmark models the steady aqueous humor (AH) flow in the posterior and anterior chambers of the human eyeball, coupled with the overall heat transfer, adapted from~\cite{ooi_simulation_2008,kilgour_operator_2021}.
The full model description is available in~\cite{saigre_coupled_2024_abstract}.
It it run with the toolbox \texttt{heatfluid} of \Feelpp.


\paragraph{Benchmarking Tools Used}

The following tools were used for performance profiling and analysis:
\begin{itemize}
    \item \textbf{\Feelpp}: the performance tools integrated into the \Feelpp framework were used to measure the execution time.
    \item \textbf{Gaya}: the benchmark was performed on the Gaya supercomputer (see \Cref{sec:arch:gaya}).
\end{itemize}

The metrics measured are the execution time to:
\begin{inparaenum}[\it (i)]
    \item load and initialize the mesh that is already partitioned on the disk,
    \item initialize the data structures,
    \item assembly algebraic objects of the linear system,
    \item solve the non-linear algebraic system, and
    \item export the results.
\end{inparaenum}


\paragraph{Input/Output Dataset Description}

\begin{itemize}
    \item \textbf{Input Data:} The input dataset consists of a family of 3D tetrahedral meshes generated through the process described in~\cite{chabannes_3d_2024}, and denoted \texttt{Mr0} to \texttt{Mr6}, with an increasing number of elements.
    \Cref{tab:feelpp:wp1:coupled:mesh} presents the characteristics of these meshes.
    The input data also provides the configuration files necessary to run the simulations.
    \item \textbf{Output Data:} The output includes the computed temperature, velocity, and pressure fields for each mesh, stored in HDF5 format, as well as the time taken to perform each step of the simulation.
    \item \textbf{Data Repository:} All input and output datasets are available in a Zenodo repository \cite{saigre_mesh_2024}, accessible through DOI: \href{https://doi.org/10.5281/ZENODO.13886143}{10.5281/ZENODO.13886143}.
\end{itemize}


\begin{table}[!ht]
    \centering
    { \setlength{\parindent}{0pt}
    \def\arraystretch{1.25}
    \arrayrulecolor{numpexgray}
    {\fontsize{9}{11}\selectfont
    \begin{tabular}{!{\color{numpexgray}\vrule}c!{\color{numpexgray}\vrule}c!{\color{numpexgray}\vrule}c!{\color{numpexgray}\vrule}c!{\color{numpexgray}\vrule}c!{\color{numpexgray}\vrule}c!{\color{numpexgray}\vrule}c!{\color{numpexgray}\vrule}c!{\color{numpexgray}\vrule}}
        \rowcolor{numpexgray}
        \multicolumn{5}{c!{\color{numpexgray}\vrule}}{\color{white}\bf Mesh properties} & \multicolumn{3}{c!{\color{numpexgray}\vrule}}{\color{white}\bf Number of degrees of freedom} \\
        \hline
        \rowcolor{numpexgray}{\color{white}\bf Tag} & {\color{white}\bf $h_\text{min}$} & {\color{white}\bf $h_\text{max}$} & {\color{white}\bf $h_\text{mean}$} & {\color{white}\bf \# elements} & {\color{white}\bf $T$} & {\color{white}\bf $\vct{u}$} & {\color{white}\bf $p$} \\
        \texttt{Mr0} & \pgfmathprintnumber{1.247583e-04} & \pgfmathprintnumber{3.997611e-03} & \pgfmathprintnumber{9.227331e-04} & \pgfmathprintnumber{191939} & \pgfmathprintnumber{37470} & \pgfmathprintnumber{84966} & \pgfmathprintnumber{4615} \\
        \rowcolor{numpexlightergray}
        \texttt{Mr1} & \pgfmathprintnumber{1.367312e-04} & \pgfmathprintnumber{3.634717e-03} & \pgfmathprintnumber{7.717604e-04} & \pgfmathprintnumber{282030} & \pgfmathprintnumber{51753} & \pgfmathprintnumber{116709} & \pgfmathprintnumber{6155} \\
        \texttt{Mr2} & \pgfmathprintnumber{6.539683e-05} & \pgfmathprintnumber{1.599067e-03} & \pgfmathprintnumber{4.668270e-04} & \pgfmathprintnumber{746664} & \pgfmathprintnumber{131327} & \pgfmathprintnumber{589992} & \pgfmathprintnumber{28548} \\
        \rowcolor{numpexlightergray}
        \texttt{Mr3} & \pgfmathprintnumber{3.294835e-05} & \pgfmathprintnumber{9.592658e-04} & \pgfmathprintnumber{4.166619e-04} & \pgfmathprintnumber{1403433} & \pgfmathprintnumber{241831} & \pgfmathprintnumber{707532} & \pgfmathprintnumber{34304} \\
        \texttt{Mr4} & \pgfmathprintnumber{2.549458e-05} & \pgfmathprintnumber{5.293352e-04} & \pgfmathprintnumber{2.883913e-04} & \pgfmathprintnumber{6038645} & \pgfmathprintnumber{1027375} & \pgfmathprintnumber{1024008} & \pgfmathprintnumber{48534} \\
        \rowcolor{numpexlightergray}
        \texttt{Mr5} & \pgfmathprintnumber{3.120124e-05} & \pgfmathprintnumber{1.501561e-04} & \pgfmathprintnumber{2.772105e-04} & \pgfmathprintnumber{43893359} & \pgfmathprintnumber{7374833} & \pgfmathprintnumber{4616967} & \pgfmathprintnumber{205342} \\
        \texttt{Mr6} & \pgfmathprintnumber{2.820610e-05} & \pgfmathprintnumber{9.940551e-07} & \pgfmathprintnumber{1.835537e-04} & \pgfmathprintnumber{150630096} & \pgfmathprintnumber{25200452} & \pgfmathprintnumber{14671089} & \pgfmathprintnumber{636943} \\
        \hline
    \end{tabular}
    }}
    \caption{Characteristics of meshes used for the convergence study and number of degrees of freedom for temperature $T$, velocity $\vct{u}$, and pressure fields $p$, with the discretization $P_1\text{--}P_2P_1$.}%
    \label{tab:feelpp:wp1:coupled:mesh}
\end{table}


\paragraph{Results Summary}

The results of the benchmark are summarized in~\Cref{fig:feelpp:wp1:coupled:time} and~\Cref{fig:feelpp:wp1:coupled:time-rel},
showing the computational time and relative computational time for each component of the simulation, respectively.
The results are presented for the three biggest meshed of the family, namely \texttt{Mr4}, \texttt{Mr5}, and \texttt{Mr6}.
Note that for \texttt{Mr6}, the simulation dit not complete on 1 node (128 cores) due to memory limitations.

We observe that the resolution of the non-linear algebraic system is the most time-consuming part of the simulation, followed by the assembly of the linear system.
Moreover, even though the relative time is globally similar when the number of cores is increased, we note a decrease in the absolute time for various components of the simulation, except for the Post process part, which involves writing the results to disk.
The assembly time remains significant compared to other parts of the simulation, with a noticeable increase in the time spent resolving the non-linear system, which forms the largest portion of the computation.
As the number of cores increases, we also observe a proportional increase in time dedicated to I/O operations, particularly in the Post process phase, due to the larger volumes of data being written to disk.

\iffalse
\pgfplotstableread{\currfiledir/data/heatfluid-time-data.dat}\data
\begin{figure}
    \centering
    \begin{tikzpicture}
        \begin{axis}[
            width=\textwidth, height=8cm,
            xlabel={Nproc}, ylabel={Computational time [s]},
            xtick={0,1,2,3,4,5,6,7,8,9,10}, xticklabels={1,2,4,8,16,32,64,128,256,512,640},
            legend style={at={(0.5,-0.15)}, anchor=north, legend columns=-1},
            ymajorgrids=true, yminorgrids=true,
            bar width=7pt, ybar stacked,
            ymode=log,
            % title={Computational time for the 3D case},
        ]
        \addplot+[ybar, bar width=0.2, fill=customdarkblue, draw=black, point meta=y] table [x=x, y=initMesh] {\data};
        \addlegendentry{Mesh}


        \addplot+[ybar, bar width=0.2, fill=customcyan, draw=black, point meta=y] table [x=x, y expr=\thisrow{init}-\thisrow{initMesh}] {\data};
        \addlegendentry{Data Structures}

        \addplot+[ybar, bar width=0.2, fill=customorange, draw=black, point meta=y] table [x=x, y=algebraic-nlsolve] {\data};
        \addlegendentry{Solve}

        \addplot+[ybar, bar width=0.2, fill=custompurple, draw=black, point meta=y] table [x=x, y expr=\thisrow{algebraic-newton-initial-guess}+\thisrow{algebraic-jacobian}+\thisrow{algebraic-residual}] {\data};
        \addlegendentry{Assembly}

        \addplot+[ybar, bar width=0.2, fill=customgreen, draw=black, point meta=y
        ] table [x=x, y=exportResults] {\data};
        \addlegendentry{Post process}

        \end{axis}
        \end{tikzpicture}
    \caption{Computational time for the coupled heat-fluid testcase, performed on Gaya with the mesh \texttt{Mr4}.}
\end{figure}

\begin{figure}
    \centering

    \begin{tikzpicture}
        \begin{axis}[
            width=\textwidth, height=8cm,
            xlabel={Nproc}, ylabel={Relative computational time [\%]},
            xtick={0,1,2,3,4,5,6,7,8,9,10}, xticklabels={1,2,4,8,16,32,64,128,256,512,640},
            legend style={at={(0.5,-0.15)}, anchor=north, legend columns=-1},
            ymajorgrids=true, yminorgrids=true,
            bar width=7pt, ybar stacked,
            ymin=0, ymax=100,
            % title={Relative computational time for the 3D case},
        ]

        % Compute the relative time for each component by dividing by the total time
        % using the correct column names from the initial plot
        \addplot+[ybar, bar width=0.2, fill=customdarkblue, draw=black, point meta=y]
            table [x=x, y expr={100*\thisrow{initMesh}/(\thisrow{initMesh} + (\thisrow{init}-\thisrow{initMesh}) + \thisrow{algebraic-nlsolve} + (\thisrow{algebraic-newton-initial-guess} + \thisrow{algebraic-jacobian} + \thisrow{algebraic-residual}) + \thisrow{exportResults})}] {\data};
        \addlegendentry{Mesh}

        \addplot+[ybar, bar width=0.2, fill=customcyan, draw=black, point meta=y]
            table [x=x, y expr={100*(\thisrow{init}-\thisrow{initMesh})/(\thisrow{initMesh} + (\thisrow{init}-\thisrow{initMesh}) + \thisrow{algebraic-nlsolve} + (\thisrow{algebraic-newton-initial-guess} + \thisrow{algebraic-jacobian} + \thisrow{algebraic-residual}) + \thisrow{exportResults})}] {\data};
        \addlegendentry{Data Structures}

        \addplot+[ybar, bar width=0.2, fill=customorange, draw=black, point meta=y]
            table [x=x, y expr={100*\thisrow{algebraic-nlsolve}/(\thisrow{initMesh} + (\thisrow{init}-\thisrow{initMesh}) + \thisrow{algebraic-nlsolve} + (\thisrow{algebraic-newton-initial-guess} + \thisrow{algebraic-jacobian} + \thisrow{algebraic-residual}) + \thisrow{exportResults})}] {\data};
        \addlegendentry{Solve}

        \addplot+[ybar, bar width=0.2, fill=custompurple, draw=black, point meta=y]
            table [x=x, y expr={100*(\thisrow{algebraic-newton-initial-guess} + \thisrow{algebraic-jacobian} + \thisrow{algebraic-residual})/(\thisrow{initMesh} + (\thisrow{init}-\thisrow{initMesh}) + \thisrow{algebraic-nlsolve} + (\thisrow{algebraic-newton-initial-guess} + \thisrow{algebraic-jacobian} + \thisrow{algebraic-residual}) + \thisrow{exportResults})}] {\data};
        \addlegendentry{Assembly}

        \addplot+[ybar, bar width=0.2, fill=customgreen, draw=black, point meta=y]
            table [x=x, y expr={100*\thisrow{exportResults}/(\thisrow{initMesh} + (\thisrow{init}-\thisrow{initMesh}) + \thisrow{algebraic-nlsolve} + (\thisrow{algebraic-newton-initial-guess} + \thisrow{algebraic-jacobian} + \thisrow{algebraic-residual}) + \thisrow{exportResults})}] {\data};
        \addlegendentry{Post process}

        \end{axis}
    \end{tikzpicture}

    \caption{Relative time spent in each component of the computation for the coupled heat-fluid testcase, performed on Gaya with the mesh \texttt{Mr4}.}
\end{figure}
\fi


\pgfplotstableread{\currfiledir/data/heatfluid-time-M4.dat}\dataMQuatre
\pgfplotstableread{\currfiledir/data/heatfluid-time-M5.dat}\dataMCinq
\pgfplotstableread{\currfiledir/data/heatfluid-time-M6.dat}\dataMSix

\begin{figure}
    \centering
    \begin{subfigure}[b]{\textwidth}
    \begin{tikzpicture}
        \begin{axis}[
            width=\textwidth, height=8cm,
            xlabel={Nproc}, ylabel={Computational time [s]},
            xtick={0,1,2,3,4,5}, xticklabels={128,256,384,512,640,768},
            legend style={at={(0.5,-0.18)}, anchor=north, legend columns=-1},
            ymajorgrids=true, yminorgrids=true, ymin=0,
            bar width=7pt, ybar stacked,
            %ymode=log,
            % title={Computational time for the 3D case},
        ]
        \addplot+[ybar, bar width=0.2, fill=customdarkblue, draw=black, point meta=y] table [x=x, y=initMesh] {\dataMQuatre};
        % \addlegendentry{Mesh}
        \addplot+[ybar, bar width=0.2, fill=customcyan, draw=black, point meta=y] table [x=x, y expr=\thisrow{init}-\thisrow{initMesh}] {\dataMQuatre};
        % \addlegendentry{Data Structures}
        \addplot+[ybar, bar width=0.2, fill=customorange, draw=black, point meta=y] table [x=x, y expr=\thisrow{algebraic-newton-initial-guess}+\thisrow{algebraic-jacobian}+\thisrow{algebraic-residual}] {\dataMQuatre};
        % \addlegendentry{Assembly}
        \addplot+[ybar, bar width=0.2, fill=custompurple, draw=black, point meta=y] table [x=x, y=algebraic-nlsolve] {\dataMQuatre};
        % \addlegendentry{Solve}
        \addplot+[ybar, bar width=0.2, fill=customgreen, draw=black, point meta=y] table [x=x, y=exportResults] {\dataMQuatre};
        % \addlegendentry{Post process}

        \resetstackedplots

        \addplot+[ybar, bar width=0.2, fill=customdarkblue, draw=black, point meta=y, forget plot] table [x=x, y=initMesh] {\dataMCinq};
        \addplot+[ybar, bar width=0.2, fill=customcyan, draw=black, point meta=y, forget plot] table [x=x, y expr=\thisrow{init}-\thisrow{initMesh}] {\dataMCinq};
        \addplot+[ybar, bar width=0.2, fill=customorange, draw=black, point meta=y, forget plot] table [x=x, y expr=\thisrow{algebraic-newton-initial-guess}+\thisrow{algebraic-jacobian}+\thisrow{algebraic-residual}] {\dataMCinq};
        \addplot+[ybar, bar width=0.2, fill=custompurple, draw=black, point meta=y, forget plot] table [x=x, y=algebraic-nlsolve] {\dataMCinq};
        \addplot+[ybar, bar width=0.2, fill=customgreen, draw=black, point meta=y, forget plot] table [x=x, y=exportResults] {\dataMCinq};

        \resetstackedplots

        \addplot+[ybar, bar width=0.2, fill=customdarkblue, draw=black, point meta=y, forget plot] table [x=x, y=initMesh] {\dataMSix};
        \addplot+[ybar, bar width=0.2, fill=customcyan, draw=black, point meta=y, forget plot] table [x=x, y expr=\thisrow{init}-\thisrow{initMesh}] {\dataMSix};
        \addplot+[ybar, bar width=0.2, fill=customorange, draw=black, point meta=y, forget plot] table [x=x, y expr=\thisrow{algebraic-newton-initial-guess}+\thisrow{algebraic-jacobian}+\thisrow{algebraic-residual}] {\dataMSix};
        \addplot+[ybar, bar width=0.2, fill=custompurple, draw=black, point meta=y, forget plot] table [x=x, y=algebraic-nlsolve] {\dataMSix};
        \addplot+[ybar, bar width=0.2, fill=customgreen, draw=black, point meta=y, forget plot] table [x=x, y=exportResults] {\dataMSix};

        \end{axis}
        \end{tikzpicture}
    \caption{Computational time for the coupled heat-fluid testcase.}
    \label{fig:feelpp:wp1:coupled:time}
    \end{subfigure}
    \hfill
    \begin{subfigure}[b]{\textwidth}
    \begin{tikzpicture}
        \begin{axis}[
            width=\textwidth, height=8cm,
            xlabel={Nproc}, ylabel={Relative computational time [\%]},
            xtick={0,1,2,3,4,5}, xticklabels={128,256,384,512,640,768},
            legend style={at={(0.5,-0.18)}, anchor=north, legend columns=-1},
            ymajorgrids=true, yminorgrids=true,
            bar width=7pt, ybar stacked,
            ymin=0, ymax=100,
            % title={Relative computational time for the 3D case},
        ]

        % Compute the relative time for each component by dividing by the total time
        % using the correct column names from the initial plot
        \addplot+[ybar, bar width=0.2, fill=customdarkblue, draw=black, point meta=y]
            table [x=x, y expr={100*\thisrow{initMesh}/(\thisrow{initMesh} + (\thisrow{init}-\thisrow{initMesh}) + \thisrow{algebraic-nlsolve} + (\thisrow{algebraic-newton-initial-guess} + \thisrow{algebraic-jacobian} + \thisrow{algebraic-residual}) + \thisrow{exportResults})}] {\dataMQuatre};
        \addlegendentry{Mesh}
        \addplot+[ybar, bar width=0.2, fill=customcyan, draw=black, point meta=y]
            table [x=x, y expr={100*(\thisrow{init}-\thisrow{initMesh})/(\thisrow{initMesh} + (\thisrow{init}-\thisrow{initMesh}) + \thisrow{algebraic-nlsolve} + (\thisrow{algebraic-newton-initial-guess} + \thisrow{algebraic-jacobian} + \thisrow{algebraic-residual}) + \thisrow{exportResults})}] {\dataMQuatre};
        \addlegendentry{Data Structures}
        \addplot+[ybar, bar width=0.2, fill=customorange, draw=black, point meta=y]
            table [x=x, y expr={100*(\thisrow{algebraic-newton-initial-guess} + \thisrow{algebraic-jacobian} + \thisrow{algebraic-residual})/(\thisrow{initMesh} + (\thisrow{init}-\thisrow{initMesh}) + \thisrow{algebraic-nlsolve} + (\thisrow{algebraic-newton-initial-guess} + \thisrow{algebraic-jacobian} + \thisrow{algebraic-residual}) + \thisrow{exportResults})}] {\dataMQuatre};
        \addlegendentry{Assembly}
        \addplot+[ybar, bar width=0.2, fill=custompurple, draw=black, point meta=y]
            table [x=x, y expr={100*\thisrow{algebraic-nlsolve}/(\thisrow{initMesh} + (\thisrow{init}-\thisrow{initMesh}) + \thisrow{algebraic-nlsolve} + (\thisrow{algebraic-newton-initial-guess} + \thisrow{algebraic-jacobian} + \thisrow{algebraic-residual}) + \thisrow{exportResults})}] {\dataMQuatre};
        \addlegendentry{Solve}
        \addplot+[ybar, bar width=0.2, fill=customgreen, draw=black, point meta=y]
            table [x=x, y expr={100*\thisrow{exportResults}/(\thisrow{initMesh} + (\thisrow{init}-\thisrow{initMesh}) + \thisrow{algebraic-nlsolve} + (\thisrow{algebraic-newton-initial-guess} + \thisrow{algebraic-jacobian} + \thisrow{algebraic-residual}) + \thisrow{exportResults})}] {\dataMQuatre};
        \addlegendentry{Post process}

        \resetstackedplots

        \addplot+[ybar, bar width=0.2, fill=customdarkblue, draw=black, point meta=y]
            table [x=x, y expr={100*\thisrow{initMesh}/(\thisrow{initMesh} + (\thisrow{init}-\thisrow{initMesh}) + \thisrow{algebraic-nlsolve} + (\thisrow{algebraic-newton-initial-guess} + \thisrow{algebraic-jacobian} + \thisrow{algebraic-residual}) + \thisrow{exportResults})}] {\dataMCinq};
        \addplot+[ybar, bar width=0.2, fill=customcyan, draw=black, point meta=y]
            table [x=x, y expr={100*(\thisrow{init}-\thisrow{initMesh})/(\thisrow{initMesh} + (\thisrow{init}-\thisrow{initMesh}) + \thisrow{algebraic-nlsolve} + (\thisrow{algebraic-newton-initial-guess} + \thisrow{algebraic-jacobian} + \thisrow{algebraic-residual}) + \thisrow{exportResults})}] {\dataMCinq};
        \addplot+[ybar, bar width=0.2, fill=customorange, draw=black, point meta=y]
            table [x=x, y expr={100*(\thisrow{algebraic-newton-initial-guess} + \thisrow{algebraic-jacobian} + \thisrow{algebraic-residual})/(\thisrow{initMesh} + (\thisrow{init}-\thisrow{initMesh}) + \thisrow{algebraic-nlsolve} + (\thisrow{algebraic-newton-initial-guess} + \thisrow{algebraic-jacobian} + \thisrow{algebraic-residual}) + \thisrow{exportResults})}] {\dataMCinq};
        \addplot+[ybar, bar width=0.2, fill=custompurple, draw=black, point meta=y]
            table [x=x, y expr={100*\thisrow{algebraic-nlsolve}/(\thisrow{initMesh} + (\thisrow{init}-\thisrow{initMesh}) + \thisrow{algebraic-nlsolve} + (\thisrow{algebraic-newton-initial-guess} + \thisrow{algebraic-jacobian} + \thisrow{algebraic-residual}) + \thisrow{exportResults})}] {\dataMCinq};
        \addplot+[ybar, bar width=0.2, fill=customgreen, draw=black, point meta=y]
            table [x=x, y expr={100*\thisrow{exportResults}/(\thisrow{initMesh} + (\thisrow{init}-\thisrow{initMesh}) + \thisrow{algebraic-nlsolve} + (\thisrow{algebraic-newton-initial-guess} + \thisrow{algebraic-jacobian} + \thisrow{algebraic-residual}) + \thisrow{exportResults})}] {\dataMCinq};

        \resetstackedplots

        \addplot+[ybar, bar width=0.2, fill=customdarkblue, draw=black, point meta=y]
            table [x=x, y expr={100*\thisrow{initMesh}/(\thisrow{initMesh} + (\thisrow{init}-\thisrow{initMesh}) + \thisrow{algebraic-nlsolve} + (\thisrow{algebraic-newton-initial-guess} + \thisrow{algebraic-jacobian} + \thisrow{algebraic-residual}) + \thisrow{exportResults})}] {\dataMSix};
        \addplot+[ybar, bar width=0.2, fill=customcyan, draw=black, point meta=y]
            table [x=x, y expr={100*(\thisrow{init}-\thisrow{initMesh})/(\thisrow{initMesh} + (\thisrow{init}-\thisrow{initMesh}) + \thisrow{algebraic-nlsolve} + (\thisrow{algebraic-newton-initial-guess} + \thisrow{algebraic-jacobian} + \thisrow{algebraic-residual}) + \thisrow{exportResults})}] {\dataMSix};
            \addplot+[ybar, bar width=0.2, fill=customorange, draw=black, point meta=y]
            table [x=x, y expr={100*(\thisrow{algebraic-newton-initial-guess} + \thisrow{algebraic-jacobian} + \thisrow{algebraic-residual})/(\thisrow{initMesh} + (\thisrow{init}-\thisrow{initMesh}) + \thisrow{algebraic-nlsolve} + (\thisrow{algebraic-newton-initial-guess} + \thisrow{algebraic-jacobian} + \thisrow{algebraic-residual}) + \thisrow{exportResults})}] {\dataMSix};
            \addplot+[ybar, bar width=0.2, fill=custompurple, draw=black, point meta=y]
                table [x=x, y expr={100*\thisrow{algebraic-nlsolve}/(\thisrow{initMesh} + (\thisrow{init}-\thisrow{initMesh}) + \thisrow{algebraic-nlsolve} + (\thisrow{algebraic-newton-initial-guess} + \thisrow{algebraic-jacobian} + \thisrow{algebraic-residual}) + \thisrow{exportResults})}] {\dataMSix};
        \addplot+[ybar, bar width=0.2, fill=customgreen, draw=black, point meta=y]
            table [x=x, y expr={100*\thisrow{exportResults}/(\thisrow{initMesh} + (\thisrow{init}-\thisrow{initMesh}) + \thisrow{algebraic-nlsolve} + (\thisrow{algebraic-newton-initial-guess} + \thisrow{algebraic-jacobian} + \thisrow{algebraic-residual}) + \thisrow{exportResults})}] {\dataMSix};
        \end{axis}
    \end{tikzpicture}

    \caption{Relative time spent in each component of the computation for the coupled heat-fluid testcase.}
    \label{fig:feelpp:wp1:coupled:time-rel}
\end{subfigure}
\caption{Absolute (\Cref{fig:feelpp:wp1:coupled:time}) and relative (\Cref{fig:feelpp:wp1:coupled:time-rel}) computational time for the coupled heat-fluid testcase, performed on Gaya with the meshes \texttt{Mr4} (left), \texttt{Mr5} (middle), and \texttt{Mr6} (right).}
\label{fig:feelpp:wp1:heatfluid:time}
\end{figure}


\paragraph{Challenges Identified}
Several challenges were encountered during the benchmarking process: \textbf{??}
\begin{itemize}
    \item \textbf{Memory Usage:}
    \item \textbf{Parallelization Inefficiencies:}
    \item \textbf{Cache and Memory Bottlenecks:}
\end{itemize}


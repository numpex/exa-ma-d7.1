\section{Software: Feel++}
\label{sec:WP1:Feel++:software}

\begin{table}[h!]
    \centering
    { \setlength{\parindent}{0pt}
    \def\arraystretch{1.25}
    \arrayrulecolor{numpexgray}
    {\fontsize{9}{11}\selectfont
    \begin{tabular}{!{\color{numpexgray}\vrule}p{.4\textwidth}!{\color{numpexgray}\vrule}p{.6\textwidth}!{\color{numpexgray}\vrule}}
        \rowcolor{numpexgray}{\rule{0pt}{2.5ex}\color{white}\bf Field} & {\rule{0pt}{2.5ex}\color{white}\bf Details} \\
        \rowcolor{white}\textbf{Consortium} & \begin{tabular}{l}
Feel++ Consortium\\
\end{tabular} \\
        \rowcolor{numpexlightergray}\textbf{Exa-MA Partners} & \begin{tabular}{l}
CNRS\\
Inria Grenoble\\
Unistra\\
\end{tabular} \\
        \rowcolor{white}\textbf{Contact Emails} & \begin{tabular}{l}
christophe.prudhomme@cemosis.fr\\
vincent.chabannes@cemosis.fr\\
\end{tabular} \\
        \rowcolor{numpexlightergray}\textbf{Supported Architectures} & \begin{tabular}{l}
CPU\\
\end{tabular} \\
        \rowcolor{white}\textbf{Repository} & \href{https://github.com/feelpp/feelpp}{https://github.com/feelpp/feelpp} \\
        \rowcolor{numpexlightergray}\textbf{License} & \begin{tabular}{l}
OSS:: GPL v*\\
OSS:: LGPL v*\\
\end{tabular} \\
        \bottomrule
    \end{tabular}
    }}
    \caption{WP1: Feel++ Information}
\end{table}

\subsection{Software Overview}
\label{sec:WP1:Feel++:summary}

Provide a brief overview of the software with respect to WP1.

\begin{table}[h!]
    \centering
    { 
        \setlength{\parindent}{0pt}
        \def\arraystretch{1.25}
        \arrayrulecolor{numpexgray}
        {
            \fontsize{9}{11}\selectfont
            \begin{tabular}{!{\color{numpexgray}\vrule}p{.25\linewidth}!{\color{numpexgray}\vrule}p{.6885\linewidth}!{\color{numpexgray}\vrule}}
    
    \rowcolor{numpexgray}{\rule{0pt}{2.5ex}\color{white}\bf Features} &  {\rule{0pt}{2.5ex}\color{white}\bf Short Description }\\ 
    
\rowcolor{white}    unstructured mesh & provide short description here \\
\rowcolor{numpexlightergray}    mesh adaptation & provide short description here \\
\rowcolor{white}    cG & provide short description here \\
\rowcolor{numpexlightergray}    dG/hdG & provide short description here \\
\rowcolor{white}    parallel in time & provide short description here \\
\rowcolor{numpexlightergray}    multiphysics coupling & provide short description here \\
\rowcolor{white}    multiscale coupling & provide short description here \\
\rowcolor{numpexlightergray}    inhouse & provide short description here \\
\rowcolor{white}    interface & provide short description here \\
\rowcolor{numpexlightergray}    finite element & provide short description here \\
\rowcolor{white}    spectral element & provide short description here \\
\end{tabular}
        }
    }
    \caption{WP1: Feel++ Features}
\end{table}


\subsection{Parallel Capabilities}
\label{sec:WP1:Feel++:performances}


\begin{itemize}
    \item describe the parallel programming  environment : MPI, OpenMP, CUDA, OpenACC, etc.
    \item describe the parallel computation environment: type of architecture and super computer used.
    \item describe the parallel capabilities of the software
    \item \textbf{Scalability:} Describe the general scalability properties of the software
    \item \textbf{Integration with Other Systems:} Describe how the software integrates with other numerical libraries and middleware in the Exa-MA framework.
\end{itemize}

\subsection{Initial Performance Metrics}
\label{sec:WP1:Feel++:metrics}

In this section, provide a summary of the initial performance metrics of the software with respect to WP1.
You can list one or more benchmarks and their associated results, the challenges , bottlenecks and the expectations for future versions of the software.



\subsubsection{Benchmark \#1}
\begin{itemize}
    \item describe the benchmark
    \item \textbf{Benchmarking Tools Used:} Describe the tools used for parallel benchmarking and the metrics mesured
    \item \textbf{Results Summary:} Results summary not available.
    \item \textbf{Challenges Identified:} No challenges identified.
\end{itemize}

\subsection{12 months roadmap}
\label{sec:WP1:Feel++:roadmap}

Describe the planned improvements  for the software in the context of WP1 and Exa-MA for the year to come that will be adressed in the next version of this deliverable.
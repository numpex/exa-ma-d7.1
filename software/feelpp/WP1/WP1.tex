%!TEX root = ../../../exa-ma-d7.1.tex
\section{Software: \texorpdfstring{\Feelpp}{Feel++}}
\label{sec:WP1:Feelpp:software}

\begin{table}[!ht]
    \centering
    { \setlength{\parindent}{0pt}
    \def\arraystretch{1.25}
    \arrayrulecolor{numpexgray}
    {\fontsize{9}{11}\selectfont
    \begin{tabular}{!{\color{numpexgray}\vrule}p{.4\textwidth}!{\color{numpexgray}\vrule}p{.6\textwidth}!{\color{numpexgray}\vrule}}
        \rowcolor{numpexgray}{\rule{0pt}{2.5ex}\color{white}\bf Field} & {\rule{0pt}{2.5ex}\color{white}\bf Details} \\
        \rowcolor{white}\textbf{Consortium} & \begin{tabular}{l}
\Feelpp{} Consortium\\
\end{tabular} \\
        \rowcolor{numpexlightergray}\textbf{Exa-MA Partners} & \begin{tabular}{l}
CNRS\\
Inria Grenoble\\
Unistra\\
\end{tabular} \\
        \rowcolor{white}\textbf{Contact Emails} & \begin{tabular}{l}
christophe.prudhomme@cemosis.fr\\
vincent.chabannes@cemosis.fr\\
\end{tabular} \\
        \rowcolor{numpexlightergray}\textbf{Supported Architectures} & \begin{tabular}{l}
CPU Only\\
\end{tabular} \\
        \rowcolor{white}\textbf{Repository} & \href{https://github.com/feelpp/feelpp}{https://github.com/feelpp/feelpp} \\
        \rowcolor{numpexlightergray}\textbf{License} & \begin{tabular}{l}
OSS:: GPL v*\\
OSS:: LGPL v*\\
\end{tabular} \\
        \rowcolor{white}\textbf{Bottlenecks roadmap} & \begin{tabular}{l}
B10 - Scientific Productivity\\
B11 - Reproducibility and Replicability of Computation\\
B12 - Pre/Post Processing and In-Situ Processing\\
B2 - Interconnect Technology\\
B6 - Data Management\\
B7 - Exascale Algorithms\\
\end{tabular} \\
\rowcolor{numpexlightergray}\textbf{Contributors} & \begin{tabular}{l}
    Christophe Prud'homme (UNISTRA)\\
    Vincent Chabannes (UNISTRA)\\
    Thomas Saigre (UNISTRA)\\
    Céline Van Landeghem (UNISTRA)\\
    Christophe Trophime (CNRS)\\
\end{tabular}\\
        \hline
    \end{tabular}
    }}
    \caption{WP1: \Feelpp{} Information}
\end{table}

\subsection{Software Overview}
\label{sec:WP1:Feelpp:summary}

\Feelpp is an open-source \Cpp{} library designed for solving partial differential equations (PDEs), it supports seamless parallel computing based on MPI, and it is designed to be highly modular and extensible.
It implements a \ac{DSEL} for variational formulations, which allows users to define complex PDEs in a concise and readable manner directly in \Cpp{} leveraging the power of modern \Cpp{} features.

In~\Cref{tab:WP1:Feelpp:features} we provide a summary of the software features relevant to the work package which are briefly discussed.

\begin{table}[!ht]
    \centering
    {
        \setlength{\parindent}{0pt}
        \def\arraystretch{1.25}
        \arrayrulecolor{numpexgray}
        {
            \fontsize{9}{11}\selectfont
            \begin{tabular}{!{\color{numpexgray}\vrule}p{.25\linewidth}!{\color{numpexgray}\vrule}p{.6885\linewidth}!{\color{numpexgray}\vrule}}

    \rowcolor{numpexgray}{\rule{0pt}{2.5ex}\color{white}\bf Features} &  {\rule{0pt}{2.5ex}\color{white}\bf Short Description }\\

\rowcolor{white}    cG&  continuous Galerkin of arbitrary order; conforming and non conforming interpolation operator;  \ac{DSEL} for cG methods and variational formulations \\
\rowcolor{numpexlightergray}    dG/hdG & support dG and HdG methods in 1D, 2D and 3D of arbitrary order; support postprocessing for increased accuracy for HdG; Static condensation multithreaded; \ac{DSEL} for dG/HdG methods and variational formulations\\
\rowcolor{white}    finite element & $H^1$, $L^2$, $H^\mathrm{div}$ and $H^\mathrm{curl}$ finite elements of arbitrary order in 1D, 2D, 3D; \\
\rowcolor{numpexlightergray}    inhouse & efficient data structures for localisation: BVH and KD-trees \\
\rowcolor{white}    interface & interfaces with MMG/ParMMG; Gmsh; Eigen3 \\
\rowcolor{numpexlightergray}    mesh adaptation & use MMG and ParMMG for mesh adaptation; mesh quality indicators to trigger adaptation\\
\rowcolor{white}    multiphysics coupling & Support for function space cartesian products; \ac{DSEL} for variational formulations  \\
\rowcolor{numpexlightergray}    multiscale coupling & multi-dimension coupling, \textit{e.g.} 0D-3D, 1D-3D or 2D-3D; support for solving coupled systems of PDEs and ODEs\\
\rowcolor{white}    parallel in time & space-time parallel implementation of original parareal algorithm \\
\rowcolor{numpexlightergray}    spectral element & high order methods on simplices and hypercubes; Gauss-Legendre, Gauss-Lobatto, Gauss-Radau, electrostatic and Fekete points;  high order geometric transformation (up to order 4 using Gmsh); construction of $P_N \mathrm{iso} P_1$; $L^2$ orthonormal basis functions used as primal basis functions on hypercubes (Gauss-Legendre) and simplices (Dubiner) \\
\rowcolor{white}    unstructured mesh & Hypercubes and Simplices meshes in 1D, 2D and 3D as well as 1D meshes in 2D and 3D and 2D meshes in 3D; efficient localisation using kd-tree; \ac{DSEL} to manipulate geometric entity collections (points,edges,faces, facets and volumes)\\
\hline
\end{tabular}
        }
    }
    \caption{WP1: \Feelpp Features}
    \label{tab:WP1:Feelpp:features}
\end{table}


\subsection{Parallel Capabilities}
\label{sec:WP1:Feelpp:performances}


\begin{itemize}
    \item describe the parallel programming  environment : MPI, OpenMP, CUDA, OpenACC, etc.
    \item describe the parallel computation environment: type of architecture and super computer used.
    \item describe the parallel capabilities of the software
    \item \textbf{Scalability:} Describe the general scalability properties of the software
    \item \textbf{Integration with Other Systems:} Describe how the software integrates with other numerical libraries in the Exa-MA framework.
\end{itemize}


\subsection{Initial Performance Metrics}
\label{sec:WP1:Feelpp:metrics}

This section provides a summary of initial performance benchmarks performed in the context of WP1. It ensures reproducibility by detailing input/output datasets, benchmarking tools, and the results. All data should be publicly available, ideally with a DOI for future reference.

\begin{itemize}
    \item \textbf{Overall Performance:} Summarize the software's computational performance, energy efficiency, and scalability results across different architectures (e.g., CPU, GPU, hybrid systems).
    \item \textbf{Input/Output Dataset:} Provide a detailed description of the dataset used for the benchmark, including:
        \begin{itemize}
            \item Input dataset size, structure, and format (e.g., CSV, HDF5, NetCDF).
            \item Output dataset format and key results.
            \item Location of the dataset (e.g., GitHub repository, institutional repository, or open access platform).
            \item DOI or permanent link for accessing the dataset.
        \end{itemize}
    \item \textbf{open-data Access:} Indicate whether the datasets used for the benchmark are open access, and provide a DOI or a direct link for download. Where applicable, highlight any licensing constraints.
    \item \textbf{Challenges:} Identify any significant bottlenecks or challenges observed during the benchmarking process, including data handling and computational performance.
    \item \textbf{Future Improvements:} Outline areas for optimization, including dataset handling, memory usage, or algorithmic efficiency, to address identified challenges.
\end{itemize}

% create latex counter
\newcounter{feelppWP1benchcounter}
% set the counter to 1
\setcounter{feelppWP1benchcounter}{1}
\subimport{../software/feelpp/WP1}{WP1-thermalbridge.tex}
% increment the counter
\stepcounter{feelppWP1benchcounter}
% get value of counter



\subimport{../software/feelpp/WP1}{WP1-heatfluid.tex}
% increment the counter
\stepcounter{feelppWP1benchcounter}
\subimport{../software/feelpp/WP1}{WP1-contact.tex}
%\subimport{./}{WP1-lap-elas}



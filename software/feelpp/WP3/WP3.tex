%!TEX root = ../../../exa-ma-d7.1.tex
\section{Software: \texorpdfstring{\Feelpp}{Feel++}}
\label{sec:WP3:Feelpp:software}

\begin{table}[!ht]
    \centering
    { \setlength{\parindent}{0pt}
    \def\arraystretch{1.25}
    \arrayrulecolor{numpexgray}
    {\fontsize{9}{11}\selectfont
    \begin{tabular}{!{\color{numpexgray}\vrule}p{.4\textwidth}!{\color{numpexgray}\vrule}p{.6\textwidth}!{\color{numpexgray}\vrule}}
        \rowcolor{numpexgray}{\rule{0pt}{2.5ex}\color{white}\bf Field} & {\rule{0pt}{2.5ex}\color{white}\bf Details} \\
        \rowcolor{white}\textbf{Consortium} & \begin{tabular}{l}
\Feelpp Consortium\\
\end{tabular} \\
        \rowcolor{numpexlightergray}\textbf{Exa-MA Partners} & \begin{tabular}{l}
CNRS\\
Inria Grenoble\\
Unistra\\
\end{tabular} \\
        \rowcolor{white}\textbf{Contact Emails} & \begin{tabular}{l}
christophe.prudhomme@cemosis.fr\\
vincent.chabannes@cemosis.fr\\
\end{tabular} \\
        \rowcolor{numpexlightergray}\textbf{Supported Architectures} & \begin{tabular}{l}
CPU Only\\
\end{tabular} \\
        \rowcolor{white}\textbf{Repository} & \href{https://github.com/feelpp/feelpp}{https://github.com/feelpp/feelpp} \\
        \rowcolor{numpexlightergray}\textbf{License} & \begin{tabular}{l}
OSS:: GPL v*\\
OSS:: LGPL v*\\
\end{tabular} \\
        \rowcolor{white}\textbf{Bottlenecks roadmap} & \begin{tabular}{l}
B10 - Scientific Productivity\\
B11 - Reproducibility and Replicability of Computation\\
B12 - Pre/Post Processing and In-Situ Processing\\
B2 - Interconnect Technology\\
B6 - Data Management\\
B7 - Exascale Algorithms\\
\end{tabular} \\
\rowcolor{numpexlightergray}\textbf{Contributors} & \begin{tabular}{l}
    Christophe Prud'homme (UNISTRA)\\
    Vincent Chabannes (UNISTRA)\\
    Thomas Saigre (UNISTRA)\\
\end{tabular}\\
        \hline
    \end{tabular}
    }}
    \caption{WP3: \Feelpp Information}
\end{table}

\subsection{Software Overview}
\label{sec:WP3:Feelpp:summary}

In~\cref{tab:WP3:Feelpp:features} we provide the summary of the \Feelpp features relevant to the work package which are briefly discussed.

\begin{table}[!ht]
    \centering
    {
        \setlength{\parindent}{0pt}
        \def\arraystretch{1.25}
        \arrayrulecolor{numpexgray}
        {
            \fontsize{9}{11}\selectfont
            \begin{tabular}{!{\color{numpexgray}\vrule}p{.25\linewidth}!{\color{numpexgray}\vrule}p{.6885\linewidth}!{\color{numpexgray}\vrule}}

    \rowcolor{numpexgray}{\rule{0pt}{2.5ex}\color{white}\bf Features} &  {\rule{0pt}{2.5ex}\color{white}\bf Short Description }\\

\rowcolor{white}    algebraic multiphysics coupling & In \Feelpp, algebraic multiphysics coupling is achieved through the fieldsplit PETSc preconditioner. This allows the construction of block preconditioners, leveraging both PETSc’s built-in preconditioners and in-house preconditioners designed for specific applications, such as computational fluid dynamics (CFD) and magnetostatic problems. These preconditioners enhance the solver's performance for coupled systems by efficiently handling the interaction between different physical fields. \\
\rowcolor{numpexlightergray}    domain decomposition methods & \Feelpp provides support for domain decomposition methods, such as hp-mortar methods. These methods utilize specialized preconditioners that exploit the structure of mortar elements to improve solver efficiency. The hp-mortar approach allows for flexible discretization and is particularly useful for non-conforming meshes and multi-domain problems, enhancing both scalability and accuracy in large-scale simulations. \\
\hline
\end{tabular}
        }
    }
    \caption{WP3: \Feelpp Features}
    \label{tab:WP3:Feelpp:features}
\end{table}


\subsection{Parallel Capabilities}
\label{sec:WP3:Feelpp:performances}

\begin{description}
    \item[Parallel Environment:] MPI.
    \item[Architectures] CPU Only: Gaya~\cref{sec:arch:gaya} and almost all JU systems~\cref{sec:arch:eurohpc-ju} except Marenostrum 5 and Deucalion.
    \item[Scalability] Only Speedups are measured
\end{description}

\subsection{Initial Performance Metrics}
\label{sec:WP3:Feelpp:metrics}

This section provides a summary of initial performance benchmarks performed in the context of WP3. It ensures reproducibility by detailing input/output datasets, benchmarking tools, and the results. All data should be publicly available, ideally with a DOI for future reference.

\begin{itemize}
    \item \textbf{Overall Performance:} Summarize the software's computational performance, energy efficiency, and scalability results across different architectures (e.g., CPU, GPU, hybrid systems).
    \item \textbf{Input/Output Dataset:} Provide a detailed description of the dataset used for the benchmark, including:
        \begin{itemize}
            \item Input dataset size, structure, and format (e.g., CSV, HDF5, NetCDF).
            \item Output dataset format and key results.
            \item Location of the dataset (e.g., GitHub repository, institutional repository, or open access platform).
            \item DOI or permanent link for accessing the dataset.
        \end{itemize}
    \item \textbf{open-data Access:} Indicate whether the datasets used for the benchmark are open access, and provide a DOI or a direct link for download. Where applicable, highlight any licensing constraints.
    \item \textbf{Challenges:} Identify any significant bottlenecks or challenges observed during the benchmarking process, including data handling and computational performance.
    \item \textbf{Future Improvements:} Outline areas for optimization, including dataset handling, memory usage, or algorithmic efficiency, to address identified challenges.
\end{itemize}

\subsubsection{Benchmark \#1: Laplacian and Linear Elasticity}
\paragraph{Description:} Briefly describe the benchmark case, including the problem size, target architecture (e.g., CPU, GPU), and the input data. Mention the specific goals of the benchmark (e.g., testing scalability, energy efficiency).

\paragraph{Benchmarking Tools Used:} List the tools used for performance analysis, such as Extrae, Score-P, TAU, Vampir, or Nsight, and specify what metrics were measured (e.g., execution time, FLOPS, energy consumption).

\paragraph{Input/Output Dataset Description:}
\paragraph{Input Data:} Describe the input dataset (size, format, data type) and provide a DOI or link to access it.

\paragraph{Output Data:} Specify the structure of the results (e.g., memory usage, runtime logs) and how they can be accessed or replicated.

\paragraph{Data Repository:} Indicate where the data is stored (e.g., Zenodo, institutional repository) and provide a DOI or URL for accessing the data.

\paragraph{Results Summary:} Include a summary of key metrics (execution time, memory usage, FLOPS) and their comparison across architectures (e.g., CPU, GPU).

\paragraph{Challenges Identified:} Describe any bottlenecks encountered (e.g., memory usage, parallelization inefficiencies) and how they impacted the benchmark.

\subsubsection{Benchmark \#2: Elliptic linear PDE : Thermal bridges}
\label{sec:WP3:Feelpp:benchmark:thermal_bridges}
\paragraph{Description:} % Briefly describe the benchmark case, including the problem size, target architecture (e.g., CPU, GPU), and the input data. Mention the specific goals of the benchmark (e.g., testing scalability, energy efficiency).
The benchmark description can be found in
\Cref{sec:WP1:Feelpp:benchmark:thermal_bridges}.

The goal of this benchmark in WP3 is to analyze the scalability performance of a linear solver
available in \Feelpp, which is a wrapper of Petsc solver.
In this benchmark, we use a GMRES solver preconditioned by an algebraic
multigrid called GAMG. This kind of preconditioner is generally well adapted to
elliptic PDE.


\paragraph{Benchmarking Tools Used:} %List the tools used for performance analysis, such as Extrae, Score-P, TAU, Vampir, or Nsight, and specify what metrics were measured (e.g., execution time, FLOPS, energy consumption).
See \Cref{sec:WP1:Feelpp:benchmark:thermal_bridges}

\paragraph{Input/Output Dataset Description:}
See \Cref{sec:WP1:Feelpp:benchmark:thermal_bridges}

\paragraph{Results Summary:}% Include a summary of key metrics (execution time, memory usage, FLOPS) and their comparison across architectures (e.g., CPU, GPU).

The metrics computed in this experiment are the time execution of the linear solver and the number of iterations of the GMRES algorithm.
The \Cref{fig:feelpp:wp3:thermal_bridges:performance_measure_splitted}
illustrate the results obtained. In each subfigure, we can see the impact of the
mesh level for a finite element approximation. We have also \Cref{fig:feelpp:wp3:thermal_bridges:performance_measure_all}
which present all results in the same charts. It allows us to see the impact of
mesh and finite element choice.

The speed-up is not very good (or ideal), particularly for the coarse mesh. The
reasons are not yet well understood because we can see that the number of
iterations is quite constant with respect to the number of CPU cores. We need to
investigate more in detail what happened. At the number of iteration levels, the
results are good, with excellent constant behavior with respect to the number of
CPU cores. We can see also the impact of the mesh level and finite element
approximation order. As expected, the number of iterations increases with the
number of degrees of freedom but not too much. However, we can notice in the
case $P_1$ that the mesh M1 takes more iteration than the other. Maybe some
phenomena are not well captured with this approximation which is the coarsen.



\begin{figure}
  \centering

  \foreach [expand list=true] \polyId in {1,2,3} {

    \pgfplotstableread[col sep=comma]{\currfiledir/../WP1/data/thermalbridges_M1_P\polyId_discoverer.csv}\dataMa
    \pgfplotstableread[col sep=comma]{\currfiledir/../WP1/data/thermalbridges_M2_P\polyId_discoverer.csv}\dataMb
    \pgfplotstableread[col sep=comma]{\currfiledir/../WP1/data/thermalbridges_M3_P\polyId_discoverer.csv}\dataMc

    \def\plotSetup{
      {table=dataMa,column=algebraic-solve,legend=M1,color=customdarkblue},
      {table=dataMb,column=algebraic-solve,legend=M2,color=customcyan},
      {table=dataMc,column=algebraic-solve,legend=M3,color=customorange}
    }


    \begin{subfigure}[c]{0.49\textwidth}
      \centering
      \barChart[ybar,xticklabels from table={\dataMa}{nProc}]{\plotSetup}
        \caption{Execution time - $P_{\polyId}$}
    \end{subfigure}
    \begin{subfigure}[c]{0.49\textwidth}
      \centering
      \begin{tikzpicture}
        \def\myLineWidth{2pt}
        \begin{axis}[
          width=\textwidth, height=0.6172\textwidth,
          % xtick=data,
          xmajorgrids=true, xminorgrids=false, minor x tick num=3,
          ymajorgrids=true, %yminorgrids=true,
          minor y tick num=2,
          % xticklabel={\pgfmathparse{100*\tick}\pgfmathprintnumber[precision=0]{\pgfmathresult}\%},
          xticklabel={\pgfmathparse{\tick}\pgfmathprintnumber[fixed,set thousands separator={},precision=0]{\pgfmathresult}},
          xlabel={Number of CPU cores}, ylabel={Number of iterations},
          legend style={at={(0.5,1)}, anchor=south,font=\small,legend columns=3}
          ]
          \addplot[color=customdarkblue,mark=o,line width=\myLineWidth] table [x=nProc, y=ksp-niter] {\dataMa};
          \addlegendentry{M1}
          \addplot[color=customcyan,mark=triangle,line width=\myLineWidth] table [x=nProc, y=ksp-niter] {\dataMb};
          \addlegendentry{M2}
          \addplot[color=customorange,mark=square,line width=\myLineWidth] table [x=nProc, y=ksp-niter] {\dataMc};
          \addlegendentry{M3}
        \end{axis}
      \end{tikzpicture}
      \caption{Number of iterations of GMRES - $P_{\polyId}$}
    \end{subfigure}
  }
  \caption{Thermal bridges benchmarks - Performance measures of algebraic solver
    (Mesh comparison) - GMRES preconditionned by GAMG - Discoverer supercomputer}
  \label{fig:feelpp:wp3:thermal_bridges:performance_measure_splitted}
\end{figure}



\begin{figure}
  \centering

  \pgfplotstableread[col sep=comma]{\currfiledir/../WP1/data/thermalbridges_M1_P1_discoverer.csv}\dataMaPa
  \pgfplotstableread[col sep=comma]{\currfiledir/../WP1/data/thermalbridges_M1_P2_discoverer.csv}\dataMaPb
  \pgfplotstableread[col sep=comma]{\currfiledir/../WP1/data/thermalbridges_M1_P3_discoverer.csv}\dataMaPc
  \pgfplotstableread[col sep=comma]{\currfiledir/../WP1/data/thermalbridges_M2_P1_discoverer.csv}\dataMbPa
  \pgfplotstableread[col sep=comma]{\currfiledir/../WP1/data/thermalbridges_M2_P2_discoverer.csv}\dataMbPb
  \pgfplotstableread[col sep=comma]{\currfiledir/../WP1/data/thermalbridges_M2_P3_discoverer.csv}\dataMbPc
  \pgfplotstableread[col sep=comma]{\currfiledir/../WP1/data/thermalbridges_M3_P1_discoverer.csv}\dataMcPa
  \pgfplotstableread[col sep=comma]{\currfiledir/../WP1/data/thermalbridges_M3_P2_discoverer.csv}\dataMcPb
  \pgfplotstableread[col sep=comma]{\currfiledir/../WP1/data/thermalbridges_M3_P3_discoverer.csv}\dataMcPc

  \def\chartLinePlot#1#2{
    \begin{tikzpicture}
      \def\myLineWidth{2pt}
      \def\myLineStyleA{loosely dashdotdotted} %dashdotdotted
      \def\myLineStyleB{dashed}
      \def\myLineStyleC{solid}
      %\def\myMarkStyle{every mark/.append style={solid}}
      %\edef\myMarkStyle{\noexpand{every mark/.append style={solid}}}

      \begin{axis}[
        width=\textwidth, height=0.6172\textwidth,
        % xtick=data,
        xmajorgrids=true, xminorgrids=false, minor x tick num=3,
        ymajorgrids=true, %yminorgrids=true,
        minor y tick num=2,
        % xticklabel={\pgfmathparse{100*\tick}\pgfmathprintnumber[precision=0]{\pgfmathresult}\%},
        xticklabel={\pgfmathparse{\tick}\pgfmathprintnumber[fixed,set thousands separator={},precision=0]{\pgfmathresult}},
        xlabel={Number of CPU cores},% ylabel={Number of iterations},
        % legend style={at={(0.5,1)}, anchor=south,font=\small,legend columns=3}
        legend style={at={(0.,1)}, anchor=north west,font=\small,legend columns=3},
        #2
        ]
        \addplot[color=customdarkblue,\myLineStyleA,mark=o,line width=\myLineWidth,every mark/.append style={solid}] table [x=nProc, y=#1] {\dataMaPa};
        \addlegendentry{M1-P1}
        \addplot[color=customdarkblue,\myLineStyleB,mark=o,line width=\myLineWidth,every mark/.append style={solid}] table [x=nProc, y=#1] {\dataMaPb};
        \addlegendentry{M1-P2}
        \addplot[color=customdarkblue,\myLineStyleC,mark=o,line width=\myLineWidth,every mark/.append style={solid}] table [x=nProc, y=#1] {\dataMaPc};
        \addlegendentry{M1-P3}
        \addplot[color=customcyan,\myLineStyleA,mark=triangle,line width=\myLineWidth,every mark/.append style={solid}] table [x=nProc, y=#1] {\dataMbPa};
        \addlegendentry{M2-P1}
        \addplot[color=customcyan,\myLineStyleB,mark=triangle,line width=\myLineWidth,every mark/.append style={solid}] table [x=nProc, y=#1] {\dataMbPb};
        \addlegendentry{M2-P2}
        \addplot[color=customcyan,\myLineStyleC,mark=triangle,line width=\myLineWidth,every mark/.append style={solid}] table [x=nProc, y=#1] {\dataMbPc};
        \addlegendentry{M2-P3}
        \addplot[color=customorange,\myLineStyleA,mark=square,line width=\myLineWidth,every mark/.append style={solid}] table [x=nProc, y=#1] {\dataMcPa};
        \addlegendentry{M3-P1}
        \addplot[color=customorange,\myLineStyleB,mark=square,line width=\myLineWidth,every mark/.append style={solid}] table [x=nProc, y=#1] {\dataMcPb};
        \addlegendentry{M3-P2}
        \addplot[color=customorange,\myLineStyleC,mark=square,line width=\myLineWidth,every mark/.append style={solid}] table [x=nProc, y=#1] {\dataMcPc};
        \addlegendentry{M3-P3}
      \end{axis}
    \end{tikzpicture}
  }

  \begin{subfigure}[c]{\textwidth}
    \centering
    \chartLinePlot{ksp-niter}{ylabel={Number of iterations}}
    \caption{Number of iterations of GMRES}
  \end{subfigure}
  \begin{subfigure}[c]{\textwidth}
    \centering
    \chartLinePlot{algebraic-solve}{ylabel={Execution time [s]}}
    \caption{Execution time}
  \end{subfigure}
  \caption{Thermal bridges benchmarks - Performance measures of algebraic solver
    (Mesh and polynomial order comparison) - GMRES preconditionned by GAMG - Discoverer supercomputer}
  \label{fig:feelpp:wp3:thermal_bridges:performance_measure_all}
\end{figure}


\paragraph{Challenges Identified:} %Describe any bottlenecks encountered (e.g., memory usage, parallelization inefficiencies) and how they impacted the benchmark.

\begin{itemize}
\item Improve the HPC scalability at large/extreme scale.
\item Reduce the memory
\end{itemize}

\subsubsection{Benchmark \#3: CFD FDA Benchmark}

\fullcite{chabannes_high_2017}
\paragraph{Description:} Briefly describe the benchmark case, including the problem size, target architecture (e.g., CPU, GPU), and the input data. Mention the specific goals of the benchmark (e.g., testing scalability, energy efficiency).

\paragraph{Benchmarking Tools Used:} List the tools used for performance analysis, such as Extrae, Score-P, TAU, Vampir, or Nsight, and specify what metrics were measured (e.g., execution time, FLOPS, energy consumption).

\paragraph{Input/Output Dataset Description:}
\paragraph{Input Data:} Describe the input dataset (size, format, data type) and provide a DOI or link to access it.

\paragraph{Output Data:} Specify the structure of the results (e.g., memory usage, runtime logs) and how they can be accessed or replicated.

\paragraph{Data Repository:} Indicate where the data is stored (e.g., Zenodo, institutional repository) and provide a DOI or URL for accessing the data.

\paragraph{Results Summary:} Include a summary of key metrics (execution time, memory usage, FLOPS) and their comparison across architectures (e.g., CPU, GPU).

\paragraph{Challenges Identified:} Describe any bottlenecks encountered (e.g., memory usage, parallelization inefficiencies) and how they impacted the benchmark.

\subsubsection{Benchmark \#4: Conjuguate heat transfer}



The benchmark has been described in the prequel of this document, in \Cref{sec:WP1:Feelpp:benchmark4}, and simulate the aqueous humor flow in the human eye, coupled with the heat transfer.
The strategy employed to simulate the non-linear model implemented in the \texttt{heatfluid} toolbox of \Feelpp consists of a fixed-point iteration.
Without efficient preconditioning, iterative solvers would not actually converge for solving the system.
Direct methods are only usable for small problems, otherwise, memory and computational costs can exceed available resources and render them unsuitable for larger problems.

Precisely, we employ a \emph{fieldsplit}\index{fieldsplit}, or PDE-based, strategy, allowing us to define what we could call ``sub-preconditioners'' for each system block.
We split the system into two blocks according to the fluid and heat unknowns.
The fluid block is solved using a Schur complement approach~\cite{elman_finite_2014}, while the heat block is solved using a few iteration of GAMG.
%
A more in-depth description of the preconditioner can be found in~\cite{saigre_coupled_2024_paper}.


\subsection{12-Month Roadmap}
\label{sec:WP3:Feelpp:roadmap}

In this section, describe the roadmap for improving benchmarks and addressing the challenges identified. This should include:
\begin{itemize}
    \item \textbf{Data Improvements:} Plans for improving input/output data management, including making datasets more accessible and ensuring reproducibility through open-data initiatives.
    \item \textbf{Methodology Application:} Implementation of the benchmarking methodology proposed in this deliverable to streamline reproducibility and dataset management.
    \item \textbf{Results Retention:} Plans to maintain benchmark results in a publicly accessible repository with appropriate metadata and documentation, ensuring long-term usability.
\end{itemize}

In~\cref{tab:WP3:Feelpp:bottlenecks}, we briefly discuss the bottleneck roadmap associated to the software and relevant to the work package.

\begin{table}[!ht]
    \centering



    \centering
    {
        \setlength{\parindent}{0pt}
        \def\arraystretch{1.25}
        \arrayrulecolor{numpexgray}
        {
            \fontsize{9}{11}\selectfont
            \begin{tabular}{!{\color{numpexgray}\vrule}p{.25\linewidth}!{\color{numpexgray}\vrule}p{.6885\linewidth}!{\color{numpexgray}\vrule}}

    \rowcolor{numpexgray}{\rule{0pt}{2.5ex}\color{white}\bf Bottlenecks} &  {\rule{0pt}{2.5ex}\color{white}\bf Short Description }\\

\rowcolor{white}    B10 - Scientific Productivity & provide short description here \\
\rowcolor{numpexlightergray}    B11 - Reproducibility and Replicability of Computation & provide short description here \\
\rowcolor{white}    B12 - Pre/Post Processing and In-Situ Processing & provide short description here \\
\rowcolor{numpexlightergray}    B2 - Interconnect Technology & provide short description here \\
\rowcolor{white}    B6 - Data Management & provide short description here \\
\rowcolor{numpexlightergray}    B7 - Exascale Algorithms & enable hpddm and better preconditionner configurations at large scale (eg. algebraic multigrid), enable algebraic saddle point preconditioners from WP3, enable hp-mortar and compression strategies; update with latest PETSc improvement and start using PETSc GPU API\\
\hline
\end{tabular}
        }
    }
    \caption{WP3: \Feelpp plan with Respect to Relevant Bottlenecks}
    \label{tab:WP3:Feelpp:bottlenecks}
\end{table}

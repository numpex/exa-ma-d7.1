%!TEX root = ../../../exa-ma-d7.1.tex
\section{Software: \texorpdfstring{\Feelpp}{Feel++}}
\label{sec:WP2:Feelpp:software}

\begin{table}[!ht]
    \centering
    { \setlength{\parindent}{0pt}
    \def\arraystretch{1.25}
    \arrayrulecolor{numpexgray}
    {\fontsize{9}{11}\selectfont
    \begin{tabular}{!{\color{numpexgray}\vrule}p{.4\textwidth}!{\color{numpexgray}\vrule}p{.6\textwidth}!{\color{numpexgray}\vrule}}
        \rowcolor{numpexgray}{\rule{0pt}{2.5ex}\color{white}\bf Field} & {\rule{0pt}{2.5ex}\color{white}\bf Details} \\
        \rowcolor{white}\textbf{Consortium} & \begin{tabular}{l}
\Feelpp Consortium\\
\end{tabular} \\
        \rowcolor{numpexlightergray}\textbf{Exa-MA Partners} & \begin{tabular}{l}
CNRS\\
Inria Grenoble\\
Unistra\\
\end{tabular} \\
        \rowcolor{white}\textbf{Contact Emails} & \begin{tabular}{l}
christophe.prudhomme@cemosis.fr\\
vincent.chabannes@cemosis.fr\\
\end{tabular} \\
        \rowcolor{numpexlightergray}\textbf{Supported Architectures} & \begin{tabular}{l}
CPU Only\\
\end{tabular} \\
        \rowcolor{white}\textbf{Repository} & \href{https://github.com/feelpp/feelpp}{https://github.com/feelpp/feelpp} \\
        \rowcolor{numpexlightergray}\textbf{License} & \begin{tabular}{l}
OSS:: GPL v*\\
OSS:: LGPL v*\\
\end{tabular} \\
        \rowcolor{white}\textbf{Bottlenecks roadmap} & \begin{tabular}{l}
B10 - Scientific Productivity\\
B11 - Reproducibility and Replicability of Computation\\
B12 - Pre/Post Processing and In-Situ Processing\\
B2 - Interconnect Technology\\
B6 - Data Management\\
B7 - Exascale Algorithms\\
\end{tabular} \\
\rowcolor{numpexlightergray}\textbf{Contributors} & \begin{tabular}{l}
    Christophe Prud'homme (UNISTRA)\\
    Vincent Chabannes (UNISTRA)\\
    Thomas Saigre (UNISTRA)\\
\end{tabular}\\
        \bottomrule
    \end{tabular}
    }}
    \caption{WP2: \Feelpp Information}
\end{table}

\subsection{Software Overview}
\label{sec:WP2:Feelpp:summary}

In~\cref{tab:WP2:Feelpp:features}, we provide a summary of the software features relevant to the work package which are briefly discussed.

\begin{table}[!ht]
    \centering
    {
        \setlength{\parindent}{0pt}
        \def\arraystretch{1.25}
        \arrayrulecolor{numpexgray}
        {
            \fontsize{9}{11}\selectfont
            \begin{tabular}{!{\color{numpexgray}\vrule}p{.25\linewidth}!{\color{numpexgray}\vrule}p{.6885\linewidth}!{\color{numpexgray}\vrule}}

    \rowcolor{numpexgray}{\rule{0pt}{2.5ex}\color{white}\bf Features} &  {\rule{0pt}{2.5ex}\color{white}\bf Short Description }\\

\rowcolor{white}    ROM-DA: GEIM PBDW & \Feelpp support data driven reduced basis method PBDW and GEIM which can work often in combination\\
\rowcolor{numpexlightergray}    ROM: RB, CRB, RB-GREEDY, POD, EIM, SCM, $\min-\theta$ & \Feelpp supports efficient reduced basis approximation thanks to affine decomposition that can be computed by hand or via EIM. Several constructions are possible: Reduced Basis from sampling, Greedy and POD. Certified Reduced Basis can be built. The online code is generated as well as a database and associated metadata. \\
\rowcolor{white}    ROM: NIRB & Nonintrusive reduced basis~\cite{CRMATH_2009__347_7-8_435_0} are available as a python layer \\
\hline
\end{tabular}
        }
    }
    \caption{WP2: \Feelpp Features}
    \label{tab:WP2:Feelpp:features}
\end{table}


\subsection{Parallel Capabilities}
\label{sec:WP2:Feelpp:performances}


\begin{description}
    \item[Parallel Programming Environment:] The model order reduction methods profit from \Feelpp parallel capabilities. MPI is used to distribute the load for offline calculation. The online stage can use multithreading.
    \item[Super computer:] The reduced basis has been run on Gaya~\Cref{sec:arch:gaya}.
    \item[Parallel Capabilities:] the offline stage can profit from the parallel capabilities of \Feelpp.
    \item[Scalability:] The reduced basis methods share similar performances as in WP1~\Cref{sec:WP1:Feelpp:software}.
    \item[Integration with Other Systems:] The reduced basis methods are often used in combination with OpenTURNS and OpenMODELICA/Dymola.
\end{description}

\subsection{Initial Performance Metrics}
\label{sec:WP2:Feelpp:metrics}

This section provides a summary of initial performance benchmarks performed in the context of WP2. It ensures reproducibility by detailing input/output datasets, benchmarking tools, and the results. All data should be publicly available, ideally with a DOI for future reference.

\begin{description}
    \item[Overall Performance] \Feelpp reduced order method profit from \Feelpp parallel capabilities.
    \item[Input/Output Dataset] The dataset for the \Feelpp model order reduction methods are composed of the finite element configuration for the offline stage and a JSON description of the parameter domain, the parameter variability and the outputs of interests.
    \item[open-data Access] \Feelpp uses json as metadata format and in-house open format for the online database. The current format require documentation.
    \item[Challenges] Some of the challenges are that the reduced basis method imply a complex workflow with many steps that need to be carefully configured for efficient computation. In particular the reduced basis functions require  robust finite element solves as well as a posteriori error bounds or estimations which means developing robust algrebraic solution strategies.
    \item[Future Improvements] performance monitoring of all the steps of the reduced order methods, documentation of the online stage data formats and large scale benchmarking.
\end{description}



\subsubsection{Benchmark \#1: Heat transfer in the eye using \ac{RB} and \ac{NIRB}}

\paragraph{Description:} \cite{saigre_model_2024} presents a 3D mathematical model to simulate heat transfer in the human eye, addressing the challenges of experimental analysis. A sensitivity analysis explores the effects of various parameters on heat distribution in ocular tissues. To improve computational efficiency, we applied a fast model reduction technique with certified error bounds. Our results align with experimental and numerical findings, highlighting the importance of blood flow and environmental factors.

In~\cite{saigre_mathematical_2024}, we extend the reduced order model to include non-intrusive reduced basis methods~\textbf{cite ?}. The benchmark assesses the software's performance in simulating heat transfer in the eye using these two methods.
This second methods consists in solving the high fidelity model on a coarse mesh and use interpolation and a reduced basis to obtain an approximation of the solution on a fine mesh.
A rectification post-process is applied to the reduced solution to improve the accuracy of the reduced solution.

\paragraph{Benchmarking Tools Used:} We used \Feelpp to monitor the time spent in the various section of the code, including the reduced basis generation and the non-intrusive reduced basis method. To assess the software's performance, we measured the following quantities:
\begin{inparaenum}[\it (i)]
    \item the time to compute the high fidelity solution $T^\text{fem}(\mu)$,
    \item the time to perform the online stage of the RBM, \emph{i.e.} to compute the reduced solution $T^{\text{rbm}, N}(\mu)$ and associated the error bound $\Delta_N(\mu)$,
\end{inparaenum}

\paragraph{Input/Output Dataset Description}

\begin{description}
    \item[Input Data:] The dataset for the \Feelpp model order reduction methods are composed of the finite element configuration for the offline stage and a JSON description of the parameter domain, the parameter variability and the outputs of interests.
    \item[Output Data:] The output data is composed of a code, performing the online stage of the methods, as well as database associated to the reduced basis.
    \item[Data Repository:] The source code to generated the data for the two methods are available in the \Feelpp repository~\cite{prudhomme_feelppfeelpp_2024}.
\end{description}

\paragraph{Results Summary:}

\Cref{tab:feelpp:wp2:eye:rbm} displays a comparative analysis of execution times for solving the heat transfer problem.
We first discuss the execution times for the high-fidelity solution, encompassing both $P_1$ and $P_2$ finite-element discretizations.
The measured time, denoted as $t_\text{exec}$, includes assembling and solving the problem.
In addition, we also evaluate the execution time of the online phase of our certified reliable reduced basis model.
It corresponds to the time spent by the application to compute the reduced solution $T^{\text{rbm},N}$, as well as the error bound $\Delta_N$, for a reduced basis of size $N=10$.
This comparison highlights a significant reduction in the time required to assemble and solve the problem using our advanced reduced basis approach.
Importantly, this efficiency does not compromise accuracy; the results from the reduced basis model are effectively sharp with respect to the high fidelity model.
More detailed insights abouth these results are available in \cite{saigre_model_2024}.

We present in \Cref{tab:feelpp:wp2:eye:nirb} the time of execution and associated speedup to run the NIRB method the bioheat transfer problem in the eye using $P_2$ discretization.
The time of execution is measured for various elements:
\begin{inparaenum}[\it (i)]
    \item The fine snapshots $T_h^\mathcal{N}$, \emph{i.e.} the high-fidelity solution,
    \item the coarse snapshots $T_H^\mathcal{N}$, \emph{i.e.} a high-fidelity solution on the coarse mesh,
    \item the NIRB approximation without the rectification post-process $T_{Hh}^{N,\text{NoRect}}$, and
    \item the NIRB approximation with the rectification post-process $T_{Hh}^{N,\text{Rect}}$. This last element is the final reduced solution.
\end{inparaenum}
%
The gain of this second method is less significant than the RBM method, but it is still a significant improvement in terms of computational time.
We remark that the problem scales well with the number of processors, as evidenced by the speedup obtained for the quantites computed.



\begin{table}
    \centering

\end{table}


\begin{table}
    \centering

    \begin{subtable}[b]{\textwidth}
    \centering
    {\setlength{\parindent}{0pt}
    \def\arraystretch{1.25}
    \arrayrulecolor{numpexgray}
    \begin{tabular}{!{\color{numpexgray}\vrule}c!{\color{numpexgray}\vrule}c!{\color{numpexgray}\vrule}c!{\color{numpexgray}\vrule}c!{\color{numpexgray}\vrule}c!{\color{numpexgray}\vrule}}
        \rowcolor{numpexgray}
                        & \multicolumn{3}{c}{\color{white}\bf Finite element resolution}                                                  & {\color{white}\bf Reduced model} \\
        \rowcolor{numpexgray}
                        & \multicolumn{3}{c}{\color{white}\bf $T^\text{fem}(\mu)$}                                                        & {\color{white}\bf $T^{\text{rbm}, N}(\mu), \Delta_N(\mu)$} \\
        \rowcolor{numpexgray}
                        & {\color{white}\bf $P_1$} & {\color{white}\bf $P_2$ (\texttt{np=1})} & {\color{white}\bf $P_2$ (\texttt{np=12})} &  \\
        \rowcolor{numpexlightergray}
        Problem size    & $\mathcal{N} = 207~845$  & \multicolumn{2}{c|}{$\mathcal{N} = 1~580~932$}                                        & $N = 10$ \\
        $t_\text{exec}$ & \qty{5.534}{\second}     & \qty{62.432}{\second}                    & \qty{10.76}{\second}                      & \qty{2.88e-04}{\second}\\
        \rowcolor{numpexlightergray}
        speed-up        & 11.69                    & --                                        & 5.80                                      & \qty{2.17e5}{}\\
        \hline
    \end{tabular}
    }
    \caption{Times of execution of the finite element model for both $P_1$ and $P_2$ discretization against the computation time of the reduced solution and error bound.}
    \label{tab:feelpp:wp2:eye:rbm}
    \end{subtable}


    \begin{subtable}[b]{\textwidth}
    \centering
    \resizebox{\textwidth}{!}{\setlength{\parindent}{0pt}
    \def\arraystretch{1.25}
    \arrayrulecolor{numpexgray}
    \begin{tabular}{!{\color{numpexgray}\vrule}rl!{\color{numpexgray}\vrule}c!{\color{numpexgray}\vrule}c!{\color{numpexgray}\vrule}c!{\color{numpexgray}\vrule}c!{\color{numpexgray}\vrule}c!{\color{numpexgray}\vrule}c!{\color{numpexgray}\vrule}c!{\color{numpexgray}\vrule}c!{\color{numpexgray}\vrule}}
        \rowcolor{numpexgray}
        \multicolumn{2}{c}{\color{white}\bf Element computed} & \multicolumn{3}{c}{\color{white}\bf Computational time}  & \multicolumn{3}{c}{\color{white}\bf Speedup}\\
        \rowcolor{numpexgray}
        \multicolumn{2}{c}{}                                  & \color{white}\bf \texttt{np = 12} & \color{white}\bf \texttt{np = 64}   & \color{white}\bf \texttt{np = 128} & \color{white}\bf \texttt{np = 12} & \color{white}\bf \texttt{np = 64} & \color{white}\bf \texttt{np = 128} \\
        \rowcolor{numpexlightergray}
        Fine snapshots         & $T_h^\mathcal{N}$            & \qty{13.6}{\second}     & \qty{4.79}{\second} & \qty{2.38e+00}{\second} & --   & 2.84  & 5.71 \\
        Coarse snapshots       & $T_H^\mathcal{N}$            & \qty{2.77e+00}{\second} & \qty{1.10}{\second} & \qty{0.859}{\second}    & 4.90 & 12.36 & 15.83 \\
        \rowcolor{numpexlightergray}
        NIRB w/o rectification & $T_{Hh}^{N,\text{NoRect}}$   & \qty{3.190}{\second}    & \qty{1.309}{\second} & \qty{1.0687}{\second}   & 4.26 & 10.39 & 12.73 \\
        NIRB w/ rectification  & $T_{Hh}^{N,\text{Rect}}$     & \qty{3.203}{\second}    & \qty{1.311}{\second} & \qty{1.0691}{\second}   & 4.25 & 10.37 & 12.72 \\
        \hline
    \end{tabular}}
    \caption{Time of execution and associated speedup to run the NIRB method for the 3D eye-model with $P_2$ discretization, mean over 64 parameters.}
    \label{tab:feelpp:wp2:eye:nirb}
    \end{subtable}
    \caption{Execution times and speedup for the heat transfer problem in the eye using \ac{RB} and \ac{NIRB}.}
    \label{tab:feelpp:wp2:eye}
\end{table}



\paragraph{Challenges Identified:} \textbf{???}


\subsection{12-Month Roadmap}
\label{sec:WP2:Feelpp:roadmap}

In this section, describe the roadmap for improving benchmarks and addressing the challenges identified. This should include:
\begin{itemize}
    \item \textbf{Data Improvements:} Plans for improving input/output data management, including making datasets more accessible and ensuring reproducibility through open-data initiatives.
    \item \textbf{Methodology Application:} Implementation of the benchmarking methodology proposed in this deliverable to streamline reproducibility and dataset management.
    \item \textbf{Results Retention:} Plans to maintain benchmark results in a publicly accessible repository with appropriate metadata and documentation, ensuring long-term usability.
\end{itemize}

In~\cref{tab:WP2:Feelpp:bottlenecks}, we briefly discuss the bottleneck roadmap associated to the software and relevant to the work package.

\begin{table}[!ht]
    \centering
    {
        \setlength{\parindent}{0pt}
        \def\arraystretch{1.25}
        \arrayrulecolor{numpexgray}
        {
            \fontsize{9}{11}\selectfont
            \begin{tabular}{!{\color{numpexgray}\vrule}p{.25\linewidth}!{\color{numpexgray}\vrule}p{.6885\linewidth}!{\color{numpexgray}\vrule}}

    \rowcolor{numpexgray}{\rule{0pt}{2.5ex}\color{white}\bf Bottlenecks} &  {\rule{0pt}{2.5ex}\color{white}\bf Short Description }\\

\rowcolor{white}    B10 - Scientific Productivity & provide short description here \\
\rowcolor{numpexlightergray}    B11 - Reproducibility and Replicability of Computation & provide short description here \\
\rowcolor{white}    B12 - Pre/Post Processing and In-Situ Processing & provide short description here \\
\rowcolor{numpexlightergray}    B2 - Interconnect Technology & provide short description here \\
\rowcolor{white}    B6 - Data Management & provide short description here \\
\rowcolor{numpexlightergray}    B7 - Exascale Algorithms & provide short description here \\
\hline
\end{tabular}
        }
    }
    \caption{WP2: \Feelpp plan with Respect to Relevant Bottlenecks}
    \label{tab:WP2:Feelpp:bottlenecks}
\end{table}
\section{Software: Feel++}
\label{sec:WP2:Feel++:software}

\begin{itemize}
    \item \textbf{Contact Email(s):} vincent.chabannes@cemosis.fr, christophe.prudhomme@cemosis.fr
    \item \textbf{Supported Architecture(s):} CPU
    \item \textbf{Repository Link:} \href{https://github.com/feelpp/feelpp}{https://github.com/feelpp/feelpp}
\end{itemize}

\subsection{Software Overview}
\label{sec:WP2:Feel++:summary}

Provide a brief overview of the software with respect to WP2.

\subsection{Parallel Capabilities}
\label{sec:WP2:Feel++:performances}


\begin{itemize}
    \item describe the parallel programming  environment : MPI, OpenMP, CUDA, OpenACC, etc.
    \item describe the parallel computation environment: type of architecture and super computer used.
    \item describe the parallel capabilities of the software
    \item \textbf{Scalability:} Describe the general scalability properties of the software
    \item \textbf{Integration with Other Systems:} Describe how the software integrates with other numerical libraries and middleware in the Exa-MA framework.
\end{itemize}

\subsection{Initial Performance Metrics}
\label{sec:WP2:Feel++:metrics}

In this section, provide a summary of the initial performance metrics of the software with respect to WP2.
You can list one or more benchmarks and their associated results, the challenges , bottlenecks and the expectations for future versions of the software.



\subsubsection{Benchmark \#1}
\begin{itemize}
    \item describe the benchmark
    \item \textbf{Benchmarking Tools Used:} Describe the tools used for parallel benchmarking and the metrics mesured
    \item \textbf{Results Summary:} Results summary not available.
    \item \textbf{Challenges Identified:} No challenges identified.
\end{itemize}

\subsection{12 months roadmap}
\label{sec:WP2:Feel++:roadmap}

Describe the planned improvements  for the software in the context of WP2 and Exa-MA for the year to come that will be adressed in the next version of this deliverable.
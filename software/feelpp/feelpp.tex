%!TEX root = ../../exa-ma-d7.1.tex
\section{Software: Feel++}
\label{sec:Feelpp:software}

\begin{table}[h!]
    \centering
    { \setlength{\parindent}{0pt}
    \def\arraystretch{1.25}
    \arrayrulecolor{numpexgray}
    {\fontsize{9}{11}\selectfont
    \begin{tabular}{!{\color{numpexgray}\vrule}p{.4\textwidth}!{\color{numpexgray}\vrule}p{.6\textwidth}!{\color{numpexgray}\vrule}}
        \rowcolor{numpexgray}{\rule{0pt}{2.5ex}\color{white}\bf Field} & {\rule{0pt}{2.5ex}\color{white}\bf Details} \\
        \rowcolor{white}\textbf{Consortium} & \begin{tabular}{l}
\Feelpp Consortium\\
\end{tabular} \\
        \rowcolor{numpexlightergray}\textbf{Exa-MA Partners} & \begin{tabular}{l}
CNRS\\
Inria Grenoble\\
Unistra\\
\end{tabular} \\
        \rowcolor{white}\textbf{Contact Emails} & \begin{tabular}{l}
christophe.prudhomme@cemosis.fr\\
vincent.chabannes@cemosis.fr\\
\end{tabular} \\
        \rowcolor{numpexlightergray}\textbf{Supported Architectures} & \begin{tabular}{l}
CPU Only\\
\end{tabular} \\
        \rowcolor{white}\textbf{Repository} & \href{https://github.com/feelpp/feelpp}{https://github.com/feelpp/feelpp} \\
        \rowcolor{numpexlightergray}\textbf{License} & \begin{tabular}{l}
OSS:: GPL v*\\
OSS:: LGPL v*\\
\end{tabular} \\
        \rowcolor{white}\textbf{Bottlenecks roadmap} & \begin{tabular}{l}
B10 - Scientific Productivity\\
B11 - Reproducibility and Replicability of Computation\\
B12 - Pre/Post Processing and In-Situ Processing\\
B2 - Interconnect Technology\\
B6 - Data Management\\
B7 - Exascale Algorithms\\
\end{tabular} \\
        \bottomrule
    \end{tabular}
    }}
    \caption{\Feelpp Information}
\end{table}

\subsection{Software summary}
\label{sec:Feelpp:summary}
\Feelpp is a C++ framework for multiphysics simulations based on the finite element method. 
It is designed to handle complex simulations in a wide variety of fields, including fluid dynamics, electromagnetism, and solid mechanics. 
It focuses on high-performance computing and reproducibility in scientific computation.

\subsection{Purpose}
\label{sec:Feelpp:purpose}
\Feelpp aims to provide a flexible and efficient environment for conducting finite element analysis in multiple scientific domains. 
It allows for the rapid prototyping of numerical models and is built with parallel computing in mind to scale on modern HPC architectures.

\subsection{Programming and Computational Environment}
\label{sec::Feelpp:environment_capabilities}

The following table summarizes these aspects for \Feelpp, providing a view of its programming and computational capabilities.

\begin{table}[h!]
%%    \centering
    {
    \setlength{\parindent}{0pt}
    \def\arraystretch{1.25}
    \arrayrulecolor{numpexgray}
    {\fontsize{9}{11}\selectfont
    \begin{longtable}{lp{.3\textwidth}p{.5\textwidth}}
        \rowcolor{gray}\textbf{\color{white}Category} & \textbf{\color{white}Details} & \textbf{\color{white}Description} \\
        \hline
        \endfirsthead % End of the first head (the header for the first page)
        
        \hline
        \rowcolor{gray}\textbf{Category} & \textbf{Details} & \textbf{Description} \\
        \hline
        \endhead % End of the head (for all other pages)
        
        \hline
        \endfoot % Footer for all but the last page
        
        \hline
        \endlastfoot % Footer for the last page

        \rowcolor{white}Languages  & \begin{tabular}{l}
C++\\
C++17\\
C++20\\
Python\\
\end{tabular} & Feel++ is primarily developed in C++ and supports modern C++ standards, including C++17 and C++20, which provide enhanced performance, safety features, and modern programming paradigms for high-performance computing applications. This allows Feel++ to leverage advanced language features such as constexpr, parallel algorithms, and enhanced lambda expressions, improving both code maintainability and computational efficiency. Additionally, Feel++ integrates with Python, enabling scripting capabilities and facilitating ease of use for rapid prototyping, automation, and integration with scientific workflows. This dual-language support provides flexibility for both performance-critical tasks and user-friendly interfaces. \\
        \rowcolor{numpexlightergray}Parallelism  & \begin{tabular}{l}
MPI\\
Parallelism - C++\\
Task based\\
\end{tabular} & Feel++ offers robust support for parallel computing, making it suitable for high-performance simulations. It utilizes MPI (Message Passing Interface) for distributed memory parallelism, enabling efficient communication between processes across different nodes in HPC environments. Additionally, Feel++ implements parallelism directly in C++, allowing fine-grained control over threading and parallel execution within shared memory systems. The framework also supports task-based parallelism thanks to specx, enabling the efficient execution of independent computational tasks, improving load balancing, and optimizing resource usage across heterogeneous computing architectures. This parallelism support ensures scalability and high performance in complex multiphysics simulations.\\
        \rowcolor{white}Data Formats  & \begin{tabular}{l}
Data-management system\\
Ensight\\
Gmsh and associated formats\\
HDF5\\
JSON\\
VTK\\
YAML\\
in-house format\\
\end{tabular} & Feel++ supports data management, including remote data management systems such as Girder and GitHub. It handles a variety of input and output formats, including Ensight, Gmsh, HDF5, JSON, VTK, YAML, and an in-house format. This flexibility enables Feel++ to align with the \textbf{FAIR} principles (Findable, Accessible, Interoperable, and Reusable), ensuring efficient data sharing and reuse across different platforms and tools.\\
        \rowcolor{numpexlightergray}Resilience  & \begin{tabular}{l}
Checkpoint restart\\
\end{tabular} & Feel++ provides resilience support through checkpoint restart functionality. This allows simulations to save their state at specific points, enabling recovery from failures without restarting from the beginning. Binary in-house formats are used for fast restarts, ensuring minimal downtime and efficient continuation of long-running computations, especially on large-scale HPC systems.\\
        \rowcolor{white}DevOps & \begin{tabular}{l}
Continuous Benchmarking\\
Continuous Delivery\\
Continuous Integration\\
\end{tabular} & Outlines the development and operational practices including continuous integration, containerization, and testing methodologies.  \\
        \rowcolor{numpexlightergray}Packaging  & \begin{tabular}{l}
Debian\\
Fedora\\
Spack\\
Ubuntu\\
\end{tabular} & Feel++ is available in various packaging formats to ensure broad compatibility across different platforms. It is packaged for popular Linux distributions and HPC environments, with specific mirrors and repositories:\\
        \rowcolor{white}Testing  & \begin{tabular}{l}
Unit\\
Validation\\
Verification\\
\end{tabular} & Testing methodologies employed to ensure software quality and correctness.\\
        \rowcolor{numpexlightergray}Containerization  & \begin{tabular}{l}
Docker\\
Singularity\\
\end{tabular} & Container technologies used to package and deploy the software.\\
        \rowcolor{white}Interfaces  & \begin{tabular}{l}
Dymola/OpenModelica/FMU\\
HPdomain decomposition methods\\
MMG/ParMMG\\
OpenTurns\\
PETSc\\
Salome\\
\end{tabular} & List of software \Feelpp has interfaces with.\\
        \bottomrule
    \end{longtable}
    }}
    \caption{\Feelpp programming and computational environment}
\end{table}


\begin{longtable}{lp{0.3\textwidth}p{0.5\textwidth}}
        \rowcolor{gray}\textbf{\color{white}Category} & \textbf{\color{white}Details} & \textbf{\color{white}Description} \\
        \hline
        \endfirsthead % End of the first head (the header for the first page)
        
        \hline
        \rowcolor{gray}\textbf{Category} & \textbf{Details} & \textbf{Description} \\
        \hline
        \endhead % End of the head (for all other pages)
        
        \hline
        \endfoot % Footer for all but the last page
        
        \hline
        \endlastfoot % Footer for the last page
    
        \rowcolor{white}Languages  & C++, C++17, C++20, Python & Feel++ is primarily developed in C++ and supports modern C++ standards, including C++17 and C++20, which provide enhanced performance and safety features for high-performance computing applications. It also integrates with Python for scripting capabilities. \\
        
        \rowcolor{numpexlightergray}Parallelism  & MPI, Parallelism - C++17 and after, Task based - C++ & Feel++ offers robust support for parallel computing, utilizing MPI for distributed memory parallelism and implementing task-based parallelism to improve load balancing and resource usage. \\
        
        \rowcolor{white}Data Formats  & JSON, YAML, HDF5, VTK, in-house format & Feel++ supports various data formats for efficient data sharing and reuse, aligning with the FAIR principles. \\
        
        \rowcolor{numpexlightergray}Resilience  & Checkpoint restart & Provides checkpoint restart functionality for fault tolerance and recovery, ensuring minimal downtime on large-scale HPC systems. \\
        
        \rowcolor{white}DevOps & Continuous Integration, Continuous Delivery, Continuous Benchmarking & Outlines the development practices including CI/CD to maintain high software quality. \\
        
        \rowcolor{numpexlightergray}Packaging  & Debian, Fedora, Spack, Ubuntu & Available in multiple packaging formats for compatibility across different platforms. \\
        
        \rowcolor{white}Testing  & Unit, Validation, Verification & Various testing methodologies employed to ensure software quality and correctness. \\
        
        \rowcolor{numpexlightergray}Containerization  & Docker, Singularity & Technologies used to package and deploy the software for consistency across different environments. \\
        
        \rowcolor{white}Interfaces  & Dymola/OpenModelica/FMU,\\ hpddm\\ MMG/ParMMG\\
        OpenTurns\\
        PETSc\\
        Salome& Lists the software that Feel++ interfaces with for enhanced capabilities. \\

    \end{longtable}
    

\subsection{Mathematics}
\label{sec:Feelpp:mathematics}
\Feelpp is based on the finite element method (FEM), which is used for solving partial differential equations (PDEs) in complex geometries. 
It leverages advanced numerical techniques to ensure accuracy and scalability, including adaptive mesh refinement, domain decomposition, and error estimation.

\subsection{Relevant Publications}
\label{sec:Feelpp:publications}
Relevant publications include:
\begin{itemize}
    \item \fullcite{prudhomme_feelppfeelpp_2024}
    \item \fullcite{saigre_model_2024}
    \item \fullcite{van_landeghem_mathematical_2024}
    \item \fullcite{prudhomme_ktirio_2024}
\end{itemize}

\subsection{Acknowledgements}
\label{sec::Feelpp:acknowledgements}
The software has been developed with the support of the following funding agencies and institutions through various research projects: 
\begin{itemize}
   \item Université de Strasbourg
   \item CNRS
   \item ANR
   \item Région Grand Est
   \item AMIES
   \item European Commission
   \item EuroHPC JU
\end{itemize}

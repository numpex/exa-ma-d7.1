\section{Software: Feel++}
\label{sec:Feel++:software}



\begin{table}[h!]
    \centering
    { \setlength{\parindent}{0pt}
    \def\arraystretch{1.25}
    \arrayrulecolor{numpexgray}
    {\fontsize{9}{11}\selectfont
    \begin{tabular}{!{\color{numpexgray}\vrule}p{.4\textwidth}!{\color{numpexgray}\vrule}p{.6\textwidth}!{\color{numpexgray}\vrule}}
        \rowcolor{numpexgray}{\rule{0pt}{2.5ex}\color{white}\bf Field} & {\rule{0pt}{2.5ex}\color{white}\bf Details} \\
        \rowcolor{white}\textbf{Consortium} & \begin{tabular}{l}
Feel++ Consortium\\
\end{tabular} \\
        \rowcolor{numpexlightergray}\textbf{Exa-MA Partners} & \begin{tabular}{l}
CNRS\\
Inria Grenoble\\
Unistra\\
\end{tabular} \\
        \rowcolor{white}\textbf{Contact Emails} & \begin{tabular}{l}
christophe.prudhomme@cemosis.fr\\
vincent.chabannes@cemosis.fr\\
\end{tabular} \\
        \rowcolor{numpexlightergray}\textbf{Supported Architectures} & \begin{tabular}{l}
CPU Only\\
\end{tabular} \\
        \rowcolor{white}\textbf{Repository} & \href{https://github.com/feelpp/feelpp}{https://github.com/feelpp/feelpp} \\
        \rowcolor{numpexlightergray}\textbf{License} & \begin{tabular}{l}
OSS:: GPL v*\\
OSS:: LGPL v*\\
\end{tabular} \\
        \bottomrule
    \end{tabular}
    }}
    \caption{Feel++ Information}
\end{table}

\subsection{Software summary}
\label{sec:Feel++:summary}
Detailed overview not available.



\subsection{Purpose}
\label{sec:Feel++:purpose}
Purpose not available.

\subsection{Programming and Computational Environment}
\label{sec::Feel++:environment_capabilities}


The following table summarizes these aspects for Feel++, providing a  view of its programming and computational capabilities.

\begin{table}[h!]
    \centering
    {
    \setlength{\parindent}{0pt}
    \def\arraystretch{1.25}
    \arrayrulecolor{numpexgray}
    {\fontsize{9}{11}\selectfont
    \begin{tabular}{lp{.3\textwidth}p{.5\textwidth}}
        \rowcolor{numpexgray}{\rule{0pt}{2.5ex}\color{white}\bf Category}  & {\rule{0pt}{2.5ex}\color{white}\bf Details} & {\rule{0pt}{2.5ex}\color{white}\bf Description}\\
        \rowcolor{white}Languages  & \begin{tabular}{l}
C++\\
C++17\\
C++20\\
Python\\
\end{tabular} & Programming languages and language standards supported by the software \\
        \rowcolor{numpexlightergray}Parallelism  & \begin{tabular}{l}
MPI\\
Parallelism - C++\\
Task based\\
\end{tabular} & Parallel computing methods and frameworks utilized by the software.\\
        \rowcolor{white}Data Formats  & \begin{tabular}{l}
Data-management system\\
Ensight\\
Gmsh and associated formats\\
HDF5\\
JSON\\
VTK\\
VTK\\
YAML\\
in-house format\\
\end{tabular} & Data formats that the software can handle or produce.\\
        \rowcolor{numpexlightergray}Resilience  & \begin{tabular}{l}
Checkpoint restart\\
\end{tabular} & Fault tolerance and recovery mechanisms employed by the software.\\
        \rowcolor{white}DevOps & \begin{tabular}{l}
Continuous Benchmarking\\
Continuous Delivery\\
Continuous Integration\\
\end{tabular} & Outlines the development and operational practices including continuous integration, containerization, and testing methodologies.  \\
        \rowcolor{numpexlightergray}Packaging  & \begin{tabular}{l}
Debian\\
Fedora\\
Spack\\
Ubuntu\\
\end{tabular} & Software packaging and distribution.\\
        \rowcolor{white}Testing  & \begin{tabular}{l}
Unit\\
Validation\\
Verification\\
\end{tabular} & Testing methodologies employed to ensure software quality and correctness.\\
        \rowcolor{numpexlightergray}Containerization  & \begin{tabular}{l}
Docker\\
Singularity\\
\end{tabular} & Container technologies used to package and deploy the software.\\
        \rowcolor{white}Interfaces  & \begin{tabular}{l}
Dymola/OpenModelica/FMU\\
HPdomain decomposition methods\\
MMG/ParMMG\\
OpenTurns\\
PETSc\\
Salome\\
\end{tabular} & List of software Feel++ has interfaces with.\\
        \bottomrule
    \end{tabular}
    }}
    \caption{Feel++ programming and computational environment}
\end{table}



\subsection{Mathematics}
\label{sec:Feel++:mathematics}
Mathematics not available.

In this section, provide a summary the mathematics used in the software.


\subsection{Relevant Publications}
\label{sec:Feel++:publications}

Here is a list of relevant publications related to the software:


\subsection{Acknowledgements}
\label{sec::Feel++:acknowledgements}

The software has been developed with the support of the following funding agencies and institutions: 




Acknowledgements not available.



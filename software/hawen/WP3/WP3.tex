\section{Software: Hawen}
\label{sec:WP3:Hawen:software}

\begin{table}[h!]
    \centering
    { \setlength{\parindent}{0pt}
    \def\arraystretch{1.25}
    \arrayrulecolor{numpexgray}
    {\fontsize{9}{11}\selectfont
    \begin{tabular}{!{\color{numpexgray}\vrule}p{.4\textwidth}!{\color{numpexgray}\vrule}p{.6\textwidth}!{\color{numpexgray}\vrule}}
        \rowcolor{numpexgray}{\rule{0pt}{2.5ex}\color{white}\bf Field} & {\rule{0pt}{2.5ex}\color{white}\bf Details} \\
        \rowcolor{white}\textbf{Consortium} & \begin{tabular}{l}
Inria\\
\end{tabular} \\
        \rowcolor{numpexlightergray}\textbf{Exa-MA Partners} & \begin{tabular}{l}
Inria BXSO\\
\end{tabular} \\
        \rowcolor{white}\textbf{Contact Emails} & \begin{tabular}{l}
florian.faucher@inria.fr\\
\end{tabular} \\
        \rowcolor{numpexlightergray}\textbf{Supported Architectures} & \begin{tabular}{l}
CPU Only\\
\end{tabular} \\
        \rowcolor{white}\textbf{Repository} & \href{https://gitlab.com/ffaucher/hawen}{https://gitlab.com/ffaucher/hawen} \\
        \rowcolor{numpexlightergray}\textbf{License} & \begin{tabular}{l}
OSS:: GPL v*\\
\end{tabular} \\
        \rowcolor{white}\textbf{Bottlenecks roadmap} & \begin{tabular}{l}
B10 - Scientific Productivity\\
B11 - Reproducibility and Replicability of Computation\\
B6 - Data Management\\
B7 - Exascale Algorithms\\
\end{tabular} \\
        \bottomrule
    \end{tabular}
    }}
    \caption{WP3: Hawen Information}
\end{table}

\subsection{Software Overview}
\label{sec:WP3:Hawen:summary}

By considering time-harmonic wave problem, the discretization in 
\hawen~results in a large sparse linear system to solve. In this
we investigate the efficient resolution of the linear system arising
from the discretization of wave problems. In particular, \hawen~currently
relies on direct solver MUMPS for the resolution of the linear system.
MUMPS is a direct solver which uses LU factorization, the main advantage
is to enable the resolution for multiple right-hand sides, which a 
necessity in the context of inversion where several sources have to 
be simulated (e.g., possibly several thousands in Earth imaging). 
In addition, the matrix factorization can be reused in some operations
of inversion, typically for the gradient computation.

The direct linear solver MUMPS follows three steps with 1) the 
analysis phase, 2) the factorization, and 3) the solve. 
In addition to the multiple right-hand side features that acts 
at the solve phase, we wish to investigate how the specificity 
of the HDG method 
(working with the degrees of freedom on the faces of the discretized cells only)
used by \hawen~can also serve in the analysis phase to simplify 
the graph that is created by MUMPS.


In~\cref{tab:WP3:Hawen:features} we provide a summary of the software 
features relevant to the work package which are briefly discussed.

\begin{table}[h!]
    \centering
    { 
        \setlength{\parindent}{0pt}
        \def\arraystretch{1.25}
        \arrayrulecolor{numpexgray}
        {
            \fontsize{9}{11}\selectfont
            \begin{tabular}{!{\color{numpexgray}\vrule}p{.25\linewidth}!{\color{numpexgray}\vrule}p{.6885\linewidth}!{\color{numpexgray}\vrule}}
    
    \rowcolor{numpexgray}{\rule{0pt}{2.5ex}\color{white}\bf Features} &  {\rule{0pt}{2.5ex}\color{white}\bf Short Description }\\ 
    
\rowcolor{white}    "reuse of Krylov subspaces for multiple right-hand sides" & 
We currently use direct solver MUMPS to solve for multiple right-hand side and 
which to investigate the performance with extremely large number ( $>\num{10000}$). \\
\end{tabular}
        }
    }
    \caption{WP3: Hawen Features}
    \label{tab:WP3:Hawen:features}
\end{table}


\subsection{Parallel Capabilities}
\label{sec:WP3:Hawen:performances}

\hawen~uses MPI and OpenMP parallelism. The HDG method is particularly 
appropriate for parallelism (as other methods in the DG family), such 
that each cell of the mesh can be treated independently in parallel. 
\hawen~has been used on several supercomputers, including GENCI Adastra
with Genoa partition for CPUs parallelism.
\hawen~is linked with MUMPS library, which is a multifrontal direct solver 
for sparse linear systems. 


%\begin{itemize}
%    \item describe the parallel programming  environment : MPI, OpenMP, CUDA, OpenACC, etc.
%    \item describe the parallel computation environment: type of architecture and super computer used.
%    \item describe the parallel capabilities of the software
%    \item \textbf{Scalability:} Describe the general scalability properties of the software
%    \item \textbf{Integration with Other Systems:} Describe how the software integrates with other numerical libraries in the Exa-MA framework.
%\end{itemize}


\subsection{Initial Performance Metrics}
\label{sec:WP3:Hawen:metrics}


The benchmarks of this WP strongly relate with the 
ones of WP1 described in \cref{sec:WP1:Hawen:metrics}.
Therefore, we consider large-scale time-harmonic wave 
problem for acoustic and elastic propagation.
In this WP, we emphasize the consideration of many 
right-hand sides.


%This section provides a summary of initial performance benchmarks performed in the context of WP3. It ensures reproducibility by detailing input/output datasets, benchmarking tools, and the results. All data should be publicly available, ideally with a DOI for future reference.
%
%\begin{itemize}
%    \item \textbf{Overall Performance:} Summarize the software's computational performance, energy efficiency, and scalability results across different architectures (e.g., CPU, GPU, hybrid systems).
%    \item \textbf{Input/Output Dataset:} Provide a detailed description of the dataset used for the benchmark, including:
%        \begin{itemize}
%            \item Input dataset size, structure, and format (e.g., CSV, HDF5, NetCDF).
%            \item Output dataset format and key results.
%            \item Location of the dataset (e.g., GitHub repository, institutional repository, or open access platform).
%            \item DOI or permanent link for accessing the dataset.
%        \end{itemize}
%    \item \textbf{open-data Access:} Indicate whether the datasets used for the benchmark are open access, and provide a DOI or a direct link for download. Where applicable, highlight any licensing constraints.
%    \item \textbf{Challenges:} Identify any significant bottlenecks or challenges observed during the benchmarking process, including data handling and computational performance.
%    \item \textbf{Future Improvements:} Outline areas for optimization, including dataset handling, memory usage, or algorithmic efficiency, to address identified challenges.
%\end{itemize}

\subsubsection{Benchmark \#1: Visco-acoustic time-harmonic wave propagation}
\label{subsec:WP3:Hawen:benchmark1}

This benchmark follows the configuration of the one 
established in WP1 in \cref{subsec:WP1:Hawen:benchmark2} and
considers the propagation of acoustic waves in a large-scale 
3D medium.
However, contrary to WP1, we now emphasize the operations 
of linear algebra to solve the linear system resulting from
the HDG discretization in \hawen. 
We highlight two main objectives: 
\begin{enumerate}
\item Study the specificity of the HDG method to fasten the 
      analysis phase of the direct solver, in particular with
      the ``analysis by block'' feature of the direct solver 
      MUMPS.
\item Study the performance when a large number of right-hand 
      sides has to be solves. 
\end{enumerate}


%\begin{itemize}
%    \item \textbf{Description:} Briefly describe the benchmark case, including the problem size, target architecture (e.g., CPU, GPU), and the input data. Mention the specific goals of the benchmark (e.g., testing scalability, energy efficiency).
%    \item \textbf{Benchmarking Tools Used:} List the tools used for performance analysis, such as Extrae, Score-P, TAU, Vampir, or Nsight, and specify what metrics were measured (e.g., execution time, FLOPS, energy consumption).
%    \item \textbf{Input/Output Dataset Description:}
%        \begin{itemize}
%            \item \textbf{Input Data:} Describe the input dataset (size, format, data type) and provide a DOI or link to access it.
%            \item \textbf{Output Data:} Specify the structure of the results (e.g., memory usage, runtime logs) and how they can be accessed or replicated.
%            \item \textbf{Data Repository:} Indicate where the data is stored (e.g., Zenodo, institutional repository) and provide a DOI or URL for accessing the data.
%        \end{itemize}
%    \item \textbf{Results Summary:} Include a summary of key metrics (execution time, memory usage, FLOPS) and their comparison across architectures (e.g., CPU, GPU).
%    \item \textbf{Challenges Identified:} Describe any bottlenecks encountered (e.g., memory usage, parallelization inefficiencies) and how they impacted the benchmark.
%\end{itemize}

\subsubsection{Benchmark \#2: Visco-elastic time-harmonic wave propagation}
\label{subsec:WP3:Hawen:benchmark2}

This benchmark follows the previous one but in the context of 
elastic wave propagation. The configuration is 
established in WP1 in \cref{subsec:WP1:Hawen:benchmark2}.

\subsection{12-Month Roadmap}
\label{sec:WP3:Hawen:roadmap}
%
%In this section, describe the roadmap for improving benchmarks and addressing the challenges identified. This should include:
%\begin{itemize}
%    \item \textbf{Data Improvements:} Plans for improving input/output data management, including making datasets more accessible and ensuring reproducibility through open-data initiatives.
%    \item \textbf{Methodology Application:} Implementation of the benchmarking methodology proposed in this deliverable to streamline reproducibility and dataset management.
%    \item \textbf{Results Retention:} Plans to maintain benchmark results in a publicly accessible repository with appropriate metadata and documentation, ensuring long-term usability.
%\end{itemize}

We wish to improve input/output data management by including more options 
to save the results of \hawen, for instance this can include the hdf5 format
which is not allowed currently.

In~\cref{tab:WP3:Hawen:bottlenecks}, we briefly discuss the bottleneck roadmap associated to the software and relevant to the work package.

\begin{table}[h!]
    \centering
    
    

    \centering
    { 
        \setlength{\parindent}{0pt}
        \def\arraystretch{1.25}
        \arrayrulecolor{numpexgray}
        {
            \fontsize{9}{11}\selectfont
            \begin{tabular}{!{\color{numpexgray}\vrule}p{.25\linewidth}!{\color{numpexgray}\vrule}p{.6885\linewidth}!{\color{numpexgray}\vrule}}
    
    \rowcolor{numpexgray}{\rule{0pt}{2.5ex}\color{white}\bf Bottlenecks} &  {\rule{0pt}{2.5ex}\color{white}\bf Short Description }\\ 
    
\rowcolor{white}    B10 - Scientific Productivity & provide short description here \\
\rowcolor{numpexlightergray}    B11 - Reproducibility and Replicability of Computation & provide short description here \\
\rowcolor{white}    B6 - Data Management & provide short description here \\
\rowcolor{numpexlightergray}    B7 - Exascale Algorithms & provide short description here \\
\end{tabular}
        }
    }
    \caption{WP3: Hawen plan with Respect to Relevant Bottlenecks}
    \label{tab:WP3:Hawen:bottlenecks}
\end{table}

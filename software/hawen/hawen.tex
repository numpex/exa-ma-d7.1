\section{Software: Hawen}
\label{sec:Hawen:software}



\begin{table}[h!]
    \centering
    { \setlength{\parindent}{0pt}
    \def\arraystretch{1.25}
    \arrayrulecolor{numpexgray}
    {\fontsize{9}{11}\selectfont
    \begin{tabular}{!{\color{numpexgray}\vrule}p{.4\textwidth}!{\color{numpexgray}\vrule}p{.6\textwidth}!{\color{numpexgray}\vrule}}
        \rowcolor{numpexgray}{\rule{0pt}{2.5ex}\color{white}\bf Field} & {\rule{0pt}{2.5ex}\color{white}\bf Details} \\
        \rowcolor{white}\textbf{Consortium} & \begin{tabular}{l}
Inria\\
\end{tabular} \\
        \rowcolor{numpexlightergray}\textbf{Exa-MA Partners} & \begin{tabular}{l}
Inria BXSO\\
\end{tabular} \\
        \rowcolor{white}\textbf{Contact Emails} & \begin{tabular}{l}
florian.faucher@inria.fr\\
\end{tabular} \\
        \rowcolor{numpexlightergray}\textbf{Supported Architectures} & \begin{tabular}{l}
CPU Only\\
\end{tabular} \\
        \rowcolor{white}\textbf{Repository} & \href{https://gitlab.com/ffaucher/hawen}{https://gitlab.com/ffaucher/hawen} \\
        \rowcolor{numpexlightergray}\textbf{License} & \begin{tabular}{l}
OSS:: GPL v*\\
\end{tabular} \\
        \rowcolor{white}\textbf{Bottlenecks roadmap} & \begin{tabular}{l}
B10 - Scientific Productivity\\
B11 - Reproducibility and Replicability of Computation\\
B6 - Data Management\\
B7 - Exascale Algorithms\\
\end{tabular} \\
        \bottomrule
    \end{tabular}
    }}
    \caption{Hawen Information}
\end{table}

% ---------------------------------------------
\newcommand{\hawen}{\textsc{Hawen}}
\subsection{Software summary}
\label{sec:Hawen:summary}
% ---------------------------------------------

Software \hawen~(\url{https://ffaucher.gitlab.io/hawen-website/})
considers the time-harmonic modeling of mechanical waves, and the 
associated quantitative inverse wave problems.
The code uses the Hybridizable Discontinuous Galerkin method for the 
discretization. 
It relies on iterative minimization algorithm to solve the quantitative
inverse wave problem via nonlinear optimization.



\subsection{Purpose}
\label{sec:Hawen:purpose}

The major applications of \hawen~are to solve large-scale inverse 
problems, in particular in the context of Earth's imaging and 
helioseismology.


\subsection{Programming and Computational Environment}
\label{sec::Hawen:environment_capabilities}


The following table~\ref{table:hawen-environment} summarizes these aspects for \hawen, providing a  view of its programming and computational capabilities.

\begin{table}[h!]
    \centering
    {
    \setlength{\parindent}{0pt}
    \def\arraystretch{1.25}
    \arrayrulecolor{numpexgray}
    {\fontsize{9}{11}\selectfont
    \begin{tabular}{lp{.3\textwidth}p{.5\textwidth}}
        \rowcolor{numpexgray}{\rule{0pt}{2.5ex}\color{white}\bf Category}  & {\rule{0pt}{2.5ex}\color{white}\bf Details} & {\rule{0pt}{2.5ex}\color{white}\bf Description}\\
        \rowcolor{white}Languages  & \begin{tabular}{l}
Fortran\\
\end{tabular} & Programming languages and language standards supported by the software \\
        \rowcolor{numpexlightergray}Parallelism  & \begin{tabular}{l}
MPI\\
Multithread\\
\end{tabular} & Parallel computing methods and frameworks utilized by the software.\\
        \rowcolor{white}Data Formats  & \begin{tabular}{l}
Gmsh and associated formats\\
VTK\\
in-house binary format\\
\end{tabular} & Data formats that the software can handle or produce.\\
        \rowcolor{numpexlightergray}Resilience  & \begin{tabular}{l}
None\\
\end{tabular} & Fault tolerance and recovery mechanisms employed by the software.\\
        \rowcolor{white}DevOps & \begin{tabular}{l}
None\\
\end{tabular} & Outlines the development and operational practices including continuous integration, containerization, and testing methodologies.  \\
        \rowcolor{numpexlightergray}Packaging  & \begin{tabular}{l}
None\\
\end{tabular} & Software packaging and distribution.\\
        \rowcolor{white}Testing  & \begin{tabular}{l}
Analytic solutions\\
\end{tabular} & Testing methodologies employed to ensure software quality and correctness.\\
        \rowcolor{numpexlightergray}Containerization  & \begin{tabular}{l}
None\\
\end{tabular} & Container technologies used to package and deploy the software.\\
        \rowcolor{white}Interfaces  & \begin{tabular}{l}
MUMPS\\ 
Metis\\ 
ARPACK (optional)\\
PARPACK (optional)\\
\end{tabular} & List of software Hawen has interfaces with.\\
        \bottomrule
    \end{tabular}
    }}
    \caption{Hawen programming and computational environment}
    \label{table:hawen-environment}
\end{table}



\subsection{Mathematics}
\label{sec:Hawen:mathematics}
Mathematics not available.

In this section, provide a summary the mathematics used in the software.


\subsection{Relevant Publications}
\label{sec:Hawen:publications}

Here is a list of relevant publications related to the software:
\begin{itemize}
\item \cite{Hawen2021}: Reference of the software in the journal of open-source software;
\item \cite{Faucher2020adjoint}:
      Mathematical details of the adjoint-state method in the framework 
      of Hybridizable Discontinuous Galerkin discretization method.
      It provides the computational steps for the implementation of the 
      inverse problem.
\item \cite{Pham2024stabilization}: Details of the numerical implementation 
      of the HDG method for anisotropic elasticity.
\item \cite{Faucher2019FRgWIGeo,Faucher2020DAS}: 
      Use of the software in the context of seismic imaging.
\item \cite{Pham2020Siam,Pham2024assembling}:
      Use of the software in the context of helioseismology.
\item \cite{Faucher2023viscoacoustic}: 
      Use of the software in the context of viscoacoustics ultrasound imaging.
\item \cite{Liu2024,Benitez2024}: 
      Use of the software in the context of data science and
      for benchmarks.
\end{itemize}


\subsection{Acknowledgements}
\label{sec::Hawen:acknowledgements}

The software has been developed with the support of the following funding agencies and institutions: 

\begin{itemize}
  \item Since 2021, F. Faucher is part of the team Makutu of INRIA Bordeaux, at the 
                    University of Pau and Pays de l'Adour.
  \item 2024--2029, F. Faucher acknowledges support of the European Research Council 
                    with ERC-StG Project INCORWAVE -- grant 101116288.
  \item 2019--2021, F. Faucher acknowledges funding by the Austrian Science Fund (FWF) 
        under the Lise Meitner grant allocation M2791-N.
\end{itemize}



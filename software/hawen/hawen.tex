\section{Software: Hawen}
\label{sec:Hawen:software}



\begin{table}[h!]
    \centering
    { \setlength{\parindent}{0pt}
    \def\arraystretch{1.25}
    \arrayrulecolor{numpexgray}
    {\fontsize{9}{11}\selectfont
    \begin{tabular}{!{\color{numpexgray}\vrule}p{.4\textwidth}!{\color{numpexgray}\vrule}p{.6\textwidth}!{\color{numpexgray}\vrule}}
        \rowcolor{numpexgray}{\rule{0pt}{2.5ex}\color{white}\bf Field} & {\rule{0pt}{2.5ex}\color{white}\bf Details} \\
        \rowcolor{white}\textbf{Consortium} & \begin{tabular}{l}
Inria\\
\end{tabular} \\
        \rowcolor{numpexlightergray}\textbf{Exa-MA Partners} & \begin{tabular}{l}
Inria BXSO\\
\end{tabular} \\
        \rowcolor{white}\textbf{Contact Emails} & \begin{tabular}{l}
florian.faucher@inria.fr\\
\end{tabular} \\
        \rowcolor{numpexlightergray}\textbf{Supported Architectures} & \begin{tabular}{l}
CPU Only\\
\end{tabular} \\
        \rowcolor{white}\textbf{Repository} & \href{https://gitlab.com/ffaucher/hawen}{https://gitlab.com/ffaucher/hawen} \\
        \rowcolor{numpexlightergray}\textbf{License} & \begin{tabular}{l}
OSS:: GPL v*\\
\end{tabular} \\
        \rowcolor{white}\textbf{Bottlenecks roadmap} & \begin{tabular}{l}
B10 - Scientific Productivity\\
B11 - Reproducibility and Replicability of Computation\\
B6 - Data Management\\
B7 - Exascale Algorithms\\
\end{tabular} \\
        \bottomrule
    \end{tabular}
    }}
    \caption{Hawen Information}
\end{table}

% ---------------------------------------------
\newcommand{\hawen}{\textsc{Hawen}}
\subsection{Software summary}
\label{sec:Hawen:summary}
% ---------------------------------------------

Software \hawen~(\url{https://ffaucher.gitlab.io/hawen-website/})
considers the time-harmonic modeling of mechanical waves, and the 
associated quantitative inverse wave problems.
The code uses the Hybridizable Discontinuous Galerkin method (HDG) 
for the discretization. 
It relies on nonlinear iterative minimization algorithm to solve 
the quantitative inverse wave problem.
The code is written in \texttt{Fortran90} and uses \texttt{mpi}
and \texttt{OpenMP} parallelism.



\subsection{Purpose}
\label{sec:Hawen:purpose}

\hawen~solves large-scale inverse wave problems in the frequency domain, 
with an emphasis on applications in the context of Earth's imaging and helioseismology.


\subsection{Programming and Computational Environment}
\label{sec::Hawen:environment_capabilities}


The following table~\ref{table:hawen-environment} summarizes these aspects for \hawen, providing a  view of its programming and computational capabilities.

\begin{table}[h!]
    \centering
    {
    \setlength{\parindent}{0pt}
    \def\arraystretch{1.25}
    \arrayrulecolor{numpexgray}
    {\fontsize{9}{11}\selectfont
    \begin{tabular}{lp{.3\textwidth}p{.5\textwidth}}
        \rowcolor{numpexgray}{\rule{0pt}{2.5ex}\color{white}\bf Category}  & {\rule{0pt}{2.5ex}\color{white}\bf Details} & {\rule{0pt}{2.5ex}\color{white}\bf Description}\\
        \rowcolor{white}Languages  & \begin{tabular}{l}
Fortran\\
\end{tabular} & Programming languages and language standards supported by the software \\
        \rowcolor{numpexlightergray}Parallelism  & \begin{tabular}{l}
MPI\\
Multithread\\
\end{tabular} & Parallel computing methods and frameworks utilized by the software.\\
        \rowcolor{white}Data Formats  & \begin{tabular}{l}
Gmsh and associated formats\\
VTK\\
in-house binary format\\
\end{tabular} & Data formats that the software can handle or produce.\\
        \rowcolor{numpexlightergray}Resilience  & \begin{tabular}{l}
None\\
\end{tabular} & Fault tolerance and recovery mechanisms employed by the software.\\
        \rowcolor{white}DevOps & \begin{tabular}{l}
None\\
\end{tabular} & Outlines the development and operational practices including continuous integration, containerization, and testing methodologies.  \\
        \rowcolor{numpexlightergray}Packaging  & \begin{tabular}{l}
None\\
\end{tabular} & Software packaging and distribution.\\
        \rowcolor{white}Testing  & \begin{tabular}{l}
Analytic solutions\\
\end{tabular} & Testing methodologies employed to ensure software quality and correctness.\\
        \rowcolor{numpexlightergray}Containerization  & \begin{tabular}{l}
None\\
\end{tabular} & Container technologies used to package and deploy the software.\\
        \rowcolor{white}Interfaces  & \begin{tabular}{l}
MUMPS\\ 
Metis\\ 
ARPACK (optional)\\
PARPACK (optional)\\
\end{tabular} & List of software Hawen has interfaces with.\\
        \bottomrule
    \end{tabular}
    }}
    \caption{Hawen programming and computational environment}
    \label{table:hawen-environment}
\end{table}



\subsection{Mathematics}
\label{sec:Hawen:mathematics}

\paragraph{Forward problem} \hawen~allows for the modeling of time-harmonic
mechanical waves for different types of media. 
By using the HDG method for the discretization, we work with first-order systems.
In the case of visco-acoustics wave propagation, the forward problem consists in
finding the scalar pressure field $p$ and vector velocity field $\boldsymbol{v}$ 
such that
\begin{subequations}\label{eq:hawen:viscoacoustic}
\begin{empheq}[left={\empheqlbrace}]{align}
 &-\mathrm{i}\omega \rho \boldsymbol{v} \,+\, \nabla p \,=\, 0 \,, \\
 &-\dfrac{\mathrm{i}\omega}{\kappa} p \,+\, \nabla\cdot\boldsymbol{v} \,=\, f \,,
\end{empheq}\end{subequations}
where $\omega$ is the angular frequency, and the medium is characterized
by the density $\rho$ and the bulk modulus $\kappa$. The source term is 
$f$. 
We further refer to \cite{Faucher2020adjoint,Faucher2023viscoacoustic}
for more details.
The code allows for different types of boundary conditions, such as 
imposing the Dirichlet trace, Neumann trace, or a Robin-type condition
(e.g., for radiation condition).


In the case of linear visco-elastic wave propagation, the forward problem consists in
finding the displacement field $\boldsymbol{u}$ and stress tensor $\boldsymbol{\sigma}$ 
solutions to, cf., e.g., \cite{Pham2024stabilization},
\begin{subequations}\begin{empheq}[left={\empheqlbrace}]{align}
 &-\omega^2\rho\boldsymbol{u} \,-\, \nabla\cdot\boldsymbol{\sigma}\,=\,\boldsymbol{f} \,, \\
 & \boldsymbol{\sigma} \,=\, \boldsymbol{C} \, \big(\nabla\boldsymbol{u} \,+\, (\nabla\boldsymbol{u})^t\big) \,.
\end{empheq}\end{subequations}
Here, the medium is characterized by the density 
$\rho$ and the stiffness tensor $\boldsymbol{C}$,
we refer to \cite{Pham2024stabilization}
for more details related to HDG implementation.

In the context of helioseismology, we consider the scalar-wave propagation 
which comes from simplification of the Galbrun's equation, 
cf.~\cite{Pham2020Siam,Pham2024assembling}, in three dimensions, the 
simplest form corresponds to finding $u$ solution to: 
\begin{equation} \label{eq:hawen:helio-scalar}
  \Big( \,-\nabla \,-\,\dfrac{\omega^2}{c^2} \,+\, q \Big) \, u \,=\, f\,,
\end{equation}
with potential $q$, sound-speed $c$ and source term $f$. 
In addition, \hawen~solves the problem under the assumption of 
spherical symmetry and we refer to \cite{Pham2020Siam} for more 
details. 
Several choices of radiation boundary conditions have been 
implemented and tested, cf.~\cite{Pham2019radiationBC,Pham2020Siam}.
Recently, more complete equations to model the Sun have 
been implemented in \hawen, see \cite{Pham2024assembling}.

\paragraph{Inversion} 
\hawen~solves the inverse problems for the reconstruction of the 
model parameters from the measurements of waves. For instance, 
considering the acoustic wave equation \eqref{eq:hawen:viscoacoustic},
it consists in finding the density $\rho$ and bulk modulus $\kappa$
from the observations of the pressure $p$ and/or velocity field $\boldsymbol{v}$
at a discrete set of positions.
\hawen~employs a nonlinear iterative minimization algorithm for the 
reconstruction of the properties, as illustrated in 
\cite{Hawen2021,Faucher2019FRgWIGeo,Faucher2020DAS,Faucher2023viscoacoustic}.
This approach is typically referred to as the \emph{Full Waveform Inversion}
in the context of seismic imaging, cf. \cite{Virieux2009} for a review.


\subsection{Relevant Publications}
\label{sec:Hawen:publications}

Here is a list of relevant publications related to the software:
\begin{itemize}
\item \cite{Hawen2021}: Reference of the software in the journal of open-source software;
\item \cite{Faucher2020adjoint}:
      Mathematical details of the adjoint-state method in the framework 
      of Hybridizable Discontinuous Galerkin discretization method.
      It provides the computational steps for the implementation of the 
      inverse problem.
\item \cite{Pham2024stabilization}: Details of the numerical implementation 
      of the HDG method for anisotropic elasticity.
\item \cite{Faucher2019FRgWIGeo,Faucher2020DAS}: 
      Use of the software in the context of seismic imaging.
\item \cite{Pham2020Siam,Pham2024assembling}:
      Use of the software in the context of helioseismology.
\item \cite{Faucher2023viscoacoustic}: 
      Use of the software in the context of viscoacoustics ultrasound imaging.
\item \cite{Liu2024,Benitez2024}: 
      Use of the software in the context of data science and
      for benchmarks.
\end{itemize}


\subsection{Acknowledgements}
\label{sec::Hawen:acknowledgements}

The software has been developed with the support of the following funding agencies and institutions: 

\begin{itemize}
  \item Since 2021, F. Faucher is part of the team Makutu of INRIA Bordeaux, at the 
                    University of Pau and Pays de l'Adour.
  \item 2024--2029, F. Faucher acknowledges support of the European Research Council 
                    with ERC-StG Project INCORWAVE -- grant 101116288.
  \item 2019--2021, F. Faucher acknowledges funding by the Austrian Science Fund (FWF) 
        under the Lise Meitner grant allocation M2791-N.
\end{itemize}



\section{Software: TRUST Platform}
\label{sec:WP3:TRUST Platform:software}

\begin{table}[h!]
    \centering
    { \setlength{\parindent}{0pt}
    \def\arraystretch{1.25}
    \arrayrulecolor{numpexgray}
    {\fontsize{9}{11}\selectfont
    \begin{tabular}{!{\color{numpexgray}\vrule}p{.4\textwidth}!{\color{numpexgray}\vrule}p{.6\textwidth}!{\color{numpexgray}\vrule}}
        \rowcolor{numpexgray}{\rule{0pt}{2.5ex}\color{white}\bf Field} & {\rule{0pt}{2.5ex}\color{white}\bf Details} \\
        \rowcolor{white}\textbf{Consortium} & \begin{tabular}{l}
CEA\\
\end{tabular} \\
        \rowcolor{numpexlightergray}\textbf{Exa-MA Partners} & \begin{tabular}{l}
CEA\\
\end{tabular} \\
        \rowcolor{white}\textbf{Contact Emails} & \begin{tabular}{l}
pierre.ledac@cea.fr\\
\end{tabular} \\
        \rowcolor{numpexlightergray}\textbf{Supported Architectures} & \begin{tabular}{l}
CPU Only\\
CPU and GPU\\
GPU Only\\
\end{tabular} \\
        \rowcolor{white}\textbf{Repository} & \href{https://github.com/cea-trust-platform}{https://github.com/cea-trust-platform} \\
        \rowcolor{numpexlightergray}\textbf{License} & \begin{tabular}{l}
OSS::\\
OSS::BSD\\
\end{tabular} \\
        \rowcolor{white}\textbf{Bottlenecks roadmap} & \begin{tabular}{l}
B10 - Scientific Productivity\\
B11 - Reproducibility and Replicability of Computation\\
B6 - Data Management\\
B7 - Exascale Algorithms\\
\end{tabular} \\
        \bottomrule
    \end{tabular}
    }}
    \caption{WP3: TRUST Platform Information}
\end{table}

\subsection{Software Overview}
\label{sec:WP3:TRUST Platform:summary}

In~\cref{tab:WP3:TRUST Platform:features} we provide a summary of the software features relevant to the work package which are briefly discussed.

\begin{table}[h!]
    \centering
    { 
        \setlength{\parindent}{0pt}
        \def\arraystretch{1.25}
        \arrayrulecolor{numpexgray}
        {
            \fontsize{9}{11}\selectfont
            \begin{tabular}{!{\color{numpexgray}\vrule}p{.25\linewidth}!{\color{numpexgray}\vrule}p{.6885\linewidth}!{\color{numpexgray}\vrule}}
    
    \rowcolor{numpexgray}{\rule{0pt}{2.5ex}\color{white}\bf Features} &  {\rule{0pt}{2.5ex}\color{white}\bf Short Description }\\ 
    
\rowcolor{white}    direct solver & MUMPS on CPU and STRUMPACK on GPU are used in TRUST \\
\rowcolor{numpexlightergray}    krylov solver & PETSc on CPU, AmgX or PETSc on Nvidia GPU, rocALUTION or PETSc on AMD GPU are used in TRUST \\
\rowcolor{white}    multiphysics coupling & TRUST is intended to be coupled with other software (structure solver for instance) through the ICoCo standard (\href{https://github.com/cea-trust-platform/icoco-coupling}{https://github.com/cea-trust-platform/icoco-coupling}), as a reference component whose performance to preserve in the partitioned coupling  \\
\end{tabular}
        }
    }
    \caption{WP3: TRUST Platform Features}
    \label{tab:WP3:TRUST Platform:features}
\end{table}


\subsection{Parallel Capabilities}
\label{sec:WP3:TRUST Platform:performances}


\begin{itemize}
    \item The parallel programming model is MPI and a mix of Kokkos and OpenMP-target directives for GPU computing  
    \item The parallel computation environment are personal computers, clusters, and super computers: e.g. Adastra (CINES), Jean-Zay (IDRIS), Topaze (CCRT)
    %\item describe the parallel capabilities of the software
    %\item \textbf{Scalability:} Describe the general scalability properties of the software
    \item \textbf{Integration with Other Systems:} TRUST integrates with the linear algebra libraries developed and improved in the Exa-Ma framework and especially those interfaced through PETSc. It thus serves for the practical evaluation of the integral performance gain provided by the project in representative physical cases of interest.
\end{itemize}


\subsection{Initial Performance Metrics}
\label{sec:WP3:TRUST Platform:metrics}

Establishing reference metrics for TRUST to serve as basic data for the methodology proposed in chapter \ref{chap:methodology} is an ongoing process that will continue during the upcoming year. However, some solid elements are available concerning the relevant tests to be implemented, emphasizing on the relation between TRUST and linear algebra libraries supported in WP3, and with PETSc in particuler:

\begin{itemize}
    \item \textbf{Weak scaling test for GPU: } using Krylov solver (ex. Hypre) through PETSc on 2 to 1024 GPUs
    \item\textbf{Weak scaling test for CPU: } using Krylov multigrid solver through PETSc on 8 to 32768 CPUs
    \item \textbf{Strong scaling for CPU: } using MUMPS direct solver through PETSc on 1 to 50 cores.
\end{itemize}

Current observations state that:

\begin{itemize}
    \item \textbf{Weak scaling test for GPU: } the scalability can be improved, in particular with :
    \begin{itemize}
        \item the fine tuning of the already tested libraries or the choice of new libraries, with a particular focus on the convergence rate,
        \item the improvement of the robustness of the MPI communications between GPUs.
    \end{itemize}
    \item\textbf{Weak scaling test for CPU: } the scalability already reaches satisfactory levels, but some improvement can yet be expected from the advanced management of MPI communications with a focus on collective communications.
    \item \textbf{Strong scaling for CPU: } these tests are great interest for component applications (i.e. complex physics, smaller models) that can be involved in multiphysics and partitioned coupling.
\end{itemize}

Tests for performance evaluation of the hybrid CPU-GPU run modes will be considered in a second step.

\subsection{12-Month Roadmap}
\label{sec:WP3:TRUST Platform:roadmap}

The 12-Month Roadmap for TRUST platform is mainly dedicated to the implementation of tests identified in the previous section in the methodology proposed in section \ref{sec:methodology-types}, in terms of scalability analysis for CPU or GPU in a first step, and of hybrid benchmarking for CPU-GPU in a second step (post-12 month perspective).\\
\\
Practically:
\begin{itemize}
    \item \textbf{Data availability:} Data for the selected benchmarks will be made available in a format readable with open-source only software, namely from the Salome environment (\href{https://www.salome-platform.org/}{https://www.salome-platform.org/}).
    \item \textbf{Methodology Application:} Implementation of the benchmarking methodology in terms of scalability measurements, with a particular focus on the selection of the relevant output data and associated tolerance for the reproducibility guarantee.
    \item \textbf{Results Retention:} Benchmark results will be stored and made available with the suitable level of metadata and documentation through a dedicated repository proposed by the Exa-MA project.
\end{itemize}

In~\cref{tab:WP3:TRUST Platform:bottlenecks}, we briefly discuss the bottleneck roadmap associated to the software and relevant to the work package.

\begin{table}[h!]
    \centering
    
    

    \centering
    { 
        \setlength{\parindent}{0pt}
        \def\arraystretch{1.25}
        \arrayrulecolor{numpexgray}
        {
            \fontsize{9}{11}\selectfont
            \begin{tabular}{!{\color{numpexgray}\vrule}p{.25\linewidth}!{\color{numpexgray}\vrule}p{.6885\linewidth}!{\color{numpexgray}\vrule}}
    
    \rowcolor{numpexgray}{\rule{0pt}{2.5ex}\color{white}\bf Bottlenecks} &  {\rule{0pt}{2.5ex}\color{white}\bf Short Description }\\ 
    
\rowcolor{white}    B10 - Scientific Productivity & Accelerate the access to large scale numerical results to enhance knowledge and speed-up engineering operations in the field of computational fluid mechanics. \\
\rowcolor{numpexlightergray}    B11 - Reproducibility and Replicability of Computation & Ensure the reliability of the results for large scale and complex simulation independently from the computer and the selected run-mode (CPU, GPU, hybrid). \\
\rowcolor{white}    B6 - Data Management & This bottleneck is indirectly addressed for TRUST through the availability of the benchmark datasets and the collection of the results. \\
\rowcolor{numpexlightergray}    B7 - Exascale Algorithms & Benefit from the latest improvement in linear algebra libraries, quantify the practical gains on representative applications and track/characterize the remaining bottlenecks for the global performance. Use TRUST as a reference component for the analysis of the performance of partitioned coupling at exascale.\\
\end{tabular}
        }
    }
    \caption{WP3: TRUST Platform plan with Respect to Relevant Bottlenecks}
    \label{tab:WP3:TRUST Platform:bottlenecks}
\end{table}

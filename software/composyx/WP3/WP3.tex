\section{Software: Composyx}
\label{sec:WP3:Composyx:software}

\begin{table}[h!]
    \centering
    { \setlength{\parindent}{0pt}
    \def\arraystretch{1.25}
    \arrayrulecolor{numpexgray}
    {\fontsize{9}{11}\selectfont
    \begin{tabular}{!{\color{numpexgray}\vrule}p{.4\textwidth}!{\color{numpexgray}\vrule}p{.6\textwidth}!{\color{numpexgray}\vrule}}
        \rowcolor{numpexgray}{\rule{0pt}{2.5ex}\color{white}\bf Field} & {\rule{0pt}{2.5ex}\color{white}\bf Details} \\
        \rowcolor{white}\textbf{Consortium} & \begin{tabular}{l}
None\\
\end{tabular} \\
        \rowcolor{numpexlightergray}\textbf{Exa-MA Partners} & \begin{tabular}{l}
Inria BXSO\\
\end{tabular} \\
        \rowcolor{white}\textbf{Contact Emails} & \begin{tabular}{l}
gilles.marait@inria.fr\\
\end{tabular} \\
        \rowcolor{numpexlightergray}\textbf{Supported Architectures} & \begin{tabular}{l}
CPU Only\\
\end{tabular} \\
        \rowcolor{white}\textbf{Repository} & \href{https://gitlab.inria.fr/composyx/composyx}{https://gitlab.inria.fr/composyx/composyx} \\
        \rowcolor{numpexlightergray}\textbf{License} & \begin{tabular}{l}
OSS: Cecill-*\\
\end{tabular} \\
        \rowcolor{white}\textbf{Bottlenecks roadmap} & \begin{tabular}{l}
B10 - Scientific Productivity\\
B11 - Reproducibility and Replicability of Computation\\
B6 - Data Management\\
B7 - Exascale Algorithms\\
\end{tabular} \\
        \bottomrule
    \end{tabular}
    }}
    \caption{WP3: Composyx Information}
\end{table}

\subsection{Software Overview}
\label{sec:WP3:Composyx:summary}

In~\cref{tab:WP3:Composyx:features} we provide a summary of the software features relevant to the work package which are briefly discussed.

\begin{table}[h!]
    \centering
    { 
        \setlength{\parindent}{0pt}
        \def\arraystretch{1.25}
        \arrayrulecolor{numpexgray}
        {
            \fontsize{9}{11}\selectfont
            \begin{tabular}{!{\color{numpexgray}\vrule}p{.25\linewidth}!{\color{numpexgray}\vrule}p{.6885\linewidth}!{\color{numpexgray}\vrule}}
    
    \rowcolor{numpexgray}{\rule{0pt}{2.5ex}\color{white}\bf Features} &  {\rule{0pt}{2.5ex}\color{white}\bf Short Description }\\ 
    
\rowcolor{white}    "singular value decomposition (SVD) and eigenvalue solver" & Provide randomized EVD and SVD partial decomposition \\
\rowcolor{numpexlightergray}    direct solver & provide interface to MUMPS, PaStiX and qr\_mumps \\
\rowcolor{white}    krylov solver & provide interface to Fabulous that implement various subspace methods and their block-counterpart \\
\end{tabular}
        }
    }
    \caption{WP3: Composyx Features}
    \label{tab:WP3:Composyx:features}
\end{table}


\subsection{Parallel Capabilities}
\label{sec:WP3:Composyx:performances}


\begin{itemize}
    \item describe the parallel programming  environment : MPI+ threads and MPI+StarPU for heterogeneous manycores.
    \item describe the parallel computation environment: distributed manycores
    %\item describe the parallel capabilities of the software
    \item \textbf{Scalability:} weak scalability on up-to $\approx$ 20~000 cores for the solution of a $\approx 10^9$ linear system. 
    \item \textbf{Integration with Other Systems:} No integration into other Exa-Ma software yet.
\end{itemize}


\subsection{Initial Performance Metrics}
\label{sec:WP3:Composyx:metrics}

This section provides a summary of initial performance benchmarks performed in the context of WP3. It ensures reproducibility by detailing input/output datasets, benchmarking tools, and the results. All data should be publicly available, ideally with a DOI for future reference.

\begin{itemize}
    \item \textbf{Overall Performance:} Summarize the software's computational performance, energy efficiency, and scalability results across different architectures (e.g., CPU, GPU, hybrid systems).
    \item \textbf{Input/Output Dataset:} we do not have I/O as we generate at runtime the local matrices used for the  benchmark
    \item \textbf{open-data Access:} Indicate whether the datasets used for the benchmark are open access, and provide a DOI or a direct link for download. Where applicable, highlight any licensing constraints.
    \item \textbf{Challenges:} Identify any significant bottlenecks or challenges observed during the benchmarking process, including data handling and computational performance.
    \item \textbf{Future Improvements:} perform more exhaustive experiments on heteroneous nodes, that is using the MPI+StarPU option.
\end{itemize}

\subsubsection{Benchmark \#1}
\begin{itemize}
    \item \textbf{Description:} Briefly describe the benchmark case, including the problem size, target architecture (e.g., CPU, GPU), and the input data. Mention the specific goals of the benchmark (e.g., testing scalability, energy efficiency).
    \item \textbf{Benchmarking Tools Used:} List the tools used for performance analysis, such as Extrae, Score-P, TAU, Vampir, or Nsight, and specify what metrics were measured (e.g., execution time, FLOPS, energy consumption).
    \item \textbf{Input/Output Dataset Description:}
        \begin{itemize}
            \item \textbf{Input Data:} Describe the input dataset (size, format, data type) and provide a DOI or link to access it.
            \item \textbf{Output Data:} Specify the structure of the results (e.g., memory usage, runtime logs) and how they can be accessed or replicated.
            \item \textbf{Data Repository:} Indicate where the data is stored (e.g., Zenodo, institutional repository) and provide a DOI or URL for accessing the data.
        \end{itemize}
    \item \textbf{Results Summary:} Include a summary of key metrics (execution time, memory usage, FLOPS) and their comparison across architectures (e.g., CPU, GPU).
    \item \textbf{Challenges Identified:} Describe any bottlenecks encountered (e.g., memory usage, parallelization inefficiencies) and how they impacted the benchmark.
\end{itemize}

\subsection{12-Month Roadmap}
\label{sec:WP3:Composyx:roadmap}

In this section, describe the roadmap for improving benchmarks and addressing the challenges identified. This should include:
\begin{itemize}
    \item \textbf{Data Improvements:} Plans for improving input/output data management, including making datasets more accessible and ensuring reproducibility through open-data initiatives.
    \item \textbf{Methodology Application:} Implementation of the benchmarking methodology proposed in this deliverable to streamline reproducibility and dataset management.
    \item \textbf{Results Retention:} Plans to maintain benchmark results in a publicly accessible repository with appropriate metadata and documentation, ensuring long-term usability.
\end{itemize}

In~\cref{tab:WP3:Composyx:bottlenecks}, we briefly discuss the bottleneck roadmap associated to the software and relevant to the work package.

\begin{table}[h!]
    \centering
    \centering
    { 
        \setlength{\parindent}{0pt}
        \def\arraystretch{1.25}
        \arrayrulecolor{numpexgray}
        {
            \fontsize{9}{11}\selectfont
            \begin{tabular}{!{\color{numpexgray}\vrule}p{.25\linewidth}!{\color{numpexgray}\vrule}p{.6885\linewidth}!{\color{numpexgray}\vrule}}
    
    \rowcolor{numpexgray}{\rule{0pt}{2.5ex}\color{white}\bf Bottlenecks} &  {\rule{0pt}{2.5ex}\color{white}\bf Short Description }\\ 
    
\rowcolor{white}    B10 - Scientific Productivity & provide short description here \\
\rowcolor{numpexlightergray}    B11 - Reproducibility and Replicability of Computation & Guix-HPC \\
\rowcolor{white}    B6 - Data Management & not applicable \\
\rowcolor{numpexlightergray}    B7 - Exascale Algorithms & Tune CPU and GPU features - Possibly add numerical resiliency \\
\end{tabular}
        }
    }
    \caption{WP3: Composyx plan with Respect to Relevant Bottlenecks}
    \label{tab:WP3:Composyx:bottlenecks}
\end{table}
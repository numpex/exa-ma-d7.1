\section{Software: Freefem++}
\label{sec:WP5:Freefem++:software}

\begin{table}[h!]
    \centering
    { \setlength{\parindent}{0pt}
    \def\arraystretch{1.25}
    \arrayrulecolor{numpexgray}
    {\fontsize{9}{11}\selectfont
    \begin{tabular}{!{\color{numpexgray}\vrule}p{.4\textwidth}!{\color{numpexgray}\vrule}p{.6\textwidth}!{\color{numpexgray}\vrule}}
        \rowcolor{numpexgray}{\rule{0pt}{2.5ex}\color{white}\bf Field} & {\rule{0pt}{2.5ex}\color{white}\bf Details} \\
        \rowcolor{white}\textbf{Consortium} & \begin{tabular}{l}
Sorbonne U\\
\end{tabular} \\
        \rowcolor{numpexlightergray}\textbf{Exa-MA Partners} & \begin{tabular}{l}
Inria PARIS\\
Sorbonne U\\
\end{tabular} \\
        \rowcolor{white}\textbf{Contact Emails} & \begin{tabular}{l}
frederic.hecht@sorbonne-universite.fr\\
pierre-henri.tournier@sorbonne-universite.fr\\
pierre.jolivet@sorbonne-universite.fr\\
\end{tabular} \\
        \rowcolor{numpexlightergray}\textbf{Supported Architectures} & \begin{tabular}{l}
CPU Only\\
\end{tabular} \\
        \rowcolor{white}\textbf{Repository} & \href{https://github.com/FreeFem/FreeFem-sources}{https://github.com/FreeFem/FreeFem-sources} \\
        \rowcolor{numpexlightergray}\textbf{License} & \begin{tabular}{l}
OSS:: LGPL v*\\
\end{tabular} \\
        \rowcolor{white}\textbf{Bottlenecks roadmap} & \begin{tabular}{l}
B10 - Scientific Productivity\\
B11 - Reproducibility and Replicability of Computation\\
B6 - Data Management\\
B7 - Exascale Algorithms\\
\end{tabular} \\
        \bottomrule
    \end{tabular}
    }}
    \caption{WP5: Freefem++ Information}
\end{table}

\subsection{Software Overview}
\label{sec:WP5:Freefem++:summary}

FreeFEM interfaces with various optimization libraries for large scale problems. IPOPT is a software library for non-linear, constrained optimization and implements a primal-dual interior point method along with filter method based line searches. The CMA-ES library implements stochastic strategies for optimization of nonlinear non-convex 'blackbox' functions. FreeFEM is also interfaced with the Toolkit for Advanced Optimization (TAO) from the PETSc library.\\

In~\cref{tab:WP5:Freefem++:features} we provide a summary of the software features relevant to the work package which are briefly discussed.

\begin{table}[h!]
    \centering
    { 
        \setlength{\parindent}{0pt}
        \def\arraystretch{1.25}
        \arrayrulecolor{numpexgray}
        {
            \fontsize{9}{11}\selectfont
            \begin{tabular}{!{\color{numpexgray}\vrule}p{.25\linewidth}!{\color{numpexgray}\vrule}p{.6885\linewidth}!{\color{numpexgray}\vrule}}
    
    \rowcolor{numpexgray}{\rule{0pt}{2.5ex}\color{white}\bf Features} &  {\rule{0pt}{2.5ex}\color{white}\bf Short Description }\\ 
    
\rowcolor{white}    Iterative methods & FreeFEM is interfaced with several optimization libraries such as IPOPT, CM-AES and TAO from PETSc.\\
\end{tabular}
        }
    }
    \caption{WP5: Freefem++ Features}
    \label{tab:WP5:Freefem++:features}
\end{table}


\subsection{Parallel Capabilities}
\label{sec:WP5:Freefem++:performances}

IPOPT is a sequential library. FreeFEM provides a MPI-parallel interface to CM-AES, and TAO relies on the parallel capabilities of the PETSc library.

\begin{itemize}
    \item \textbf{Parallel Environment:} MPI

    \item \textbf{Architectures:} CPU only ; local computing clusters, french supercomputers (Irene, Jean Zay, ...)
 
    \item \textbf{Scalability:} relies on optimization librairies such as IPOPT, CM-AES and TAO from PETSc. Scalability will be measured in terms of number of iterations of the optimization algorithm and computing time.
\end{itemize}


\subsection{Initial Performance Metrics}
\label{sec:WP5:Freefem++:metrics}

In the near future, the plan is to provide results comparing the performance of the different methods available in FreeFEM for a constrained minimal volume optimization problem described below.

\subsubsection{Benchmark \#1: Minimal volume}

The Plateau problem is named after the Belgian natural scientist J. Plateau, who made numerous experiments with soap films, realizing a large variety of minimal surfaces.\\
In this well known problem, the aim is to find a minimal surface or volume whose boundary is fixed. We restrict ourselves to closed and $C^2$ surfaces, parametrized by means of the graph of a given function $u:\bar{\Omega} \rightarrow \mathbb{R}$ (with $\Omega\subset\mathbb{R}^d$, where $d=2$ or $3$) belonging to the admissible set
\[
	S(\Omega) = \left\{ u:\bar{\Omega} \rightarrow \mathbb{R},\quad u_{\vert\Omega} \in C^d(\Omega,\mathbb{R}),\quad u_{\vert\partial\Omega} \in C^1(\partial\Omega,\mathbb{R}) \right\},
\]
so that an admissible surface/volume is given by the parametric surface 
$$
X(\boldsymbol{x}) = (\boldsymbol{x},u(\boldsymbol{x})), \quad \boldsymbol{x} \in \bar{\Omega} .
$$
The cost functional is
\[
	J(u) = \int_{\Omega} (1+\vert\nabla u\vert^2)^\frac{1}{2}\,d\boldsymbol{x} .
\]
Given a function $\gamma\in C^1(\partial\Omega,\mathbb{R})$ along the boundary of $\Omega$, we seek an optimal solution of the Plateau optimization problem 
\begin{align}
    \nonumber
	\min_{u \in S(\Omega)} J(u) \mbox{ such that } u_{\vert\partial\Omega} = \gamma .
\end{align}
This problem can be solved by finding the critical points of the functional $J$ thanks to an iterative procedure, e.g. combining a fixed point method and a Newton method.\\

Such an iterative procedure requires the computation of the gradient, and if possible, the Hessian. Here, the first derivative of the cost functional $J$ is given by
\begin{align}
    \nonumber
	DJ(u)v = \int_{\Omega}  \frac{\nabla u\cdot\nabla v}{(1+\vert\nabla u\vert^2)^{\frac{1}{2}}} \,d\boldsymbol{x} \quad \forall v\in S(\Omega)
\end{align}
and its second order derivative is given by
\begin{align}
    \nonumber
	D^2J(u)(v,w) = \int_{\Omega}  \frac{\nabla v\cdot\nabla w}{(1+\vert\nabla u\vert^2)^{\frac{1}{2}}}  -\frac{(\nabla u \cdot \nabla v)(\nabla u \cdot \nabla w)}{(1+\vert\nabla u\vert^2)^{\frac{3}{2}}} \,d\boldsymbol{x}	\quad \forall v,w \in S(\Omega).
\end{align}	
As an example, we can choose $\Omega$ as the unit square and $\gamma(x,y) = \cos(\pi x)\cos(\pi y)$. This problem is solved using TAO from PETSC in a FreeFEM script available in the distribution at~\url{https://github.com/FreeFem/FreeFem-sources/blob/master/examples/hpddm/minimal-surface-Tao-2d-PETSc.edp}.

\subsection{12-Month Roadmap}
\label{sec:WP5:Freefem++:roadmap}

The plan is to provide results comparing the performance of the different methods available in FreeFEM for a constrained minimal volume optimization problem. In particular, the performance and scalability of TAO solvers will be assessed, both in terms of number of iterations and computing time. The script will be made publicly available.
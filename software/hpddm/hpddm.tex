\section{Software: HPDDM}
\label{sec:HPDDM:software}



\begin{table}[h!]
    \centering
    { \setlength{\parindent}{0pt}
    \def\arraystretch{1.25}
    \arrayrulecolor{numpexgray}
    {\fontsize{9}{11}\selectfont
    \begin{tabular}{!{\color{numpexgray}\vrule}p{.4\textwidth}!{\color{numpexgray}\vrule}p{.6\textwidth}!{\color{numpexgray}\vrule}}
        \rowcolor{numpexgray}{\rule{0pt}{2.5ex}\color{white}\bf Field} & {\rule{0pt}{2.5ex}\color{white}\bf Details} \\
        \rowcolor{white}\textbf{Consortium} & \begin{tabular}{l}
None\\
\end{tabular} \\
        \rowcolor{numpexlightergray}\textbf{Exa-MA Partners} & \begin{tabular}{l}
Inria PARIS\\
Sorbonne U\\
\end{tabular} \\
        \rowcolor{white}\textbf{Contact Emails} & \begin{tabular}{l}
pierre@joliv.et\\
\end{tabular} \\
        \rowcolor{numpexlightergray}\textbf{Supported Architectures} & \begin{tabular}{l}
CPU or GPU\\
\end{tabular} \\
        \rowcolor{white}\textbf{Repository} & \href{https://github.com/hpdomain decomposition methods/hpdomain decomposition methods}{https://github.com/hpdomain decomposition methods/hpdomain decomposition methods} \\
        \rowcolor{numpexlightergray}\textbf{License} & \begin{tabular}{l}
OSS:: LGPL v*\\
\end{tabular} \\
        \rowcolor{white}\textbf{Bottlenecks roadmap} & \begin{tabular}{l}
B10 - Scientific Productivity\\
B11 - Reproducibility and Replicability of Computation\\
B6 - Data Management\\
B7 - Exascale Algorithms\\
\end{tabular} \\
        \bottomrule
    \end{tabular}
    }}
    \caption{HPDDM Information}
\end{table}

\subsection{Software summary}
\label{sec:HPDDM:summary}

HPDDM (High-Performance unified framework for Domain Decomposition Methods) is a library of
advanced Krylov methods and multilevel domain decomposition methods, used as preconditioners, designed for high-performace computing.
The available Krylov methods include CG, block CG, GMRES, and block GMRES.
The available domain decomposition methods include one- and multi-level restricted additive Schwarz (RAS) methods.
The efficiency of these methods has been shown on elliptic PDEs such as the scalar diffusion equation, the linear elasticity equations,
or the Helmhotz equation for wave propagation.
The library can be used through various software for scientific computing:
PETSc, SLEPc, FreeFEM, Feel++, or HTOOL.

\subsection{Purpose}
\label{sec:HPDDM:purpose}

HPDDM aims at providing the scientific community a set of robust and efficient domain decomposition methods
to solve linear systems in parallel.

\subsection{Programming and Computational Environment}
\label{sec::HPDDM:environment_capabilities}


The following table summarizes these aspects for HPDDM, providing a  view of its programming and computational capabilities.

\begin{table}[h!]
    \centering
    {
    \setlength{\parindent}{0pt}
    \def\arraystretch{1.25}
    \arrayrulecolor{numpexgray}
    {\fontsize{9}{11}\selectfont
    \begin{tabular}{lp{.3\textwidth}p{.5\textwidth}}
        \rowcolor{numpexgray}{\rule{0pt}{2.5ex}\color{white}\bf Category}  & {\rule{0pt}{2.5ex}\color{white}\bf Details} & {\rule{0pt}{2.5ex}\color{white}\bf Description}\\
        \rowcolor{white}Languages  & \begin{tabular}{l}
C\\
C++\\
Fortran\\
Python\\
\end{tabular} & Programming languages and language standards supported by the software \\
        \rowcolor{numpexlightergray}Parallelism  & \begin{tabular}{l}
GPU\\
MPI\\
Multithread\\
\end{tabular} & Parallel computing methods and frameworks utilized by the software.\\
        \rowcolor{white}Data Formats  & \begin{tabular}{l}
in-house format\\
\end{tabular} & Data formats that the software can handle or produce.\\
        \rowcolor{numpexlightergray}Resilience  & \begin{tabular}{l}
None\\
\end{tabular} & Fault tolerance and recovery mechanisms employed by the software.\\
        \rowcolor{white}DevOps & \begin{tabular}{l}
Continuous Integration\\
\end{tabular} & Outlines the development and operational practices including continuous integration, containerization, and testing methodologies.  \\
        \rowcolor{numpexlightergray}Packaging  & \begin{tabular}{l}
None\\
\end{tabular} & Software packaging and distribution.\\
        \rowcolor{white}Testing  & \begin{tabular}{l}
Unit\\
Verification\\
\end{tabular} & Testing methodologies employed to ensure software quality and correctness.\\
        \rowcolor{numpexlightergray}Containerization  & \begin{tabular}{l}
None\\
\end{tabular} & Container technologies used to package and deploy the software.\\
        \rowcolor{white}Interfaces  & \begin{tabular}{l}
Feel++\\
Freefem++\\
MUMPS\\
PETSc\\
PaStiX\\
HTOOL\\
\\
\end{tabular} & List of software HPDDM has interfaces with.\\
        \bottomrule
    \end{tabular}
    }}
    \caption{HPDDM programming and computational environment}
\end{table}



\subsection{Mathematics}
\label{sec:HPDDM:mathematics}
\fullcite{dolean_introduction_2015} provides the mathematical foundation of the library, most importantly when it comes to the definition of parameter-robust preconditioner for symmetric positive definite systems.

\subsection{Relevant Publications}
\label{sec:HPDDM:publications}

Here is a list of relevant publications related to the software:
\begin{itemize}
\item \fullcite{jolivet_ddprecond_2013} presents a two-level domain decomposition preconditioner tested to solve elliptic problems in parallel with up to 22 billion unknowns in 2D and 2 billion unknowns in 3D.
\item \fullcite{jolivet_blockrecycling_2016} presents block iterative methods and recycling strategies to solve linear systems with multiple right-hand sides and millions of unknwons while preserving a good scalability.
\item \fullcite{jolivet_ksphpddm-pchpddm_2021} presents KSPHPDDM (Krylov methods) and PCHPDDM (preconditioners) available in PETSc, when configured with HPDDM support.
\end{itemize}
  

\subsection{Acknowledgements}
\label{sec::HPDDM:acknowledgements}

The software has been developed with the support of the following funding agencies and institutions: 

\begin{itemize}
\item Sorbonne Université
\item Institut de Recherche en Informatique de Toulouse
\item INRIA
\item ANR
\item Eidgenössische Technische Hochschule Zürich
\item Université Grenoble Alpes
\end{itemize}

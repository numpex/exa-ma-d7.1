%!TEX root = ../exa-ma-d7.1.tex
%%%%%%%%%%%%%%%%%%%%%%%%%%%%%%%%%%%%%%%%%
%%% Conclusions
%%%%%%%%%%%%%%%%%%%%%%%%%%%%%%%%%%%%%%%%%

\clearpage
\section{Conclusions}
\label{sec:conclusions}

In this deliverable, we have presented the initial benchmarking efforts and methodologies developed within the Exa-MA project, a component of the PEPR NumPEx program. 
The Exa-MA project aims to advance high-performance computing towards and beyond the exascale barrier by developing relevant numerical methods and ensuring that software is production-ready for exascale computing by the project's conclusion.

We introduced a  benchmarking methodology designed to  address the challenges and bottlenecks associated with exascale computing. 
This methodology focuses on several critical aspects:

\begin{itemize}
    \item \textbf{Development of Numerical Methods and Algorithms}: Advancing state-of-the-art methods optimized for exascale architectures.
    \item \textbf{Testing and Validation Processes}: Establishing structured non-regression testing, verification, and validation to ensure correctness and performance integrity.
    \item \textbf{Benchmarking Strategy}: Implementing a phased approach to measure performance, scalability, and energy efficiency on advanced computational architectures.
    \item \textbf{Data Management and I/O Strategies}: Addressing I/O bottlenecks through efficient data management techniques.
    \item \textbf{Profiling and Measurement Tools}: Utilizing advanced tools to gather detailed performance insights.
    \item \textbf{Containerization and Packaging}: Ensuring reproducible and portable execution environments.
    \item \textbf{Continuous Integration and Deployment (CI/CD)}: Maintaining reproducibility and sustained high performance through continuous benchmarking and regression testing.
    \item \textbf{Fault Tolerance Strategies}: Enhancing system resilience and data integrity.
\end{itemize}

The document provided an overview of the software developed within Exa-MA, highlighting their features, parallel capabilities, and the technologies employed. General statistics offered insights into their characteristics and technological choices, emphasizing the diversity and adaptability of the project's software components.

In the work package chapters (WP1 to WP6), we presented software relevant to each focus area, detailing the numerical methods and algorithms they implement, parallel capabilities, benchmarks and metrics developed, challenges identified, and a 12-month roadmap for each. 
This  approach ensures that all aspects of exascale computing, from discretization to uncertainty quantification, are addressed cohesively.

The initial benchmarking results present the initial state in developing exascale-ready numerical methods and software tools. 
The challenges identified during this process have informed our future work.
Depending on the software, the presentation of the results vary and we have in general an unbalanced presentation not only within each workpackage  but also across the workpackages.
We do not feel that this is a problem, as the various software are in different stages of development and benchamarking and the benchmarking results are not yet all readily available.
However, during the coming months we should strive to setup the methodology for all software and be able any moment to present current results.
The next release of this documentation should have a more balanced systematic presentation of the results within and across the workpackages.


\section*{Future Work}

Looking ahead, the Exa-MA project will continue to refine and expand upon the methodologies and tools presented in this deliverable. 
Our future efforts will focus on:

\begin{itemize}
    \item \textbf{Incorporating new methods and algorithms}: Developing and integrating novel numerical methods and algorithms optimized for exascale architectures.
    \item \textbf{Enhancing Scalability and Performance}: Further optimizing numerical methods and software tools to fully leverage emerging exascale platforms, ensuring efficient utilization of computational resources.
    
    \item \textbf{Extending Benchmarking Metrics}: Incorporating new metrics and evaluation criteria relevant to exascale computing, such as energy consumption, resilience, and data movement efficiency.
    \item \textbf{Fault Tolerance Mechanisms}: Implementing fault tolerance strategies to maintain resilience in the face of hardware and software failures common in exascale environments.
    \item \textbf{Strengthening Community Collaboration}: Engaging with the broader HPC community to share insights, tools, and best practices, fostering a collaborative ecosystem that accelerates progress.
    \item \textbf{Addressing Identified Challenges}: Focusing research and development efforts on overcoming the specific bottlenecks and challenges identified in this initial benchmarking phase.
\end{itemize}

\section*{Final Remarks}

This deliverable marks a significant step towards advancing exascale computing by providing a structured approach to the development and benchmarking of numerical methods, algorithms, and software. By adhering to the proposed benchmarking methodology and continuously improving our approaches, we aim to make substantial contributions to the field of high-performance computing.

The progress achieved thus far in some of the software demonstrates the feasibility and importance of a methodical approach to overcoming the challenges of exascale computing. 


%%%%%%%%%%%%%%%%%%%%%%%%%%%%%%%%%%%%%%%%%
%!TEX root = ../exa-ma-d7.1.tex
%%%%%%%%%%%%%%%%%%%%%%%%%%%%%%%%%%%%%%%%%
%%% List of Abbreviations
%%%%%%%%%%%%%%%%%%%%%%%%%%%%%%%%%%%%%%%%%

\clearpage
\section*{List of Abbreviations}
\label{sec:abbreviations}

\begin{acronym}[ABCDEF]
	\acro{exama}[\textsc{Exa-MA}]{Methods and Algorithms for Exascale Computing}
	\acro{DoA}{Description of Action}
	\acro{EC}{European Commission}	
	\acro{FAIR}{Findable, Accessible, Interoperable, and Reusable}
	\acro{WP}{Work Package}
	%% Bottlenecks for Exa-MA
	\acro{B1}{Energy Efficiency \acroextra{: Develop energy-efficient technologies to meet the 20 MW target for exascale systems.}}
	\acro{B2}{Interconnect Technology \acroextra{: Improve vertical (intra-node) and horizontal (inter-node) data movement for better energy efficiency and performance.}}
	\acro{B3}{Memory Technology \acroextra{: Integrate new memory technologies (e.g., PCRAM, NOR Flash, ReRAM, memristor) to improve capacity, bandwidth, resiliency, and energy efficiency.}}
	\acro{B4}{Scalable System Software \acroextra{: Increase the scalability, power sensitivity, and resiliency of system software, including operating systems, runtime systems, and monitoring systems.}}
	\acro{B5}{Programming Systems \acroextra{: Develop new programming paradigms to express fine-grained concurrency, locality, and resilience.}}
	\acro{B6}{Data Management \acroextra{: Develop software that handles massive amounts of data, addressing both offensive I/O (e.g., data analysis and compression) and defensive I/O (e.g., fault tolerance).}}
	\acro{B7}{Exascale Algorithms \acroextra{: Redesign algorithms to improve scalability by reducing communication, avoiding or hiding synchronization, and enhancing computational efficiency on accelerators.}}
	\acro{B8}{Discovery, Design, and Decision Algorithms \acroextra{: Focus research on ensembles of small runs, as used in uncertainty quantification and parameter optimization, rather than only on single heroic simulations.}}
	\acro{B9}{Resilience, Robustness, and Accuracy \acroextra{: Ensure that computations are correct, reproducible, and verifiable, even in the presence of software and hardware errors.}}
	\acro{B10}{Scientific Productivity \acroextra{: Provide scientists with tools to use exascale systems productively, including program development, application execution, input preparation, output collection, and result analysis.}}
	\acro{B11}{Reproducibility and Replicability of Computation \acroextra{: Ensure that research results are reproducible and that data and codes are provided so others can re-obtain the same results.}}
	\acro{B12}{Pre/Post Processing \acroextra{: Enable efficient visualization and in situ processing for large-scale simulations.}}
	\acro{B13}{Integration of Uncertainties \acroextra{: Create opportunities to integrate uncertainties directly into the core of calculations for unseen impacts.}}
	%% Objectives for Exa-MA
	\acro{O1}{Objective 1 \acroextra{: Develop methods, algorithms, and implementations that exploit exascale architectures to enhance modeling, solving, data assimilation, optimization, and uncertainty quantification beyond current capabilities.}}
	\acro{O2}{Objective 2 \acroextra{: Develop or contribute to software libraries that assemble critical reusable components, hiding hardware complexity while exposing only the methodological interfaces.}}
	\acro{O3}{Objective 3 \acroextra{: Identify and co-design Methodological and Algorithmic Patterns at exascale that can be efficiently reused in large-scale applications (e.g., weather forecasting).}}
	\acro{O4}{Objective 4 \acroextra{: Enable AI algorithms to achieve exascale performance by leveraging the methods (O1) and libraries (O2) developed in the project.}}
	\acro{O5}{Objective 5 \acroextra{: Provide demonstrators through mini-apps and proxy-apps that will be openly available and benchmarked to showcase exascale readiness.}}

\end{acronym}

%%%%%%%%%%%%%%%%%%%%%%%%%%%%%%%%%%%%%%%%%
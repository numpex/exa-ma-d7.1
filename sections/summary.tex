%!TEX root = ../exa-ma-d7.1.tex
%%%%%%%%%%%%%%%%%%%%%%%%%%%%%%%%%%%%%%%%%
%%% Executive Summary
%%%%%%%%%%%%%%%%%%%%%%%%%%%%%%%%%%%%%%%%%
\clearpage
\section*{Executive Summary}
\addcontentsline{toc}{section}{Executive Summary}
\label{sec:summary}

\noindent
\textbf{Scope.} This edition (v2025) of Deliverable D7.1 consolidates, in a single annual
\emph{mega-deliverable}, the scientific advances of WP1--WP6 and their software production,
benchmarking and methodology (WP7). It provides a 360° view: methods \textrightarrow\ software
\textrightarrow\ mini-apps/demonstrators \textrightarrow\ metrics, with traceable artefacts (code, containers,
datasets, scripts, logs). % see Exec. Summary and ToC for structure
% (source: D7.1 structure and methodology)
% (PDF executive summary and Chapter 8)
% 

\medskip
\noindent
\textbf{Highlights (2024--2025).}
\begin{itemize}
  \item \emph{WP1 Discretization.} Robust meshing and adaptation at scale, including
        feature-aware workflows and fully parallel anisotropic adaptation with ParMmg (Fichera),
        plus a high-order wave proxy-app (CPU/GPU). % WP1§2.1--2.3
  \item \emph{WP2 MOR/Surrogates/SciML.} PINNs with domain decomposition, FNO-based
        warm-starts for nonlinear solves, nonlinear ROM with neural implicit representations. % WP2§3.2
  \item \emph{WP3 Solvers.} GenEO extensions to non-SPD systems; mixed-precision and
        variable-accuracy Krylov (PROMISE); fault-tolerant PCG patterns; partitioned coupling
        building blocks. % WP3§4.2
  \item \emph{WP4 Data Assimilation.} Launch of stochastic inversion (EKI mini-app spec) with
        manufactured tests and ensemble-MPI design. % WP4§5.3
  \item \emph{WP5 Optimization.} Fractal continuous optimization at scale; parallel Bayesian
        optimization for expensive objectives; neural level-set shape optimization. % WP5§6.2
  \item \emph{WP6 UQ.} Multi-output GP surrogates with physical constraints; graph-indexed
        kernels (SWWL) for mesh/graph outputs; categorical-kernel benchmarks. % WP6§7.2--7.5
\end{itemize}

\medskip
\noindent
\textbf{Applications and demonstrators (selection).}
Thermal bridges validation (Feel++), FDA nozzle (Navier–Stokes), geometry distance/BVH vs FMM,
parallel anisotropic mesh adaptation (ParMmg), reduced-basis thermal fin, and HPDDM–DFN. % Ch.10

\medskip
\noindent
\textbf{Software production and benchmarking.}
We apply a unified methodology (Chap.~8): CI/CD with nightly/non-regression and ReFrame,
containerized runs (Docker/Apptainer), FAIR data/DOIs (Zenodo), energy/perf/scaling metrics,
and acceptance criteria per application tier (mini-app, extended mini-app, demonstrator). % Ch.8

\medskip
\noindent
\textbf{KPI snapshot (targets and current status).}
\begin{center}
\begin{tabular}{p{4cm} p{4cm} p{6cm}}
\toprule
\textbf{KPI} & \textbf{Target} & \textbf{Current Status (2025)} \\
\midrule
Level 1 demonstrators at scale & At least 3 per WP (except WP7) & 2 mini-apps defined specified (WP1), implemented and  part of the benchmarking process \\
Level 2 demonstrators at scale & At least 5 total &  6 extended mini-apps and demonstrators specified, 2 implemented and 1 part of the benchmarking process \\
Production applications with Exa-MA tools & 5 mini-apps, 4 proxy apps; ~20 libraries impacted &  about 12 frameworks impacted, 1 proxy app under development, \\
Software stack validation scale & Full Exascale level & Not measured for now \\
Training participants & More than 100 trainees (PC0 KPI-5) &  1 training on CI/CD in HPC environment March 2025 (40 particpants via EuroHPC)\\
Scientific publications & More than 50 papers & 36 Publications comprise 16 preprints/working papers, 8 journal articles, 4 conference papers, 3 proceedings, 2 poster communications, 1 lecture, 1 book, and 1 report. \\
NumPEx ecosystem contribution & Conformity with PC0 KPI-7--10 & collaborations between PC1/PC2/PC3; upstream patches merged into identified frameworks \\
\bottomrule
\end{tabular}
\end{center}

\medskip
\noindent
\textbf{Why it matters.} (i) \emph{Sovereign, open, reproducible} stack at scale; (ii) \emph{Measurable
impact} via transparent KPIs and re-usable artefacts; (iii) \emph{Ecosystem leverage} across FR/EU/US
software with upstream contributions.

\medskip
\noindent
\textbf{Outlook (next 12 months).}
Include new application specifications, implementations and benchmarking, broaden GPU portability, densify auto-bench
pipelines; keep annual refresh with KPI deltas; common milestones with PC1/PC2/PC3.



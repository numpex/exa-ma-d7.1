%!TEX root = ../exa-ma-d7.1.tex
%%%%%%%%%%%%%%%%%%%%%%%%%%%%%%%%%%%%%%%%%
%%% Introduction
%%%%%%%%%%%%%%%%%%%%%%%%%%%%%%%%%%%%%%%%%

\clearpage
\section{Introduction}
\label{sec:introduction}

%%%%%%%%%%%%%%%%%%%%%%%%%%%%%%%%%%%%%%%%%
%%% Section content, please change!
%%%%%%%%%%%%%%%%%%%%%%%%%%%%%%%%%%%%%%%%%

\subsection{Purpose and Scope of the Document}
\label{sec:purpose}

The Exa-MA project, part of the PEPR NumPEx program funded under France’s Plan d’Investissement d’Avenir, is dedicated to advancing the frontier of high-performance computing towards and beyond the exascale barrier.
This deliverable, D7.1, represents the first in a series of annual benchmarking reports designed to evaluate the performance of key software tools and libraries that will underpin the next generation of scientific computing.

Exascale computing promises significant breakthroughs across a wide range of disciplines, from climate science to materials physics, by enabling simulations and analyses at unprecedented scales and resolutions.
However, realizing the full potential of exascale computing requires overcoming substantial challenges in software architecture, scalability, and efficiency.
Benchmarking plays a crucial role in this context by providing a rigorous framework to measure, analyze, and optimize the performance of computational tools on state-of-the-art hardware architectures.

In this deliverable, we will assess several key software components on their ability to leverage advanced computational architectures (CPU, GPU, and Hybrid systems) effectively.
Our focus is on elucidating each software’s features, its parallel computing capabilities, and initial performance metrics as these tools move towards exascale readiness.
This evaluation will inform ongoing development efforts and help streamline the transition of these tools to fully exploit the capabilities of emerging exascale systems.

Through testing and analysis, we aim to ensure that the Exa-MA suite of tools not only meets the high computational demands of future exascale applications but also adheres to the principles of energy efficiency, scalability, and robustness required in the exascale era.


\subsection{Structure of the Document}
\label{sec:structure}

% This document is organised as follows: \Cref{sec:introduction} provides information about the report content whereas \Cref{sec:guidelines} introduces guides for writing \projacronym{} deliverables. The deliverable is concluded in \Cref{sec:conclusions}. Finally, additional information is provided in the \nameref{sec:app:architectures} and \nameref{sec:appendix-b}.

%%%%%%%%%%%%%%%%%%%%%%%%%%%%%%%%%%%%%%%%%
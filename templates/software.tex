\clearpage
\chapter{Software}
\label{sec:software}

This chapter presents the software developed within Exa-MA, focusing on features, mathematics, functionalities, publications, acknowledgments, and contact details.

First, we present some general statistics about the software identified for Exa-MA that will present some benchmarking results. 
Other software (new or existing) will be included in the future versions of this document.


\section{General Statistics}
\label{sec:software:statistics}

In this section, we provide an overview of the key characteristics and technological choices for the software developed and benchmarked within Exa-MA. 
These statistics offer insights into the diversity of hardware architectures, programming languages, and parallel computing technologies utilized across the different software packages. 

The aim is to highlight the widespread usage of various technologies, demonstrating both the flexibility and the breadth of approaches within the project. 
Additionally, the DevOps practices employed, such as continuous integration, testing, and deployment, are presented to underscore the commitment to ensuring quality, reliability, and maintainability of the software developed under Exa-MA.

The following subsections shows different aspects of the software involved in the project, from supported architectures and programming languages to specific parallelism technologies, data formats, and DevOps strategies. 
It helps to assess the readiness of the software for large-scale simulations and benchmarks in exascale computing environments.


\subsection{Architectures}

The following pie chart~\ref{fig:arch} shows the distribution of hardware architectures used for the benchmarks.

\VAR{software.arch_chart}

\subsection{Programming Languages}

The following pie chart~\ref{fig:languages} shows the distribution of programming languages used, highlighting the variety of computational solutions employed.

\VAR{software.languages_chart}

\subsection{Parallelism Technology}


The pie chart~\ref{fig:parallelism} below represents the parallelism techniques used in Exa-MA software selected for this document.

\VAR{software.parallelism_chart}


\subsection{Data Formats}

The chart~\ref{fig:data} shows the supported data formats, for flexibility and compatibility in data handling, supported by Exa-MA software selected for this document.

\VAR{software.data_chart}

\subsection{DevOps - CI/CD}

The pie chart~\ref{fig:devops-cicd} below displays the support of continuous integration and deployment practices as well as continuous benchmarking, showcasing systematic software updates, quality maintenance and performance regression.

\VAR{software.devops_cicd_chart}

\subsection{DevOps - Packaging}

The next chart~\ref{fig:devops-packaging} shows different packaging methods used, which help in the distribution and management of software.

\VAR{software.devops_packaging_chart}

\subsection{DevOps - Containers}

The pie chart~\ref{fig:devops-containers} displays the use of container technologies, which help encapsulate the software to run reliably in various environments.

\VAR{software.devops_containers_chart}

\subsection{DevOps - Testing}

The following pie chart~\ref{fig:devops-testing} details the testing practices adopted, illustrating the commitment to software reliability and functionality.

\VAR{software.devops_testing_chart}

\VAR{software.software}

